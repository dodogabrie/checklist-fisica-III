\documentclass[10pt,a4paper]{article}
\usepackage{float}
\usepackage{verbatim}
\usepackage[utf8]{inputenc}
\usepackage[italian]{babel}
\usepackage{amsmath}
\usepackage{amsbsy}
\usepackage{amsfonts}
\usepackage{amssymb}
\usepackage{graphicx}
\usepackage[left=2cm,right=2cm,top=2cm,bottom=2cm]{geometry}
\usepackage{xcolor}
\usepackage{amsthm}
\usepackage{mhchem}
\usepackage[hypertexnames=false]{hyperref}
\usepackage{nameref}
\usepackage{multicol}
% per le immagini
\usepackage{import}
\usepackage{xifthen}
\usepackage{pdfpages}
\usepackage{transparent}




\hypersetup{
    pdftitle={title},
    pdfsubject={subject},
    pdfauthor={author},
}

% comandi richiamati
\newcommand{\numberset}{\mathbb}
\newcommand{\Q}{\numberset{Q}}
\newcommand{\I}{\numberset{I}}
\renewcommand\thepart{\arabic{part}}
\renewcommand\thesection{\alph{section}}
\renewcommand\thesubsection{\thepart.\thesection.\arabic{subsection}}
\newcommand{\parallelsum}{\mathbin{\|}}
\newcommand{\bs}{\boldsymbol}

\newcommand{\incfig}[1]{%
    \def\svgwidth{\columnwidth}
    \import{./figures/}{#1.pdf_tex}
}

\author{Edoardo Gabrielli}
\title{Checklist per l'esame di Fisica 3}

\begin{document}

\maketitle
Ispirata alla Checklist creata in precedenza da Giuseppe Bogna e Edoardo Centamori. 

\begin{center}
	\paragraph{Sommario}
\end{center}
Questo pdf contiene le domande con risposta per l'esame di fisica 3 poste nell'anno accademico 2019/2020. È presente la risposta alle sole 4 domande di tipo C da me selezionate ed alcune delle risposte alle domande B può essere incompleta. 

\clearpage

\tableofcontents

\listoffigures
\clearpage


\pagenumbering{arabic}
\part{Prerequisiti}
\section{Domande a}
\subsection[]{ Definire le quantità $\beta$ e $\gamma$ per le trasformazioni di Lorentz.} 
Presi due sistemi di riferimento inerziali $O$ ed $O'$ si ha che $\beta$ è la velocità del sistema $O'$ rispetto ad $O$ (in unità di c) mentre:
\[
	\gamma = \frac{1}{\sqrt{1-\beta^2}}
.\] 

\subsection[]{ Definire il prodotto scalare di due 4-vettori.} 
Presi $x^{\mu} = (x^0,\boldsymbol{x})$ e $y^{\mu} = ( y^0, \boldsymbol{y})$ si definisce il loro prodotto come $x^{\mu}y_{\nu}$ come:
\[
x^{\mu}y_{\nu} = x^{0}y^{0}-\boldsymbol{x} \cdot \boldsymbol{y} 
.\] 
\subsection[]{ Definire il modulo di un 4-vettore.}
Se $x^{\mu}$ è un 4-vettore il suo modulo è definito secondo la metrica:
\[
|x|^2 = x^{\mu}x_{\mu}
.\] 
Dato che il tensore metrico non è definito positivo il modulo di un 4-vettore può esser positivo, negativo o nullo. 
\subsection[]{ Scrivere le trasformazioni di Lorentz per il boost lungo un asse (asse x).}
Per un boost lungo l'asse x si ha:
\[
\begin{cases}
	ct' = \gamma ( ct - \beta x) \\	
	x' = \gamma (x - \beta ct ) \\
	y' = y \\
	z' = z
\end{cases}
\]
o in forma matriciale:

\[
\left( \begin{array}{c} ct' \\ x' \\ y' \\ z' \end{array} \right)
= 
\begin{pmatrix}
	\gamma  & -\gamma \beta  & 0 & 0\\
	- \gamma \beta & \gamma & 0 & 0\\
	0 & 0 & 1 & \\
	0 & 0 & 0 & 1 \\
\end{pmatrix}
\left( \begin{array}{c} ct \\ x \\ y \\ z  \end{array} \right)
\] 

\subsection[]{ Definire le derivate in 4-dimensioni, la quadridivergenza, il differenziale di uno scalare di Lorentz, l’operatore di D’Alembert.}
Si definisce gli operatori $\partial^{\mu}$ e $\partial_{\mu}$ come:
\[
	\partial^{\mu} = \left( \frac{1}{c} \frac{\partial}{\partial t} , - \nabla  \right) ,\quad
	\partial_{\mu} = \left(\frac{1}{c} \frac{\partial}{\partial t} ,  \nabla  \right)
.\] 
Preso un generico campo tensoriale, la sua 4-divergenza (rispetto a qualche indice) è la contrazione tra l'operatore $\partial^{\mu}$ e l'indice stesso (se ques'ultimo è covariante) oppure tra l'operatore $\partial_{mu}$ e l'indice stesso (se quest'ultimo è controvariante). Ad esempio preso il campo vettoriale $v^{\mu} \equiv \left(v^{0}, \boldsymbol{v}\right)$ la sua 4-divergenza sarà:
\[
\partial_{\mu}v^{\mu} = \frac{1}{c} \frac{\partial v^{0}}{\partial t} + \nabla \boldsymbol{v} 
.\]
Se invece $\phi$ è un invariante di Lorentz il suo differenziale è:
 \[
	 d\phi = dx^{\mu}\partial_{\mu}\phi = \frac{\partial\phi}{\partial t}dt + \left(d \boldsymbol{x}\cdot \nabla \right)\phi  
.\]
L'operatore di D'Alembert invece è: 
\[
	\Box = \partial_{\mu}\partial^{\mu} = \frac{1}{c^{2}}\frac{\partial^{2}}{\partial t^{2}}    
.\] 
\subsection[]{ Definire il tempo proprio e dare la relazione (differenzale) fra tempo proprio e tempo nel sistema in cui si osserva il moto.}
Si sa che l'intervallo $ds^{2}$ è uno scalare di Lorentz, inoltre per un sistema solidale (o tangente) si ha $dx^{\mu} = \left( cd\tau, 0 \right)$ con $d\tau$ il tempo proprio infinitesimo. Si ha quindi: 
\[
		c^{2}d\tau^{2} = c^{2}d\tau^{2} - |dx|^{2} = c^{2}dt^{2} -|\boldsymbol{v}|^{2}dt^{2} \implies d\tau = \frac{dt}{\gamma} 
.\] 
\subsection[]{ Dare la definizione di invariante di Lorentz.} 
Un invariante di Lorentz è una grandezza che viene lasciata invariata dalle trasformazioni di Lorentz: è uguale in tutti i sistemi di riferimento inerziali.
\subsection[]{ Definire la 4-velocità ed il 4-impulso di un punto materiale di massa m, esprimere le loro unità di misura nei sistemi MKS e $\hbar$ = c = 1 , dimoststrare che il loro modulo è costante.} 
In MKS la 4-velocità ed il 4-impulso sono rispettivamente:
\[
	u^{\mu} = \frac{dx^{\mu}}{d\tau} = \left( \gamma c, \gamma \boldsymbol{v}  \right) \left( \frac{[m]}{[s]}\right) , \quad \quad 
	p^{\mu} = mu^{\mu} \left( \frac{[kg] \cdot [m]}{[s]} \right) 
.\]
Mentre in unità di $\hbar$ = c = 1 si ha che $u^{\mu}$ è adimensionale mentre $p^{\mu}$ di misura in $kg$.
Dimostriamo che il modulo di questi due è costante:
\[
	u^{\mu}u_{\mu} = \gamma^{2}\left( c^{2} - \boldsymbol{v^{2}} \right) = c^{2} 
\]
 \[
	p^{\mu}p_{\mu} = mc^2
\] 
\subsection[]{ Enunciare la legge di conservazione del 4-impulso.}
Per un sistema isolato (ovvero non sottoposto a forze esterne) il 4-impulso totale si conserva nel tempo.
\subsection[]{ Dare la definizione di 4-vettore covariante e controvariante; definire un tensore covariante di rango 2 e la sua traccia.}
\label{sec:1.a.10}
Un tensore covariante è il prodotto tensoriale di due quadrivettori covarianti. Di conseguenza un tensore covariante $F_{\mu \nu}$ trasforma sotto trasformazioni di Lorentz come:
\[
	F'_{\mu \nu} = \Lambda_{\mu}^{\alpha}\Lambda_{\nu}^{\beta} F_{\alpha \beta}  
.\]
La traccia di $F$ è 
\[
	F^{\mu}_{\mu} = g^{\mu \nu} F_{\mu \nu}
\] 
\subsection[]{Definire il tensore metrico $g_{\mu \nu}$}
Il tensore metrico si definisce come: $g_{\mu \nu} = $ diag$\left( 1,-1,-1,-1 \right) $ 
\subsection[]{Dare la definizione di tensore antisimmetrico di rango 2 ed indicare quali dei suoi elementi siano le componenti di un vettore polare e quali quelle di un vettore assiale tridimensionale.}
Un tensore antisimmetrico $F^{\mu \nu}$ di rango 2 è un tensore che cambia segno sotto scambio di indici:
\[
	F^{\mu \nu} = -F^{\nu \mu}
\]
Nel caso particolare in cui $F^{\mu \nu}$ è della forma:
\[
	F^{\mu \nu} = 
\left( 
\begin{array}{c|ccc}
	0 & v_{x} & v_{y} & v_{z} \\
	\hline
	-v_{x} & 0 & -w_{z} & w_{y} \\
	-v_{y} & w_{z} & 0 & -w_{x} \\
	-v_{z} & -w_{y} & w_{x} & 0
\end{array}
\right) 
\] 
Allora il vettore $\boldsymbol{v}$ è polare (invariante sotto parità) mentre il vettore $\boldsymbol{w}$ è assiale ("contro"variante sotto parità: cambia segno).
\subsection[]{ Definire quando una legge è scritta in modo relativisticamente covariante.}
Una legge fisica è scritta in modo relativisticamente covariante se è una uguaglianza tra due oggetti che trasformano allo stesso modo sotto cambi di sistema di riferimento, ossia se i due oggetti hanno gli stessi indici covarianti e controvarianti. 
\subsection[]{ Enunciare o ricavare la legge relativistica di composizione delle velocità.}
Supponiamo di avere un corpo puntiforme e siano $\boldsymbol{v}$ e $\boldsymbol{v'}$ le sue velocità nei sistemi inerziali $O$ e  $O'$.
Se  $\boldsymbol{w} = w \hat{x}$ è la velocità di $O'$ rispetto ad $O$ allora si ha 
\[
	v'_{x} = \frac{v_{x} - w}{1 - v_{x}w/c^{2}} \quad \quad
	v'_{y} = \frac{v_{y}}{\gamma \left( 1 - v_{x}w/c^{2} \right) } \quad \quad
	v'_{z} = \frac{v_{z}}{\gamma \left( 1 - v_{x}w/c^{2} \right) }
\]
con $\gamma = \frac{1}{\sqrt{1 - w^{2}/c^{2}}}$

\subsection[]{ Dimostrare che il modulo di un 4-vettore ed il prodotto di due 4-vettori sono invarianti di Lorentz.} 
È sufficiente dimostrare che il prodotto di due 4-vettori è invariante:
\[
	x'^{\mu} y'_{\mu} = \Lambda^{\mu}_{\alpha} g_{\mu \beta} \Lambda^{\beta}_{\gamma} x^{\alpha} y^{\gamma}
.\]
Inoltre il gruppo di Lorentz può esser definito come il gruppo che lascia invariato la metrica di Minkowsky, quindi:
\[
	\Lambda^{\mu}_{\alpha} g_{\mu \beta} \Lambda^{\beta}_{\gamma} = g_{\alpha \gamma} \implies x'^{\mu}y'_{\mu} = x^{\mu}y_{\mu}
\]  
\subsection[]{ Spiegare il paradosso dei gemelli.} 
Consideriamo i gemelli Bob e Alice. Supponiamo che la prima rimanga sulla terra (supposta sistema inerziale) e che Bob parta per una stella lontana a velocità costante. Per Alice l'orologio di Bob è rallentato dunque lei pensa che al ritorno di Bob ella sarà più giovane di lui.\\ 
Dal punto di vista di Bob è invece l'orologio di Alice ad essere rallentato (che nel suo sistema si allontana da lei alla velocità della nave) quindi pensa in maniera opposta ad Alice: crede che sarà lui il più giovane al suo ritorno.\\
Il paradosso nasce dalla erroneità della seconda affermazione (quella fatta da Bob): il sistema di Bob non può essere inerziale perchè dovrà necessariamente accelerare per tornare indietro. Quindi al ritorno Bob è più vecchio di Alice e, facendo un diagramma di Minkowsky, si vede che l'invecchiamento di Bob è tutto dovuto alla fase di accelerazione e decelerazione della nave.
\subsection[]{Dimostrare che l'operatore di D'Alembert è un invariante di Lorentz.} 
Se si considera l'operatore di D'Alembert come il prodotto di quadrivettori allora la dimostrazione è già stata effettuata nel punto $(1.a.15)$.
\subsection[]{Dimostrare che la 4-accelerazione e la 4-velocità sono perpendicolari.} 
Visto che $u^{\mu}u_{\mu} = c^{2}$ possiamo derivare a destra e sinistra rispetto al tempo proprio ottenendo:
\[
	0 = \frac{\mbox{d} u^{\mu}u_{\mu}}{\mbox{d} \tau} = u^{\mu}a_{\mu} + a^{\mu}u_{\mu} = 2 u^{\mu}a_{\mu}
\] 
Da quest'ultima relazione si evince che 4-velocità e 4-accelerazione sono perpendicolari.
\subsection[]{ Quanto valgono in MKS e in CGS le costanti: c, $\epsilon_0$,  $\mu_0$, $e^2/4\pi$,  $\hbar$?} 
In MKS si ha:
\[
	c = 3 \cdot 10^{8} \quad \text{m/s}
\] 
\[
	\epsilon_0 = 8.854 \cdot 10^{-12} \quad \text{F/m}
\]
\[
	\mu_0 = 4\pi \cdot 10^{-7} \quad \text{H/m}
\] 
\[
	\frac{e^{2}}{4\pi} = 2.04 \cdot 10^{-39} \quad \text{C}^{2}
\] 
\[
	\hbar = 1.05 \cdot 10^{-34} \quad \text{J} \cdot \text{s}
\] 
In CGS invece:
\[
	c = 3 \cdot 10^{10} \quad \text{cm/s}
\] 
\[
	\epsilon_0 = \frac{1}{4\pi}
\]
\[
	\mu_0 = \frac{4\pi}{c^{2}} 
\] 
\[
	\frac{e^{2}}{4\pi} = 1.83 \cdot 10^{-20} \quad esu^{2} \quad \text{(con 1 esu = 1 Am/c = $10^{-8}$ cm/c)} 
\] 
\[
	\hbar = 1.05 \cdot 10^{-27} \quad \text{erg} \cdot \text{s}
\]

\subsection[]{Quanto vale entro il 5\% la costante $\hbar c$ in eV-nm e in MeV-fm?}
La costante $\hbar c$ vale:
\[
	\hbar c = 197 \text{ eV/nm} = 197 \text{ MeV/fm}
\]  

\subsection[]{Spiegare la differenza tra le seguenti categorie di fotoni: infrarossi - visibili - ultravioletti - raggi X - raggi $\gamma$. } 
La differenza tra le categorie sta nella energia (o equivalentemente nella frequenza):
\begin{table}[H]
	\centering
	\label{tab: fotoni}
	\begin{tabular}{ccc}
		Fotoni & Frequenza [Hz] & Energia [eV] \\
		\hline
		infrarosso & $5 \cdot 10^{11} - 4 \cdot 10^{14}$ & $2 \cdot 10^{-3} \sim 1.5$  \\
		visibile & $4 \cdot 10^{14} - 8 \cdot 10^{14}$ & $1.5 \sim  3$ \\
		ultravioletto & $8 \cdot 10^{14} - 3 \cdot 10^{17}$ & $3 \sim 10^3$ \\
		raggi X & $3 \cdot 10^{17} - 5 \cdot 10^{19}$ & $ 10^3 \sim 2 \cdot 10^{5} $ \\
		raggi $\gamma$ & $\ge 5 \cdot 10^{19}$ & $\ge 10^5$
	\end{tabular}
\end{table}

\subsection[]{ Quanto vale la massa del fotone?}
Il fotone ha massa nulla.
\subsection[]{ Quanto valgono, entro il 5\%, la carica elettrica dell’elettrone e del protone (in MKSA)?}
La carica dell'elettrone vale 
\[
	e = -1.602 \cdot 10^{-19} \quad \text{C}
\]
 mentre quella del protone vale l'opposto. 

\subsection[]{ Quanto vale, entro il 5\%, la costante di struttura fine ($\alpha$)?}
\[
	\alpha = \frac{e^{2}}{4\pi \epsilon_0 \hbar c} = 7.29 \cdot 10^{-3} \approx \frac{1}{137}
\] 	
\subsection[]{ Quanto valgono, entro il 10\%, la massa dell’elettrone e del protone (in MKS e in MeV/$c^2$)?}
\[
m_{e} = 0.511 \text{ MeV/$c^{2}$} = 9.11 \cdot 10^{-31} \text{ kg}
\] 
\[
m_{p} = 938 \text{ MeV/$c^{2}$} = 1.67 \cdot 10^{-27} \text{ kg}
\] 
\subsection[]{Dire se la differenza fra la massa del neutrone e la somma della massa del protone e dell’elettrone sia: 1 MeV; 10 MeV; 100 MeV oppure negativa.}
\[
	m_n - \left( m_p + m_e \right) \approx 1 \text{ Mev} 
\] 
\subsection[]{Quanto è l’ordine di grandezza dell’energia media di legame di un elettrone all’interno di un atomo?}
L'energia di legame di un elettrone all'interno di un atomo varia tra 1 e 100 eV, tale energia è tendenzialmente più vicina ad 1 eV.
\subsection[]{Spiegare la differenza fra ottica fisica ed ottica geometrica}
L'ottica geometrica studia i fenomeni ottici assumendo che la luce si propaghi mediante raggi rettilinei (riflessione, rifrazione).\\ 
L'ottica fisica è la branca dell'ottica che studia i fenomeni in cue emerge la natura ondulatoria della luce (interferenza, diffrazione).\\
Nel limite in cui le dimensioni lineari deglio oggetti studiati siano molto maggiori della lunghezza d'onda della luce incidente l'ottica fisica è approssimata sempre meglio dall'ottica geometrica .
\subsection[]{ Esprimere tutte le relazioni fra campo elettrico, magnetico direzione di propagazione di un’onda e.m. piana.}
I campi ed il vettore d'onda formano una terna ortogonale; in particolare si ha (in CGS):
\[
	\boldsymbol{B} = \hat{k} \wedge \boldsymbol{E} \quad \quad 
	\boldsymbol{E} = \boldsymbol{B} \wedge \overline{k} 
\] 
In particolare i cambi sono trasversali, ovvero:
\[
	\hat{k} \cdot \boldsymbol{E} = 0 \quad \quad 
	\hat{k} \cdot \boldsymbol{B} = 0
\] 

\subsection[]{ Dare la definizione di onda piana elettromagnetica monocromatica e delle seguenti quantita’: ampiezza, frequenza angolare, vettore d’onda, frequenza, periodo, lunghezza
d’onda. Scrivere le relazioni esistenti fra le grandezze sopra definite.}
Un onda piana monocromatica è un onda le cui componenti dei campi sono della forma 
\[
	f\left( \boldsymbol{r}, t \right) = f_{0} e^{i \boldsymbol{k} \cdot \boldsymbol{r} - i \omega t }
\] 
L'ampiezza è il modulo dei campi, $\omega$ è la frequenza angolare, $\boldsymbol{k}$ è il vettore d'onda, $f = \omega/2\pi$ è la frequenza, $T = 1/f$ è il periodo, $\lambda = 2\pi/\boldsymbol{|k|}$ è la lunghezza d'onda. Nel vuoto si ha la relazione $\omega = k c$.

\subsection[]{ Definire la relazione di dispersione, la velocità di fase e la velocità di gruppo per un’onda e.m. e spiegarne il loro significato fisico. }
Dalle equazioni di Maxwell in MKS
\[
	\nabla \boldsymbol{E} = \frac{\rho}{\epsilon_0} \quad \quad \quad \quad 
	\nabla \boldsymbol{B} = 0 
\] 
\[
	\nabla \times \boldsymbol{E} = - \frac{\partial \boldsymbol{B} }{\partial t} \quad \quad \quad
	\nabla \times \boldsymbol{B} = \mu_0 \boldsymbol{J} + \mu_0 \epsilon_0 \frac{\partial \boldsymbol{E} }{\partial t} 
\] 
si ricava che, per campi monocromatici:	
\[
	\nabla^{2}\boldsymbol{E} - \frac{\epsilon\left( \omega \right) \mu \left( \omega \right) \omega^{2}}{c^{2}}\boldsymbol{E} = 0 
\]
\[
	\nabla^{2}\boldsymbol{B} - \frac{\epsilon\left( \omega \right) \mu \left( \omega \right) \omega^{2}}{c^{2}}\boldsymbol{B} = 0
\] 
Quindi esplicitando anche il laplaciano delle equazioni si trova la relazione funzionale che lega $\omega$ a k: la relazione di dispersione
\[
	c^{2}k^{2} = \epsilon \left( \omega \right) \mu \left( \omega \right) \omega^{2}
\] 
La velocità di gruppo e di fase possono allora essere definite come
\[
v_{f} = \frac{\omega}{k} \quad \quad 
v_{g} = \frac{\partial \omega}{\partial k} 
\]
La prima rappresenta la velocità con cui si propaga la fase dell'onda mentre la seconda quella con cui si propaga l'inviluppo del pacchetto. 
\subsection[]{ Definire la polarizzazione di un’onda e.m.}
La polarizzazione di un'onda EM è la direzione in cui oscilla il campo elettrico. Questa può essere lineare (se la direzione di oscillazione non varia nel tempo) , circolare o ellittica.
\subsection[]{ In un sistema Oxyz scrivere l‘espressione del campo elettrico e del campo magnetico di un’onda e.m. piana monocromatica, polarizzata linearmente lungo y e che si propaga lungo x, sia utilizzando il formalismo reale, sia utilizzando il formalismo complesso complesso.}
In CGS:
\[
	\boldsymbol{E} = E_{0} \hat{y} e^{ikx - i\omega t} = E_{0} \hat{y} \cos\left( kx - \omega t \right)   
\] 
\[
	\boldsymbol{B} = E_{0} \hat{z} e^{ikx - i\omega t} = E_{0} \hat{z} \cos\left( kx - \omega t \right)  
\] 
\subsection[]{ Enunciare e spiegare il principio di Huygens.}
Ogni elemento di un fronte d'onda si può considerare come sorgente secondaria di onde sferiche in fase con l'onda primaria e di ampiezza proporzionale all'ampiezza dell'onda primaria e all'area dell'elemento di fronte d'onda. \\
La distribuzione angolare di ampiezza è data dal fattore di obliquità:
\[
	f\left( \theta \right) = \frac{1 + \cos\left( \theta \right) }{2} 
\] 
\subsection[]{ Definire e calcolare l'impedenza del vuoto, e chiarire il suo significato fisico. }
Consideriamo un'onda e.m. monocromatica polarizzata linearmente che propaga (nel vuoto) nella direzione $\hat{z}$, i campi saranno in MKS:
\[
	\boldsymbol{E} = E_{0} \hat{x} \cos\left( \omega t - kz \right) \quad \quad
	\boldsymbol{B} = \frac{E_{0}}{c} \hat{y} \cos\left( \omega t - kz \right) 
\] 
Da cui il vettore di Poynting:
\[
	\boldsymbol{S} = \frac{E_{0}^2}{c \mu_{0}}\cos^2\left( \omega t - kz \right) = \sqrt{\frac{\mu_0}{\epsilon_0}}E_0^2 \cos^2\left( \omega t - kz \right)   
\] 
Nell'ultima espressione il termine sotto radice ha le dimensioni di $\sqrt{\frac{H}{F}} = \Omega$: è una resistenza.  
\[
	Z_{0} = \sqrt{\frac{\mu_{0}}{\epsilon_0}} \approx 120 \pi \text{ }\Omega
\]
Questa è chiamata impedenza del vuoto e rappresenta la resistività $\rho$ superficiale di un materiale cheassorbe senza riflettere le onde e.m. piane (tipicamente nella regione di microonde: $\sim 3 \text{GHz} < f < \sim 300 \text{GHz}$) 

\subsection[]{ Definire - in CGS e in MKSA - per un sistema di cariche e correnti elettriche: momento di dipolo elettrico; momento di quadrupolo elettrico;  momento di dipolo magnetico.}
\[
	\boldsymbol{p} = \int \rho \left( \boldsymbol{r} \right) d^{3}r     
\] 
\[
	\Q = \int \rho \left( r \right) \left( 3 \boldsymbol{r} \otimes \boldsymbol{r} -  r^2 \I  \right) d^3 r 
\] 
\[
	\boldsymbol{m} = \frac{1}{2[c]}\int \boldsymbol{r} \wedge \boldsymbol{J}\left(\boldsymbol{r}\right) d^3r
\] 
Dove $\rho$ e  $\boldsymbol{J}$ sono le densità di carica e di corrente, le quantità tra parentesi [ ] sono quelle da aggiungere per il sistema CGS.

\subsection[]{ Calcolare, a partire dalle EDM, la velocità delle onde elettromagnetiche in un mezzo omogeneo, lineare ed isotropo.}
In un mezzo così descritto vale la relazione:
\[
	\nabla^{2}\boldsymbol{E} - \frac{\epsilon\left( \omega \right) \mu \left( \omega \right) \omega^{2}}{c^{2}}\boldsymbol{E} = 0 
\] 
Quindi la velocità di fase cercata è (considerando la relazione di dispersione trovata sopra):
\[
v_{f} = \frac{\omega}{k} =  \frac{c}{\sqrt{\epsilon \mu} }
\] 
\subsection[]{ Esprimere la densita’ di energia di un’onda e.m. piana in funzione dei campi elettrico e/o magnetico.}
In unità MKS:
\[
	u = \frac{1}{2} \left( \boldsymbol{E}\boldsymbol{D} + \boldsymbol{B}\boldsymbol{H} \right) = \epsilon |\boldsymbol{E}|^2    
\] 
In CGS invece:
\[
	u = \frac{\boldsymbol{E}\boldsymbol{D} + \boldsymbol{B}\boldsymbol{H}}{8\pi} = \frac{\epsilon}{4\pi} |\boldsymbol{E}|^2 
\]
\subsection[]{ Dare la definizione ed esprimere il vettore di Poynting di un’onda e.m. piana in funzione del campo elettrico e/o magnetico.}
\[
	\boldsymbol{S} = \frac{c}{[4\pi]}\boldsymbol{E} \wedge \boldsymbol{H} = \frac{c}{[4\pi]}Z_{0} E_0^2 \hat{k}     
\] 
Le quantità in [ ] indicano i pezzi da aggiungere in CGS.

\subsection[]{ Esprimere la pressione (di radiazione) che un campo e.m. esercita su una superficie piana.}
Supponendo la superficie perfettamente riflettente si ottiene:
\[
	p = \frac{2I}{c} 
.\] 
con I = $\left<S\right>$ è l'intensità dell'onda. 
\subsection[]{ Dare la definizione di interferenza e diffrazione; di interferenza costruttiva e distruttiva.}
L'interferenza è un fenomeno in cui le intensità di due onde coerenti non si sommano linearmente. La diffrazione è un fenomeno in cui un fascio di radiazione si allarga (emette onde sferiche) dopo aver superato una fenditura o un ostacolo.

 % risposto


\part{Indagine della materia tramite collisioni e decadimenti}
\setcounter{section}{0}
\renewcommand*{\theHsection}{chX.\the\value{section}}
\section{Domande a}
\subsection[ Assorbimento, diffusione di onda E.M.]{Descrivere qualitativamente il fenomeno dell'assorbimento, il fenomeno della diffusione elastica ed il fenomeno della diffusione inelastica di un'onda e.m. su un sistema.}
Schematizzando il sistema come una scatola su cui facciamo incidere onde e.m. e osservando la radiazione emessa dall'oggetto pssiamo distinguere tre fenomeni:
\begin{figure}[H]
	\centering
	\includegraphics[width=0.4\textwidth]{immagini/1.png}
	\caption{Assobimento e diffusione e.m. di un sistema}
	\label{fig:1}
\end{figure}
Una parte della potenza irraggiata dall'onda sorgente può essere assorbita dall'oggetto (quindi dissipata con qualche meccanismo interno): fenomeno dell'assorbimento.\\
Una parte della potenza dell'onda incidente può essere diffusa sempre con la medesima frequenza: fenomeno della diffusione elastica.\\
Una parte della potenza dell'onda incidente può essere diffusa con frequenze differenti dalla incidente stessa: fenomeno della diffusione inelastica.\\
Un esempio di sistema di questo tipo è l'atomo in cui l'onda incidente eccita gli elettroni che, accelerando, possono irraggiare e dare luogo ai tre fenomeni citati.

\subsection[ Sezioni d'urto di onda E.M. su bersaglio]{Per un’onda e.m. monocromatica che incide su un bersaglio (per esempio un circuito o un atomo) definire le sezioni d’urto: di assorbimento, elastica differenziale, totale elastica; inelastica differenziale; inelastica totale; totale.}
\begin{itemize}
\item Sezione d'urto di assorbimento: 
	\[
		\sigma_{abs} = \frac{\left< P_{abs} \right> }{\left< \left|S_{in}\right| \right> }
	\] 	
\item Sezione d'urto elastica:
	\[
		\sigma_{el} = \frac{\left<P_{el} \right>}{\left< \left| S_{in} \right|  \right>} 
	\] 
	\[
		\frac{\mbox{d} \sigma_{el}}{\mbox{d} \Omega} = R^2 \frac{\left< \left| S_{el}\left( \theta,\phi \right)  \right|\right>}{ \left< \left| S_{in} \right|  \right> } 
	\] 
\item Sezione d'urto inelastica:\\
per ogni frequenza angolare $\omega_{i}$ a cui avviene la diffusione si ha:
\[
	\sigma_{\omega_{i}} = \frac{\left<P_{\omega_{i}} \right>}{\left< \left| S_{in} \right|  \right>} 
\] 
\[
	\frac{\mbox{d} \sigma_{\omega_{i}}}{\mbox{d} \Omega} = R^2 \frac{\left< \left| S_{\omega_{i}}\left( \theta,\phi \right)  \right|\right>}{ \left< \left| S_{in} \right|  \right> }
\] 
\item Sezione d'urto totale: 
	\[
	\sigma_{tot} = \sigma_{abs} + \sigma_{el} + \sum_{n=i} \sigma_{\omega_{i}}
	\] 
\end{itemize}
Da notare che l'unità di misura della sezione d'urto è quella di un'area.

\subsection[ Ampiezza di scattering]{Definire la ampiezza di scattering per un’onda e.m. monocromatica che incide su un bersaglio fisso (per esempio un circuito o un atomo).}
Per uno stato finale del sistema (con l'onda diffusa generata dal bersaglio) a grandi distanze il campo elettrico può esser scritto come il prodotto di un'onda sferica e di un termine che tenga conto della dinamicha del processo:
\[
	\boldsymbol{E}_{f} = \boldsymbol{f}\left( \theta, \varphi \right) \frac{e^{-i\left( \omega_{f}t - k_{f}R + \phi \right)}}{R}
\]
Con $\omega_{f}$ frequenza uscente, $k_{f}$ vettore d'onda uscente, $\phi$ fase.\\
L'ampiezza di scattering $\boldsymbol{f}$ è quindi l'ampiezza dell'onda sferica riemessa dall'oggetto scatterante (che interagisce con un'onda piana monocromatica). Questa è legata alla sezione d'urto differenziale dalla relazione:
\[
	\frac{\mbox{d} \sigma_{\omega_{f}}}{\mbox{d} \Omega} = \frac{\left| \boldsymbol{f}\left( \theta, \varphi \right) \right|^2}{\left| \boldsymbol{E_{0}} \right|^2 }
\] 


\subsection[ Potenza irraggiata da dipolo elettrico, magnetico e quadrupolo elettrico]{Descrivere la situazione in cui la legge 
\[
	P = \frac{2}{3c^3} \ddot{\boldsymbol{p_{e}}}^2 + \frac{1}{180 c^5} \dddot{Q_{ij}}^2 + \frac{2}{3c^3} \ddot{\boldsymbol{p_m}}^2
\] 
(espressa in CGS) è applicabile e spiegare il significato e l'unità di misura di ogni grandezza fisica ivi indicata; trascrivere poi l'espressione in MKSA.}
P è la potenza irraggiata da un sistema in cui sono presenti un dipolo elettrico (1), un quadrupolo magnetico (2) ed un dipolo magnetico (3):
\begin{enumerate}
	\item Dipolo elettrico:
	\[
		P_{1}^{CGS} = \frac{2}{3 c^3} \ddot{\boldsymbol{p}}_{el}^2 \quad \quad 
		P_{1}^{MKS} = k_{0}\frac{2}{3 c^3} \ddot{\boldsymbol{p}}_{el}^2
	\]
	con
	\[
		\boldsymbol{p}_{e} = \sum_{cariche} q \boldsymbol{r} 
	\] 
	\item Quadrupolo elettrico:
	\[
		P_{2}^{CGS} = \frac{1}{180 c^{5}}\dddot{Q_{ij}}^2 \quad \quad 
		P_{2}^{MKS} = k_{0} \frac{1}{180 c^{5}} \dddot{Q_{ij}}^2
	\]
	con
	\[
		\Q = \sum_{cariche}q\left( 3 \boldsymbol{r} \wedge \boldsymbol{r} - \boldsymbol{r}^2\I \right) 
	\] 
	\item dipolo magnetico:
	\[
		P_{3}^{CGS} = \frac{2}{3c^{3}} \ddot{\boldsymbol{p}}_{m}^2 \quad \quad 
		P_{3}^{MKS} = k_{0} \frac{2}{3c^5} \ddot{\boldsymbol{p} }_{m}^2
	\]
	con
	\[
		p_{m} = \frac{1}{2[c]} \sum_{cariche} q \boldsymbol{r} \wedge \boldsymbol{v}   
	\] 
\end{enumerate}
Queste sono applicabili se le dimensioni caratteristiche dell'oggetto che emette sono molto più piccole della lunghezza d'onda incidente. 


\subsection[ Distribuzione angolare della radiazione di dipolo elettrico e magnetico (non relativistico)]{Scrivere la distribuzione angolare della radiazione di dipolo elettrico e di dipolo magnetico nel caso non relativistico.}
\begin{figure}[H]
	\centering
	\includegraphics[width=0.4\textwidth]{immagini/2.png}
	\caption{Dipolo magnetico oscillante.}
	\label{fig:2}
\end{figure}
Sulla base della notazione di figura si ha:
\paragraph*{MKSA} I campi e la potenza irraggiata si scrivono come:\\
Dipolo magnetico:
\[
	\boldsymbol{E} = -k_{0}\frac{\ddot{\boldsymbol{p}}_m\left( t_{rit} \right) \wedge \hat{r} }{\left| \boldsymbol{r} \right| c^3} \quad \quad 
	\boldsymbol{B} = k_{0}\frac{\left( \ddot{\boldsymbol{p}}_{m}\left( t_{rit} \right) \wedge \hat{r}  \right) \wedge \hat{r}  }{ \left| \boldsymbol{r}  \right| c^3 }
\] 
\[
	P_m = \frac{\left| \ddot{\boldsymbol{p} }_{m} \right|^2}{6 \pi \epsilon_0 c^5} \quad  \quad \quad 
	\frac{\mbox{d} P_{m} }{\mbox{d} \Omega} = \frac{\mbox{d} P_m}{\mbox{d} \cos{\alpha \mbox{ d}\beta}} = \frac{1}{16 \pi ^2 \epsilon_0 c^5} \ddot{\boldsymbol{p}}_m^2( t_{rit})  \sin^2{\alpha} 
\] 
Dipolo elettrico:
\[	
	\boldsymbol{E} = k_{0}\frac{\left( \ddot{\boldsymbol{p}}_{e}\left( t_{rit} \right) \wedge \hat{r}  \right) \wedge \hat{r}  }{ \left| \boldsymbol{r}  \right| c^2 } \quad \quad 
	\boldsymbol{B} = k_{0}\frac{\ddot{\boldsymbol{p}}_e\left( t_{rit} \right) \wedge \hat{r} }{\left| \boldsymbol{r} \right| c^2} 
\] 
\[
	P_e = \frac{\left| \ddot{\boldsymbol{p} }_{e} \right|^2}{6 \pi \epsilon_0 c^3} \quad  \quad \quad 
	\frac{\mbox{d} P_{e}}{\mbox{d} \Omega} = \frac{\mbox{d} P_e}{\mbox{d} \cos{\alpha \mbox{ d}\beta}} = \frac{1}{16 \pi ^2 \epsilon_0 c^3} \ddot{\boldsymbol{p}}_e^2( t_{rit})  \sin^2{\alpha} 
\] 
\paragraph*{CGS} I campi e la potenza irraggiata si scrivono come:\\
Dipolo magnetico:
\[
	\boldsymbol{E} = -\frac{\ddot{\boldsymbol{p}}_m\left( t_{rit} \right) \wedge \hat{r} }{\left| \boldsymbol{r} \right| c^2} \quad \quad 
	\boldsymbol{B} = \frac{\left( \ddot{\boldsymbol{p}}_{m}\left( t_{rit} \right) \wedge \hat{r}  \right) \wedge \hat{r}  }{ \left| \boldsymbol{r}  \right| c^2 }
\] 
\[
	P_m = \frac{2}{3}\frac{\left| \ddot{\boldsymbol{p} }_{m} \right|^2}{c^3} \quad \quad \quad 
	\frac{\mbox{d} P_{m} }{\mbox{d} \Omega} = \frac{\mbox{d} P _m}{\mbox{d} \cos{\alpha \mbox{ d}\beta}} = \frac{1}{4 \pi c^3} \ddot{\boldsymbol{p}}_m^2( t_{rit})  \sin^2{\alpha} 
\] 
Dipolo elettrico:
\[	
	\boldsymbol{E} = \frac{\left( \ddot{\boldsymbol{p}}_{e}\left( t_{rit} \right) \wedge \hat{r}  \right) \wedge \hat{r}  }{ \left| \boldsymbol{r}  \right| c^2 } \quad \quad 
	\boldsymbol{B} = \frac{\ddot{\boldsymbol{p}}_e\left( t_{rit} \right) \wedge \hat{r} }{\left| \boldsymbol{r} \right| c^2} 
\] 
\[
	P_e = \frac{2}{3}\frac{\left| \ddot{\boldsymbol{p} }_{e} \right|^2}{c^3} \quad \quad \quad 
	\frac{\mbox{d} P_{e}}{\mbox{d} \Omega} = \frac{\mbox{d} P_e}{\mbox{d} \cos{\alpha \mbox{ d}\beta}} = \frac{1}{4 \pi c^3} \ddot{\boldsymbol{p}}_e^2( t_{rit})  \sin^2{\alpha} 
\]
Si potrebbe fare una verifica con: 
\[
	P = \int{ \frac{\mbox{d} P}{\mbox{d} \Omega}} \text{d}\Omega
\] 
\subsection[ Resistenza di irraggiamento per un circuito ad una maglia]{Definire la "resistenza di irraggiamento" di un circuito elettrico a una maglia e fornire un esempio.}
Nota la corrente ($I$) che scorre nel circuito la resistenza di irraggiamento è la resistenza dovuta alla dissipazione per irraggiamento:
\[
	R_{irr} = \frac{P_{irr}}{I^2}
\] 
Ad esempio per un circuito planare si ha (CGS):
\[
	\boldsymbol{m}  = IS \hat{n} / c \implies 
	R_{irr} = \frac{2}{3}\frac{S^2\omega^{4}}{c^{5}}
\] 

\subsection[ Urto elastico ed inelastico tra particelle con esempi]{Definire urto elastico ed urto inelastico fra due particelle; fornire poi almeno un esempio di reazione elastica ed una inelastica fra: 
	\begin{enumerate}
		\item un fotone ed un atomo
		\item due particelle cariche
		\item un protone ed un nucleo.
	\end{enumerate}
}
Un urto è elastico se la natura delle particelle non varia nell'urto stesso, ovvero se per ogni costituente
\[
	p^{\mu}_{i}p_{i,\mu} = m_{i}^2
\] 
è costante nel tempo. Urti di questo tipo sono della forma:
\[
	a \ + \ b \implies a \ + \ b
\] 
In tutti gli altri casi l'urto è anaelastico e si hanno situazion del tipo:
\[
	a \ + \ b \implies \sum_{i} p_{i}
\] 
Diamo adesso esempi di reazioni:
\paragraph{Fotone e Atomo}
\begin{itemize}
	\item Elastico: Scattering Thomson. 
	\[ 
		\gamma \ + \ H \implies \gamma \ + \ H 
	\]
	\item Inelastico: Effetto Compton.
	\[
		\gamma \ + \ A \implies \gamma \ + \ e^- \ + \ A^+
	\] 
\end{itemize}
\paragraph{Due particelle cariche}
\begin{itemize}
	\item Elastico: 
	\[
		p \ + \ p \implies p \ + \ p 
	\] 
	\item Inelastico: Produzione del bosone di Higs
	\[
		p \ + \ p \implies H \ + \ \ldots
	\] 
\end{itemize}
\paragraph{Protone e Nucleo}
\begin{itemize}
	\item Elastico: 
	\[
		\text{da trovare ancora\ldots}
	\]
	\item Inelastico: Reazione degli alchimisti
	\[
		p \ + \ {}^{198}_{80}\text{Hg}_{118} \implies p \ + \ p \ + \ {}^{197}_{79}\text{Au}^{-}_{118} 
	\] 
\end{itemize}
\subsection[ Conservazione di alcune grandezze per interazioni elettromagnetiche/forti]{Dire quali fra le seguenti grandezze si conservano sempre nei processi di urto in cui avvengano interazioni elettromagnetiche e/o forti, ma non deboli.} 
\begin{enumerate}
	\item \textbf{carica elettrica.} Si conserva.
	\item \textbf{numero barionico.} Si conserva.
	\item \textbf{numero leptonico elettronico.} Si conserva. 
	\item \textbf{numero leptonico muonico.} Si conserva.
	\item \textbf{numero di elettroni.} Non si conserva: Formazione di coppie.
		\[
			\ce{ \ce{\gamma} -> \ce{e^+} + \ce{e^-}}
		.\]  
	\item \textbf{differenza fra il numero di elettroni ed il numero di positroni.} Si conserva.
	\item \textbf{numero di protoni.} Non si conserva: produzione di protoni su nucleo.
	\item \textbf{differenza fra il numero di protoni ed il numero di antiprotoni.} Si conserva.
\end{enumerate}


\subsection[ Conservazione di alcune grandezze per interazioni forti]{Dire quali fra le seguenti grandezze si conservano sempre nei processi di urto in cui avvengano esclusivamente interazioni forti:  Fornire almeno un esempio per ogni situazione in cui vi sia una grandezza non conservata.}
\begin{enumerate}
	\item \textbf{carica elettrica.} Si conserva.
	\item \textbf{numero barionico.} Si conserva.
	\item \textbf{numero leptonico elettronico.} Si conserva.
	\item \textbf{numero leptonico muonico.} Si conserva.
	\item \textbf{numero di elettroni.} Si conserva.
	\item \textbf{differenza fra il numero di elettroni ed il numero di positroni.} Si conserva.  
	\item \textbf{numero di protoni.} Si conserva.
	\item \textbf{differenza fra il numero di protoni ed il numero di antiprotoni.} Si conserva.
\end{enumerate}


\subsection[$\ $ Conservazione di alcune grandezze per interazioni deboli]{Dire quali fra le seguenti grandezze si conservano sempre nei processi di urto in cui avvengano interazioni deboli, fornire almeno un esempio quando si ha qualcosa di non conservato.}
\begin{enumerate}
	\item \textbf{carica elettrica.} si conserva.
	\item \textbf{numero barionico.} si conserva.
	\item \textbf{numero leptonico elettronico.} non si conserva.
	\[
		\mu^- \implies e^- \ + \ \gamma
	\]
	\item \textbf{numero leptonico muonico.} Non si conserva, stessa interazione di sopra.
	\item \textbf{numero di elettroni.} Non si conserva: Decadimento del Neutrone.
	\[
		n \implies p \ + \ e^- \ + \ \overline{\nu}_{e}
	\] 
	\item \textbf{differenza fra il numero di elettroni ed il numero di positroni.} Non si conserva: Decadimento $\beta^+$
	\[	
		\ce{\ce{^{A}_{Z}X_{N}} -> \ce{^{A}_{Z-1}Y^{-}_{N+1}} + e^{+} + \nu_{e} }
	\] 
	\item \textbf{numero di protoni.} Non si conserva: Decadimento $\beta^-$
	\[
		\ce{\ce{^{A}_{Z}X_{N}} -> \ce{^{A}_{Z+1}Y^{+}_{N-1}} + e^{-} + \overline{\nu}_{e} }
	\] 
	\item \textbf{differenza fra il numero di protoni ed il numero di antiprotoni.} Non si conserva: Decadimento $\beta^-$ 
\end{enumerate}

\subsection[$\ $ Processi esclusivi e inclusivi, esotermici e endotermici, Q-value.]{Definire i processi esclusivi e inclusivi, il Q-valore di un processo e i processi esotermici o endotermici.}
Un processo si dice esclusivo se in esso viene misurato il 4-impulso di tutti i prodotti. Un processo si dice inclusivo se in esso vengono misurati solo i 4-impulsi di alcuni prodotti.\\
Il Q-valore di un processo è definito come: 
\[
	Q = \left( m_{i} - m_{f} \right) c^2 = T_f - T_i
\] \label{eq:Q-valore}
Un processo è esotermico se $Q > 0$, endoterimco altrimenti.

\subsection[$\ $ Tre definizioni di sezione d'urto equivalenti nei processi corpuscolari]{Definire la sezione d’urto nei seguenti tre casi, e dimostrare come da ognuno di essi si possano dedurre gli altri due: 
\begin{enumerate}
	\item Particelle incidenti su un unico bersaglio [dati: $j_{\text{incidenti}}$; $N_f$] 
	\item Sottile fascio di particelle incidenti su una lastra contenente i bersagli [dati: $\Phi_{incidenti}$, $\hat{\sigma}_{bersagli}$, $N_f$] 
	\item Urti nel volume fra particelle di due specie diverse e differenti concentrazioni [dati: $N_{eventi}$ per unità di tempo e di volume, $n_1$ e $n_2$ concentrazione delle particellle interagenti, $v_{rel}$ (si ipotizza che tutte le particelle di una specie abbiano la stessa velocità).]
\end{enumerate}
Con $\hat{\sigma}$ densità superficiale di eventi,  $j$ densità di corrente, $N_f$ frequenza di eventi osservati (o numero di eventi per unità di tempo),  $\Phi$ flusso di particelle. 
}
Nel primo caso si ha:
\[
	\sigma = \frac{1}{\left| \boldsymbol{j}\right|} \frac{\mbox{d} N_{f}}{\mbox{d} t} 
\] 
Nel secondo invece:
\[
	\sigma = \frac{1}{n_s \Phi} \frac{\mbox{d} N_f}{\mbox{d} t} 
\] 
Nel terzo:
\[
	\sigma = \frac{1}{n_1 n_2 v_{rel}} \frac{\mbox{d} n_f}{\mbox{d} t} 
\] 
Nell'ultimo la sezione d'urto dipende dalla velocità relativa. Inoltre se si ha la funzione di distrubuzione per la velocità relativa:
\[
	\frac{\mbox{d} N_f}{\mbox{d} t} = n_1 n_2 \int_0^\infty \sigma\left( v_{rel} \right) f\left( v_{rel} \right) dv_{rel}
\] 
Per passare dal primo al secondo caso basta osservare che:
\[
	\Phi = \left| \boldsymbol{j} \right| S, \quad \quad 
	n_{s} = \frac{1}{S} \quad \left( \text{un bersaglio} \text{} \right) 
\] 
Con S area del bersaglio.
Se mi metto in un sistema in cui una delle particelle è ferma, è evidente mostrare l'equivalenza tra il caso (2) e (3).

\subsection[$\ $ Sezione d'urto elastica, inclusiva, esclusiva e totale per urti tra particelle]{Per gli urti fra due particelle definire le sezioni d'urto: elastica, inelastica, inclusiva, esclusiva, totale.}
Consideriamo la reazione:
\[
	a \ + \ b \implies p_1 \ + \ p_2 \ + \ \ldots \ + \ p_n
\] 
e sia $f_i\left( E_i \right) $ la distrubuzione di probabilità dell'energia del prodotto i-esimo. La sezione d'urto inclusiva di tale prodotto è:
\[
	\sigma_i = \int_{E_{i, min}}^{E_{i,max}} f\left( E_{i} \right) dE_{i}
\]
Se invece si considera la distribuzione degli impulsi di tutte le particelle finali: $f\left( P_{1}, \ldots , P_{n} \right) $ si ha una sezione d'urto esclusiva:
\[
	\sigma_e = \int f\left( P_{1}, \ldots P_{n} \right) \prod_{i = 1}^{n}d^{4}P_i  
\] 
La sezione d'urto elastica è la sezione d'urto reltiva as un urto elastico, la sezione d'urto anaelastica è la sezione d'urto relativa ad un urto anaelastico.

\subsection[$\ $ Probabilità di interazione di particella su lamina sottile]{Calcolare la probabilità di interazione per una particella che incide su una lamina sottile [dati: $\sigma$ processo, $N_{bersagli}$ per unità superficie].\\
Che significato avrebbe una probabilità maggiore di uno? Quest'ultima risposta dipende dalle tipologie degli urti?}
La probabilità di interazione è: $P = N \sigma $. Se  questa è maggiore di 1 significa che è venuta meno l'approssimazione di lamina sottile. In ogni caso questa non dipende dal tipo di interzaione.

\subsection[$\ $ Forza di reazione radiativa]{Indicare le condizioni per cui la forza di reazione radiativa per una particella (di massa m e carica unitaria) $F_{rad} = m \tau \dot{\boldsymbol{a}}$ è da considerarsi valida ed utilizzabile.} \label{subsec: 2.a.15}
Si ritiene necessario, per indicare le approssimazioni, ricavare la forza in questione. 
La forza di radiazione può essere defita come la forza il cui lavoro è responsabile della perdita di energia per irraggiamento noto nella formula di Larmor (CGS):
\[
	\int_{t_1}^{t_2} \boldsymbol{F}_{rad} \cdot \boldsymbol{v} dt = -\frac{2}{3} \frac{e^2}{c^3} \int_{t_1}^{t_2} \dot{\boldsymbol{v}}^2
\]
Posso scrivere:
\[
	\dot{\boldsymbol{v}}^2 = \frac{\mbox{d} \left( \boldsymbol{v} \dot{\boldsymbol{v}} \right) }{\mbox{d} t} - \boldsymbol{v} \ddot{\boldsymbol{v}} 
\]
Inserendo nel secondo membro della equazione si ha:
\[
	\int_{t_1}^{t_2}{ \boldsymbol{F_{rad}} \cdot \boldsymbol{v} dt } = -
\left. \frac{2}{3}\frac{e^2}{c^3} \boldsymbol{v} \cdot \dot{\boldsymbol{v}}\right|_{t_{1}}^{t_{2}} + \frac{2}{3}\frac{e^2}{c^3} \int_{t_1}^{t_2} \boldsymbol{v}  \cdot \ddot{\boldsymbol{v}} dt
\] 
Se il moto è periodico di ha: 
\[
	\int_{t_1}^{t_2} \left( \boldsymbol{F}_{rad} - \frac{2}{3}\frac{e^2}{c^3}\ddot{\boldsymbol{v}} \right) \cdot \boldsymbol{v} dt = 0  
\] 
Poiche velocità e accelerazione sono ortogonali in tal caso, abbiamo quindi un candidato per la $\boldsymbol{F}_{rad}$.
\[
	\boldsymbol{F}_{rad} = \frac{2}{3}\frac{e^2}{c^3} \ddot{\boldsymbol{v}}.  
\] 
che in MKSA si scrive come:
\[
	\boldsymbol{F}_{rad} = \frac{q^2}{6\pi \epsilon_0 c^3} \dddot{\boldsymbol{x}} = m_{e}\frac{q^2}{6\pi \epsilon_0 c^3} \dddot{\boldsymbol{x}} = m_e \frac{2}{3}\frac{r_e}{c} \dddot{\boldsymbol{x}} = m_e \tau \dddot{\boldsymbol{x}} 
.\] 
Possiamo quindi dare una stima dei valori tipici di questa forza:
\[
r_e = \frac{q^2}{4\pi \epsilon_0 m_e c^2} = 2.82 fm \implies \tau = \frac{2}{3}\frac{r_e}{c} = 6.2 \cdot 10^{-24} s 
.\] \label{eq: raggio classico} 
Tenendo conto della forza viscosa che agisce classicamente sull'elettrone 
\[
	\boldsymbol{F}_{visc} = -\beta \dot{\boldsymbol{x}} = - m_e \Gamma' \dot{\boldsymbol{x}} 
\]
con valori tipici: $\Gamma' \sim 10^{10} \ s^{-1}$.\\
Infine ipotizzando anche una forza "elastica" attrattiva nucleare con valore tipoco $\omega_{0} \sim 10^{14} - 10^{16}$ otteniamo la relazione:
\[
\Gamma' \ll \omega_{0} \ll \frac{1}{\tau}
.\] \label{eq:relazione-parametri-elettrone} 
Questo conto sarà utile per la Domanda \hyperref[subsec: 2.b.12]{2.b.12}.

\subsection[$\ $ Significato fisico della Breight-Wiegner]{Spiegare il significato e indicare l'unità di misura di ogni grandezza fisica nelle seguenti leggi:
\[
	\left( 1 \right) \to \frac{\mbox{d} \sigma_{el}}{\mbox{d} \Omega} = r_{e}^2 L \left( \omega \right)  \sin^2 \left( \alpha \right)   \quad \quad
	\left( 2 \right) \to  \frac{\mbox{d} \sigma_{el}}{\mbox{d} \Omega} = r_{e}^2 L \left( \omega \right)  \frac{1 + \cos^2 \left( \alpha \right) }{2}   
\]
\[
	\left( 3 \right) \to  \sigma_{el} = \sigma_{Th} L \left( \omega \right) \quad \quad 
	\left( 4 \right) \to  \sigma_{tot} = 4 \pi r_{e} c \frac{\omega^2 \Gamma }{\left( \omega_{0}^2 - \omega^2 \right)^2 + \omega^2 \Gamma_{tot} } 
\] 
con 
\[
	L \left( \omega \right) = \frac{\omega^4}{\left( \omega_{0}^2 - \omega^2 \right)^2 + \omega^2 \Gamma_{tot} } \quad \quad 
	\sigma_{Th} = \frac{8}{3} \pi r_{e}^2 = 0.66 \text{ barn}
\]
inerenti l'interazione di un’onda e.m. piana e monocromatica su un elettrone legato elasticamente.} \label{subsec: 2.a.16}
L'equazione (1) è la sezione d'urto differenziale per l'interazione tra l'elettrone ed un onda polarizzata linearmente. Può essere ottenuta dalla equazione del moto dell'elettrone legato elasticamente soggetto alle forze della domanda precedente (di richiamo, di attrito viscoso e radiazione radiativa) immerso nel campo di una onda e.m. piana monocromatica (vedi Domanda \hyperref[subsec: 2.b.12]{2.b.12}):
\[
	\boldsymbol{x} = \frac{e \boldsymbol{E_0}}{m_e}\frac{1}{\omega_0^2-\omega^2- i\omega \Gamma' - i \tau \omega^2} = \frac{eE_0}{m_e}\frac{1}{\omega_0^2-\omega^2- i\omega\Gamma_{tot}}
.\] 
In cui si definiscono 
\[
	\Gamma_{tot} = \Gamma' + \Gamma \frac{\omega^2}{\omega_0^2} \quad \quad 
	\Gamma = \omega_0^2 \tau \quad \text{ con $\tau$ quello ottenuto nella domanda precedente}
\]
Quindi basta adesso ricordare che la distribuzione angolare di potenza irraggiata da un dipolo oscillante $ \boldsymbol{p} = \boldsymbol{p}_0 e^{-i\omega t}$ è:
\[
	\frac{\mbox{d} P}{\mbox{d} \Omega} = \frac{\omega^{4} \left| \boldsymbol{p}_0 \right|^2 }{4\pi c^3} \sin^2\left( \alpha \right) 
.\] 
Quindi inserendo $\left| \boldsymbol{p}_0 \right| = \left| e \boldsymbol{x} \right| $  e dividendo per il vettore di Poynting incidente si ha la tesi (1).

L'equazione (2) è la sezione d'urto differenziale con l'onda incidente non polarizzata: bisogna in questo caso mediare su tutte le possibili polarizzazioni dell'onda incidente ottendendo il fattore finale.\\
l'espressione (3) si ottinene integrando sull'angolo solido la (1) o la (2).\\
infine la (4) è la sezione d'urto totale definita come:
\[
	\sigma_{tot} = \frac{e \left< \dot{\boldsymbol{x}}\boldsymbol{E}\right>}{\left<\boldsymbol{S_{in}} \right>}
.\]
Possiamo notare anche che:
\[
	\frac{\sigma_{el}}{\sigma_{tot}} = \frac{1}{1 + \Gamma'/\left( \omega^2\tau \right)  }
.\] 
Che tende all'unità quando non c'è dissipazione.

\subsection[$\ $ osservazioni sullo scattering Rutherford]{Discutere qualitativamente le osservazioni sperimentali dello scattering di Rutherford.}
Inviando un fascio di particelle $\alpha$ su una lamina d'oro si osserva che circa 1 particella su 8000 viene deviata a grandi angoli o rimbalza. Queste osservazioni non sono spiegabili con il modello a panettone di Thomson ma si spiegano bene con il modello di Bhor.
\[
	\left. \frac{\mbox{d} \sigma}{\mbox{d} \Omega} \right|_{Ruth} = \frac{1}{\left(4\pi \epsilon_0\sin^{2}\left( \frac{\theta}{2} \right) \right)^2 } \left(\frac{zZe^2}{4T} \right)^2 
.\] 
Si deduce facilmente che le particelle più energetiche sono più penetranti, le particelle più cariche vengono maggiormente deflesse.

\subsection[$\ $ Differenza tra Scattering Rutherford e Mott]{Spiegare la differenza fra lo scattering Rutherford e lo scattering Mott.}
Lo Scattering Mott:
\[
	\left. \frac{\mbox{d} \sigma}{\mbox{d} \Omega} \right|_{Mott} = \left. \frac{\mbox{d} \sigma}{\mbox{d} \Omega} \right|_{Ruth} \left( 1- \beta^2 \sin^2\left( \frac{\theta}{2} \right)  \right) 
.\] 
Tiene conto dello spin degli elettroni e degli effetti relativistici che questi hanno nel loro moto.

\subsection[$\ $ Significato dei termini nelle sezioni d'urto Mott e Rutherford]{Spiegare il significato di tutti i termini delle seguenti espressioni delle sezioni d’urto differenziali Rutherford e Mott: 
\[
	\left. \frac{\mbox{d} \sigma}{\mbox{d} \Omega} \right|_{Ruth} = \left( \frac{zZe^2}{4 \pi \epsilon_0} \right)^2 \left( \frac{1}{4T} \right)^2 \frac{1}{\sin^4\left( \theta/2 \right) }  
\] 
\[
	\left. \frac{\mbox{d} \sigma}{\mbox{d} \Omega} \right|_{Mott} = \left. \frac{\mbox{d} \sigma}{\mbox{d} \Omega} \right|_{Ruth} \left( 1 - \beta^2 \sin^2 \theta/2  \right) \quad \quad
	\text{con } T \to \frac{pV}{2}
\] 
}
Sono tutti termini di banale comprensione, ricordiamo però che $\theta$ è l'angolo di scatterig: l'angolo tra la direzione iniziale della particella e quello finale.

\subsection[$\ $ Definizione di raggio nucleare tramite lo Scattering Rutherford]{Dare la definizione operativa di raggio nucleare mediante lo scattering di Rutherford.}
Fissando un angolo di scatternig $\hat{\theta} = 60^o$ si ha che per energie maggiori di $T_{soglia} \approx 30 MeV$ divena importante l'interzione forte con il nucleo: si ha un discostamento dalla legge che lega la sezione d'urto Rutherford a T. Quindi possiamo ipotizzare che la distanza minima che si ottiene in questo caso sia una buona stima del raggio nucleare.
\[
	d \approx \frac{zZ e^2}{4 \pi \epsilon_0 T_{soglia}} \implies R = \frac{d}{2}\left( 1 + \frac{1}{\sin\left( \frac{\hat{\theta}}{2} \right) } \right) 
.\] 
È necessario notare che $R$ è la distanza tra i nuclei degli atomi coinvolti nello scattering, non il nucleo dell'atomo (che vorremmo definire), è quindi necessario incidere con particelle il cui nucleo sia di dimensioni attese molto inferirori rispetto a quelle del nucleo in esame per dare una buona stima del raggio di quest'ultimo (ipotesi verificata nel caso di particelle $\alpha$ su Piombo, ad esempio).

\subsection[$\ $ Definizione di A, Z, N, nuclei isotopi, isobari, isotoni, stabili ed instabili]{Definire le quantità che in un nucleo usualmente si indicano con A, Z, N (simbologia ${}^A_Z X_N$ ). Dare la definizione di nuclei isotopi, isobari, isotoni, stabili, instabili.}
A è il numero di nucleoni (protoni e neutroni), Z è il numero di protoni, N è il numero di neutroni.\\
Due nuclei sono:
\begin{itemize}
	\item Isotopi: hanno lo stesso Z.
	\item Isobari: hanno lo stesso A.
	\item Isotoni hanno lo stesso N.
	\item Instabili: hanno vita media finita.
	\item Stabili: hanno vita media "infinita".
\end{itemize}

\subsection[$\ $ Unità di massa atomica, energia di legame B e difetto di massa $\Delta$]{Dopo avere definito l’unità di massa atomica e avere dato il suo valore in MeV/$c^2$, definire l’energia di legame (B) di un atomo ed il “difetto di massa” ($\Delta$) di un atomo.}
\[
	\text{1 u.m.a.} = \frac{1}{12}m\left( {}^{12}_{6}C_{6} \right) = 931.49 \text{ MeV/}c^2 
.\]
La B è l'energia necessaria per separare un nucleone dal suo nucleo.\\
L'energia di legame B di un nucleo X con A nucleoni e Z protoni è:
\[
	B\left( A, Z \right) = Z\left( m_p + m_e - m_u \right) - N \left( m_n - m_u \right) - \Delta_{A, Z} = 7.29 \text{MeV} \cdot Z + 8.07 \text{MeV} \cdot N - \Delta_{A,Z} 
.\] 
Dove 
\[
	m_u = 1 \text{ u.m.a.}\quad \quad 
	m_p = 938.2 \text{ MeV/}c^2\quad \quad 
	m_e = 0.511 \text{ MeV/}c^2
\]
Il difetto di massa è invece:
\[
	\Delta = m_{x} - \frac{A}{12}m\left( {}^{12}_{6}C_{6} \right) = m_x - Am_u 
.\] 
E lo si pio trovare nelle tabelle in rete.\\
Il difetto di massa è particolarmente utile per ricavare il Q-valore: è immediato esprimere la massa del nucleo coinvolto in una interazione mediante tale quantita e l'unità di massa atomica $m_u$.

\subsection[$\ $ Formula semiempirica B e termini spiegati dal modello a goccia]{Enunciare la formula semiempirica B = B(A,Z) ed indicare i suoi termini che sono spiegati dal modello a goccia. Spiegare le ipotesi su cui tale modello è basato e
fornire l'ordine di grandezza dell’energia media di legame di un nucleone all’interno di un nucleo.}
Nel modello a goccia si ha in prima approssimazione (per nuclei abbastanza grandi) una energia di legame proporzionale al numero di nucleoni e quindi al volume del nucleo stesso:
\[
	B_1\left( A, Z \right) = a_V A \quad \quad \text{Termine correttivo di volume}
.\] 
In seconda approssimazione possiamo considerare che i nucleoni che si trovano sulla superficie del nucleo non sono circondati da altri nucleoni, vi sarà allora un termine correttivo superficiale:
\[
	B_2\left( A,Z \right) = a_S A^{2/3} \quad \quad \text{Termine correttivo di superficie} 
.\] 
Poi possiamo aggiungere una ulteriore correzione per tener conto della repulsione columbiana tra i nucleoni carichi (Z tiene conto della carica, A tiene conto del raggio):
\[
	B_3\left( A,Z \right) = a_C \frac{Z^2}{A^{2/3}} \quad \quad \text{Termine correttivo Columbiano}
.\] 
Si hanno infine altri termini correttivi che tengono di conto di effetti quantistici e del principio di Pauli:
\[
	B\left( A,Z \right) = a_{V}A - a_{S}A^{2/3} - a_{C} \frac{Z^2 }{A^{1/3}} + a_{sym}\frac{\left( Z - N \right)^2}{A} + \delta_{pair} 
.\] \label{eq:B-energy}
Il modello in considerazione è quello "A Goccia", l'ipotesi di questo è che il nucleo di numero atomico Z e peso atomico A occupi un volume sferico di raggio:
\[
	R = r_0 A^{1/3} + r_{skin} \approx \left( 1.25 A^{1/3} + 2.0  \right) fm
.\]
Il modello prevede che l'energia di legame tra due nucleoni sia dell'ordine di $\sim 2.2$ MeV, ovvero la differenza tra la massa del deutone e la somma delle masse del protone e neutrone. 

\subsection[$\ $ Definizione dei decadimenti $\alpha$, $\beta^{+}$,$\beta^{-}$, $\gamma$, cattura elettronica con rispettivi Q-Value]{Definire i decadimenti $\alpha$, $\beta$ , $\gamma$ e il decadimento tramite cattura elettronica in un nucleo. Calcolare il Q-valore per il decadimento $\beta+$, $\beta-$, e per la cattura elettronica a partire dal difetto di massa delle specie coinvolte.} \label{sec:decadimenti}
\paragraph{Decadimento $\alpha$}
\[
	\ce{ \ce{^{A}_{Z}X_{N}} -> \ce{^{A-4}_{Z-2}Y^{2-}_{N-2} + \alpha}} \quad \quad \text{con $\alpha$ nucleo di  \ce{^{4}_{}He^{2+}_{}}}
\]
\paragraph{Decadimento $\beta+$}
\[
	\ce{\ce{^{A}_{Z}X_{N}} -> \ce{^{A}_{Z-1}Y^{-}_{N+1}} + e^{+} + \nu_{e} }
.\]
in cui si ha la transizione:
\[
	\ce{p -> n + e^+ + \overline{\nu}_e}
.\]
Il Q-valore del decadimento è : $Q = \Delta_{A,Z} - \Delta_{A, Z-1} - 2m_e $\\
Il Q-valore della reazione è: $ Q = m_p - m_n - m_e = - 1.804 $MeV.

\paragraph{Decadimento $\beta-$}
\[
	\ce{\ce{^{A}_{Z}X_{N}} -> \ce{^{A}_{Z+1}Y^{+}_{N-1}} + e^{-} + \overline{\nu}_{e} }
.\]
in cui si ha la transizione:
\[
	\ce{n -> p + e^- + \nu_e}
.\]
Il Q-valore del decadimento è : $Q = \Delta_{A,Z} - \Delta_{A, Z+1} $\\
Il Q-valore della reazione di transizione è: $ Q = m_n - m_p - m_e = 0.782 $MeV.
\paragraph{Cattura elettronica}
\[
	\ce{\ce{^{A}_{Z}X_{N}} -> \ce{^{A}_{Z-1}Y_{N+1}} + \nu_e} \quad \quad Q = \Delta_{A,Z} - \Delta_{A, Z-1} 
.\] 
in cui si ha:
\[
\ce{p + e ->n + \nu_e }
.\] 

\subsection[$\ $ Scoperta del neutrino nel decadimento $\beta^{-}$]{Come si e' arrivati alla conclusione che nel decadimento beta deve essere emessa una particella neutra non rivelata? }
Il problema in questione risale al 1934 (con Pauli che ipotizza l'esistenza della particella), venne formalizzato successivamente da Fermi e da Bhor.\\
Ciò che destava sgomento era lo spettro di emissione dell'elettrone. Spieghiamo a grandi linee il problema: \\
All'inzio si pensava che avvenisse il processo
\[
	\ce{n -> p + e-}
.\] 
In cui sicuramente si hanno le relazioni (di qui in avanti c = 1) 
\[
	m_e \text{ (0.511 MeV) }\ll m_p \text{ (938.3 MeV) } \approx m_n \text{ (939.6 MeV)}
.\] 
Conviene quindi ipotizzare che il neutrone si trovi inizialmente fermo, in tal caso anche il protone viene praticamente creato fermo. Per la conservazione dell'energia:
\[
	m_n =E_p + E_e
.\]
con 
\[
	E_p = \sqrt{m_p^2 + p_p^2} \quad \quad 
	E_e = \sqrt{m_e^2 + p_e^2} 
.\]
Se si trascura il rinculo del protone, come ipotizzato sopra:
\[
	m_n \approx m_p + \sqrt{m_e^2 + p_e^2} 
.\]
Quest'ultima relazione fissa l'impulso e l'energia dell'elettrone, quindi ci aspettiamo che lo spettro di quest'ultimo abbia un unico picco, invece sperimentalmente si ottengono curve del tipo:
\begin{figure}[H]
	\centering
	\includegraphics[width=0.5\textwidth]{immagini/Decadimento_beta_(spettro).jpg}
	\caption{Spettro di energia dell'elettrone.}
	\label{fig:beta}
\end{figure} 
Tale grafico mostra uno spettro completo che parte da energie nulle fino ad arrivare ad annullarsi di nuovo a $\sim 5.5 \ m_e$.\\
Per spiegarlo è quindi necessario ipotizzare che vi sia un'altra particella tra i prodotti che è appunto il neutrino.


\subsection[$\ $ Spiegazione dell'esistenza del neutrone nell'atomo]{Spiegare perchè, sebbene il neutrone libero sia instabile, esso non possa decadere quando è all'interno di taluni nuclei.}
Affinchè il neutrone in un nucleo possa decadere è necessario che
\[
	Q = \sum_{in} M_k - \sum_{fin} M_k > 0
.\] 
Questo in alcuni nuclei può non essere verificato, ad esempio:
\paragraph{Stabilità del Deuterio}
\[
	\ce{\ce{^{2}_{1}H_{1}} -> \ce{^{1}_{1}H_{0}} + p + e- + \overline{\nu}_e }
.\] 

con:
\[
	\ce{n -> p + e- + \overline{\nu}_e}
.\] 
si ha che la reazione non avviene perchè: 
\[
	Q = \Delta_{2,1} - 2\Delta_{1,0} = (13.136 -14.578)\text{ MeV} = -1.442 \text{ MeV} 
.\] 

\subsection[$\ $ Particelle per la misura del fattore di forma]{Quali particelle incidenti e di quale energia si utilizzano per misurare i fattori di forma nucleari?}
Si usano in genere Scattering elastici, in particolare la particella adatta è l'elettrone, ad esempio:  
\[
\ce{e- + p  -> e- + p} \quad \quad \text{Per misurare il fattore di forma e.m. del protone}
.\] 
Se funziona con il protone allora funziona anche con i nuclei avendo cura di scegliere le giuste energie con il metodo seguente.\\
Per quanto riguarda l'energia dell'elettrone dobbiamo tener di conto di che lunghezza vogliamo ispezionare: Per sondare oggetti di dimensioni caratteristiche del fm è necessario sondare con energie aventi le stesse dimensioni in termini di lunghezza d'onda di De Broglie:
\[
	E = \hbar \omega = \frac{2\pi\hbar c}{\lambda} =  \frac{1240 \text{ MeV} \cdot \text{fm}}{1 \text{ fm}} = 1240 \text{ MeV}
.\]

\subsection[$\ $ Larghezza di vita media, tempo di dimezzamento, branching fraction]{Dare le definizioni di: larghezza, vita media, semivita (o tempo di dimezzamento), rapporto di decadimento (“Branching fraction” o “Branching ratio”) per il decadimento di una particella.}
Se N è il numero di particelle non ancora decadute la vita media è definita da:
\[
	\dot{N} = -\frac{N}{\tau}
.\] 
La larghezza di vita media è la probabilità di decadimento nell'unità di tempo:
\[
	\Gamma = 1/\tau
.\]
La larghezza viene solitamente espressa in eV tramite:
\[
	\Gamma \text{ [eV]} = \Gamma \text{ [s]}^{-1} \cdot \hbar
.\] 
Perchè una particella che decade si può interpretare come una risonanza piccata nella massa di quest'ultima e di larghezza prorio $\Gamma$.\\
Il tempo di dimezzamento è
\[
	T = \tau \ln\left( 2 \right)  
.\] 
In fine il rapporto di decadimento di un "canale" è il rapporto tra i decadimenti di quel canale ed il numero totale di decadimenti.
\[
	B_{\text{f}} = \frac{\Gamma_{\text{f}}}{\Gamma}
.\] 

\subsection[$\ $ Ordini di grandezza di sezioni d'urto forti o deboli]{ Quali sono gli ordini di grandezza tipici delle sezioni d’urto delle interazioni forti e delle interazioni deboli?}
Per le interazioni forti si hanno sezioni d'urto dell'ordine di $10 - 100$ mb, per le interazioni deboli invece $1$ fb.

\subsection[$\ $ Vite medie per interazioni forti, elettromagnetiche, deboli]{Quali sono, approssimativamente, gli ordini di grandezza delle vite medie dovute ad interazioni deboli, elettromagnetiche, forti?}
\begin{itemize}
	\item Interazioni deboli: dai 15 minuti per il decadimento $\beta$ del neutrone fino a $10^{-8}$ s del decadimento del  $\pi$ carico.
	\item Interazioni eletromagnetiche: tempi tipici sono dell'ordine di $10^{-16}$ s.
	\item Interazione forte: tempi tipici sono $10^{-23}$ s.
\end{itemize}

\subsection[$\ $ Cinematica dell'effetto Mossbauer]{Spiegare la cinematica di un decadimento $\gamma$ nucleare e spiegare qualitativamente l'effetto Mossbauer. }
\paragraph{Definizione di decadimento $\gamma$.}
I decadimenti $\gamma$ sono delle transizioni fra uno stato eccitato di un nucleo ed uno stato di energia inferiore con l'emissione di un fotone (raggi X: 10keV-1MeV).\\
Considerando la reazione:
\[
	\ce{\ce{^{57}_{26}F^*_e} ->[T_{1/2} = 97.7 \text{ ns}] \ce{^{57}_{26}F_e} + \gamma ( 14.4 \text{ keV} ) }
.\] 
\begin{itemize}
	\item $M$: la massa dello stato fondamentale del nucleo.
	\item $M^* = M + E_0$ : la massa dello stato eccitato. 
	\item $E_\gamma$: l'energia del fotone nel laboratorio e nel centro di massa se il decadimento avviene da fermo.	
\end{itemize}
Siamo interessati all'energia dei fotoni uscenti, a prima vista sembrerebbe $E_0$ ma in realtà è una sovrastima: il rinculo del nucleo si porta via energia, vediamolo cinematicamente.\\ 
La quantità di moto del fotone nel centro di massa (e quindi anche dell'atomo di $F_e$):
\[
	E_\gamma = P^\gamma_{cm}
.\] 
Quindi si ha anche che dalla conservazione dell'energia:
\[
	E_{in} = M^* = M + E_0 = E_{fin} = \sqrt{M^2 + E_\gamma^2} + E_\gamma 
.\]
Possiamo quindi ricavare $E_\gamma$ in funzione di tutto il resto:
\[
	E_\gamma = \frac{E_0\left( E_0 + 2M \right)}{2\left( E_0 + M \right)} 
.\]
Essendo la massa del $\ce{^{57}_{26}F_e} = 53.05 \text{ GeV}$ ed $E_0 = 14.4 \text{ keV}$ possiamo approssimare:
\[
	E_\gamma  \approx E_0 - \frac{E_0^2}{2M}
.\] 
Quindi l'energia persa è:
\[
	\Delta E_\gamma \approx - \frac{E_0^2}{2M} \implies \frac{\Delta E_\gamma}{E_\gamma} \approx - \frac{E_0}{2M}  
.\]
Essendo $E_0 \approx E_\gamma$. Nel caso in esame si ha:
\[
	\left| \Delta E_\gamma \right| \approx 1.9 meV \implies \left| \frac{\Delta E_\gamma}{E_\gamma}\right| \approx 1.3 \cdot 10^{-7} 
.\] 
Quest'ultima è molto maggiore di:
\[
	\frac{\Gamma}{E_0} = 3.2 \cdot 10^{-13} 
.\] 
Quindi sarà difficile che un fotone partito dal decadimento riesca ad eccitare un nuovo atomo di $F_e$ poichè il fotone è emesso in un range di energia molto grande rispetto alla larghezza del processo.\\
Se invece prendo un blocco di atomi di ferro (o in gergo: nu bll pezz de ferragl) la massa che va al denominatore nelle equazioni sopra non è più quella del singolo atomo ma quella di tutto il blocco. Succede quindi che:
\[
	\Delta E_\gamma \ll \Gamma 
.\] 
Si può quindi avere un effetto coerente se il fotone che esce urta contro un altro atomo di ferro: Effetto Mosbauer.

\subsection[$\ $ Variabili indipendenti con due reagenti e N prodotti]{Quante sono le variabili indipendenti nello stato finale di una reazione in cui due particelle collidono ed N particelle sono prodotte?}
Sia il processo di decadimento:
\[
	\ce{a + b -> p_1 + p_2 + \ldots \text{+} p_n }
.\] 
Si ha che il numero di osservabili indipendenti è dato da le variabili ed i vincoli in gioco:
\begin{itemize}
	\item n 4-impulsi $\implies$ 4n variabili
	\item n vincoli dovuti alla massa delle singole particelle: $m_i^2 = P_{0, i}^2 - \boldsymbol{P}_{i}^2$
	\item 4 vincoli per la conservazione impulso-energia: $P_{in} = \sum_i P_i$
\end{itemize}
quindi le variabili indipendenti sono 3n - 4.
\paragraph{Nota}%
In quanto affermato non c'è stato bisogno di esplicitare il numero di reagenti, questo numero vale in generale per qualsiasi numero di particelle iniziali (se si hanno n prodotti).

\subsection[$\ $ Variabili indipendenti per un decadimento a due, considerazioni sul caso di Spin nullo]{Quante sono le variabili indipendenti nello stato finale di una reazione in cui una particella decade in due particelle? Quali implicazioni avremmo se la particella che decade avesse un momento angolare nullo? }
Le variabili indipendenti sono 3*2 - 4 = 2, se la particella ha spin nullo allora si ha una isotropia spaziale che ci permette di integrare l'espressione per il decadimento a due corpi:
\[
	\frac{\mbox{d} \Gamma}{\mbox{d} \Omega_1} = f_{dec}\left( \Omega_1 \right) \frac{P_{cm}}{4M} 
.\] 
sull'angolo solido:
\[
	\Gamma = f_{dec} \frac{P_{cm}}{4M}4\pi
.\] 

\subsection[$\ $ Variabili del Dalitz Plot]{Definire le variabili utilizzate nel "Dalitz plot".}
Le variabili del Dalitz plot sono:
\begin{itemize}
	\item $s_{12}$: il quadrato della massa invariante delle particelle 1 e 2
	\item $s_{23}$: il quadrato della massa invariante delle particelle 2 e 3
\end{itemize}

\subsection[$\ $ Variabili indipendenti per il decadimento a tre corpi, considerazioni sul caso di Spin nullo]{Quante sono le variabili indipendenti nello stato finale di una reazione in cui una particella decade in tre particelle? Quali implicazioni avremmo se la particella che decade avesse un momento angolare nullo?}
Le variabili indipendenti sono 3*3-4 = 5.\\
La larghezza di decadimento si può esprimere come:
\[
	\Gamma = \int{f_{dec}\left( s_{12}, s_{23}, \alpha ,\beta,\gamma \right) dL_p}
.\] 
Con $\alpha, \beta,\gamma$ angoli di Eulero e $dL_p$ elemento infinitesimo dello spazio delle fasi definito da:
\begin{align*}
	dL_p &= \frac{\mbox{d}^3 \bs{P}_1}{2E_1}\cdot\frac{\mbox{d}^3 \bs{P}_2}{2E_2}\cdot\frac{\mbox{d}^3 \bs{P}_3}{2E_3}\cdot 
	\delta^4\left(\bs{P}_{\text{in}}- \sum_{n=1}^{3} \bs{P}_i  \right)=\\
	     &= \frac{1}{32s}\text{d}s_{12}\text{d}s_{23}\text{d}\alpha \text{d}\left( \cos\beta \right)\text{d}\gamma 
.\end{align*}
Se la particella ha momento angolare nullo allora lo stato iniziale non ha una direzione privilegiata, quindi $f_{dec}$ non dipende dagli angoli di Eulero e si può integrare su questi ultimi:
\[
	d\Gamma = f_{dec}\left( s_{12}, s_{23} \right) \frac{ds_{12}ds_{23}}{32s} \int_0^{2 \pi}{d\alpha \int_{-1}^1{d\cos\beta \int_0^{2\pi}{ d\gamma }}} = \frac{\pi^2}{4s} f_{dec}\left( s_{12}, s_{23} \right) ds_{12}ds_{23}
.\]
\subsection[$\ $ Funsione di distribuzione esclusiva dei 4-impulsi]{ Definire la funzione di distribuzione esclusiva dei 4-impulsi delle particelle emergenti dopo la collisione di due particelle (oppure dopo il decadimento di una particella).}
Per la collisione di due particelle si ha:
\[
	d\sigma = f_{urto}\left( P_1 \ldots P_n \right)dL_p = f_{urto}\left( P_1 \ldots P_n \right) \frac{\mbox{d}^3 \boldsymbol{P}_1}{2E_1}\ldots \frac{\mbox{d}^3 \boldsymbol{P}_n}{2E_n} \delta^4\left( P_{in}-\sum_{i}P_i \right) 
.\] 
con $\sigma$ sezione d'urto del processo, $f_{urto}$ probabilità di misurare la sezione d'urto $d\sigma$ in un intorno di $P_1 \ldots P_n$ (contenente tutte le informazioni dinamiche del processo) e $dL_p$ è l'elemento infinitesimo dello spazio delle fasi.

\subsection[$\ $ Metodo della massa invariante]{Spiegare il metodo della ‘massa invariante’ per identificare una particella instabile e misurarne la sua massa.}
Il metodo della massa invariante è un metodo utile ad individuare particelle instabili tramite l'analisi del Dalitz Plot.\\
Ricominciamo dall'inizio: per il decadimento a 3 corpi si hanno 3n-4 = 5 variabili indipendenti, mettendosi nel sistema del centro di massa (dove il decadimento avviene in un piano) si possono scegliere gli angoli di eulero come 3 delle 5 variabili, le altre due sono aribitrarie.
È stata però adottata la convenzione di scegliere come variabili quelle che andranno a comporre il Dalitz Plot definite come:
\[
	\text{Massa inv. di 1 e 2: }  \quad \implies \quad 
	s_{12} = \left( P_1 + P_2 \right)^2 = \left( P_{in} - P_3 \right)^2 = s + m_3^2 - 2\sqrt{s} E_3 
.\] 
\[
	\text{Massa inv. di 2 e 3: } \quad \implies \quad  
	s_{23} = \left( P_2 + P_3 \right)^2 = \left( P_{in} - P_1 \right)^2 = s + m_1^2 -2 \sqrt{s} E_1 
.\] 
Può essere infine utile definire anche (non è variabile del Dalitz):
\[
	\text{Massa inv. di 1 e 3: }  \quad \implies \quad 
	s_{13} = \left( P_1 + P_3 \right)^2 = \left( P_{in} - P_2 \right)^2 = s + m_2^2 - 2\sqrt{s} E_2 
.\] 

Nelle relazioni $\sqrt{s}= E_1+E_2+E_3$ è l'energia nel centro di massa del sistema. si può notare che tutte e tre le quantità sopra sono vincolate:
\[
	\text{1 e 2 ferme} \quad \implies \quad \left( m_1 + m_2 \right)^2 \le s_{12} \le \left( \sqrt{s} - m_3 \right)^2 \quad \impliedby \quad \text{3 ferma}  
.\]
\[
	\text{1 e 3 ferme} \quad \implies \quad \left( m_1 + m_3 \right)^2 \le s_{13} \le \left( \sqrt{s} - m_2 \right)^2 \quad \impliedby \quad \text{2 ferma}  
.\] 
\[
	\text{2 e 3 ferme} \quad \implies \quad \left( m_2 + m_3 \right)^2 \le s_{23} \le \left( \sqrt{s} - m_1 \right)^2 \quad \impliedby \quad \text{1 ferma}  
.\] 
Possiamo quindi interpolare le prime 3 relazioni:
\[
	s_{12} + s_{13} + s_{23} = 3s + m_1^2 + m_2^2 + m_3^2 - 2 \sqrt{s}\left( E_1 + E_2 + E_3 \right) = s + m_1^2 + m_2^2 + m_3^2   
.\] 
E aggiungendo il vincolo su $s_{13}$:
\[
	m_1^2 + m_2^2 + 2m_2 \sqrt{s} \quad \le \quad  ( s_{12} + s_{23} ) \quad  \le  \quad  s + m_2^2 - 2m_1 m_3 
.\] 
Quindi la distribuzione di particelle finali è vincolata a stare in una porzione dello spazio delle fasi di forma rettangolare:
\begin{figure}[H]
	\centering
	\includegraphics[width=0.5\textwidth]{immagini/Dalitz.png}
	\caption{Esempio di Dalitz Plot.}
	\label{fig:Dalitz}
\end{figure}
Adesso manca di osservare che nelle variabili scelte l'elemento infinitesimo dello spazio delle fasi di può scrivere come:
\[
	dL_p = \frac{1}{32s} \ ds_{12} \ ds_{23} \ d\alpha \ d\left( \cos(\beta) \right) \ d\gamma 
.\] 

Questo risultato mostra che lo spazio delle fasi è uniformemente popolato nella zona permessa (piatto) se si utilizzano le variabili descritte.\\

Venendo al dunque si ha che, sperimentalmente, quando questo spazio non è uniformemente popolato si può dedurre che vi sia un processo intermedio non previsto: un decadimento a due corpi in cui uno dei prodotti decade a sua volta in cue corpi, si hanno allora degli addensamenti nello spazio delle fasi in zone che ci indicano la massa invariante della particella instabile (dal decadimento a due). Questo è il metodo della massa invariante.



 % risposto 
\section{Domande b}
\subsection[ Calcolo della resistenza di irraggiamento per spira investita da onda E.M.]{Calcolare la "resistenza di irraggiamento" di un circuito elettrico quadrato di lato L, piccolo rispetto alla lunghezza d'onda $\lambda$ della radiazione monocromatica incidente, se il circuito è puramente resistivo con resistenza R.\\
Calcolare anche la sezione d'urto di assorbimento e la sezione d'urto elastica se l'onda incidente ha campo magnetico perpendicolare al piano del circuito e di modulo massimo $B_0$. }
Nomi a parte il problema è schematizzato in Figura:
\begin{figure}[H]
	\centering
	\includegraphics[width=0.5\textwidth]{immagini/spira_onda.png}
	\caption{Spira immersa nel campo di onda e.m.}
	\label{fig:spira1}
\end{figure}
\paragraph{Calcolo della resistenza di irraggiamento.}
I campi ed il vettore di Poynting dell'onda sono:
\[
	\boldsymbol{E} = E_x \hat{i} = E_0 \cos\left( \omega t - kz \right) \hat{i} 
.\] 
\[
	\boldsymbol{B} = B_y \hat{j} = B_0 \cos\left( \omega t - kz \right) \hat{j} 
.\] 
\[
	\boldsymbol{S}_{in} = \frac{E_0^2}{Z_0}\cos\left( \omega t - kz \right) \hat{k} 
.\] 
Sia $I\left( t \right) $ la corrente che scorre nel circuito; possiamo sfruttare le ipotesi di dimensioni piccole (rispetto a $\lambda$) per dire che tale corrente è uniforme in tutta la spira. Trascurando anche autoinduttanza e capacità parassite possiamo affermare che il circuito ha momento di dipolo nullo.\\
Non vale lo stesso per il momento di dipolo magnetico:
\[
	\boldsymbol{p}_m = I\left( t \right) l^2 \hat{j}
.\] 
Adesso aggiungendo le ipotesi di perfetta monocromaticita dell'onda incidente e di nessuna perdita di energia per irraggiamento del circuito si calcola la corrente $I\left( t \right)$ applicando Faraday:   
\[
	\varepsilon\left( t \right) = - \frac{\mbox{d} \Phi\left( \boldsymbol{B} \right) }{\mbox{d} t} = R_{load} I\left( t \right) 
.\]
Mettiamo quindi in mezzo la geometria del circuito:
\[
	I\left( t \right) = \frac{\varepsilon\left( t \right)}{R_{load}} = 
	- \frac{1}{R_{load}} \frac{\mbox{d} \Phi\left( \boldsymbol{B} \right) }{\mbox{d} t} 
	= - \frac{1}{R_{load}} \frac{\mbox{d}}{\mbox{d}t} \left[ \int_{-l/2}^{l/2} B_0 \cos\left( \omega t - kz \right) l dz \right] 
	= \frac{\omega l^2B_0\sin\left( \omega t\right)}{R_{load}} \frac{\sin\left( kl/2 \right)}{kl/2}      
.\] 
Agginungendo l'approssimazione:
\[
	\frac{kl}{2} = \frac{\pi l}{\lambda} \ll 1 \implies I\left( t \right) = \frac{\omega l^2 B_0 \sin\left( \omega t \right) }{R_{load}}  
.\] 
Possiamo allora calcolare la potenza irraggiata:
\[
	P_{el} = \frac{\left| \ddot{\boldsymbol{p_m}} \right| ^2}{6\pi \epsilon_0 c^{5}} = \frac{\ddot{I}^2\left( t \right) l^4}{6\pi \epsilon_0 c^5}
.\] 
Se si effettua un bilancio energetico del circuito:
\[
	\varepsilon I = R_{load}I^2 + P_{el} = R_{load} I^2 + \frac{l^4}{6\pi \epsilon_0 c^5}\ddot{I}^2
.\] 
Nel caso in analisi la f.e.m. è armonica 
\[
	\varepsilon = \varepsilon_0 \sin\left( \omega t \right) \quad 
	\text{ con } \quad  
	\varepsilon_0 = \omega l^2 B_0
\]
Quindi la soluzione stazionaria per la corrente sarà anch'essa armonica: $I = I_0 \sin\left( \omega t \right) $, in conclusione:
\[
	\varepsilon I = R_{load}I^2 + \frac{\omega ^{4}l^{4}}{6\pi \epsilon_0 c^{5}} \implies \varepsilon = \left( R_{load} + R_{irr} \right) I
.\] 
Dove è stata definita la resistenza di irraggiamento (dipendente dalla frequenza):
\[
	R_{irr} = \frac{\omega^4l^4}{6\pi \epsilon_0 c^5}
.\]
Espressa in funzione della lunghezza d'onda:
\[
	R_{irr} = \omega^4 \frac{l^4}{6\pi \epsilon c^5} = \left( \frac{2\pi c}{\lambda} \right) ^4 \frac{l^4\sqrt{\mu_0 \epsilon_0} }{6 \pi \epsilon_0 c^4} = \frac{8}{3}\pi^3 Z_0\left( \frac{l}{\lambda} \right)^4 = 31.1 \text{ k}\Omega \left( \frac{l}{\lambda} \right)^4  
.\] 
\paragraph{Calcolo delle sezioni d'urto.}
Notando che 
\[
	I\left( t \right) = \frac{\varepsilon\left( t \right) }{\left( R_{load} + R_{irr} \right) }
.\] 
Possiamo ottenere la potenza assorbita e la potenza "elastica":
\[
	P_{abs} = R_{load} I^2 = \frac{R_{load}}{\left( R_{load} + R_{irr} \right) ^2}\varepsilon^2
.\] 
\[
P_{el} = R_{irr} I^2 =  \frac{R_{load}}{\left( R_{load} + R_{irr} \right) ^2}\varepsilon^2
.\] 
Quindi la potenza trasferita al carico è massima per $R_{load} = R_{irr}$.\\
Adesso basta mediare il vettore di Poynting per concludere:
\[
\left< \left| \boldsymbol{S}_{in} \right| \right> = \frac{\boldsymbol{E}_0^2}{2Z_0}
.\] 
llora le sezioni d'urto sono:
\[
	\sigma_{abs} = \frac{4\pi^2}{\lambda^2}l^4Z_0 \frac{R_{load}}{\left(R_{load} + R_{irr}\right)^2}
.\] 
\[
	\sigma_{irr} = \frac{4\pi^2}{\lambda^2}l^4Z_0 \frac{R_{irr}}{\left(R_{load} + R_{irr}\right)^2}
.\]
\[
	\sigma_{tot} = \sigma_{irr} + \sigma_{load} = \frac{4\pi^2}{\lambda^2}l^4Z_0 \frac{1}{\left(R_{load} + R_{irr}\right)}
.\] 

\subsection[ Q-valore ed energia di soglia considerando l'interazione Coulombiana]{Utilizzando le apposite tabelle che forniscono le masse dei nuclei, determinare il Q-valore o l'energia di soglia dei seguenti processi, valutando l'eventuale ruolo della interazione coulombiana nello stato iniziale:
\begin{enumerate}	
	\item	p + 40Ar $\implies$ p + 39Ar + n 
	\item	p + 14N $\implies$ X + n
	\item	p + 16O $\implies$ X + n
	\item	n + 14N $\implies$ 14C + p
	\item	4He + 14N $\implies$ 17O + p
	\item	2H + 3H $\implies$ 4He + n
	\item	2H + 2H $\implies$ 4He + $\gamma$
	\item	p + 198Hg $\implies$ 197Au + p + p
\end{enumerate}
}
Partiamo con un pò di teoria:
\[
	Q = \sum M_{in} - \sum M_{fin} 
.\] 
Se il Q-valore è positivo allora la reazione avviene in modo spontaneo: l'energia di soglia è nulla.\\
Se il Q-valore è negativo allora l'energia di soglia è maggiore di zero e dipende dalla carica del proiettile.
Se la particella proiettile è neutra allora l'energia di soglia è il modulo del Q-valore, altrimenti è necessario calcolare l'energia necessaria a vincere l'interazione columbiana per arrivare al nucleo (essendo le interazioni sopra scritte tutte forti) nel sistema del laboratorio.\\
Per effettuare il calcolo sfruttiamo la conservazione dell'energia e della quantità di moto non relativistiche in una dimensione, chiariamo la notazione:
\begin{itemize}
	\item $R$: raggio del nucleo colpito 
	\item $m_{prt}$: massa del proiettile.
	\item v$_0$: velocità iniziale del proiettile.
	\item $T$: energia cinetica iniziale del proiettile.
	\item $Z$: protoni del nucleo colpito.
	\item  $M$: massa del nucleo colpito.
	\item $d$: distanza in cui i nuclei si urtano definita dalla somma dei raggi delle due particelle coinvolte:
		 \[
			 d = R + r_{prt} \approx \left( 1.25 A^{1/3} + r_{\text{skin}} + r_{prt} \right) \text{ fm} = \left( 1.25 A^{1/3} + 2 + r_{prt} \right) \text{ fm}  
		.\] 
\end{itemize}
Facendo il conto:
 \[
\begin{cases}
	m_{prt}\text{v}_0 = \left( m_p + M \right)V_{cm}\\
	T \ge \frac{1}{2}\left( m_{prt} + M \right)V_{cm} + \frac{Ze^2}{4\pi \epsilon_0 d} \\
	T = \frac{1}{2}m_{\text{prt}}\text{v}_0^2
\end{cases}
\]
Quindi sviluppando per $T$ si ottinene l'energia cinetica necessaria per la reazione:
\[
	T \ge \left( 1 + \frac{m_{prt}}{M} \right) \frac{Ze^2}{4\pi \epsilon_0 d} = T_{\text{min}} 
.\] 
e per l'energia di soglia dobbiamo soltanto sommare il modulo del Q-valore:
\[
	E_{\text{soglia}} = T_{\text{min}} + \left| Q \right| 
.\] 
In questo modo possiamo risolvere tutte le interazioni elencate.

\paragraph{Interazione 1.}
Bisogna notare che $40 \text{Ar} = \ce{^{40}_{18}\text{Ar}}$ (vedi tabelle con difetti di massa), quindi:
\[
	\ce{p + 40\text{Ar} -> p + 39\text{Ar} + n}
.\] 
\[
	Q = \left( m_p + 40m_u + \Delta_{40, 18} \right) - \left( m_p + 39 m_u + \Delta_{39, 18} + m_n  \right) = m_u + \Delta_{40, 18} - \Delta_{39, 18} - m_n      
.\]
Numericamente:
\[
	Q \approx \left( 931.49 + \left( -35.04 \right) - \left( -33.24 + 939.57 \right)  \right)\text{Mev} \approx -8.3 \text{ MeV} \quad \quad \text{Reazione endotermica}
.\]
Per il calcolo dell'energia di soglia applichiamo subito quando visto sopra:
\[
	E_{\text{soglia}} = \left( 1 + \frac{m_p}{M_{40\text{Ar}}} \right) \frac{Ze^2}{4 \pi \epsilon_0 d} + \left| Q \right| \quad \quad 
	\text{con } M_{40Ar} = 40m_u + \Delta_{40,18}
.\]
si calcola la distanza minima d (il raggio del protone è circa 1.25 fm): 
\[
	d \approx \left( 1.25 \cdot \left( 40 \right)^{ 1/3 } + 2 + r_{protone} \right)\text{fm} \approx 6.25 \text{ fm} 
.\] 
Quindi l'energia di soglia (calcolo numerico approssimato a mente\ldots):
\[
	E_{\text{soglia}} \approx 4 \text{ MeV} + 8 \text{ MeV} \approx 12 \text{ MeV}   
.\] 
\paragraph{Interazione 2.}%
\[
	\ce{p + 14N -> 14O + n}
.\]
\[
	Q = -6.54 \text{ MeV} \quad \quad E_{\text{soglia}}=8.26 \text{ MeV}
.\] 
\paragraph{Interazione 3.}%
\[
	\ce{p + 160 -> 16F + n }
.\] 
\[
	Q = -16.8 \text{ MeV} \quad \quad E_{\text{soglia}}=18.7 \text{ MeV}
.\] 
\paragraph{Interazione 4.}%
\[
	\ce{n + 14N -> 14C + p}
.\] 
\[
	Q = -1.557 \text{ MeV} \quad \quad E_{\text{soglia}}=1.557 \text{ MeV}
.\] 
\paragraph{Interazione 5.}%
\[
	\ce{4He + 14N -> 17O + p}
.\] 
\[
	Q =-4.432  \text{ MeV} \quad \quad E_{\text{soglia}}=4.432 \text{ Miv}
.\] 
eccetera eccetera\ldots
\subsection[ Equivalenza tra sezione d'urto elastica con fotoni incidenti o onda monocromatica incidente]{Dimostrare la relazione fra la definizione della sezione d’urto elastica nel caso di fotoni incidenti su un unico bersaglio e la definizione di sezione d'urto elastica per un’onda e.m. monocromatica su un unico bersaglio.}
Nel caso di onda monocromatica su un bersaglio si ha:
\[
	\sigma_{\text{el}} = \frac{\left< P_{el}\right>}{\left<\left| \boldsymbol{S}_{in} \right|  \right>}
.\]
Mentre per un fascio di fotoni incidenti:
\[
	\sigma_{\text{el}} = \frac{1}{\left| \boldsymbol{j}_{\gamma}\right|}\frac{\mbox{d} N_{el}}{\mbox{d} t}
.\] 
L'equivalenza delle due deriva dal fatto che se si moltiplica e si divide la seconda per $ \hbar \omega$:
\[
	\sigma_{\text{el}} = \frac{\frac{\mbox{d} N_{el}}{\mbox{d} t}}{\left| \boldsymbol{j}_{\gamma}\right|} = \frac{\frac{\mbox{d} N_{el}}{\mbox{d} t}\hbar \omega }{\left| \boldsymbol{j}_{\gamma}\right|\hbar \omega} = \frac{\left< P_{el}\right>}{\left<\left| \boldsymbol{S}_{in} \right|  \right>}
.\] 	

\subsection[ Numero di eventi su volume e tempo negli urti tra particelle con $v_{\text{rel}}$ distribuita con $f\left( v_{\text{rel}} \right) $]{Quale calcolo si deve effettuare per determinare il numero di eventi per unità di tempo e di volume che si producono negli urti fra particelle di due specie diverse e differenti concentrazioni le cui velocità relative sono distribuite con un funzione $f(V_{rel})$, normalizzata all'unità, e la cui sezione d'urto è $\sigma(V_{rel})$?}
Date due specie con densità volumica $n_a$ e  $n_b$ che si scontrano e con densità di prodotti $n_f$, se la $ f\left( V_{\text{rel}} \right) $ è normalizzata allora si ha:
\[
	\frac{\mbox{d} n_{\text{f}}}{\mbox{d} t } = n_a n_b \int_0^{\infty} f\left( v_{rel} \right) \sigma_{\text{f}}\left( v_{rel} \right) v_{rel} dv_{rel}
.\] 
Tipicamente la $f\left( V_{rel} \right)$ è gaussiana. 

\subsection[ Attenuazione di fascio di particelle su materiale omogeneo con atomi di una specie]{Calcolare l'attenuazione di un fascio di particelle incidenti su un materiale omogeneo e composto da atomi di una sola specie in funzione della profondità [dati: sezione d'urto del processo su ogni atomo del bersaglio, densità di massa del mezzo, numero atomico del mezzo].}
\label{sec:2.b.5}
\paragraph{Lamina sottile}
Data una lastra di materiale omogeneo definiamo alcune (tante, forse troppe) quantità utili: spessore $\Delta x$, densità $\rho$, area $\Delta S$, volume di lastra considerato $V$ , sezione d'urto su ogni atomo bersaglio (totale) $ \sigma_{\text{tot}}$, $n_b$ la concentrazione di besagli nel materiale e $n_a$ la concentrazione di particelle incidenti. Possiamo riscrivere queste in funzione delle quantità date dal testo e sfruttare qualche utile relazione.\\ 
Partiamo da $n_b$: definendo $M_{\text{tot}}$ la massa totale di lastra nel volume, $M_A$ la massa di una mole della sostanza del materiale in grammi, $N_a$ il numero di avogadro si ha:
\[
	V \cdot n_b = N_{\text{tot}} = moli\cdot N_a= \frac{M_{\text{tot}}}{M_A} N_a \implies n_b =\frac{N_{\text{tot}}}{V} = \frac{\rho}{M_A} N_a
.\]
La probabilità di interazione nel volume considerato è definita da:
\[
	P_{\text{int}} = n_b \cdot \sigma_{\text{tot}} \Delta x  = \frac{\rho N_a }{M_A} \sigma_{\text{tot}}\Delta x
.\] 
A questo punto si vede come è definita la profondità di penetrazione (essendo la $P_{\text{int}}$ adimensionale):
\[
	\mathcal{L} = \frac{M_A}{\rho N_a\sigma_{\text{tot}}}
.\] 
quindi l'attenuazione è data dalla frazione di particelle che riescono a passare, quantità che si può ricavare come il complementare della probabilità di interagire: la probabilità di passare.
\[
	A = 1 - \frac{\Delta x}{\mathcal{L}} \quad \quad 
	\text{Nel caso di lamina sottile}
.\] 
Quanto visto fin'ora funziona finche la lamina si può considerare sottile: $\Delta x < \mathcal{L}$, se questa approssimazione viene meno è necessario rivedere alcuni conti.
\paragraph{Materiale generico} Possiamo pensare ad un materiale generico come la sovrapposizione di tante lamine sottili di diversa superficie. Si può quindi vedere la probabilità di interazione come una funzione della posizione $P_{\text{int}}\left( x \right)$, quindi anche la stessa attenuazione sarà funzione della posizione $A\left( x \right) $. Calcoliamo l'attenuazione alla posizione $x + \Delta x$, dobbiamo utilizzare le regole delle probabilità combinate:
\[
	A\left( x + \Delta x \right) = A\left( x \right) A\left( \Delta x \right) = A\left( x \right) \left( 1- P_{\text{int}}\left( \Delta x \right)  \right) = A\left( x \right) \left( 1 - \frac{\Delta x }{\mathcal{L}} \right)  
.\]
Applicando il rapporto incrementale risulta quindi evidente che il risultato sarà esponenziale ( cosa che goffamente sapevamo già dal momento che si applicano le proprietà della probabilità di eventi ripetuti).
\[
	\frac{A_{\text{int}}\left( x + \Delta x \right) - A_{\text{int}}\left( x \right)}{\Delta x} = -\frac{A\left( x \right) }{\mathcal{L}} 
.\]
Facendo tentere $\Delta x$ a zero:
\[
	\dot{A} = -\frac{A}{\mathcal{L}} \implies A\left( x \right) = e^{-x/\mathcal{L}} \quad \quad 
	\text{Materiale generico}
.\] 

\subsection[ Attenuazione di fascio di particelle su materiale omogeneo con atomi di più specie]{Calcolare l'attenuazione di un fascio di particelle incidenti su un materiale omogeneo e composto da atomi di diverse specie in funzione della profondità [dati: sezione d'urto del processo su ogni atomo del bersaglio, densità di massa del mezzo, numeri atomici, composizione chimica del mezzo]}
\paragraph{Lamina sottile}
Essendo nota la composizione chimica del mezzo è noto anche la percentuale di atomi che compongono il materiale:
\[
	\text{Composizione del mezzo: } \ X^{\left( 1 \right) }_{a_1}\ldots X^{(N)}_{a_N}
.\]
Con $X^{i}$ specie atomica, $a_j$ pedice che indica la composizione chimica nella formula del composto (per non appesantire si trascurano le formule con simboli ripetuti tipo $CH_3COOH$).
Si può quindi ragionare come se avessimo N copie del nostro materiale ognuno interamente composto da una singola specie preservando le densità $\rho_i$ che sono presenti nel materiale originale, successivamente si applica il principio di sovrapposizione sommando tutte le attenuazioni e normalizzando sulle N specie:
\[
	A_i = 1 - \frac{\Delta x}{\mathcal{L}_i} \implies A_{\text{tot}} = \frac{\sum_{n=1}^{N} A_n}{N} = 1 - \sum_{n=1}^{N} \frac{\Delta x}{N \mathcal{L}_n}
.\] 
con la penetrazione definita a partire dalla densità e dalla massa molare delle singole specie:
\[
	\mathcal{L}_i = \frac{M_{A}^{\left( i \right) }}{\rho_i N_a \sigma_{\text{tot}}}
.\]
\paragraph{Materiale generico}
Con passaggi del tutto analoghi alla domanda precedente si arriva alla conlcusione:
\[
	A_{\text{tot}} =\frac{\sum_{n=1}^{N} e^{-x/\mathcal{L}_n}}{N} 
.\] 
\subsection[ Stima numerica della sezione d'urto forte in alcuni processi]{Effettuare una stima numerica della sezione d'urto totale forte per i seguenti urti: 
\begin{enumerate}
	\item	p + 40Ar
	\item	n + 14N
	\item	4He + 14N
	\item	2H + 3H
\end{enumerate}
}
Se si considera solo la sezione d'urto forte non c'è bisogno di preoccuparsi di interazioni elettrodeboli: si suppone che l'energia sia sufficiente da poter trascurare questo tipo di interazioni. Il calcolo si riduce alla stima della superficie di possibile impatto tra i due oggetti assunti come sferici:
\[
	\sigma_{\text{strong}} \approx \pi\left( R_1 + R_2 \right)^{2} 
.\] 
Con $R_1$ e  $R_2$ raggi delle particelle interagenti. Per tutti gli atomi in gioco ho scelto di considerare sempre $r_\text{skin}$.
\paragraph{Interazione 1.}
\[
	\sigma_1 \approx \pi\left( r_p + R_{40\text{Ar}} \right)^{2}  \approx \pi (1.25 \cdot (40)^{1/3} + 2 + 1.25 )^{2} \text{ fm}^{2} \approx \pi \cdot (6.25)^{2} \text{ fm}^{2}
\]
Quindi 
\[
	\sigma_1 \approx \pi \cdot 39 \text{ fm}^{2} \approx 120 \text{ fm}^{2} = 1.2 \text{b}
\]
\paragraph{Interazione 2.} 
$\sigma_2 \approx 1.23 \text{b}$

\paragraph{Interazione 3.}
$\sigma_3 \approx 2.54 \text{b} $

\paragraph{Interazione 4.}
$\sigma_4 \approx 1.7 \text{b}$

\subsection[ Protone con energia tipica di LHC su protone fermo per ottenere 14 TeV]{Calcolare l’energia che dovrebbe avere un protone che incide su un protone fermo per ottenere una energia nel centro di massa pari a quella di LHC (14TeV).}
Sia $E$ l'energia del protone nel laboratorio, si ha:
\[
	E^2_{\text{cm}} = \left( E_{lab} \right)^{2} -  \boldsymbol{P}_{\text{lab}}^{2} = \left( E + m_p \right)^{2} - \left(E^2 - m_p^2\right) = 2m_p^2 + 2m_pE 
.\]
Quindi l'energia necessaria è circa $E = 10^{6} $ TeV

\subsection[ Energia per innescare $\ce{e^{+} + e- -> p + \bar{p}}$]{Calcolare l'energia di degli elettroni/positroni per innescare la reazione:
\[
	e^{+} \ + \ e^{-} \implies p \ + \ \overline{p}
\] 
in cui i due leptoni collidono con 3-impulsi opposti e di modulo diverso.}

Se $E_1$ e $E_2$ sono le energie dei due leptoni si ha 
\[
	\left( E_1 + E_2 \right) ^2 - \left( \sqrt{E_1^2 - m_e^2} - \sqrt{E_2^2 - m_{e}^2}  \right) \ge 4 m_p^2
.\]
L'energia di soglia si ottinene studiando la funzione nelle due variabili sopra, da lì si evince che il minimo si ha per $E_1 = E_2$, quindi:
\[
	E_{\text{min}} = m_p
.\] 

\subsection[\hspace{2mm} Energia di soglia per reazioni a due (in cui un reagente è fermo)]{Calcolare l’energia di soglia nel laboratorio per le seguenti reazioni (la seconda particella è inizialmente ferma):
\begin{enumerate}
	\item	$\gamma \ + \ {}^{16} O \implies e^{+} \ + \ e^{-} \ + \ {}^{16}O $
	\item	$\gamma \ + \ e^{-} \implies e^{-} \ + \ e^{+} \ + \ e^{-}$
	\item	$p \ + \ p \implies p \ + \ p \ + \ p \ + \ \overline{p}$
	\item	$p \ + \ {}^{16}O \implies p \ + \ p \ + \ \overline{p} \ + \ {}^{16}O$
	\item	$e^{+} \ + \ e^{-} \implies p \ + \ \overline{p}$
	\item	$e^{-} \ + \ p \implies n \ + \ \nu_e$
	\item	$\overline{\nu_e} \ + \ p  \implies n \ + \ e^{+}$
\end{enumerate}
}

Per tali calcoli è necessaria una generalizzazione della risposta al quesito precedente, è infatti richiesto che:
\[
	\left( \sum_i P_{i,\text{in}} \right)^2 \ge \left(\sum_{i} m_{i, \text{fin}} \right)^2  
.\] 
Adesso si può sfruttare il fatto che i reagenti sono soltanto due e che il secondo è sempre a riposo (generalizzazione della domanda 2.b.8):
\[
	E_1 \ge \frac{1}{2m_2}\left[ \left( \sum_{i} m_{i, \text{fin}} \right)^2 - \left( m_1^2 + m_2^2 \right)  \right] 
.\]
E adesso si tratta solo di infilare dentro i numeri.

\subsection[\hspace{2mm} Probabilità che il neutrino interagisca attraversando la terra lungo il diametro]{Calcolare la probabilità che un neutrino interagisca nell’attraversare la Terra lungo un diametro.\\
Nota: sia assuma che l’energia del neutrino sia tale che la sezione d’urto totale su un singolo nucleone sia 1 fb $= 10^{-15}\cdot 10^{-24}$ cm$^2$. } 
Si assume per il calcolo $\rho_{\text{T}} \approx 5.5 \text{g}/\text{cm}^3$.\\
Possiamo ipotizzare una bassa probabilità di interazione per il neutrino, è quindi possibile assumere la terra come una lamina sottile (con immenso piacere dei terrapiattisti) e calcolare la probabilità cercata come:
\[
P_{\text{int}} = n \sigma d \approx 4.2 \cdot 10^{-6} 
.\] 
Con $d \approx 12.76 \cdot 10^{3} \text{ km}$ diametro terrestre e $n \approx = \rho_{\text{T}}\cdot N_{\text{A}} / 1 \text{[g]} \approx 5.5\cdot 6 \cdot 10^{23} \text{cm}^{-3}$ densità media di protoni sulla terra.

\subsection[\hspace{2mm} Oscillazione dell'elettrone con forze di richiamo, viscosa, radiativa nel campo di un'onda E.M.]{Dimostrare che un elettrone (non relativistico) soggetto ad una forza elastica di richiamo, ad una forza di attrito viscoso ed alla forza di reazione radiativa, nel campo di un’onda e.m. piana polarizzata linearmente oscilla con la legge:
\[
	\boldsymbol{x} = \frac{e \boldsymbol{E_0}}{m_{e}} \frac{1}{\omega_0^2-\omega^2-i\omega\Gamma_{tot}} e^{-i \omega t} \quad \quad 
	\text{ con }  \quad \quad
	\Gamma_{tot} = \Gamma' + \Gamma \frac{\omega^2}{\omega_{0}^2}
\] 
} \label{subsec: 2.b.12}
Facendo riferimento ai risultati delle Domande \hyperref[subsec: 2.a.15]{2.a.15}, \hyperref[subsec: 2.a.16]{2.a.16}  si prosegue con il calcolo.
L'equazione di moto dell'elettrone in questo caso è (considerando anche la $F_{\text{rad}}$):
\[
m_e \ddot{\boldsymbol{x}} = q \boldsymbol{E_0}e^{-i \omega t} - m_e \tau \dddot{\boldsymbol{x}} - m_e \Gamma' \dot{\boldsymbol{x}} - m_e\omega_0^2 \bs{x}
.\] 
Spostando l'incognita vettoriale a destra si ha:
\[
-\tau \dddot{\boldsymbol{x}} + \ddot{\boldsymbol{x}} + \Gamma' \dot{\boldsymbol{x}} + \omega _0^2 \boldsymbol{x} = \frac{q \boldsymbol{E_0}}{m_e} e^{-i \omega t}
.\] 
Cercando la soluzione stazionaria $\boldsymbol{x} = \boldsymbol{x}_0 e^{-i \omega t}$ si ha:
\[
	-\tau \left( -i \omega \right)^3 \boldsymbol{x}_0e^{-i \omega t} + \left( - i \omega \right) \Gamma'\boldsymbol{x}_0 e^{-i \omega t} + \omega _0^2 \boldsymbol{x}_0 e^{-i \omega t} = \frac{q \boldsymbol{E}_0}{m_e} e^{-i \omega t}
.\] 
Quindi:
\[
\boldsymbol{x}_0 = \frac{q \boldsymbol{E}_0/m_e}{ \omega _0^2 - \omega ^2 -i \omega \Gamma' - i \tau \omega ^2 }
.\]
È quindi utile definire $\Gamma_{\text{tot}} = \Gamma' + \tau \omega ^2 = \Gamma' + \Gamma \frac{\omega ^2}{\omega _0^2}$ per giungere alla conclusione:
\[
\boldsymbol{x} = \frac{q \boldsymbol{E}_0}{m_e} \frac{e^{-i \omega t}}{\omega _0^2 - \omega ^2 - i \omega \Gamma_{\text{tot}}} 
.\] 

\subsection[\hspace{2mm} Sezione d'urto differenziale elastica per onda monocromatica su elettrone legato elasticamente]{Dimostrare che la sezione d’urto differenziale elastica per un’onda e.m. piana e monocromatica su un elettrone legato elasticamente vale 
\[
	\frac{\mbox{d} \sigma_{el}}{\mbox{d} \Omega} = r_e^2 L\left( \omega \right) \sin^2\left( \alpha  \right)    
\]
con $\alpha$ angolo fra la direzione di osservazione e direzione di polarizzazione (lineare) dell'onda.} \label{subsec: 2.b.13}
La risposta al quesito è stata prematuramente scritta alla Domanda \hyperref[subsec: 2.a.16]{2.a.16} facendo uso del risultato ottenuto nella Domanda \ref{subsec: 2.b.12}.
Si riaccenna solo al fatto che è stata definita $L\left( \omega  \right)$ come:
\[ 
	L \left( \omega \right) = \frac{\omega^4}{\left( \omega_{0}^2 - \omega^2 \right)^2 + \omega^2 \Gamma_{tot}^2 } \quad \quad 
.\] 
Vediamo di dimostrare quanto scritto, sopra abbiamo ottenuto:
\[
\boldsymbol{x} = \frac{q \boldsymbol{E}_0}{m_e} \frac{e^{-i \omega t}}{\omega _0^2 - \omega ^2 - i \omega \Gamma_{\text{tot}}} 
.\] 
Calcoliamo il vettore di Poynting (mediato nel tempo) associato alla potenza irraggiata dall'elettrone:
Esplicitando l'espressione del campo di radiazione di dipolo elettrico in funzione della variabile $\alpha$ del problema si ha:
\[
	\boldsymbol{E_{\text{dip}}} =k_0 \frac{\left( e\ddot{\boldsymbol{x}} \wedge \hat{r}\right) \wedge \hat{r}}{\left| \boldsymbol{r} \right| c^2} 
.\] 
\[
	\left<\boldsymbol{S}_{\text{el}} \right> = \left< \boldsymbol{E} \wedge H\right> = \left<\frac{\left| \boldsymbol{E}_{\text{dip}} \right|^2}{c \mu_0}  \right> = 
	k_0^2\frac{\left| e \ddot{\boldsymbol{x}}  \right|^2 \sin^2\left( \alpha  \right) }{c^4 r^2 \mu_0 } =
	\frac{1}{32 \pi^2 \epsilon_0} \frac{ \left| e \ddot{\boldsymbol{x}}_0 \right|^2 \sin^2\left( \alpha \right) }{c^3 r^2} \hat{r}
.\] 
Che in CGS risulta più elegante:
\[
\left<\boldsymbol{S}_{\text{el}} \right> = \frac{1}{8 \pi} \frac{ \left| e \ddot{\boldsymbol{x}}_0 \right|^2 \sin^2\left( \alpha \right) }{c^3 r^2} \hat{r} 
.\] 
Dividendo questo vettore per il vettore di poynting iniziale (da qui in poi CGS per semplicità) si ottiene l'espressione cercata:
\begin{align*}
\frac{\mbox{d} \sigma_{\text{el}}}{\mbox{d} \Omega} = \frac{\left< \boldsymbol{S}_{\text{el}} \right> \cdot r^2 \hat{r}}{\frac{c}{8 \pi}\left| E_0 \right|^2 } = 
\frac{ \omega^{4} e^{4}}{c^{4}m_e^2} \frac{\sin^2\left( \alpha  \right) }{\left( \omega _0^2 - \omega ^2 \right)^2 + \omega ^2 \Gamma_{\text{tot}}^2} =
r_e^2 \frac{\omega ^{4}\sin^2\left( \alpha \right)}{\left( \omega _0^2 - \omega ^2 \right)^2 + \omega ^2 \Gamma_{\text{tot}}^2}
.\end{align*}
Dove è necessario riconoscere il raggio classico in CGS:
\[
	r_e^{\left( \text{CGS} \right) } = \frac{e^2}{c^2m_e}
.\] 
Va da se che in MKSA:
\[
	\frac{\mbox{d} \sigma_{\text{el}}}{\mbox{d} \Omega} = \left( \frac{k_0 e^2}{m_e c^2} \right) ^2 \frac{\omega ^{4}\sin^2\left( \alpha \right)}{\left( \omega _0^2 - \omega ^2 \right)^2 + \omega ^2 \Gamma_{\text{tot}}^2}
.\] 

%2.b.14
\subsection[\hspace{2mm} Dimostrare il valore della sezione d'urto Thompson]{Dimostrare che la sezione d’urto Thomson vale $\sigma_{Th} = \frac{8}{3}\pi r_{e}^2  = 0.66 \text{ barn}$.}
Si può arrivare allo Scattering Thompson riscrivendo l'equazione di moto dell'elettrone e trascurando la pressione di radiazione di cui si è invece fatto uso nelle precedenti domande:
\[
m_e \ddot{\boldsymbol{x}}= e \boldsymbol{E}_0 e^{-i \omega t}
.\] 
La potenza irraggiata di dipolo sarà quindi:
\[
P_{\text{irr}} = k_0 \frac{e^2 \left| \ddot{\boldsymbol{x}} \right|^2}{3c^3} 
.\]
Sviluppando i conti e ricordando la definizione di sezione d'urto si arriva banalmente a:
\[
\sigma_{\text{Th}}=\frac{8}{3}\pi r_e^2
.\] 
Ricordando ancora che il \hyperref[eq: raggio classico]{Raggio classico dell'elettrone} è: $r_e = k_0 \frac{e^2}{m_e c^2} \approx 2.82$ fm.

%2.b.15
\subsection[\hspace{2mm} Sezione d'urto elastica per onda elettromagnetica piana su elettrone legato elasticamente]{Dimostrare che la sezione d’urto elastica per un’onda e.m. piana e monocromatica su un elettrone legato elasticamente vale:
\[
	\sigma_{el} = \sigma_{Th} L\left( \omega \right) 
\] 
}
È in pratica richiesto di trovare la Funzione di Brieght-Wiegner. Visto che abbiamo la sezione differenziale tuttavia è sufficiente integrarla su tutti gli angoli. Ci si riduce allora al calcolo dell'integrale:
\[
	\int \sin^2\left( \alpha  \right) d \Omega = \int \frac{1 + \cos^2\left( \alpha \right) }{2} d\Omega = 2\pi + \frac{1}{2} 2 \pi \int_0^{2\pi} \cos^2\left( \alpha  \right) \sin\left( \alpha  \right) d \alpha = \frac{8}{3}\pi  
.\] 
Tutto il resto esce indisturbato dall'integrale sull'angolo solido, da cui la tesi.

%2.b.16
\subsection[\hspace{2mm} Sezione d'urto totale per onda piana su elettrone legato elasticamente]{Dimostrare che la sezione d’urto totale per un’onda e.m. piana e monocromatica su un elettrone legato elasticamente vale:
\[
	\sigma_{\text{tot}} = 4 \pi r_{e} c \frac{\omega^2 \Gamma_{\text{tot}}}{\left( \omega_0^2-\omega^2 \right)^2+\omega^2\Gamma_{\text{tot}}^2}
\] 
}
Abbiamo gia risolto l'equazione del moto:
\[
\boldsymbol{x}= \frac{e \boldsymbol{E}_0}{m} \frac{e^{- i \omega t}}{\omega _0^2 - \omega ^2 - i \omega \Gamma_{\text{tot}}}
.\] 
Si tratta quindi solo di ricordare che la potenza totale dissipata può essere espressa come:
\[
	P_{\text{tot}} = \left<\boldsymbol{v} \cdot \boldsymbol{F} \right> = \left<\dot{\boldsymbol{x}}\cdot q \boldsymbol{E} \right> = \left< e \dot{\boldsymbol{x}} \cdot \boldsymbol{E} \right> = \frac{1}{2} q \mathcal{R}e \left\{ \dot{\boldsymbol{x}} \cdot \boldsymbol{E}^* \right\} 
.\]
Quindi:
\[
	P_{\text{tot}} = \frac{q^2 \omega ^2 \left| \boldsymbol{E}_0 \right|^2 \Gamma_{\text{tot}}}{2m\left( \left(\omega _0^2 - \omega ^2 \right) ^2 + \omega ^2 \Gamma_{\text{tot}}^2  \right) }
.\] 
Resta adesso da dividere per il vettore di Poynting incidente:
\[
\sigma_{\text{tot}} = \frac{P_{\text{tot}}}{\left<\left| \boldsymbol{S}_{\text{in}} \right|\right>} = 4 \pi r_{e} c \frac{\omega^2 \Gamma_{\text{tot}}}{\left( \omega_0^2-\omega^2 \right)^2+\omega^2\Gamma_{\text{tot}}^2}
.\] 
Per completezza si scrive anche:
\[
\sigma_{\text{abs}} = \sigma_{\text{tot}} - \sigma_{\text{el}} = 4 \pi r_{e} \omega^2 L\left( \omega \right) \left[ c \Gamma_{\text{tot}} - \frac{2}{3}r_e \omega^2 \right]
.\] 
\begin{figure}[H]
	\centering
	\includegraphics[width=1.1\textwidth]{immagini/b-w.png}
	\caption{Andamento delle sezioni d'urto della Bright-Wiegner}
	\label{fig:Andamento Bright-Wiegner}
\end{figure}
Nella figura si mostra come vanno le sezioni d'urto, è necessario notare che, per un rendering più fedele, sarebbero serviti più punti plottati (la funzione non ha raggiunto il massimo atteso). Per pigrizia è stato testato che arrivasse al massimo ma non è stata riportata l'immagine; insomma fidarsi o provare per credere.

%2.b.17
\subsection[\hspace{2mm} Approssimare la Brieght-Wiegner con una Lorentziana]{Dimostrare che la sezione d’urto elastica per un’onda e.m. piana e monocromatica su un elettrone legato elasticamente in prossimità di una risonanza stretta (specificare il criterio) si può approssimare con una curva lorentziana
\[
	\sigma_{el} = \sigma_{Th} \frac{\omega_{0}^2 / 4}{ \left( \omega_0 - \omega \right)^2 + \frac{ \left( \Gamma' + \Gamma  \right)^2 }{4} }
\] 
}
La B-W si approssima con una lorenziana in un intorno (dell'ordine della larghezza $\Gamma + \Gamma'$) di $\omega_0$: $\omega \approx \omega_0$.\\
La risonanza è stretta se la larghezza a metà altezza è molto inferiore alla frequenza di risonanza: $\Gamma + \Gamma' \ll \omega_0$.
Passiamo alle approssimazioni allora:
\begin{align*}
	\sigma_{\text{el}} = \sigma_{\text{Th}}\frac{\omega ^{4}}{\left( \omega _0 - \omega\right)^2 \left( \omega_0 + \omega \right)^2 + \omega^2\left( \Gamma + \Gamma' \frac{\omega^2}{\omega_0^2} \right)^2} \approx \\
	\approx \sigma_{\text{Th}} \frac{\omega _0^4}{4 \omega _0^2 \left( \omega _0 - \omega  \right)^2 + \left( \Gamma + \Gamma' \right)^2 \omega _0^2  } \approx \\
	\approx \sigma_{\text{Th}}\frac{\omega_0^2/4}{\left( \omega - \omega_0 \right)^2 + \left(\frac{\Gamma + \Gamma'}{2}\right)^2 }
.\end{align*}

% 2.b.18
\subsection[\hspace{2mm} Valori nel picco per le sezioni d'urto di onda su elettrone legato elasticamente]{Dimostrare che per un’onda e.m. piana e monocromatica su un elettrone legato elasticamente le sezioni d'urto al picco valgono:
\[
	\sigma_{el} = \frac{3 \lambda_{0}^2}{2 \pi} \left( \frac{\Gamma}{\Gamma + \Gamma'} \right)^2 \quad \quad \quad \quad \quad \quad \text{ }
\]
\[
	\sigma_{TOT} = \frac{3 \lambda_{0}^2}{2 \pi} \frac{\Gamma}{\Gamma + \Gamma'} \quad \quad \text{Con $\lambda_{0} = \frac{2 \pi c}{\omega_{0}}$}
\]
\[
	\sigma_{inel} = \frac{3 \lambda_{0}^2}{2 \pi} \frac{\Gamma \Gamma'}{\left( \Gamma + \Gamma' \right)^2 } \quad \quad \quad \quad \quad \quad \text{ } 
\]
}
Le tre sezioni d'urto in questione sono:
\begin{align*}
	&\sigma_{\text{el}}= \frac{8}{3}\pi r_e^2 \frac{\omega^4}{\left( \omega_0^2-\omega^2 \right)^2+\omega^2\Gamma_{\text{tot}}^2} \\
	&\sigma_{\text{inel}}= \frac{4\pi r_e\omega^2}{\left(\omega_0^2-\omega^2\right)^2-\omega^2\Gamma_{\text{tot}}^2}
	\left(c\Gamma_{\text{tot}}-\frac{2}{3}\omega^2r_e\right)  \\
	&\sigma_{\text{tot}}= 4\pi r_e c \frac{\omega^2\Gamma_{\text{tot}}}{\left( \omega_0^2-\omega^2 \right)^2-\omega^2\Gamma_{\text{tot}}^2}
.\end{align*}
Accettanto il fatto che tutte le tre sezioni d'urto hanno un massimo per $\omega = \omega_0$ basta prendere le sezioni d'urto e inserire $\omega = \omega_0$. Per ottenere la forma di sopra è necessario scrivere il raggio classico come nella derivazione della forza di radiazione:
\[
	r_e = \frac{3}{2}\tau c= \frac{3}{2} \frac{\Gamma}{\omega_0^2}
.\] 

%2.b.19
\subsection[\hspace{2mm} Legge di smorzamento di energia ed ampiezza per elettrone legato elasticamente]{Dimostrare che un elettrone (moto non relativistico) soggetto ad una forza elastica di richiamo, ad una forza di attrito viscoso ed alla forza di reazione radiativa, se viene lasciato libero di oscillare da una posizione iniziale perde energia con una 1 legge esponenziale in cui la costante tempo vale $\frac{1}{\Gamma' + \Gamma}$. Come si chiama questa costante tempo? Quale sarebbe la costante tempo con cui, invece, si smorza l'ampiezza delle oscillazioni?}
Riprendiamo l'equazione di moto dell'elettrone, tuttavia adesso togliamo la forzante dovuta all'onda incidente.
\[
	- \tau \dddot{\boldsymbol{x}} + \ddot{\boldsymbol{x}} + \Gamma' \dot{\boldsymbol{x}} + \omega_0^2 \boldsymbol{x} = 0 
.\]  
Adesso è necessario ricordare le \hyperref[eq:relazione-parametri-elettrone]{relazioni} tra i vari parametri in gioco:
\[
	\Gamma' \ll \omega_0 \ll \frac{1}{\tau}, \quad \quad \quad  
	\Gamma \ll \omega_0
.\] 
La prima è una questione puramente di grandezze fisicamente tipiche, la seconda deriva dalla prima $\left( \omega_0 \tau \ll 1 \right) $ e dal fatto che $\Gamma = \tau \omega_0^2 \ll 1 \cdot \omega_0$.\\
È quindi ragionevole cercare una soluzione oscillante e smorzante con smorzamento debole rispetto alla pulsazione:
\[
	\boldsymbol{x} = \boldsymbol{x}_0 e^{-i\left( \omega_0 - i \gamma/2 \right)t } = \boldsymbol{x}_0 e^{-i \omega_0t} e^{-\gamma t /2}, \quad \quad \quad 
	\gamma \ll \omega_0
.\]
Adesso la festa è nel sostituire questa soluzione nella equazione buttando via i termini trascurabili:
\[
	- \tau \left( -i\left( \omega_0 - i \gamma /2 \right)  \right)^3 + \left( -i\left( \omega_0 - i \gamma /2 \right) \right)^2 - \left( -i\left( \omega_0 - i \gamma /2 \right) \right) \Gamma' - \omega_0^2 = 0 
.\]
Sostituendo $\tau = \Gamma / \omega_0^2$:
\[
	-i \frac{\Gamma}{\omega_0^2}\left( \omega_0^3 - \frac{3}{2} i \gamma \omega_0^2 + \ldots \right) - \left( \omega_0^2 - i \gamma \omega_0 + \ldots \right) + 
	\left( -i \Gamma' \omega_0 - \frac{1}{2} \gamma \Gamma'  \right) + \omega_0^2 \approx 0 
.\] 
Sempre sulla base delle approssimazioni sopra è possibile notare che i termini reali sono trascurabili (raggruppare alcuni $\Gamma$ o $\Gamma'$ nei punti giusti per vederlo), ci si riduce alla forma:
\[
	-i \Gamma \omega_0 - \frac{3}{2} \Gamma \gamma - i \Gamma' \omega_0 + i \gamma \omega_0 - \frac{1}{2} \gamma \Gamma' \approx 0 \implies 
	- \Gamma \omega_0 - \Gamma' \omega_0 + \gamma \omega_0 \approx 0 
.\] 
Che ci porta alla conclusione:
\[
\gamma \approx \Gamma + \Gamma' 
.\]
Quindi l'ampiezza delle oscilazioni è smorzata con una costante tempo data da:
\[
	\boldsymbol{x} = \ldots \cdot e^{- t /\tau_{\text{osc}}} \quad \quad \text{con} \quad \quad
\tau_{osc} = \frac{2}{\gamma} = \frac{2}{\Gamma + \Gamma'}
.\]
Se consideriamo invece l'andamento della energia è necessario tener conto del fatto che essa è quadratica nella velocità (cinetica) e nella posizione (potenziale): 
\[
E = E_0 e^{- \gamma t} \quad \implies \quad \tau_{\text{energia}} = \frac{1}{\gamma} = \frac{1}{\Gamma + \Gamma'}
.\]
In tutto questo macello il risultato importante è uno: la larghezza totale di uno stato risonante è il reciproco della sua vita media, risonanza stretta = particella longeva e viceversa.

% 2.b.20
\subsection[\hspace{2mm} Relazione tra parametro di impatto (b) e angolo di scattering ($\theta$) nel caso Rutherford]{Calcolare la relazione tra parametro d'impatto (b) e angolo di scattering ($\theta$) nel caso di scattering di Rutherford (Coulombiano) e di scattering su sfera rigida.} 
\label{sec:2.b.20}
Prima di iniziare con questi argomenti è bene dare una rilucidatà ad alcune grandezze tipiche in esame: le dimensioni atomiche.
\begin{figure}[H]
	\centering
	\includegraphics[width=0.6\textwidth]{immagini/dim-atomo.png}
	\caption{Dimensioni atomiche tipiche}
	\label{fig:atomo}
\end{figure}
Veniamo quindi agli scattering discussi, la situazione è modellizzata in Figura \ref{fig:rutherford}:
\begin{figure}[H]
	\centering
	\includegraphics[width=0.8\textwidth]{immagini/rutherford.png}
	\caption{Schema dello scattering Rutherford.}
	\label{fig:rutherford}
\end{figure}
Per tutte le seguenti affermazioni viene considerto il centro scatterante come fisso (con massa molto maggiore del proiettile).
\paragraph{Interazione Columbiana.}
Inizialmente la particella ha una velocità $v_0$ che per la conservazione dell'energia e per simmetria deve essere uguale a quella finale $v_f$:
 \[
\left| \boldsymbol{v}_0 \right|  = \left| \boldsymbol{v}_f \right| 
.\] 
Si può trovare la variazione di impulso dopo interazione:
\[
	\Delta p = \left| \Delta \boldsymbol{p} \right| = \sqrt{\left( m \boldsymbol{v}_f - m \boldsymbol{v}_0 \right)^2} = m \sqrt{v_0^2 + v_0^2 -2v_0^2 \cos\left( \theta \right)} = 2m v_0 \sin\left( \frac{\theta}{2} \right)   
.\]
 
È utile anche ricordare la relazione che lega $\mu$ (angolo di riferimento che giace sul piano perpendicolare alla $\boldsymbol{v}_0$) all'elemento infinitesimo di sezione d'urto:
\[
d \sigma = b db d\mu
.\] 
\begin{figure}[H]
    \centering
    \incfig{elemento-infinitesimo-sez}
    \caption{Elemento infinitesimo della sezione d'urto}
    \label{fig:elemento-infinitesimo-sez}
\end{figure}

Essendoci qua una simmetria sotto rotazioni attorno all'asse $\hat{v}_0$ possiamo integrare sull'angolo $\mu$ della relazione:
\[
d\sigma = 2 \pi b db
.\] 
Altra relazione differenziale che ci è utile adesso è: 
\[
\frac{\mbox{d} \sigma}{\mbox{d} \mu} = b db
.\] 
La sezione d'urto differenziale si può scrivere allora in funzione di $b\left( \theta \right)$ :
\[
	\frac{\mbox{d} \sigma}{\mbox{d} \Omega} = \frac{\mbox{d} \sigma}{\mbox{d}\mu \mbox{d}\cos\left( \theta \right) } = \frac{ b \mbox{d}b}{\mbox{d}\cos\left( \theta \right)} = - \frac{b \text{d}b}{ \sin\left( \theta \right) d \theta }  
.\]
Consideriamo adesso la variabile $\varphi$ nella Figura \ref{fig:rutherford} indice della posizione angolare dell'oggetto durante l'interazione, è definita  tra:
\[
\varphi_{\text{min}} = -\frac{\pi - \theta}{2} \le \varphi \le \frac{\pi -\theta}{2} = \varphi_{\text{max}}
.\] 
Possiaom giocarci la conservazione del momento angolare durante il processo sfruttando la variabile sopra definita come cordinata polare.
\[
	L_z = m \left( \boldsymbol{r \wedge \boldsymbol{v}}\right)_z = m \left[ \boldsymbol{r} \left( \dot{r} \hat{r} + r \dot{\varphi} \hat{\varphi}\right)\right]=
	mr^2\dot{\varphi}
.\] 
Quindi considerando anche il momento angolare iniziale abbiamo una prima relazione per parametrizzare il differenziale nel tempo:
\[
m v_0 b = m r^2 \frac{\mbox{d} \varphi}{\mbox{d} t} \quad \quad \implies \quad \quad dt = \frac{r^2}{b v_0} d\varphi
.\] 
Perche parametrizzare il tempo? Perche ci è utile per relazionare la variabile $b$ a $\theta$! Infatti la variazione di impulso calcolata sopra può essere scritta come:
\[
\left| \Delta \boldsymbol{p} \right| = \int_{-\infty}^{\infty} \left| \boldsymbol{F}_{\perp}\right| dt = 
\int_{-\infty}^{\infty} \left| \boldsymbol{F}\right| \cos\left( \varphi \right)  dt 
.\]
La forza citata è quella columbiana tra i due corpi:
\[
\left|\boldsymbol{F}\right| = \frac{zZe^2}{4 \pi \epsilon_0 r^2} 
.\] 
Mettiamo tutto nell'integrale (compreso il dt):
\[
	\left| \Delta \boldsymbol{p} \right| = \int_{\varphi_{\text{min}}}^{\varphi_{\text{max}}} \frac{zZe^2}{4 \pi \epsilon_0 b r^2} \frac{\cos\left( \varphi \right) }{b} \frac{r^2}{v_0} d \varphi 
	=  \frac{zZe^2}{2 \pi \epsilon_0 b v_0} \cos\left( \frac{\theta}{2} \right) 
.\] 
Senza mollare eliminiamo il $\left| \Delta \boldsymbol{p} \right|$ ricavato all'inizio:
\[
	2mv_0 \sin\left( \frac{\theta}{2} \right) = \frac{zZe^2}{2 \pi \epsilon_0 b v_0} \cos\left( \frac{\theta}{2} \right) 
.\] 
E le nostre fatiche vengono ripagate perche abbiamo la prima risposta:
\[
	b\left( \theta \right) = \frac{zZe^2}{4 \pi \epsilon_0 m  v_0^2} \cot\left( \frac{\theta}{2} \right) 
.\] 
È quindi possibile definire anche una minima distanza di urto centrale $d$ quando la cotangente è unitaria:
\[
	d = \frac{zZe^2}{4 \pi \epsilon_0 m  v_0^2} = \frac{zZ\left( \alpha \hbar c \right)}{T} \approx zZ \frac{1.44[\text{MeV}]}{T[\text{MeV}]} [\text{fm}]
	\quad \quad \implies \quad \quad b\left( \theta \right) = d \cot\left( \frac{\theta}{2} \right) 
.\] \label{eq:d-rutherford}
Apprezziamo la grandezza del risultato, siamo in grado di prevedere, note la carica e la massa del proiettile, la distanza ortogonale ($b$) tra i centri incidenti dal momento che scegliamo l'angolo di osservazione $\theta_0$ (o meglio, tutto ciò che arriva all'osservatore è partito da una posizione con parametro di impatto noto).\\
Si può infine trovare la sezione d'urto Rutherford differenziale come:
\[
	\frac{\mbox{d} \theta}{\mbox{d} \Omega} = -\frac{b db}{\sin\left( \theta \right) d \theta} =
	-\frac{b}{2 \cos\left( \frac{\theta}{2} \right) \sin\left( \frac{\theta}{2} \right)} \cdot \frac{d}{2}\left( \frac{d \left( \cot\left( \frac{\theta}{2} \right)\right)}{d \theta} \right) = 
	\ldots = \frac{d^2}{16 \sin^{4}\left( \frac{\theta}{2} \right) } 
.\] 

\paragraph{Interazione con sfera rigida}
In questo caso basta sfruttare alcune semplici considerazioni geometriche:
\[
	\frac{\text{d}\sigma}{\text{d} b} = 2 \pi b \quad \quad \text{ con } \quad b<R 
.\] 
A fare la differenza qua è la semplice definizione di $\theta$:
\[
	\frac{b}{R} = \sin\left( \frac{\pi - \theta}{2} \right) = \cos\left( \frac{\theta}{2} \right)  \implies b = R \cos\left( \frac{\theta}{2} \right) 
.\] 

% 2.b.21
\subsection[\hspace{2mm} ]{Calcolare la minima distanza fra le due particelle in uno scattering Rutherford.}
Definiamo $v_m$ come la velocità del proiettile nell'istante in cui ha raggiunto la distanza minima $x$. Sfruttando la conservazione dell'energia e del momento angolare si ottiene:
\[	
	\frac{1}{2}m v_0^2 = \frac{1}{2} m v_f^2 + \frac{zZe^2}{4 \pi \epsilon_0 x}\\
.\] 
\[
	mv_0b = m v_m x
.\]
Si ricava $v_m$ dalla seconda sostituendola nella prima, ne risulta una equazione del secondo grado per $x$:
 \[
	 x^2 - d \cdot x - b^2 = 0 \implies x = \frac{d}{2} \left( 1 + \frac{1}{\sin\left( \frac{\theta}{2} \right) } \right) 
.\] 
%2.b.22
\subsection[\hspace{2mm} ]{Calcolare l'energia minima affinché un protone possa avere una interazione forte "toccando" un nucleo di ${}^{12}C$ o di ${}^{28} Si$.}
È inanzitutto necessario per risolvere il problema di minimo considerare l'urto centrale: $b = 0$.\\ 
In tal caso l'angolo di scatternig è $\theta = \pi$ e $x = d$ dove $d$ è la distanza tra i nuclei aventi raggio:
\[
	R = \left( 1.25 A^{1 /3} + 2 \right) \text{fm} 
.\] 
L'energia minima è allora proprio l'energia potenziale necessaria ad arrivare alla distanza $d$:
 \[
	 E_{\text{min}} = \frac{zZe^2}{4 \pi \epsilon_0 \left( R_1 + R_2 \right) } 
.\] 
O molto più semplicemente (vista la fatica fatta in precedenza per \hyperref[eq:d-rutherford]{definire d}:
\[
	d = \left( R_1 + R_2 \right) \approx zZ\frac{1.44 [\text{MeV}]}{T[\text{Mev}] } [\text{fm}] \implies T = zZ\frac{1.44}{\left( R_1+R_2\right) } [\text{MeV}]
.\] 
Mettendo i numeri si ha:
\[
E_{\text{C}} \approx 1.4 \text{MeV} ,\quad \quad 
E_{\text{Si}} = 11.2 \text{MeV}
.\] 

%2.b.23
\subsection[\hspace{2mm} ]{Discutere le differenze tra lo scattering di Rutherford (particelle $\alpha$ su nuclei) e lo scattering di elettroni su bersaglio puntiforme.}
La differenza principale è che gli elettroni si muovono solitamente a velocità relativistiche, questo invalida i conti fatti prima. Nel caso discusso quindi è necessario tirare in ballo la sezione d'urto Mott.

%2.b.24
\subsection[\hspace{2mm} ]{Cercando i dati nelle apposite tabelle (reperibili sul web ) si indichino gli stati finali e si calcoli il Q-valore per i decadimenti delle seguenti specie instabili: ${}^8B, {}^{39}Ar, {}^{7}Be, {}^{64}Cu, {}^{76}Ge$. }
Per rispondere a questa domanda è necessario ricordarsi la natura dei \hyperref[sec:decadimenti]{Decadimenti} ed avere sottomano il Nuclear Wallet Card.
\paragraph{Isotopo del Boro}
\label{par:8B}
\[
	\ce{\ce{^{8}_{5}B_{3}} ->[\epsilon] \ce{^{8}_{4}\text{Be}_{4}} ->[\alpha] \ce{^{4}_{2}\text{He}^{2-}_{2}} } 
.\]
\[
	\Delta Q_{\epsilon} = \Delta_{A, Z} - \Delta_{A, Z-1} \approx 18 \text{ Mev} \quad \quad  
\]
\[
	\Delta Q_{\alpha} = ( 4m_u +  \Delta_{A,Z-1}) - \left( \Delta_{A-4, Z-2} + 4m_u + \Delta _{\alpha} \right) \approx 0.1 \text{ MeV}
.\] 
Quindi l'energia complessiva del processo è 
\[
	\Delta Q_{\epsilon}+ \Delta Q_{\alpha} \approx 18.1 \text{ MeV}.
\]
Notare che in questo caso particcolare i prodotti di decadimento $\alpha$ e $\ce{^{4}_{2}\text{He}^{2-}_{2}}$ hanno approssimativamente la stessa massa.
\paragraph{Isotopo dell'Argon}
\label{par:39Ar}
\[
	\ce{\ce{^{39}_{18}\text{Ar}_{21}}  ->[\beta^-] \ce{^{39}_{19}K_{20}}}
.\] 
\[
	\Delta Q_{\beta^-} = \Delta_{A,Z} - \Delta_{A, Z+1} = 0.57 \text{ MeV} 
.\] 
\paragraph{Isotopo del Berillio}%
\label{par:7Be}
\[
	\ce{\ce{^{7}_{4}\text{Be}_{3}} ->[\epsilon] \ce{^{7}_{3}\text{Li}_{4}} }
.\] 
\[
\Delta Q_{\epsilon} = \Delta_{A, Z} - \Delta_{A, Z-1} \approx 1.861 \text{ MeV}
.\] 
\paragraph{Isotopo del Rame}%
\label{par:64Cu}
Nelle tabelle sono indicate due possibilità, esaminiamo entrambe:
\[
	\ce{ \ce{^{64}_{30}\text{Zn}_{34}} <-[\beta^-][38.5 \%] \text{ } \ce{^{64}_{29}\text{Cu}_{35}} \text{ } ->[\epsilon][61.5\%] \ce{^{64}_{28}\text{Ni}_{36}} }
.\] 
Senza essere ripetitivi (stesse cose viste sopra) si riportano i risultati:
\[
\Delta Q_{\epsilon} = 1.67 \text{ MeV}
.\] 
\[
\Delta Q_{\beta^-} = 0.68 \text{ MeV}
.\] 

\paragraph{Isotopo del Germanio}%
\label{par:76Ge}
\[
	\ce{^{76}_{32}\text{Ge}_{44}} \implies \tau_{1 /2} \sim 2 \cdot 10^{21} 
.\]
Al gran sasso si studia il decadimento $\beta^- \beta^-$ di Questo isotopo (nome in codice GERDA).
\paragraph{Nota:} I conti della domanda sono stati fatti senza calcolatore per allenamento, siete fortemente invitati a fare lo stesso e riportare gli errori a chi ha accesso al source code.

% 2.b.25
\subsection[\hspace{2mm} ]{Cercando i dati nelle apposite tabelle (reperibili sul web) si trovino i Q-valori per le reazioni 
\begin{enumerate}
	\item $n + {}^{154}Gd \implies \gamma + {}^{155}Gd$ 
	\item $n + {}^{155}Gd \implies \gamma + {}^{156}Gd$
\end{enumerate}
}
\[
	\Delta Q_1 = \left( m_n + 154 m_u + \Delta_{154, 64} \right) - \left( 155m_u + \Delta_{154, 64 } \right) \approx 6.43 \text{ MeV} 
.\]
\[
	\Delta Q_2 = \ldots \approx m_{n} - m_{u} + \Delta_{155,64}-\Delta_{156,64} \approx 7.4 \text{ MeV} 
.\] 

% 2.b.26
\subsection[\hspace{2mm} ]{Dimostrare che $\frac{d^3 p  }{2E}$ è un invariante relativistico effettuando esplicitamente la trasformazione di Lorentz (si consideri il boost lungo un asse, per esempio l'asse x)}
Facciamo questo boot lungo x:
\begin{align*}
	\frac{\mbox{d}^3 \boldsymbol{P}'}{2E'}  =& \frac{dP'_x dP'_y dP'_z}{2Ex'} = \\
	=& \frac{\gamma d\left( P_x + \beta E \right)dP_y dP_z }{2 \gamma\left( E + \beta P_x \right)} = \\
	=& \frac{ \left[ dP_x + \beta d \sqrt{P_x^2 + m^2}\right]dP_y dP_z  }{2\left( E + \beta P_x \right) } = \\ 
	=& \frac{\left( dP_x + \beta\frac{2P_x dP_x}{2\sqrt{P_x^2 + m^2} } \right)dP_y dP_z }{2\left( E + \beta P_x \right)} = \\
	= & \frac{\mbox{d}^3 \boldsymbol{P}}{2E}
.\end{align*}

% 2.b.27
\subsection[\hspace{2mm} ]{Dimostrare che 
\[
d^{4}P \delta\left( P^2-m^2 \right) \theta\left( P_0 \right) = \frac{d^3 \boldsymbol{P} }{2E}
\]
e sfruttare questo risultato per semplificare la scrittura dell’elemento infinitesimo dello spazio dei 4-impulsi di N particelle emergenti dopo la collisione di due particelle (oppure dopo il decadimento di una particella).}
È sufficiente rimembrare una delle proprietà della $\delta$:
\[
	\delta\left( f\left( x \right)  \right) = \sum_{j} \frac{\delta\left( x-x_j \right) }{\left| \left[ f'\left( x \right)  \right]_{x = x_j} \right| }
.\] 
Da inserire opportunamente nella espressione a sinistra dell'uguale nella richiesta: dobbiamo ricordare che $P$ è in realtà un quadrivettore.
\begin{align*}
	d^{4}P \cdot \delta\left( P^2-m^2 \right) \theta\left( P_0 \right) =& d^4P\cdot  \delta\left(\boldsymbol{P}^2 - P_0^2 - m^2 \right) \theta\left( P_0 \right)  = \\
	=& d^4 P \cdot  \theta\left( P_0 \right) \left[ \frac{\delta\left( P_0 - E \right) }{\left| \left[ 2P_0 \right]_{P_0 = E} \right| } + 
		\frac{\delta\left( P_0 + E \right) }{\left| \left[ 2P_0 \right]_{P_0 = -E}  \right| } \right] =\\
		=& d^4P \frac{\delta\left( P_0 - E \right)}{2E} = \frac{d^3 \boldsymbol{P}}{2E}   
.\end{align*}
\paragraph{Nota:}%
Stiamo parlando di distribuzioni, la notazione leggera utilizzata nasconde significati matematici profondi e non scontati (vedi corso di metodi matematici per la fisica II).

% 2.b.28
\subsection[\hspace{2mm} ]{Dimostrare che nel centro di massa l’elemento infinitesimo dello spazio dei 4-impulsi, nel caso di 2 sole particelle nello stato finale, si scrive come $\frac{|\boldsymbol{p_{cm}}}{4\sqrt{s}} d\Omega_{cm}.$}
Ricordiamo la forma dell'elemento infinitesimo dello spazio delle fasi:
\[
	dL_p = \left[ d^4P_1 \cdot \delta\left( P_{0,1}^2 - \boldsymbol{P}_1^2 - m_1^2 \right) \cdot \theta\left( P_{0,1} \right) \right] \cdot \ldots 
	\left[ \cdot d^4P_n \cdot \delta\left( P_{0,n}^2 - \boldsymbol{P}_n^2 - m_n^2 \right) \cdot \theta\left( P_{0,n} \right) \right] \cdot 
	\delta^4\left(  P_{in}-\sum_{i}P_i  \right) 
.\]
Che utilizzando la risposta alla domanda precedente si riscrive come:
\[
dL_p = \frac{\mbox{d}^3 \boldsymbol{P}_1}{2E_1} \ldots \frac{\mbox{d}^3 \boldsymbol{P}_n}{2E_n} \delta^4\left( P_{in}-\sum_{i}P_i \right) 
.\] 
Decisamente più carino.\\
Se si hanno due prodotti definiamo le loro masse come $m_1$ e  $m_2$, l'energia del centro di massa $\sqrt{s}$ e ricordiamo che il 3-impulso totale nel centro di massa è nullo. L'elemento infinitesimo $dL_p$ si scrive come:
\begin{align*}
	dL_p =& \frac{d^3 \boldsymbol{P}_1}{2E_1} \frac{d^3 \boldsymbol{P}_2}{2E_2} \cdot \delta^4\left(P_{\text{in}} - P_1 - P_2 \right) = \\ 
	= & \frac{d^3 \boldsymbol{P}_1}{2E_1} \frac{d^3 \boldsymbol{P}_2}{2E_2} \cdot \delta\left( \sqrt{s}-E_1-E_2\right) \cdot 
	\delta^3\left( \boldsymbol{0} + \boldsymbol{P}_1 + \boldsymbol{P}_2 \right) = \\
	= & \frac{ d \boldsymbol{P}_1 }{4E_1E_2} \cdot \delta\left( \sqrt{s}-E_1-E_2\right)   
\end{align*}
Passando in cordinate spefiche:
\[
	dL_p = \frac{P_1^2 dP_1 d \Omega_1 }{4E_1E_2} \cdot \delta\left( \sqrt{s} -E_1-E_2  \right) 
.\]
Possiamo adesso sfruttare il fatto che nel centro di massa l'impulso totale è nullo, chiamiamo $P_{\text{cm}}$ l'impulso della particella singola (le due particelle hanno in modulo lo stesso impulso in questo sistema $\left| \boldsymbol{P}_1 \right| = \left| \boldsymbol{P}_2 \right| = P_{\text{cm}}$):
\[
	\sqrt{s}= \sqrt{P_{\text{cm}}^2 + m_1^2} + \sqrt{ P_{\text{cm}}^2 + m_2^2} 
.\] 
Che può essere risolta ( Hint: servono due elevazioni al quadrato) per $P_{\text{cm}}$ :
\[
	P_{\text{cm}} = \sqrt{\frac{\left( s - \left( m_1+m_2 \right)^2 \right) \left( s - \left( m_1 - m_2 \right)^2 \right)}{4s}} 
.\]
Non esplicitiamo $P_{\text{cm}}$ nei calcoli a seguire, è bene però sappere che può essere scritto in funzione di variabili più "comuni". La relazione sopra sarà sfruttata al secondo passaggio del seguente calcolo.
\begin{align*}
	dL_p =& \frac{P_1^2 dP_1 d \Omega_1}{4E_1E_2} \cdot \delta\left( \sqrt{s} - \sqrt{P_1^2 + m_1^2} - \sqrt{P_2^2 + m_2^2}  \right) \\
	=& \frac{P_1^2 dP_1 d \Omega_1}{4E_1E_2} \cdot 
	\frac{\delta\left(P_1-P_{\text{cm}}\right)}{\left|\left[\frac{\mbox{d}}{\mbox{d}P_1}\left(\sqrt{s}-\sqrt{P_1^2+m_1^2}-\sqrt{P_1^2+m_2^2}\right)\right]_{P_1=P_{\text{cm}}}\right|} = \\
	=& \frac{P_1^2 dP_1 d \Omega_1}{4E_1E_2} \cdot \frac{ \delta \left( P_1-P_{\text{cm}} \right) }{ \left| \left[ \frac{P_1}{E_1}+\frac{P_1}{E_2} \right]_{P_1=P_{\text{cm}}} \right|}=\\
	=& \frac{P_1^2 dP_1 d \Omega_1}{4E_1E_2} \cdot \frac{\delta\left(P_1-P_{\text{cm}}\right)}{P_{\text{cm}} \frac{E_1 + E_2}{E_1E_2}} = \\
	=& \frac{P_{\text{cm}} }{4\sqrt{s}} d\Omega_{1}
\end{align*}
Noto l'impulso delle particelle nel centro di massa (o, se si vuole, $\sqrt{s}, m_1, m_2$) si ha che l'elemento infinitesimo dello spazio delle fasi dei due prodotti dipende solo dall'angolo solido di un prodotto (che in caso di isotropia spaziale come qua è l'opposto di $\Omega_2$). Questo è molto utile per la sezione d'urto differenziale ad esempio.
\[
	\frac{\mbox{d} \sigma}{\mbox{d} \Omega_1} = f_{\text{urto}}\left( \Omega_1 \right) \frac{P_{\text{cm}}}{4\sqrt{s} }
.\] 

% 2.b.29
\subsection[\hspace{2mm} ]{Nel caso di 3 particelle nello stato finale di una reazione, dimostrare che fra il quadrato della massa invariante di due di esse e l'energia della terza (nel centro di massa) sussite una relazione lineare.}
Prendiamo la massa invariante citata $\sqrt{s_{12}}$ e l'energia della restante paticella come $E_3$, con ovvio significato dei simboli si ha (definizione di massa invariante):
\[
	s_{12} = \left( P_1 + P_2 \right)^2
.\] 
Che per la conservazione dell'energia e dell'impulso si può scrivere come:
\[
	s_{12} = \left( P_{\text{in}} - P_3 \right)^2 = P_{\text{in}}^2 + P_3^2 - 2 P_{\text{in}}P_3 =s + m_3^2 - 2 \sqrt{s}E_3 
.\] 
Ecco la relazione lineare.

% 2.b.30
\subsection[\hspace{2mm} ]{Come si trasforma una funzione di distribuzione del 3-impulso 
\[
	f\left( \boldsymbol{p} \right)d^3 \boldsymbol{p}
\]
di una particella per una trasformazione di Lorentz?}
Premessa: misurare in un sistema la funzione di distribuzione del 3-impulso significa misurare il numero di eventi dn nell'elemento di volume nello spazio degli impulsi $dn = f\left( \boldsymbol{P} \right) d^3 \boldsymbol{P}$. È inoltre naturale che il numero di eventi registrati deve essere lo stesso in ogni sistema di riferimento 
\[
dn = f\left( \boldsymbol{P} \right) d^3 \boldsymbol{P} =  f\left( \boldsymbol{P}' \right) d^3 \boldsymbol{P}' = dn'
\]
Abbiamo dimostrato anche che 
\[
\frac{d^3 \boldsymbol{P}}{2E} 
.\] 
è un invariante relativistico, quindi si può concludere:
\[
	\frac{E}{E} f\left( \boldsymbol{P} \right) d^3 \boldsymbol{P} = \frac{E'}{E'} f\left( \boldsymbol{P}' \right) d^3 \boldsymbol{P}' \quad \implies \quad 
	f\left( \boldsymbol{P}' \right) = \frac{E}{E'} f\left( \boldsymbol{P} \right) 
.\] 

% 2.b.31
\subsection[\hspace{2mm} ]{Come si trasforma una funzione di distribuzione nello spazio delle fasi 
\[
	f\left(\boldsymbol{p},\boldsymbol{r}\right)d^3 \boldsymbol{p} \ d^3 \boldsymbol{r} 
\]
di una particella per una trasformazione di Lorentz?}
È necessario notare che anche qua il numero di eventi $dn$ nel volume infinitesimo dello spazio delle fasi si deve conservare come sopra, ragionando quindi allo stesso modo si trova la legge di trasformazione richiesta.\\
Mettiamoci per utilità nel sistema in cui la particella è a riposo e consideriamo un boost lungo uno dei 3 assi di riferimento, possiamo sfruttare la contrazione delle lunghezze
\[
\text{d}^3 \boldsymbol{r}' =\frac{\text{d}^3\boldsymbol{r}}{\gamma} =  \frac{\text{d}^3\boldsymbol{r}}{E'} mc^2 = \frac{\text{d}^3\boldsymbol{r}}{E'} E
.\] 
Quindi $Ed^3 \boldsymbol{r}$ è un invariante relativistico. Avendo trovato che anche $d^3 \boldsymbol{P} / E$ è invariante si ha che
\begin{align*}
	dn = f\left( \boldsymbol{P}, \boldsymbol{r} \right) \text{d}^3 \boldsymbol{P} \text{d}^3 \boldsymbol{r} =  f\left( \boldsymbol{P}', \boldsymbol{r}' \right) 
	\text{d}^3 \boldsymbol{P}' \text{d}^3 \boldsymbol{r}' = dn' & \quad \quad
	\text{Invarianza del numero di eventi} \\
	\frac{\text{d}^3 \boldsymbol{P}}{E} E \text{d}^3 \boldsymbol{r} =
	\frac{\text{d}^3 \boldsymbol{P}'}{E'} E' \text{d}^3 \boldsymbol{r}' & \quad \quad 
	\text{Invarianti trovati sopra}
.\end{align*}
Dalle quali deriva l'invarianza di $f\left( \boldsymbol{P}, \boldsymbol{r} \right)$.

% 2.b.32
\subsection[\hspace{2mm} ]{Dimostrare che se la probabilità di decadimento di una particella per unità di tempo non dipende dal tempo, la probabilità di trovare la particella non decaduta al tempo t segue una legge esponenziale.}
Sia $S\left( t \right) $ la probabilità di trovare la particella al tempo $t$, considerando la invece la probabilità $P\left( t \right)$ di decadere dopo l'unità di tempo $dt$ si ha (per ipotesi):
 \[
	 \frac{P\left( dt \right)}{dt}= \frac{1}{\tau} \text{ (costante)} \implies  S\left( dt \right) = 1 - \frac{dt}{\tau}
.\] 
Quindi all'istante $t + dt$ si ha (per le regole delle probabilità combinate):
\[
	S\left( t + dt \right) = S\left( t \right) S\left( dt \right) = S\left( t \right) \left( 1 - \frac{dt}{\tau} \right) 
.\] 
Che ha come soluzione proprio un esponenziale (vedi \hyperref[sec:2.b.5]{Domanda 2.b.5}):
\[
	S\left( t \right) = e^{- t / \tau}
.\] 

% 2.b.33
\subsection[\hspace{2mm} ]{Dire quali fra le seguenti particelle sono soggette ad interazioni forti: $ p, \overline{p} $, $\pi^{+}, \pi^{-}, \mu^{+}, \mu^{-}, e^{+}$, $e^{-}$, $\alpha,$ Nucleo di Azoto, $\nu, \overline{\nu}$ }
\label{sec:2.b.33}
Solo gli adroni (composti da Quark) possono fare interazione forte, quindi la risposta è: $p$ , $\overline{p}$ , $\pi^{+}$ , $\pi^{-}$ , $\alpha$ , Nucleo di Azoto.

\subsection[\hspace{2mm} ]{Pioni neutri, di energia E nel sistema del laboratorio, decadono in due fotoni. La distribuzione è isotropa ne centro di massa. Si calcoli la distribuzione dell’energia di uno dei due fotoni nel laboratorio e gli angoli, rispetto alla direzione di volo del pione, dei due fotoni nel sistema del laboratorio in funzione dell’angolo nel sistema del centro di massa.}
\begin{figure}[H]
	\centering
	\includegraphics[width=0.35\textwidth]{immagini/decadimento-pi1.png}
	\includegraphics[width=0.35\textwidth]{immagini/decadimento-pi2.png}
	\caption{Decadimento del $\pi^0$ in due fotoni: a sinistra nel centro di massa e a destra nel sistema del laboratorio.}
	\label{fig:Decadimento del Pione in fotoni.}
\end{figure}
Fissando idee e notazione come in Figura \ref{fig:Decadimento del Pione in fotoni.} procediamo al calcolo richiesto.
\[
	\ce{\ce{\pi} -> \ce{\gamma} + \ce{\gamma}}
.\] 
La massa del $\pi^0$ è 134.96 MeV, assumiamo che nel laboratorio questo si muova con velocità $\beta$ nell'istante precedente al decadimento. Nel sistema del $\pi^0$ si ha:
 \[
P_1' = \frac{m}{2} \cdot 
\begin{pmatrix}
	1 \\
	\cos\left( \theta_{\text{cm}} \right) \\
	\sin\left( \theta_{\text{cm}} \right) \\
	0
\end{pmatrix} \quad \quad \quad 
P_2' = \frac{m}{2}\cdot 
\begin{pmatrix} 
	1 \\
	-\cos\left( \theta_{\text{cm}}\right) \\
	- \sin\left( \theta_{\text{cm}} \right) \\
	0
\end{pmatrix} 
.\] 
Essendo inoltre la distrubuzione di prodotti isotropa nel centro di massa possiamo scrivere:
\[
	\frac{\text{d} \Gamma}{\text{d}\Omega_{\text{cm}}} = \frac{\Gamma}{4 \pi} \quad \quad \implies \quad \quad 
	\frac{\text{d} \Gamma}{\text{d}\cos\left( \theta_{\text{cm}} \right) } = \frac{\Gamma}{2}
.\] \label{eq:isotropia}
Nel laboratorio si ha invece:
\[
P_1 = E_1 \cdot 
\begin{pmatrix}
	1 \\
	\cos\left( \theta_1 \right) \\
	\sin\left( \theta_1 \right) \\
	0
\end{pmatrix} \quad \quad \quad 
P_2 = E_2\cdot
\begin{pmatrix} 
	1 \\
	\cos\left( \theta_2 \right) \\
	 \sin\left( \theta_2 \right) \\
	0
\end{pmatrix} 
.\] 
È possibile limitare le energie dei prodotti nel laboratorio tramite una trasformazione di Lorentz:
\begin{align*}
	E_1=&\gamma\frac{m}{2}\left(1+\beta\cos\left(\theta_{\text{cm}}\right)\right)&
		\quad E_2=&\gamma\frac{m}{2}\left(1-\beta\cos\left(\theta_{\text{cm}}\right)\right)\\
	P_{1,x} = E_1 \cos\left( \theta_{1} \right) =& \gamma \frac{m}{2} \left( \beta + \cos \left(\theta_{\text{cm}} \right) \right)   &
		\quad P_{2,x} = E_2 \cos\left( \theta_2 \right) =& \gamma \frac{m}{2} \left( \beta - \cos\left( \theta_{\text{cm}} \right)  \right) \\ 
	P_{1,y} = E_1 \sin\left( \theta_1 \right) =& \frac{m}{2}\sin\left( \theta_{\text{cm}}\right) &
		\quad P_{2, y} =E_2\sin\left( \theta_2 \right) =& - \frac{m}{2} \sin\left( \theta_{\text{cm}}\right) 
.\end{align*}
Guardando alla prima colonna facendo variare il $\cos\left( \theta_{\text{cm}} \right) $ con la fantasia è facile vedere che:
\[
	\gamma \frac{m}{2}\left( 1 - \beta \right) \le E_1 \le \gamma \frac{m}{2} \left( 1 + \beta \right) 
.\] 
Possiamo adesso calcolarci come è distribuita l'energia $E_1$ (della particella 1 nel laboratorio):
\[
	\frac{\mbox{d} \Gamma}{\mbox{d} E_1} = 
	\frac{\mbox{d} \Gamma}{\mbox{d} \cos\left( \theta_{\text{cm}} \right) } \frac{\mbox{d} \cos\left( \theta_{\text{cm}} \right) }{\mbox{d} E_1} =
	\frac{\Gamma}{2} \frac{2}{m \gamma \beta} = \frac{\Gamma}{m \gamma \beta}
.\] 
La distribuzione è quindi piatta (non dipende da $E_1$) e ben normalizzata se integrata nei limiti di $E_1$.\\
Per quanto riguarda la seconda richiesta procediamo calcolando la distribuzione angolare di uno dei fotoni.
Per farlo possiamo sfruttare ancora le trasformazioni di Lorentz di cui sopra isolando gli angoli nel laboratorio (essenzialmente la risposta al quesito è questa):
\begin{align*}
	\cos\left(\theta_1\right)=&\frac{P_{1,x}}{E_1}=\frac{\beta+\cos\left(\theta_{\text{cm}}\right)}{1+\beta\cos\left(\theta_{\text{cm}}\right)}\quad&  
	\cos\left(\theta_2\right)=&\frac{P_{2,x}}{E_2}=\frac{\beta-\cos\left(\theta_{\text{cm}}\right)}{1-\beta\cos\left(\theta_{\text{cm}}\right)} \\  
	\sin\left(\theta_1\right)=&\frac{P_{1,y}}{E_1}=\frac{\sin\left(\theta_{\text{cm}}\right)}{\gamma\left(1+\beta\cos\left(\theta_{\text{cm}}\right)\right)}\quad&
	\sin\left(\theta_2\right)=&\frac{P_{1,y}}{E_2}=\frac{-\sin\left(\theta_{\text{cm}}\right)}{\gamma\left(1-\beta\cos\left(\theta_{\text{cm}}\right)\right)}
.\end{align*}
Dalla prima in alto si ricava la cosa che ci serve: $\cos\left( \theta_{\text{cm}} \right)$ in funzione di $\cos\left( \theta_1 \right)$:
\[
	\cos\left( \theta_{\text{cm}} \right) = \frac{-\beta + \cos\left( \theta_1 \right) }{1 - \beta\cos\left( \theta_1 \right) }
.\] 
Sfruttiamo adesso l'isotropia nel centro di massa:
\begin{align*}
	\frac{\text{d}\Gamma}{\text{d}\cos\left( \theta_{1} \right) } =&
	\frac{\mbox{d}\Gamma}{\mbox{d}\cos\left(\theta_{\text{cm}}\right)}\frac{\mbox{d}\cos\left(\theta_{\text{cm}}\right)}{\mbox{d}\cos\left(\theta_1\right)}=\\
	=&\frac{\Gamma}{2}\frac{1-\beta\cos\left(\theta_1\right)-(-\beta)(-\beta+\cos(\theta_1))}{\left(1-\beta\cos(\theta_1)\right)^2} =\\
	=& \frac{\Gamma}{2}\frac{1-\beta^2}{\left( 1-\beta\cos\left(\theta_1\right)\right)^2}=\\
	=&\frac{\Gamma}{2\gamma^2}\frac{1}{\left(1-\beta\cos(\theta_1)\right)^2}
.\end{align*}
Nell'ultima equazione si evidenzia come siano prediletti angoli piccoli (funzione crescente al tendere di $\ce{\ce{\theta_1} -> 0 }$).\\
Infine possiamo dire qualcosa in più sulla differenza degli angoli nel laboratorio, nonchè "l'apertura" del cono formato dalla direzione dei due fotoni nel lab:
\[
	\sin\left( \theta_1-\theta_2 \right) = \sin\left( \theta_1 \right) \cos\left( \theta_2 \right) -\sin\left( \theta_2 \right) \cos\left( \theta_1 \right) =
	\frac{2 \beta\sin\left( \theta_{\text{cm}} \right) }{\gamma\left( 1-\beta^2\cos^2\left(\theta_{\text{cm}}\right)\right) }
.\]
Questa funzione vista la dipendenza da $\beta$ fa cose diverse a seconda della velocità del $\pi^0$. 

%2.b.35
\subsection[\hspace{2mm} ]{Calcolare la funzione di distribuzione in energia ed in angolo nel sistema del laboratorio di un fascio di neutrini o di muoni prodotto nel decadimento di pioni carichi di energia 14 GeV.}
\[
	\ce{\ce{\pi^+} -> \ce{\mu^+} + \ce{\nu_{\mu}}}
.\] 
Con  $m_{\pi^+} = 139.57$, $m_{\mu}=105.66$ e $m_{\nu}=0$. Se il pione ha energia 14 GeV allora avrà
\[
\gamma = \frac{E}{mc^2} = 100 \implies \beta \approx 0.99994
.\] 
Mettiamoci nel sistema del $\pi^+$, i 4-vettori energia-impulso dei prodotti sono:
\[
P'_{\mu} =  
\begin{pmatrix}
	\sqrt{P_{\text{cm}}^2+m_{\mu}^2}  \\
	P_{\text{cm}}\cos\left( \theta_{\text{cm}} \right) \\
	P_{\text{cm}}\sin\left( \theta_{\text{cm}} \right) \\
	0
\end{pmatrix} \quad \quad \quad 
P'_{\nu} = 
\begin{pmatrix} 
	P_{\text{cm}} \\
	-P_{\text{cm}}\cos\left( \theta_{\text{cm}} \right) \\
	-P_{\text{cm}}\sin\left( \theta_{\text{cm}} \right) \\
	0
\end{pmatrix} 
.\] 
Applichiamo la conservazione dell'energia nel centro di massa per esprimere $P_{\text{cm}}$ in funzione delle sole masse invarianti:
\[
	m_{\pi} = E'_{\mu} + E'_{\nu} =
	\sqrt{P_{\text{cm}}^2 + m_{\mu}^2} + P_{\text{cm}} \quad \implies
	\quad P_{\text{cm}} = \frac{m_{\pi}^2 - m_{\mu}^2}{2m_{\pi}} \approx 30\text{ Mev}
.\]
Visto che il pione ha spin nullo si ha anche qua l'isotropia dei prodott, quindi il \hyperref[eq:isotropia]{risultato} della domanda precedente su $\frac{\text{d}\Gamma}{\text{d}\Omega}$.
Trasformiamo allora per l'energia del neutrino nel laboratorio:
\[
	E_{\nu}^{\text{lab}}=\gamma P_{\text{cm}} \left(1-\beta\cos\left( \theta_{\text{cm}}\right)\right) 
.\] 
In cui si trovano di nuovo limiti inferiori essendoci $\cos\left( \theta_{\text{cm}} \right) $ di mezzo:
\[
	0 \approx \gamma P_{\text{cm}}\left( 1-\beta \right) \le E_{\nu}^{\text{lab}}\le \gamma P_{\text{cm}}\left( 1+\beta \right) \approx 6.2 \text{ GeV}
.\] 
Abbiamo tutte le carte per calcolare la distribuzione del neutrino:
\[
	\frac{\mbox{d} \Gamma}{\mbox{d} E_{\nu}^{lab}} = 
	\frac{\mbox{d} \Gamma}{\text{d}\cos\left( \theta_{\text{cm}} \right) } \frac{\text{d}\cos\left( \theta_{\text{cm}} \right) }{\mbox{d} E_{\nu}^{lab}} =
	= \frac{\Gamma}{2}\frac{1}{P_{\text{cm}}\beta \gamma}
.\] 
Analogamente si può fare per il $\mu$:
\[
	E_{\mu}^{\text{lab}}=\gamma\left( \sqrt{P_{\text{cm}}^2+m_{\mu}} + \beta P_{\text{cm}}\cos\left( \theta_{\text{cm}}\right)\right)
.\] 
\[
	7.8 \text{ GeV} \approx \gamma\left( \sqrt{P_{\text{cm}}^2 + m_{cm}^2} - \beta P_{\text{cm}} \right) \le E_{\mu}^{\text{lab}} \le 
	\gamma\left( \sqrt{P_{\text{cm}}^2 + m_{cm}^2} + \beta P_{\text{cm}} \right)  \approx 14 \text{ GeV}
.\] 
\[
\frac{\mbox{d} \Gamma}{\mbox{d} E_{\mu}^{\text{lab}}} = \frac{\Gamma}{2}\frac{1}{P_{\text{cm}}\beta \gamma}
.\] 

%2.b.36
\subsection[\hspace{2mm} ]{Qual e' l'andamento delle masse nucleari a parità di A in funzione di Z?}
I nuclei isobari (a parità di A) hanno una massa con andamento quadratico in Z:
\begin{align*}
	M_{A,Z}^{\text{atomo}} =& ZM_H^{\text{atomo}} + Nm_n - B_{A,Z} =\\
	=&ZM_H^{\text{atomo}}+\left(A-Z\right)m_u-\text{cost}\left(A\right)-a_s\frac{Z^2}{A^{1/3}}-a_{\text{sym}}\frac{\left(Z-N\right)^2}{A}-\delta_{\text{pair}}
.\end{align*}

% 2.b.37
\subsection[\hspace{2mm} ]{Dimostrare che in un tipico decadimento $\alpha$, la particella $\alpha$ emerge con circa il 98\% dell'energia disponibile.}
In un decadimento $\alpha$ si ha:
\[
\ce{\ce{^{A}_{Z}X_{N}} -> \ce{^{A-4}_{Z-2}Y_{N-2}} + \alpha}
.\]
\paragraph{Metodo Complicato}%

In decadimento tipico di questi si ha anche che $A\gg 4$, ,$Z\gg 2$. Inoltre $B\left( 4,2 \right) = 28.3$ MeV. Forti di queste informazioni immagineremo un piano $A,Z$ pensando queste come variabili indipendenti per poter fare alcune approssimazioni in seguito. Calcoliamo intanto il Q-valore.\\
Nel calcolo del $Q$-valore del processo utilizziamo la massa dell'atomo scritta come:
\[
M_{A,Z} = ZM_H + N m_n - B_{A,Z} 
.\] 
Notiamo che, conservandosi il numero di protoni e neutroni il $Q$-valore si riduce alla differenza delle Biniding Energy $B$ con particolare attenzione ad i segni nella \hyperref[eq:Q-valore]{Definizione di $Q$}:
 \[
	 Q_{\alpha} = B\left( A-4,Z-2 \right) + B\left( 4,2 \right) - B\left( A,Z \right) 
.\] 
Ragionando sempre in termini di $A,Z$ generigi sfruttiamo le ipotesi $A\gg 4$ e $Z\gg 2$ per sviluppare il termine $B\left( A-4,Z-2 \right) $ attorno al punto $\left[ A , Z \right]$:
\[
	B\left( A-4,Z-2 \right) \approx B\left( A, Z \right) - 4 \frac{\partial B\left( A,Z \right)}{\partial A}-2\frac{\partial B\left( A,Z \right) }{\partial Z}   
.\] 
Quindi reinserendo nella formula per $Q_{\alpha}$ ed utilizzando anche la \hyperref[eq:B-energy]{Definizione di B}:
\begin{align*}
	Q_{\alpha} \approx& - 4 \frac{\partial B\left( A,Z \right)}{\partial A}-2\frac{\partial B\left( A,Z \right) }{\partial Z} + 28.3 \text{ MeV} =\\
	=& -4a_V + \frac{8}{3} a_S A^{-1 /3} +4 a_C \frac{Z}{A^{1 /3}}\left( 1- \frac{Z}{3A} \right) - 4a_{sym} \frac{\left( A-2Z \right)^2 }{A^2} + 28.3 \text{ MeV} 
.\end{align*}
Quest'ultima ci permette di notare che i valori tipici del $Q$ del decadimento non si discotano dai pochi MeV (tipicamente 5 MeV), niente di relativistico insomma.\\
Se ci mettiamo adesso in un sistema solidale alla particella che decade si ha che le due particelle finiali hanno impulsi di modulo uguale ed opposta direzione (ipotesi di spin nullo), quindi:
\[
m_{\alpha}^2 \left| \boldsymbol{v_{\alpha}} \right|^2 = m_{y}^2 \left| \boldsymbol{v_{y}} \right|^2 \implies m_{\alpha} T_{\alpha} = m_{y}T_{y}
.\]
E visto che il $Q$-valore è definito anche come  $Q = T_{\alpha} + T_{y}$ se ne conclude che:
\[
	T_{\alpha} = \frac{Q}{1 + \frac{m_{\alpha}}{m_{y}}} \approx Q\left( 1-\frac{m_{\alpha}}{m_y} \right) \approx Q\left( 1-\frac{4}{A-4} \right) 
.\] 
Basta avere $A = 200$ per  $T_{\alpha} = 0.98\cdot Q$. \\
\paragraph{Metodo Semplice}%
Alle stesse conclusioni si arriva con molta meno sofferenza utilizzando il fatto che i prodotti di decadimento sono limitati da funzioni del Q-value e stimando il Q-Value con i difetti di massa:
\[
	Q = \Delta_{A,Z} - \Delta_{A-4,Z-2} -\Delta_{4,2} \quad \left( \sim \text{ MeV} \right) 
.\]
Andando nel sistema del centro di massa in cui X decade a riposo l'energia cinetica di ogni singolo prodotto è limitata superiormente dal Q-value essendo $Q = T_{\alpha}+ T_{y}$.\\
Questo pone dei limiti anche all'impulso dei prodotti, per la particella $\alpha$ si ha:
\[
	p_{\alpha}= \sqrt{E_{\alpha}^2-m_{\alpha}}= \sqrt{\left( T_{\alpha}+m_{\alpha}\right)^2- m_{\alpha}^2 } < \sqrt{Q^2+2Qm_{\alpha}} 
.\] 
Nel centro di massa naturalmente si ha $\left| \bs{p}_{\alpha} \right| = \left| \bs{p}_{y} \right| $. Quindi esprimiamo l'energia cinetica dei prodotti in  funzione dell'impulso come:
\begin{align*}
	&T_{\alpha} = \frac{\left| \bs{p}_{\alpha} \right|^2 }{2m_{\alpha}}< \frac{Q^2+2Qm_{\alpha}}{2m_{\alpha}} 
	&T_{y} =  \frac{\left| \bs{p}_{y} \right|^2 }{2M_{y}}< \frac{Q^2+2Qm_{\alpha}}{2M_{y}} 
.\end{align*}
In realtà a noi interessa che:
\[
	\frac{T_{y}}{T_{\alpha}}= \frac{m_{\alpha}}{M_{y}}
.\] 
E visto che $m_{\alpha} = 4 m_{u} + \Delta_{4.2} ]\sim$ 4 GeV e $M_{y}=\left(A-4\right)m_{u}+ \Delta_{A-4, Z-2}\sim$ 200 GeV si ha:
\[
	T_{y} \approx 0.02 T_{\alpha}
.\] 
Da cui si conclude come sopra.


\subsection[\hspace{2mm} ]{Dimostrare che in un decadimento $\beta$ la somma delle energie dell'elettrone e dell'antineutrino emessi é praticamente uguale al Q-valore della reazione.}
Ricordando il decadimento $\beta^-$ :
\[
\ce{\ce{^{A}_{Z}X_{N}} -> \ce{^{A}_{Z+1}Y^+_{N-1}} + e^- + \overline{\nu}_e}
.\]
Mettiamoci in un sistema in cui $X$ decade a riposo. In questo sistema le energie dei 3 prodotti sono tutte minori del $Q$-valore poichè:
\[
T_Y + T_{e^-} + T_{\overline{\nu}_e} = Q
.\] 
Questo pone dei limiti anche agli impulsi dei 3 prodotti:
\[
	\left| \boldsymbol{p_{e^-}} \right| = \sqrt{E_e^2 - m_e^2}  = \sqrt{\left( T_e + m_e \right)^2 - m_e^2 } < \sqrt{Q^2+2m_eQ}  
.\] 
\[
	\left| \boldsymbol{p_{\overline{\nu}_e}} \right| < Q 
.\]
\[
\left| \boldsymbol{p}_Y \right| <  \left| \boldsymbol{p_{\overline{\nu}_e}} \right| + \left| \boldsymbol{p_{e^-}} \right| < Q + \sqrt{Q^2+2m_eQ} 
.\] 
Come nel caso precedente si può dimostrare che $Q \sim \text{ MeV}$, quindi l'energia del nucleo uscente (la cui massa è almeno tre ordini superiori a Q) nel sistema del centro di massa:
\[
	T_{Y} = \frac{\left| \boldsymbol{p}_Y \right|^2 }{2m_Y} < \frac{\left( Q + \sqrt{Q^2+2m_e Q}  \right) }{2m_Y}  \sim \frac{Q}{1000} \quad \implies
	\quad Q \approx T_{e^-} + T_{\overline{\nu}_e} 
.\] 
 % risposto


\part{Elettromagnetismo classico e acceleratori di particelle}
% azzerare i contatori delle sezioni (a,b,c...) 
\setcounter{section}{0}
\renewcommand*{\theHsection}{chX.\the\value{section}}
\section{Domande a}
\subsection[]{Dare la definizione di quadri-corrente e di quadri-potenziale del campo elettromagnetico.}
\label{sec:3.a.1}
\[
	j^{\mu}=\left( c \rho, \boldsymbol{j} \right) \quad \quad 
	A^{\mu}=\left( \varphi, \boldsymbol{A}\right) 
.\] 

\subsection[]{Dare la definizione del tensore del campo elettromagnetico e scriverne le componenti.}
\label{seq:3.a.2}
Il tensore discusso è il tensore antisimmetrico di rango 2
\[
F^{\mu \nu} = \partial^{\mu}A^{\nu} - \partial^{\nu}A^{\mu}  
.\] 
avente componenti:
\[
F^{\mu \nu}=
\left(
\begin{array}{c|ccc}
	0 & -Ex & -E_y & -E_z \\
	\hline
	Ex & 0 & -B_z & B_y \\
	E_y & B_z & 0 & -B_x \\
	E_z  & -B_y & B_x & 0 \\
\end{array}
\right)
.\] 

\subsection[]{Dare la definizione della "densità di energia" del campo elettromagnetico, del "vettore di Poynting" e del "tensore degli sforzi di Maxwell"}
Trattiamo qua di leggi di conservazione provenienti da equazioni di continuità:
\label{sec:3.a.3}
\[
	\frac{\partial}{\partial t} \left(\text{dentità di energia}\right) + \nabla \cdot \left( \text{vettore} \right) = \left( \text{densità di potenza} \right) 
.\] 
Si può dimostrare (vedi Fisica II, Teorema di Poynting) che la densità di energia nell'equazione nel caso di campo elettromagnetico è:
\[
\rho_{E}=\epsilon_0 \frac{E^2}{2}+\epsilon_0c^2 \frac{B^2}{2} 
.\] 
Mentre il vettore, detto Vettore di Poynting:
\[
\boldsymbol{S}= \frac{\boldsymbol{E}\times \boldsymbol{B}}{\mu_0}
.\] 
Il tensore di Maxwell deriva invece da un'altra legge di conservazione: quella dell'impulso del campo elettromagnetico. Tale tensore è così definito:
\[
	T_{i,j}=\left( \epsilon_0 \frac{E^2+c^2B^2}{2}\delta_{ij}-\epsilon_0\left( E_iE_j+c^2B_iB_j \right)  \right) 
.\] 
\subsection[]{Scrivere le equazioni di Maxwell (sia quelle non omogenee che quelle omogenee) in forma covariante.}
\label{sec:3.a.4}
\[
	\partial_{\mu}F^{\mu \nu} = \frac{j^{\mu}}{\epsilon_0c} \quad \quad \text{Disomogenee}
.\]  
\[
	\partial_{\mu} \tilde{F}^{\mu \nu}= \frac{1}{2} \epsilon^{\mu\nu\alpha\beta}F_{\alpha\beta} = 0 \quad \quad \text{Omogenee}
.\]  

\subsection[]{Scrivere l'equazione di continuità per la quadri-corrente in forma covariante (e verificarne la consistenza con le equazioni di Maxwell)}
\label{sec:3.a.5}
Prendiamo le equazioni di Maxwell disomogenee (in forma covariante) e sfruttiamo l'antisimmetricità del tensore elettromagnetico:
\[
	\partial_{\mu}\partial_{\nu}F^{\mu\nu}=0
.\] 
facendo la stesa operazione anche al di là dell'uguale si trova quindi l'equazione richiesta
\[
	\partial_{\mu}j^{\mu}= \frac{\partial \rho}{\partial t} + \nabla\cdot \boldsymbol{j}=0
.\] 
\subsection[]{Dare la definizione di "gauge di Lorenz" e di "gauge di Coulomb".}
\label{sec:3.a.6}
Nella Gauge di Lorenz è richiesto:
\[
	\partial_{\mu}A^{\mu}=0
.\] 
Mentre nella Gauge di Columb:
\[
	\nabla\cdot \boldsymbol{A}=0
.\] 
Le due condizioni si equivalgono nel caso elettrostatico.

\subsection[]{Scrivere la legge di trasformazione di Lorentz del campo elettrico e del campo magnetico (distinguendo fra componenti parallele e componenti perpendicolari al "boost").}
\label{sec:3.a.7}
\begin{align*}
	&\boldsymbol{E}'_{\|} = \boldsymbol{E}_{\|}\\
	&\boldsymbol{E}'_{\bot}=\gamma\left( \boldsymbol{E}_{\bot}+ \boldsymbol{\beta} \wedge \boldsymbol{B} \right)\\
	&\boldsymbol{B}'_{\|}=\boldsymbol{B}_{\|}\\
	&\boldsymbol{B}'_{\bot}=\gamma\left( \boldsymbol{B}_{\bot}-\boldsymbol{\beta}\wedge \boldsymbol{E}  \right) 
.\end{align*}

\subsection[]{Dare la definizione del quadri-vettore "densità di forza di Lorentz" .}
\label{sec:3.a.8}
\[
	f^{\mu}=\frac{\mbox{d} p^{\mu}}{\mbox{d} t \text{d}^3r} 
.\] 
Il fatto che questo sia un quadrivettore deriva dal fatto che il denominatore è un invariante relativistico, si ha inoltre che:
\[
	f^{\mu}=\frac{1}{c}F^{\mu\nu}j_{\nu}=\left(\frac{\boldsymbol{j}\cdot\boldsymbol{E}}{c},\rho\boldsymbol{E}+\frac{1}{c}\boldsymbol{j}\wedge\boldsymbol{B}\right)
.\] 

\subsection[]{Ricavare le espressioni dell’effetto Doppler relativistico (calcolo della frequenza e dell’angolo misurati dal rivelatore nel caso di moto relativo fra sorgente e rivelatore stesso).}
\label{sec:3.a.9}
Si sa che esiste il 4-vettore:
\[
	k^{\mu}=\left( \frac{\omega}{c}, \boldsymbol{k} \right) 
.\]
Consideriamo un moto bidimensionale e mettiamoci nel sistema di una sorgente di onde elettromagnetiche che si muove (la sorgente) a velocità $\boldsymbol{v} =c\beta \hat{x}$, in tale sistema il quadrivettore $k^{\mu}$ lo definiamo come:
\[
	k^{\mu}=\frac{\omega}{c}\left( 1, \ \cos\theta \cdot \hat{x}, \ \sin\theta\cdot  \hat{y} \right) 
.\] 
Invece nel sistema del rivelatore questo quadrivettore è:
\[
	k^{\mu}_{R} = \frac{\omega}{c}\left( 1, \ \cos\theta_{R}\cdot \hat{x}, \ \sin\theta_{R} \cdot \hat{y} \right) 
.\] 
Legati dalla trasformazione di Lorentz tra i due sistemi:
\begin{align*}
	\omega_{R}&=\gamma\omega\left( 1 +\beta\cos\theta \right)\\ 
	\tan\theta_{R}&=\frac{\sin\theta}{\gamma\left( \cos\theta+\beta \right) }
.\end{align*}

\subsection[]{Scrivere l’espressione per i potenziali ritardati ( $\phi$ ed $A$ ) per una qualunque distribuzione di cariche ( $\rho$ ) e correnti ( $j$ ).}
\label{sec:3.b.10}
Potenziale scalare:
\[
	\varphi\left( \boldsymbol{r},t \right) = \int 
	\frac{\rho\left( \boldsymbol{r}', t-\left| \boldsymbol{r}-\boldsymbol{r}' \right| /c  \right) }{\left| \boldsymbol{r}-\boldsymbol{r}' \right| } d^3r'
.\] 
Potenziale vettore:
\[
	\boldsymbol{A}\left( \boldsymbol{r},t \right) =\frac{1}{c}\int 
	\frac{\boldsymbol{j}\left( \boldsymbol{r}', t-\left| \boldsymbol{r}-\boldsymbol{r}' \right| /c  \right) }{\left| \boldsymbol{r}-\boldsymbol{r}' \right| }d^3r'
.\] 

\subsection[]{Spiegare tutti i termini dell’espressione 
\label{sec:3.a.11}
	\[
	\boldsymbol{E} = \left[ \frac{q}{R^2}\frac{\hat{n}- \boldsymbol{\beta}}{\gamma^2 \left( 1 - \hat{n} \cdot \boldsymbol{\beta} \right)^3 } + \frac{q}{Rc} \frac{\hat{n} \wedge \left[ \left( \hat{n} -\boldsymbol{\beta}  \right) \wedge \dot{\boldsymbol{\beta}} \right] }{\left( 1 - \hat{n} \boldsymbol{\beta}\right)^3 } \right]_{t' = t - R/c}  
\] 
per il campo elettrico generato da una carica puntiforme in moto arbitrario.}
Sia $\boldsymbol{s}\left( t \right) $ la legge oraria della carica q e sia $\boldsymbol{r}$ la posizione dell'osservatore, allora possiamo definire le quantità dell'espressione:
\[
	\boldsymbol{R}=\boldsymbol{r}-\boldsymbol{s}\left( t' \right) \quad \quad \hat{n}=\boldsymbol{R}/R
.\] 
Il primo termine della equazione nella richiesta è il campo generato da una carica che si muove in moto rettilineo uniforme e non è un termine radiativo (va giu come $R^{-2}$), il secondo termine è invece radiativo e dipende dalla accelerazione della carica.
\subsection[]{Dare la definizione di “solido di radiazione” e di “diagramma di radiazione” per una carica accelerata.}
\label{sec:3.a.12}
\paragraph{Solido di radiazione}
\label{par:Solido di radiazione.}
Preso un oggetto che irraggia si dice solido di radiazione la figura tridimensiona costruita posizionando nell'origine la sorgente di radiazione e nelle diverse direzioni spaziali frecce di lunghezza proporzionale all'intensità del campo elettrico nella direzione indicata dalla freccia stessa.

\paragraph{Diagramma di radiazione.}%
\label{par:Diagramma di radiazione.}
Sezione planare del solido di radiazione in direzioni opportune a comprenderne la forma. 

\subsection[]{Quanto vale il campo magnetico generato da una carica puntiforme in moto arbitrario se è noto il campo elettrico?}
\label{sec:3.a.13}
\[
	\boldsymbol{B}= \frac{n}{c} \hat{n} \wedge \boldsymbol{E} 	
.\] 
Con n indice di rifrazione del mezzo, $\hat{n}$ direzione dell'onda elettromagnetica, nel vuoto si ha $n=1$.
\subsection[]{Spiegare tutti i termi della espressione 
\[
	\frac{\mbox{d} I_{\omega}}{\mbox{d} \Omega} = \frac{q^2}{4 \pi^2 c} \left| \int{ \frac{\hat{n} \wedge \left[ \left( \hat{n}-\boldsymbol{\beta}\right) \wedge \dot{\boldsymbol{\beta}} \right]]}{\left( 1- \hat{n} \boldsymbol{\beta}\right)^2} e^{i\omega \left( t' - \frac{\boldsymbol{r'} \cdot \hat{n}}{c} \right) }  }  \right|^2
\] 
}
\label{sec:3.a.14}
Parliamo della famigerata Radiazione di sincrotrone.
Quella scritta sopra è la distribuzione angolare dell'energia irraggiata per unità di frequenza $I_{\omega}$ di una carica $q$ in moto con velocità  $\boldsymbol{\beta}$. Inoltre $\hat{n}$ è la direzione di osservazione e il sistema usato è il CGS. 
Anche se non richiesto possiamo dimostrarla a partire dalla formula del campo di radiazione della \hyperref[sec:3.a.11]{Domanda 3.a.11}.:
\[
	\boldsymbol{E}= \left. \frac{q}{cr}
	\frac{\hat{n}\wedge\left[\left(\hat{n}-\boldsymbol{\beta}\right)\wedge\dot{\boldsymbol{\beta}}\right]}
	{\left(1-\hat{n}\cdot \boldsymbol{\beta}\right)^3} \right|_{t=t_{\text{rit}}}
\]
ricordiamo che $\boldsymbol{B}=\hat{n}\wedge \boldsymbol{E}$, e che come conseguenza la densità di energia irraggiatà $\varepsilon$ è proporzionale a $\left| \boldsymbol{E} \right| ^2$, quindi l'energia irraggiata per unità di angolo solido si esprime come:
\begin{align*}
	\frac{\mbox{d} \varepsilon}{\mbox{d} \Omega} =& \int_{-\infty}^{\infty}r^2 \left[ \boldsymbol{S}\cdot\hat{n}\right] dt =\\
	=& \int r^2 \frac{c}{4\pi}\left| \boldsymbol{E} \right|^2 dt = \\
	=& \frac{q^2}{4\pi c} \int \left[ \left. \frac{ \hat{n} \wedge \left( \left( \hat{n}-\boldsymbol{\beta} \right)\wedge\dot{\boldsymbol{\beta}}\right)}
		{\left( 1- \hat{n}\cdot \boldsymbol{\beta} \right)^3 } \right|_{t = t_{\text{rit}}} \right]^2 dt
.\end{align*}
È utile adesso avvalersi dell'itentità di parseval sulla trasformata di Fourier (attenzione che a quella su Wikipedia gli manca il fattore $2\pi$):
\[
	\int_{-\infty}^{\infty} \left| x\left( t \right)  \right|^2 dt = \frac{1}{2\pi} \int_{-\infty}^{\infty} \left| \hat{x}\left( \omega \right)\right|^2d\omega 
.\] 
Portiamo allora l'integranda nel dominio delle frequenze:
\[
	\frac{\mbox{d} \varepsilon}{\mbox{d} \Omega} = \frac{1}{2\pi}  \frac{q^2}{4\pi c} 2 \cdot\int_0^{\infty} \left| \hat{f}\left( \omega \right) \right|^2 d \omega 
.\] 
con $\hat{f}\left( \omega \right) $ trasformata dell'integranda nell'equazione sopra, l'integrale va solo sulle frequenze positive perchè la funzione che trasformiamo è reale ($\hat{f}\left( -\omega \right) = \hat{f}^*\left( \omega \right) $), da questo viene il fattore 2.\\
Adesso la funzione integranda si avvicina molto all'oggetto che stavamo cercando (un oggetto definito per unità di frequenza), infatti:
\[
	\frac{\mbox{d} I_{\omega}}{\mbox{d} \Omega} = \frac{q^2}{4\pi^2 c} \left| \hat{f}\left( \omega \right)  \right|^2 =  \frac{q^2}{4\pi^2 c} 
	\left| \int_{-\infty}^{\infty} \left. \frac{ \hat{n} \wedge \left( \left( \hat{n}-\boldsymbol{\beta} \right)\wedge\dot{\boldsymbol{\beta}}\right)}
		{\left( 1- \hat{n}\cdot \boldsymbol{\beta} \right)^3 } \right|_{t = t_{\text{rit}}} e^{-i\omega t} dt \right|^2 
.\]
Basta adesso cambiare variabili: 
\[
	\ce{\ce{dt} -> \ce{dt_{\text{rit}}}} = dt' = d \left(t - \frac{R\left( t' \right) }{c} \right)
.\] 
con $R\left( t \right)$ distanza tra particella e punto di osservazione, è bene ricordare che per distanze "grandi" come quelle dei campi di radiazione si ha:
\[
	R\left( t' \right) \approx x - \hat{n}\cdot \boldsymbol{r}\left( t' \right)  
.\] 
Con $x$ distanza del punto di osservazione dall'origine e $\boldsymbol{r}\left( t \right)$ traiettoria della particella. In questo modo è evidente che lo Jacobiano del cambio di variabili sia:
\[
	\frac{dt}{dt'}= 1-\hat{n} \cdot \boldsymbol{\beta}
.\] 
rimettendo tutto dentro si ottiene l'espressione nella richiesta.


\subsection[]{Una carica elettrica Q si muove con velocità costante (relativistica) di modulo V su una retta, a distanza b da tale retta si trova un osservatore che misura il campo elettrico e magnetico generato dalla carica. Quanto è l'ordine di grandezza del tempo in cui l'osservatore misura un campo elettrico che sia almeno la metà del campo elettrico massimo misurato?}
\label{sec:3.a.15}
È evidentemente necessario per rispondere alla domanda parametrizzare il campo elettrico in funzione del tempo nel riferimento dell'osservatore. Per far questo ci mettiamo da prima nel riferimento della carica (O') e poi con una trasformazione di Lorentz dei campi andiamo nel sistema dell'osservatore:
\begin{align*}
&									&&						&\boldsymbol{E}_{\|}=&\boldsymbol{E}'_{\|} \\
& \bs{E}'= e \frac{\bs{R}'}{R'^3}, \quad \quad \quad \bs{B}'= \ 0	 &\ce{->[\quad \text{Lorentz}\quad]}&		&\bs{E}_{\bot}=&\gamma\bs{E}'_{\bot}\\ 
&									&&						&\bs{B}\ =&\bs{\beta}\wedge \bs{E} 
.\end{align*}
Tuttavia come è ben noto i campi sono espressi ancora in funzione delle variabili del sistema solidale alla carica (t',x',y',z'), sarà necessario un altro boost per cambiare parametrizzazione.
\begin{align*}
	&x\left( t \right) = v\cdot t	&&					&x'=&\gamma\left( x-vt \right) \\
	&y\left( t \right) = 0		&\ce{ ->[\quad \text{Lorentz}] \quad}&	&y'=&y \\
	&z\left( t \right) = 0		&&					&z'=&z
.\end{align*}
Rimane la dipendenza del campo elettrico dalla distanza dalla carica anche dopo la trasformazione di Lorentz, tale distanza è tuttavia parametrizzata con le variabili primate: 

\begin{align*}
	R'& = \sqrt{x'^2 + y'^2 + z'^2} =\\ 
	&=\sqrt{\gamma^2\left( x-vt \right)^2 + y^2 + z^2} =\\ 
	&=\gamma \sqrt{\left( x-vt \right)^2 + \left( 1-\beta^2 \right)\left( y^2+z^2 \right)}=\gamma R^{*} 
.\end{align*}	
Inseriamola nelle componenti dei campi: 
\begin{align*}
	&E_x = e \frac{\gamma\left( x-vt \right) }{\gamma^3\left(R^{*}\right)^3} = e \frac{x-vt}{\gamma^2 \left(R^*\right)^3}\\
	&E_y = e \gamma \frac{y}{\gamma^3\left(R^*\right)^3} = e \frac{y}{\gamma^2\left(R^*\right)^3} \\ 
	&E_y = e \gamma \frac{z}{\gamma^3\left(R^*\right)^3} = e \frac{z}{\gamma^2 \left(R^*\right)^3} 
.\end{align*}
In maniera più compatta si può scrivere:
\[
	\bs{E}=e \frac{\bs{R}_i}{\gamma^2 \left( R^* \right)^3}
.\] 
Con $\bs{R}_i$ defito dalle equazioni precedenti.

\subsection[]{Enunciare il principio di Babinet.}

\subsection[]{Definire il fattore di forma per un'onda che incide su su sistema.}

\subsection[]{Spiegare qualitativamente il funzionamento di un acceleratore elettrostatico.}

\subsection[]{Quali sono, approssimativamente, le energie per unità di lunghezza che attualmente si ottengono nell’accelerazione di protoni con la tecnica dei "drift tube"? E delle cavità superconduttrici?}
\subsection[]{Spiegare qualitativamente il funzionamento di un acceleratore lineare, indicando le differenze importanti fra l’accelerazione di elettroni e di protoni.}

\subsection[]{Spiegare qualitativamente il funzionamento di un acceleratore circolare, indicando le differenze importanti fra l’accelerazione di elettroni e di protoni.}

\subsection[]{Effettuare un disegno, qualitativo, del solido di radiazione per una carica in un acceleratore lineare o circolare.}

 % risposto 
\section{Domande b}
\subsection*{3.b.1. Dimostrare l’espressione dt ' = 1− n̂ ⋅ β , dove t e t’ sono il tempo di osservazione ed il tempo ‘ritardato’, rispettivamente.}

\subsection*{3.b.2. Date le definizioni ‘standard’ delle variabili n̂ , β , R, r, r ', t e t ' , dimostrare le seguenti relazioni:}

\subsection*{3.b.3. Calcolare la distribuzione in potenza in funzione dell’angolo di emissione per una carica accelerata in moto non relativistico.}

\subsection*{3.b.4. Ricavare esplicitamente le leggi di trasformazione di Lorentz del campo elettrico e del campo magnetico. Discutere, in particolare, il caso in cui, in un certo sistema di riferimento inerziale, il campo magnetico è nullo e il caso in cui il campo elettrico è nullo.}

\subsection*{3.b.5. Dire quali sono gli "invarianti di Lorentz" che si possono costruire con il tensore del campo elettromagnetico e ricavarne le espressioni esplicite in termini dei campi elettrico e magnetico. Ridiscutere, usando gli invarianti, il caso discusso nel punto precedente e discutere il caso in cui gli invarianti sono nulli.}

\subsection*{3.b.6. Una carica elettrica Q si muove con velocità costante su una retta con velocità costante: x=Vt, y=b, z=0. Calcolare in funzione del tempo il campo elettrico ed il campo magnetico generato dalla carica nel punto O e produrre il grafico di ognuna delle 6 componenti trovate in funzione del tempo t.}

\subsection*{3.b.7. Scrivere in forma covariante l'equazione del moto di una carica in un campo radiativa ed indicare il campo di applicazione di questa formula.}

\subsection*{3.b.9. Dare la definizione del "tensore energia-impulso" del campo elettromagnetico e scrivere la sua relazione con la "densità di forza di Lorentz".}

\subsection*{3.b.10. Dire come si generalizzano i teoremi di conservazione dell'energia e dell'impulso a situazioni in cui sia presente un campo elettromagnetico.}

\subsection*{3.b.11. Scrivere il tensore degli sforzi per un’onda e.m. piana che si propaga in una direzione n̂ .}

\subsection*{3.b.12. Scrivere esplicitamente il 4-tensore impulso-energia per un’onda e.m. piana monocromatica che si propaga lungo l’asse x con densita’ di energia u em .}

\subsection*{3.b.13. Ricavare l’espressione per i potenziali di Lienard-Wiechert ( φ ed A per una carica puntiforme in moto arbitrario) a partire dai potenziali ritardati.}

\subsection*{3.b.14. Calcolare la potenza totale irraggiata da una carica accelerata in moto non relativistico. Esprimere i risultati in MKSA e nelle unità “naturali”.}

\subsection*{3.b.15. Ricavare la formula di Larmor relativistica P = 3 c 3 γ a − a ∧ β a partire dalla formula non relativistica ed utilizzando argomenti di invarianza relativistica.}

\subsection*{3.b.16. Dimostrare che la radiazione di sincrotrone ha uno spettro di emissione con una frequenza “critica” ω C ≈ ω o γ 3}

\subsection*{3.b.17. Calcolare il fattore di forma elettromagnetico per una sfera uniformente carica di raggio a.}

\subsection*{3.b.18. Calcolare il fattore di forma per una suprficie sferica uniformente carica di raggio a. [nota: l'interno è vuoto]}

\subsection*{3.b.19. Calcolare la velocità di un elettrone [e successivamente di un protone] posto in una struttura acceleratrice che abbia: i) campo elettrico longitudinale costante ii) campo elettrico longitudinale oscillante. Inserendo valori numerici ragionevoli, calcolare il tempo affinché la particella, partendo da ferma, raggiunga una energia pari al doppio della sua massa a riposo.}

\subsection*{3.b.20. A partire dai campi ritardati, dimostrare che è la potenza (MKSA) irraggiata da una carica accelerata in un moto rettilineo.}

\subsection*{3.b.21. Calcolare l’energia persa in una rivoluzione per una carica in moto uniforme su una circonferenza (acceleratore circolare). Calcolare la frazione di energia persa
in un giro rispetto alla sua energia cinetica, effettuando una valutazione numerica, nel caso di elettroni a LEP (energia 50 GeV, raggio ~4km) o protoni ad LHC (energia 7 TeV, raggio ~4km). Nota: utilizzare la formula di Larmor in GCS:}

\subsection*{3.b.22. Calcolare la potenza emessa in funzione dell’angolo per una carica oscillante armonicamente in linea retta (termine di dipolo elettrico)}

\subsection*{3.b.23. Calcolare, a partire dalla formula di Larmor relativistica, la potenza totale dissipata in un acceleratore lineare in funzione di (energia fornita per unita’ di lunghezza). Dimostrare che la frazione di energia persa nell’accelerazione e’ trascurabile, fornendo adeguati valori numerici nel caso di accelerazione di elettroni o protoni.}

\subsection*{3.b.24. Calcolare la lunghezza d’onda critica della radiazione di sincrotrone (elettroni) nei casi seguenti: i) energia=50GeV, raggio=4km; ii) energia=5GeV, raggio=30m}

\subsection*{3.b.25. Enunciare il teorema ottico e spiegarne il significato fisico nel caso di radiazione elettromagnetica su un ostacolo opaco.}

\subsection*{3.b.26. Come si ricava la sezione d'urto differenziale Rayleigh a partire dalla sezione d'urto Thomson ?}
 % risposto



\part{Interazione radiazione-materia}
\setcounter{section}{0}
\renewcommand*{\theHsection}{chX.\the\value{section}}

\section{Domande a}
\subsection[ Spessore di un materiale espresso in due modi]{Quale e' la relazione fra lo spessore di un materiale, espresso in cm, e lo spessore espresso in g/cm$^2$ ?
} \label{sec:4.a.1}
La relazione tra $X$ [cm] e $X'$ [g/cm$^2$] consiste in:
\[
	X\rho = X'
.\] 
Dove $\rho$ è la densità del mezzo, il vantaggio di esprimere uno spessore in g/cm$^2$ è che in questa unità lo spessore risulta indipendente dallo stato (solido, liquido, gassoso) del materiale che si usa (solitamente per rilevare le radiazioni di particelle che attraversanola materia).
\subsection[ Effetto fotoelettrico e scattering Compton]{Spiegare qualitativamente l’effetto fotoelettrico e lo scattering Compton, indicandone le differenti caratteristiche
}\label{sec:4.a.2}
L'effetto fotoelettrico constiste nella interazione:
\[
	\ce{\ce{\gamma} +  \text{atomo} -> \ce{e^-} + \text{atomo}+}
.\] 
In pratica un fotone con energia $E$ cede tutta la sua energia ad un elettrone atomico che viene espulso dall'atomo con energia $E-E_{\text{lev}}$ dove $E_{\text{lev}}$ è l'energia del livello a cui si trova l'elettrone. La soglia di energia per far avvenire il processo è quindi $E_{\text{lev}}$, mentre il range di energia in cui il processo è il prevalente è tra $E^{\text{min}}_{\text{lev}}$ e $E^{\text{max}}_{\text{lev}}$. Per il carbonio l'energia di livello massima è $10$ keV mentre per il piombo arriva anche a $500$ keV.\\
L'effetto compton invece sussiste in un urto elastico tra un fotone ed un elettrone, considerato libero:
\[
	\ce{\ce{\gamma} + e- -> \ce{\gamma} + e- }
.\] 
In cui vi può essere uno scambio di energia tra il fotone e l'atomo. Questo è l'effetto prevalente per energie del fotone incidente dell'ordine dei MeV ed a differenza del Rayleigh la sua sezione d'urto va come $Z$.
\subsection[ Lunghezza d'onda Compton]{Dare l'espressione ed il valore numerico della lunghezza d'onda Compton.
}\label{sec:4.a.3}
Visto che nell'effetto Compton vi è scambio di energia tra il fotone e l'elettrone allora la frequenza del fotone dopo lo scattering risulta cambiata, applicando la conservazione dell'energia e dell'impulso si ottiene che:
\[
	\omega-\omega'= \frac{\hbar \omega\omega'}{mc^2}\left( 1-\cos\theta \right) 
.\]
Dove $\omega$ è la frequenza del fotone prima dello scattering, $\omega'$ la frequenza dopo lo scattering, $\theta$ l'angolo di scattering.
Esprimendo tale relazione in termini di $\lambda= 2\pi c/ \omega$ si ha che:
\[
	\lambda'-\lambda= \lambda_{c}\left( 1-\cos\theta \right) 
.\] 
Abbiamo quindi definito la lunghezza d'onda Compton dell'elettrone:
\[
	\lambda_{c}=\frac{h}{mc} =2.43 \cdot 10^{-12} \text{ m} =  2.43 \text{ pm}
.\] 
\subsection[ Scattering Rayleigh]{Spiegare qualitativamente lo scattering Rayleigh
}\label{sec:4.a.4}
Lo scattering Rayleigh corrisponde all'interazione elastica tra un fotone ed un atomo. È un effetto importante fino ad energie $\sim 1$ keV (sezione d'urto che va come $Z^2$), dopo la sua sezione d'urto decresce come il reciproco del quadrato dell'energia.

\subsection[ Creazione di coppie su nucleo]{Spiegare qualitativamente il fenomeno della creazione di coppie  da parte di un raggio gamma che incide su un atomo.
}\label{sec:4.a.5}
La creazione di coppie 
\[
	\ce{\ce{\gamma} -> e+ + e-} 
\]
è cinematicamente proibita, infatti prima della reazione il sistema del centro di massa non esiste, dopo si, ma la velocità di tale sistema non cambia dopo l'urto (o più in generale, il centro di massa prima dell'urto non era definito, quindi non può essere scritta una legge di conservazione).
C'è quindi bisogno di una particella spettatrice carica.
Supponiamo che vi sia un atomo spettatore, si ha:
\[	
	\ce{\ce{\gamma} + \text{atomo} -> e+ + e- +\text{atomo}} 
.\] 
Se $M$ è la massa di tale atomo allora l'energia di soglia per il fotone è (conservazione del modulo del quadrimpulso):
\begin{align*}
	\left( E_{\gamma}+M \right)^2-E^2_{\gamma}=\left( 2m_e + M \right)^2
.\end{align*}
\[
	E_{\gamma}\ge 2m_e\left( 1+ \frac{m_e}{M} \right) 
.\] 
In particolare $E_{\gamma}\ge 2m_e \approx 1$ MeV, da cui la necessità di avere un raggio $\gamma$.


\subsection[ Effetto Cherenkov e angolo Cherenkov]{Descrivere qualitativamente l’effetto Cherenkov e dimostrare tramite il principio di Huygens che la radiazione Cherenkov è emessa ad un solo angolo.
}\label{sec:4.a.6}
L'effetto Cherenkov consiste nella emissione di onde elettromagnetiche da parte di una particella carica che si muove a velocità $v$ superiore a quella della luce.
\[
	\beta c> \frac{c}{n}
.\] 
Con $n$ parte reale dell'indice di rifrazione. L'emissione è dovuta all'irraggiamento coerente di dipoli elettrici che sono messi in oscillazione dal campo elettromagnetico generato dalla particella nel suo passaggio.\\
Analizziamo nel dettaglio la direzione di propagazione di quest'onda utilizzando il principio di Huygens.\\
Ipotizziamo che al tempo $t'$ si propaghi dalla particella un'onda elettromagnetica con fronte d'onda sferico. Dopo un tempo $\Delta t$ tale fronte ha un raggio 
\[
	R = \frac{c}{n} \Delta t
.\] 
Tra il tempo in cui l'onda discussa ha iniziato a propagarsi ($t'$) ed il tempo in cui ha raggiunto il raggio $R$ ($t = t'+ \Delta t$) si propagano perturbazioni sferiche del tutto analoghe a quella sopra, che al tempo $t$ avranno un raggio intermedio tra $R$ e $0$.\\
Di conseguenza la perturbazione elettromagnetica si sviluppa su un fronte che è la sovrapposizione di tutte le superfici sferiche di onde formatesi nel tempo $\Delta t$ (effetto motoscafo).
Chiamiamo la direzione di propagazione dell'onda $\theta$, ci aspettiamo che il coseno di quest'angolo abbia proporzionalità inversa da $\beta$ e da $n$. In questo modo all'aumentare di $n$ l'effetto risulta più evidente (luce "più lenta") mentre al diminuire di $\beta$ (nei limiti in cui vale ancora $\beta > 1 /n$) l'effetto tenda a svanire (si torna in situazioni in cui le cose vanno più lentamente della luce). 
 \[
	\cos\theta = \frac{1}{\beta n}
.\] 
Ipotizziamo che in un punto $P$ vi sia un osservatore che rivela le onde elettromagnetiche, la situazione sarebbe questa:
\begin{figure}[H]
    \centering
    \incfig{cherenkov-huygens}
    \caption{Schema di osservazione dell'effetto Cherenkov}
    \label{fig:cherenkov-huygens}
\end{figure}
Nell'immagine l'osservatore in $P$ rileva l'onda arrivare al tempo $t$, quando la particella è "già oltre", in un certo senso possiamo pensare che l'osservatore sente prima il dolore del proiettile che il rimbombo della pistola.\\
Facendo riferimento alla figura \ref{fig:cherenkov-huygens} dimostriamo la relazione angolare citata sopra.
\begin{align*}
	&R = \left| \bs{R} \right| = \frac{c}{n} \Delta t	&\bs{\rho} + \bs{\beta} c \Delta t=\bs{R} 
.\end{align*}
Facendo il modulo quadro della relazione vettoriale si ottiene:
\[
	R^2=\frac{c^2}{n^2}\Delta t^2=\beta^2c^2\Delta t^2 + \rho^2 + 2\rho\beta c\Delta t \cos\alpha 
.\] 
Adesso dividiamo per $\rho^2$ questa espressione, ciò che si ottiene è una equazione di secondo grado nell'incognita $\frac{\Delta t}{\rho}$ con dei coefficenti interessanti:
\[
	c^2\left( \beta^2- \frac{1}{n^2} \right) \left( \frac{\Delta t^2}{\rho^2} \right) + 2\beta c \cos\alpha\left( \frac{\Delta t}{\rho} \right) +1=0
.\] 
I coefficenti sono interessanti perchè, per ipotesi noi abbiamo assunto che la particella viaggia ad una velocità tale che $\beta>\frac{1}{n}$, condizione che ritroviamo espressa nel segno del primo coefficente. Quindi la soluzione della equazione esiste se:
\[
	\frac{\Delta}{4}=\beta^2c^2\cos^2\alpha - c^2\left( \beta^2-\frac{1}{n^2} \right) \ge 0
.\] 
Quindi serve avere $\sin\alpha\le \frac{1}{\beta n}$ e, imponendo anche $\Delta t>0$ si deve avere $\cos\alpha<0$ (si prende il cono all'indietro, non quello in avanti).
Quindi abbiamo due soluzioni distinte per ogni punto interno al cono delimitato dai vincoli su $\alpha$.\\
Con il principio di Huygnes sappiamo però che le soluzioni interne al cono non troveranno sovrapposizione coerente di onde sferiche, cosa che invece si ottiene sul bordo del cono dove tutte le onde sferiche avanzano coerentemente a formare una unica perturbazione.
Ci concentriamo quindi sulla soluzione al bordo ($\sin \alpha = \frac{1}{\beta n}$).\\
Facciamo una considerazione geometrica su questa particolare scelta dell'angolo $\alpha$:
\begin{figure}[H]
    \centering
    \incfig{geometria-cherenkov}
    \caption{Situazione geometrica della radiazione cherenkov}
    \label{fig:geometria-cherenkov}
\end{figure}
Sfruttando un teorema sui triangoli:
\[
	\frac{R}{\sin\left( \pi-\alpha \right) }= \frac{\Delta x}{\sin\hat{P}}
.\]
E inserendo i valori di $\Delta x$, $R$ usati sopra si dimostra che l'angolo sotteso al punto $P$ è rettangolo per $\sin\alpha= \frac{1}{\beta n}$.\\
Quindi possiamo ricavare l'angolo Cherenkov semplicemente dalle proprietà dei triangoli rettangoli:
\[
	\theta_{c}=\frac{\pi}{2}-\left( \pi-\alpha \right) = \alpha-\frac{\pi}{2}
.\] 
In particolare
\[
	\sin\alpha =\sin\left( \theta_{c}+ \frac{\pi}{2} \right) = \cos\theta_{c}=\frac{1}{\beta n}
.\] 
Abbiamo quindi trovato l'angolo Cherenkov, indica la direzione dal quale l'osservatore in $P$ vede arrivare l'onda.



\subsection[ Situazioni di applicazione della formula di Frank-Tamm]{Descrivere la situazione in cui la legge 
\[
	\frac{\mbox{d} N_{\gamma}}{\mbox{d} E_{\gamma}\text{d}x} = z^2 \frac{\alpha}{\hbar c} \sin^2\theta_{c}
.\] 
inerente la radiazione Cherenkov, è applicabile e spiegare il significato e l'unità di misura di ogni grandezza fisica ivi indicata.
}\label{sec:4.a.7}
Quella espressa è detta la formula di Frank-Tamm, si applica quando una particella di carica $z$ (in unità di $e$) attraversa uno spessore $dx$ infinitesimo di materiale con indice di rifrazione $n\left( \omega \right)$ ed esprime il numero di fotoni emessi per unità di energia $E_{\gamma}= \hbar \omega$. 
Questa equazione si riscrive in termini dell'indice di rifrazione sostituendo l'equivalente dell'angolo Cherenkov discusso nella domanda precedente:
\[	
	\frac{\mbox{d} N_{\gamma}}{\mbox{d} E_{\gamma}\text{d}x} = z^2 \frac{\alpha}{\hbar c} \left( 1- \frac{1}{\beta^2 n^2} \right)  
.\] 
Ricordiamo adesso alcuni valori delle costanti in gioco:
\begin{align*}
	&\alpha = \frac{e^2}{\hbar c}= \frac{1}{137}	&\hbar c = 197 \text{ MeV}\cdot \text{fm}
.\end{align*}
Quindi \[
	\frac{\alpha}{\hbar c}= \frac{10^{-6}\cdot 10^{13}}{137\cdot 197} \left[ \frac{\text{fotoni}}{\text{eV}\cdot \text{cm}} \right] \approx 
	370 \left[ \frac{\text{fotoni}}{\text{eV}\cdot \text{cm}} \right]
.\] 
Ricordiamo anche che l'indice di rifrazione è in realtà $n\left( \omega \right) $, da cui deriva la dipendenza della formula da $E_{\gamma}$.

\subsection[ Situazioni di applicazione della formula di Frank-Tamm integrale]{Descrivere la situazione in cui la legge 
\[
	N_{\gamma}= z^2 \frac{\alpha}{\hbar c}L \int_{E_1}^{E_2} \left[ 1 - \frac{1}{\beta^2 \epsilon_{r}\left( E \right)}\right]P_{\text{det}}dE 	
.\] 
inerente la radiazione Cherenkov, è applicabile e spiegare il significato e l'unità di misura di ogni grandezza fisica ivi indicata.
}\label{sec:4.a.8}
Le quantità sono le stesse della domanda precedente, tuttavia questa esprime il numero di fotoni rilevati da un detector avente una efficienza $P_{\text{det}}$ nel range di energia tra $E_1$ ed $E_2$, il mezzo attraversato ha inoltre larghezza $L$.
\subsection[ Cause ed effetti della radiazione di frenamento]{Descrivere qualitativamente le cause e gli effetti del fenomeno della radiazione di frenamento da parte di una particella carica nella materia.
}\label{sec:4.a.9}
Una particella carica di massa $M$ e carica $q=ze$ che attravera un materiale subisce l'interazione coulombiana con i nuclei che lo compongono di carica $Ze$.\\
Per ogni interazione la particella accelera (a causa della repulsione o attrazione) e quindi irraggia: Radiazione di frenamento o Bremsstrahlung.
\begin{figure}[H]
    \centering
    \incfig{bremsstrahlung}
    \caption{Schema dello scattering coulombiano per la Bremsstrahlung}
    \label{fig:bremsstrahlung}
\end{figure}
La perdita di energia dovuta a questo irraggiamento è importante per elettroni (aventi una certa energia critica $E_{c}$) ed è invece trascurabile per le altre particelle. Inoltre la radiazione di frenamento avviene in modo indipendente su ciascun nucleo del materiale attraversato, è allora facile immaginare che darà luogo a processi di emissione incoerenti.\\
Nel sistema del laboratorio la radiazione può esssere rilevata in modi diversi a seconda della velocità del proiettile:
\paragraph{Particella non relativistica}%
La radiazione è proporzionale a $\sin^2\alpha$, con $\alpha$ angolo tra la media della accelerazione nel processo e la direzione di osservazione.
\begin{figure}[H]
    \centering
    \incfig{bremsstrahlung-non-relativistica}
    \caption{Emissione della radioazione nel caso di Bremsstrahlung non relativistica}
    \label{fig:bremsstrahlung-non-relativistica}
\end{figure}
\paragraph{Particella ultrarelativistica}%
La radiazione viene rilevata in avanti, con angolo di $\frac{1}{\gamma}$ rispetto alla direzione di $\bs{\beta}$ (si dimostra con una trasformazione dal sistema in cui la particella è ferma a quello del laboratorio).
\begin{figure}[ht]
    \centering
    \incfig{bremsstrahlung-relativistica}
    \caption{Emissione della radioazione nel caso di Bremsstrahlung relativistica}
    \label{fig:bremsstrahlung-relativistica}
\end{figure}

\subsection[\hspace{1mm} Importanza della Bremsstralhung per gli elettroni]{Spiegare perchè la perdita di energia per irraggiamento è significativa per elettroni e non per le altre particelle cariche.
}\label{sec:4.a.10}
La forza di Coulomb dipende dalla carica elettrica e non dalla massa della particella, quindi dall'equazione del moto si ottiene che:
\[
	a \sim \frac{1}{M}
.\]
Visto che l'irraggiamento dipende dall'accelerazione al quadrato si ha anche che:
\[
	I\sim \frac{1}{M^2}
.\] 
e banalmente questo è l'unico motivo per cui gli elettroni irraggiano maggiormente di Bremsstrahlung.

\subsection[\hspace{1mm} Situazione di applicazione della energia irraggiata per unità di frequenza e di angolo solido]{Descrivere la situazione in cui la legge
\[
	\frac{\mbox{d}I_{\omega}}{\mbox{d}\Omega}=
	\frac{q^2}{4\pi^2c}\left|\int\frac{\hat{n}\wedge\left[\left(\hat{n}-\bs{\beta}\right)\wedge\dot{\bs{\beta}}\right]}{\left(1-\hat{n}\cdot\bs{\beta}\right)^2}
	e^{i\omega\left(t'-\frac{\bs{r}'\cdot\hat{n}}{c}\right)}dt'\right|^2 
.\] 
è applicabile e spiegare il significato e l'unità di misura di ogni grandezza fisica ivi
indicata.
}\label{sec:4.a.11}
L'equazione descrive l'energia irraggiata per unità di frequenza ed angolo solido, deriva dalla più generale 
\[
	\frac{\mbox{d} I_{\omega}}{\mbox{d} \Omega} = \frac{\epsilon_0 c r^2}{\pi} \left| \bs{E}\left( \bs{r}, \omega \right)  \right| ^2
.\] 
Dove $\bs{E}\left( \bs{r}, \omega \right)$ è la trasformata del campo elettrico di radiazione (valida per ogni tipo di irraggiamento).
Nel caso in questione è quello di una carica puntiforme accelerata in moto relativistico, tale equazione è stata dimostrata nella \hyperref[sec:3.a.14]{Domanda 3.a.14}.
\subsection[\hspace{1mm} Energia irraggiata per unità di frequenza per Bremsstralhung non relativistica]{Descrivere la situazione in cui la legge
\[
	I_{\omega}\left( b \right) = 
	\begin{cases}
		\frac{8z^4Z^2 \alpha \hbar c^2}{3\pi} \left( \frac{m_e}{M} \right) ^2 \frac{r^2_e}{V^2} \frac{1}{b ^2}  & \text{per } \omega<\frac{V}{b}\\
		0 & \text{per } \omega> \frac{V}{b}
	\end{cases}
.\] 
applicabile e spiegare il significato e l'unità di misura di ogni grandezza fisica ivi indicata.
}\label{sec:4.a.12}
$I_{\omega}\left( b \right) $ indica l'energia irraggiata per unità di frequenza in funzione del parametro di impatto (J$\cdot$s). Tutte le altre quantità sono state ampliamente preserntate nel corso della lettura ma facciamo un ripassino su quelle che non vediamo da un pò:
\begin{align*}
	&r_e = k_0 \frac{e^2}{m_e c^2} \approx 2.82 \text{ fm}\\
	&m_e = 0.511 \text{MeV/c}^2 = 9.12 \cdot 10^{-31} \text{kg}
.\end{align*}
Detto questo la formula rappresenta $I_{\omega}$ per una Bremstrahlung nel caso non relativistico nella approssimazione che il moto complessivo della carica possa essere assunto come rettilineo, quindi angoli di scattering piccoli. In tal modo viene assunta la variazione di quantità di moto nella sola direzione ortogonale alla direzione di incidenza del proiettile ed i conti per la dimostrazione diventano pochi.\\
È stata inoltre approssimata la funzione ad uno scalino, quando in realtà i termini sopra esrpessi corrispondono all'andamento asintodico di quest'ultima.

\subsection[\hspace{1mm} Sezione d'urto di irraggiamento]{Definire la sezione d'urto di irraggiamento e spiegare il suo significato fisico.
}\label{sec:4.a.13}
La sezione d'urto di irraggiamento è definita come
\[
	\chi_{\omega}= \int_{b_{min}}^{b_{max}} 2\pi I_{\omega}b db \quad \left[ \frac{\text{J}}{\text{Hz}}\cdot \text{m}^2 \right] 
.\] 
ed è la grandezza che moltiplicata per il numero di nuclei per unità di superficie fornisce l'energia irraggiata per intervallo di frequenza $I_{\omega}$.

\subsection[\hspace{1mm} Sezione d'urto di irraggiamento per schermaggio incompleto]{Descrivere la situazione in cui la legge
	\[
		\chi_{\omega}=\frac{16 z^4 Z^2 \alpha \hbar c^2}{3} \left( \frac{m_e}{M} \right)^2 \frac{r_e^2}{V^2} \ln\left( \frac{MV^2}{\hbar \omega} \right) 
	.\] 
e spiegare il significato e l'unità di misura di ogni grandezza fisica ivi indicata.
}\label{sec:4.a.14}
Questa sezione d'urto di irraggiamento è quella derivante dalla equazione della domanda precedente integrata sul parametro di impatto. Questa è applicabile nel caso in cui: 
\[
	\frac{\hbar}{M V} < b < \frac{V}{\omega}
.\] 
Con $M$ la massa della particella incidente.\\
Saremmo portati ad pensare che il limite inferiore su $b$ sia il raggio nucleare, in realtà il principio di indeterminazione risulta più stringente per particelle lente:
\[
	\Delta x \Delta p = \hbar 
.\] 
L'indeterminazione sull'impulso della particella è sicuramente inferiore all'impulso iniziale di questa: $\Delta p \lesssim MV$ (siamo a distanze tali per cui la probabilità di schianto nel nucleo è alta), mentre sul $\Delta x$ l'indeterminazione è proprio il parametro di impatto $b$, quindi:
\[
	b MV \gtrsim \hbar \implies b \gtrsim \frac{\hbar}{MV}
.\] 
Come limite superiore invece saremmo portati a pensare che sia il raggio atomico, in realtà il tempo tipico di interazione stringe ancora il campo nel caso non relativistico:
\[
	\Delta t \approx \frac{b}{V} \implies \omega < \frac{V}{b} 
.\] 
Nel caso di elettroni relativistici il parametro di impatto inferiore è proprio la lunghezza d'onda Compton.

\subsection[\hspace{1mm} Sezione d'urto di irraggiamento relativistica]{Descrivere la situazione in cui la legge
\[
	\frac{\mbox{d}E^{\text{irr}}}{\mbox{d}x}=
	n_{\text{nuclei}}\frac{16}{3}z^4Z^2\alpha\left(\frac{m_e}{M}\right)^2r_e^2\ln\left(\frac{192}{Z^{1 /3}}\frac{M}{m_e}\right) E
.\] 
applicabile e spiegare il significato e l'unità di misura di ogni grandezza fisica ivi indicata.
}\label{sec:4.a.15}
La formula descrive l'energia irraggiata per unità di lunghezza da una particella ultrarelativistica ($V \sim c$) con energia iniziale $E$ e massa $M$, questa espressione si ottiene integrando la sezione d'urto di irraggiamento di tale particella nel dominio detto di Screening attivo, ricaviamo intanto tale sezione d'urto. Nel sistema $\Sigma'$ in cui la particella è a riposo si ha che durante lo Scattering (in cui il nucleo va incontro alla particella) $M$ irraggia in manera non relativistica, quindi la formula per la $\chi'_{\omega}$ è:
\[
	\chi'_{\omega}=\frac{16}{3}z^4Z^2\alpha\hbar c^2 \left( \frac{m_e}{M} \right)^2  \frac{r^2_e}{V^2}\ln\left( \frac{b_{\text{MAX}}}{b_{\text{min}}} \right) 
.\] 
È bene ricordare che per questa interazione avremo una larghezza spettrale di \[
	\omega'< \gamma \frac{c}{b}
\]
Come minimo del parametro di impatto si può usare il principio di indeterminazione come per la domanda precedente, c'è tuttavia da stare attenti al fatto che adesso $\Delta p \lesssim MV\gamma$, quindi:
\[
	b \gtrsim \frac{\hbar}{MV\gamma}
.\] 
Considerando la grandezza di $\gamma$ questa stima risulta decisamente inferiore del raggio nucleare, è quindi erronea.\\
IL tempo di interazione è $\tau \approx b /\left( c\gamma \right)$, possiamo allora pensare che il campo elettrico agisce sul proiettile in una regione di estensione $\tau \cdot c = b /\gamma$. Se questa fosse più piccola di $\Delta x$ allora non avrebbe senso parlare di interazione e irraggiamento (si presenterebbero effetti di interferenza), quindi poniamo:
\[
	\Delta x \gtrsim \frac{b}{\gamma} \implies b_{\text{min}} = \frac{\hbar}{MV}
.\] 
Come nel caso non relativistico.\\
Dobbiamo adesso trovare il limite superiore, si hanno due possibili casi:
\begin{align*}
	b^{\left( 1 \right) }_{\text{max}} = \gamma \frac{v}{\omega'}
.\end{align*}
Considerando il limite dovuto al tempo di interazione, oppure
\[
	b^{\left( 2 \right) }_{\text{max}}= a =1.4 \frac{a_0}{Z^{1 /3}}
.\] 
Che corrisponde al raggio atomico con:
\[
	a_0 = \frac{\hbar}{m_e c \alpha}
.\] 
Per valutare quale dei due sia quello giusto consideriamo il fatto che, nel sistema del laboratorio, tutta la radiazione viene emessa in avanti con un angolo rispetto alla velocità iniziale della particella:
\[
	\hat{\theta}=\frac{1}{\gamma} \implies \cos \hat{\theta}\approx 1-\frac{1}{2\gamma^2}
.\] 
Quindi la corrispondente frequenza nel sistema primato sarà 
\[
	\omega'=\gamma\omega\left(1-\beta\cos\hat{\theta}\right)\approx\gamma\omega\left(1-\left(1-\frac{1}{2\gamma^2}\right)\left(1-\frac{1}{2\gamma^2}\right)\right) 
	= \frac{\omega}{\gamma}
.\] 
Siccome il fotone irraggiato nel laboratorio avrà sicuramente energia inferiore all'energia cinetica dell'elettrone che lo ha partorito si ha (conservazione dell'energia):
\[
	\hbar\omega\le \left( \gamma-1 \right)Mc^2\implies\omega'\lesssim \frac{Mc^2}{\hbar}
.\] 
Questo ci aiuta a valutare chi tra i due parametri è più grande:
\[
	\frac{b^{\left( 1 \right)}_{\text{max}}}{b^{\left( 2 \right)}_{\text{max}}}=\frac{\gamma v /\omega'}{a}\gtrsim \frac{\gamma\hbar}{aMc}=
	=\gamma Z^{1 /3} \frac{m_e \alpha}{1.4 M}  \approx \gamma\frac{Z^{1 /3}}{192}\frac{m_e}{M}
.\] 
Vista la dipendenza diretta da $\gamma$ è facile vedere che, per elettroni ultrarelativistici si ha sempre $\gamma v /\omega' > a$, quindi il parametro giusto da scegliere è il secondo, ovvero il raggio atomico $a$.
Il caso in cui lavoriamo è detto screening attivo, espresso graficamente come segue:
\begin{figure}[H]
    \centering
    \incfig{screening-bremsstrahlung}
    \caption{Tipi di screening per la Bremsstrahlung.}
    \label{fig:screening-bremsstrahlung}
\end{figure}
In ogni caso si deve rispettare il fatto che la frazione calcolata poco fa deve essere maggiore di 1 per andare in Screening attivo, quindi:
\[
	\gamma > \frac{192M}{Z^{1 /3}m_e}
.\] 
Inserendo tutto nella sezione d'urto si ottiene l'espressione:
\[
	\chi'_{\omega}=\frac{16}{3}z^4Z^2\alpha\hbar c^2 \left( \frac{m_e}{M} \right)^2  \frac{r^2_e}{c^2}\ln\left( \frac{192\cdot M}{Z^{1 /3}\cdot m_e} \right) 
.\] 
Adesso basta notare che, come visto sopra, $\omega = \gamma \omega'$. Inoltre $dE_{\text{irr}} = \gamma dE_{\text{irr}}'$ che viene dal fatto che questa quantità infinitesima trasforma come un tempo:
\[
\frac{\mbox{d} E}{\mbox{d} t} = -\frac{2}{3}\frac{e^2}{c^3}\frac{\mbox{d} u^{\mu}}{\mbox{d} \tau} \frac{\mbox{d} u_{\mu}}{\mbox{d} \tau}\implies
\text{d}E \sim  G \text{d}t
.\]
Con G invariante di Lorentz.\\
Prendiamo allora la definizione di $\chi_{\omega}$:
\[
	\chi_{\omega}=\int_{b_{\text{min}}}^{b_{\text{max}}}2\pi b \frac{\mbox{d} E}{\mbox{d} w} db
.\] 
Con tutte le considerazioni fatte sopra e considerando in più che il parametro di impatto è invariante (essendo ortogonale al boost) otteniamo $\chi_{\omega}= \chi'_{\omega}$, quindi:
\[
	\frac{\mbox{d} E^{\text{irr}}}{\mbox{d} x} = n_{\text{nuclei}} \int_{0}^{\omega_{\text{max}}} \chi_{\omega} d\omega = 
	n_{\text{nuclei}} \int_{0}^{E /\hbar} d\omega \chi_{\omega} 
	= n_{\text{nuclei}} \frac{E}{\hbar} \chi_{\omega}
.\] 
Che è proprio l'espressione cercata (E è l'energia cinetica della particella!!).
\subsection[\hspace{1mm} Sezione d'urto di irraggiamento in funzione della sezione d'urto espressa in $m^2$]{Descrivere la situazione in cui la legge
\[
	\chi_{\omega} = \hbar \omega \frac{\text{d}\sigma_{y}}{\text{d}\omega}
.\] 
 è applicabile e spiegare il significato e l'unità di misura di ogni grandezza fisica ivi indicata.
}\label{sec:4.a.16}
La formula è una manipolazione algebrica della definizione di $\chi_{\omega}$:
\[
	\chi_{\omega}= \frac{1}{n} \frac{\mbox{d} E_{\text{irr}}}{\mbox{d} x \text{d}\omega} 
.\] 
Con $n$ densità dei centri scatteranti. Proviamo a manipolare l'espressione:
\[
	\chi_{\omega} = \hbar \omega\frac{\mbox{d}}{\mbox{d} \omega} \left( \frac{\mbox{d}E_{\text{irr}}}{\mbox{d}t}\frac{1}{n\cdot \hbar \omega \frac{\mbox{d} x}{\mbox{d} t} } \right)  
.\] 
Possiamo allora definire una corrente di particelle proiettili (l'inverso del secondo termine in parentesi) a moltiplicare la potenza irraggiata mediata (è mediata perchè $E_{\text{irr}}$ è l'energia irraggiata totale), questa altro non è che la sezione d'urto del processo di irraggiamento (definita come una superficie, non come quella finta sezione d'urto $\chi_{\omega}$).
\[
	\chi_{\omega}= \hbar \omega \frac{\mbox{d} }{\mbox{d} \omega} \left( \sigma_{y} \right) 
.\] 
\subsection[\hspace{1mm} Definizione di lunghezza di radiazione]{Dare la definizione di lunghezza di radiazione.
}\label{sec:4.a.17}
La lunghezza di radiazione $X_0$ è la distanza tipica che una particella percorre in un materiale. Tale distanza infatti subentra nella legge esponenziale con cui varia l'energia della particella:
\[
	E = E_0 e^{-x /X_0}
.\] 
Tale lunghezza è tipica del materiale che la particella percorre. Per lo Screening attivo la lunghezza di radiazione è l'inverso del coefficiente che moltiplica $E$ (l'equazione della domanda 4.a.15):
\[
	X_0^{\text{attivo}} = \frac{1}{n_{\text{nuclei}}\frac{16}{3}z^4Z^2\alpha\left(\frac{m_e}{M}\right)^2r_e^2\ln\left(\frac{192}{Z^{1 /3}}\frac{M}{m_e}\right)} 
.\] 

\subsection[\hspace{1mm} Definizione di energia Critica]{Dare la definizione di "Energia critica"
}\label{sec:4.a.18}
L'energia critica è l'energia oltre il quale l'energia dell'elettrone si attenua per il solo effetto della Bremstrahlung.\\
Nello screening attivo abbiamo visto esserci un limite inferiore a $\gamma$ per la particella incidente, questo implica che in tal caso l'energia della particella ha anc'essa un limite inferiore che corrisponde proprio alla sua energia critica:
\[
	E_{c} = Mc^2\gamma_{\text{lim}}= Mc^2 \frac{192}{Z^{1 /3}}
.\] 
\subsection[\hspace{1mm} Legge di attenuazione per irraggiamento dell'energia in un mezzo]{Descrivere la situazione in cui la legge \[
	E = E_0 e^{-x /X_0}
.\] 
è applicabile e spiegare il significato e l'unità di misura di ogni grandezza ivi indicata.
}\label{sec:4.a.19}
La legge è applicabile nel caso di Bremstrahlung con particella incidente di energia $E_0$ superiore alla energia $E_{c}$ e descrive l'andamento dell'energia all'interno del materiale in cui tutta l'energia persa va in irraggiamento.
\subsection[\hspace{1mm} Formula di Tsai]{Descrivere la situazione in cui la legge
	\[
		\frac{1}{\rho X_0} = 4 \alpha r^2_e\frac{Z^2}{A\left(g\right)}N_A\left[\ln\left(\frac{184}{Z^{1/3}}\right)-f\left(Z\right)+\frac{L'}{Z}\right] 
	.\] 
applicabile e spiegare il significato e l'unità di misura di ogni grandezza ivi indicata. [formula di Tsai non dimostrata a lezione]
}\label{sec:4.a.20}
La formula descrive in modo più preciso l'inverso della lunghezza di radiazione tenendo condo della struttura del materiale che il proiettile attraversa (con gli appositi coefficienti $f\left( Z\right) $ e $L'$).\\
Il termine $f\left( Z \right) $ corrisponde ad una serie in $a = \alpha Z$, i primi termini sono :
\[
	f\left( Z \right) = a^2\left[ \left( 1+a^2 \right)^{-1}+0.0202 + \ldots \right] 
.\] 
Il termine $L'$ invece dipende dal tipo di materiale, tuttavia per i materiali con $Z\ge 5$ vale:
\[
	L' = \ln\left( \frac{194}{Z^{2 /3}} \right) 
.\] 
Ricordiamo che $A\left( g \right) $ è la massa atomica del materiale attraversato in grammi e $N_A = 6.022 \cdot 10^{23} $ è il numero di avogadro.
\subsection[\hspace{1mm} Perdita di energia per collisioni]{Descrivere qualitativamente il meccanismo della perdita di energia per collisioni da parte di una particella carica nella materia.
}\label{sec:4.a.21}
Una particella carica interagisce con gli elettroni atomici di un materiale cedendo loro energia tramite interazioni coulumbiane. Si ipotizza in queste interazioni che la variazione di impulso di ogni elettrone sia interamente ortogonale alla direzione di incidenza della particella facendo sostanzialmente uso di una approssimazione impulsiva. \\
L'effetto che il passaggio della particella provoca sul materiale può essere di vari tipi:
\begin{itemize}
	\item Il materiale si ionizza producendo coppie elettrone-ione nei gas o elettrone-lacuna nei semiconduttori.
	\item Il materiale emette fotoni di scintillazione (sostanzialmente dovuti al salto in alto e successivamente in basso in energia dell'elettrone partecipe dello scattering).
	\item Riscaldamento del materiale per agitazione termica nel passaggio.
\end{itemize}
Tale processo trova applicazione nella misura simultanea della quantità di moto e della massa di una pariticella ed allo studio del percorso residuo di particelle cariche per uso medico (adroterapia) o protettivo per la schermatura di radiazione (degli acceleratori di particelle).

\subsection[\hspace{1mm} Formula di Bethe-Bloch]{Spiegare il significato di ogni termine dell’espressione per la perdita di energia per collisioni (formula di Bethe-Bloch, non dimostrata):
	\[
		\frac{1}{\rho}\frac{\mbox{d} E_{\text{coll}}}{\mbox{d} x} = 
		z^2 \frac{Z}{A\left( g \right) }4\pi \frac{m_e c^2}{\beta^2} N_A r^2_e 
		\left( \frac{1}{2}\ln\left( \frac{2m_e c^2 \beta^2 \gamma^2}{I^2}T_{\text{Max}} \right) - \beta^2 + \frac{\delta}{2} \right) 
	.\] 
}\label{sec:4.a.22}
Abbiamo alcune quantità già viste nella formula di Tsai, in più c'è $I$ che è l'energia potenziale media di ionizzazione, $T_{\text{max}}$ che è l'energia massima trasferibile dalla particella all'elettrone:
\[
	T_{\text{Max}}= \frac{2m_e c^2 \beta\gamma}{1+ \frac{2m_e \gamma}{M}+\left( \frac{m_e}{M} \right) ^2}
.\] 
ed il fattore $\delta /2$ è il fattore di correzione della densità dovuto a fermi, quest'ultimo dipende da $\beta$ e dal tipo di materiale attraversato.

\subsection[\hspace{1mm} Andamento della Bethe-Bloch]{Disegnare qualitativamente la funzione di Bethe-Bloch indicando i valori dei punti significativi.
}\label{sec:4.a.23}
\begin{figure}[H]
    \centering
    \incfig{bethe}
    \caption{Andamento commentato della Bethe-Bloch.}
    \label{fig:bethe}
\end{figure}
Sull'asse delle $x$ abbiamo $\beta\gamma = \frac{\left| \bs{p} \right| }{Mc}$.
Si  aggiunge soltanto che il minimo della funzione si ottiene per:
\[
	\frac{1}{\rho}\left.\frac{\mbox{d} E_{\text{coll}} }{\mbox{d} x} \right|_{\text{min}} \sim 2 \frac{\text{MeV}}{g /\text{cm}^2} 
.\] 
Le particelle aventi impulso $\left| \bs{p} \right| \sim 3.5$ Mc si dicono al minimo di ionizzazione e sono importanti per l'effetto detto Picco di Brag.

\subsection[\hspace{1mm} Range o percorso residuo di una particella]{Definire il "percorso residuo" ("range") per una particella carica in un materiale.
}\label{sec:4.a.24}
Il percorso residuo $R$ (anche chiamato Range) è il percorso dopo il quale una particella attraversando un materiale esaurisce la sua energia e si ferma.\[
	R\left( E_0 \right) = \int_0^{R} dx = \int_0^{E_0} dE \cdot \frac{\mbox{d} x}{\mbox{d} E} =
	\int_0^{E_0}\frac{1}{\frac{\mbox{d} E}{\mbox{d} x} } dE
.\] 
Il calcolo precedente viene fatto numericamente poichè l'espressione per $\frac{\mbox{d} x}{\mbox{d} E}$ detto Stopping Range è complicata e soprattutto tiene conto non solo della energia persa in collisioni coulombiane ma anche quella persa in irraggiamento e in interazioni nucleari forti.
\subsection[\hspace{1mm} Stopping Power per i vari processi nella materia]{Definire gli "stopping power" totale, collision, radiative e nuclear; indicare per quali particelle ognuno di essi sia o meno rilevante.
}\label{sec:4.a.25}
Partiamo dal totale:
\[
	\left( \frac{\mbox{d} x}{\mbox{d} E} \right)_{\text{tot}}= \left( \frac{\mbox{d} x}{\mbox{d} E} \right)_{\text{collision}}+\left( \frac{\mbox{d} x}{\mbox{d} E} \right)_{\text{radiative}}+\left( \frac{\mbox{d} x}{\mbox{d} E} \right)_{\text{nuclear}}
.\] 
Il primo termine è l'inverso della formula di Bethe-Block, il secondo è l'inverso della formula di Tsai mentre il terzo è rilevante solo per interazioni forti, ovvero interessa gli adroni.

\subsection[\hspace{1mm} Calcolo del percorso residuo nota la curva di stopping power.]{Come si calcola il "percorso residuo" ("range"), nota la curva $\frac{\mbox{d} E}{\mbox{d} x}$, in funzione dell'energia E della particella?
}\label{sec:4.a.26}
Si calcola con l'integrale numerico della \hyperref[sec:4.a.24]{Domanda 4.a.24}. 


\subsection[\hspace{1mm} Picco di Brag]{Spiegare qualitatativamente il cosiddetto "picco di Bragg"
}\label{sec:4.a.27}
Quando una particella carica è rallentata da un materiale fino ad arrestarsi la distribuzione di energia depositata dipende in maniera marcata dalla forma dello stopping power $ST\left( E \right) = \left( \frac{\mbox{d} E}{\mbox{d} x}  \right)^{-1}$.\\
Se la particella entra nel mezzo con una energia superiore a quella del minimo di ionizzazione $\beta \gamma > 3.5$ allora l'energia deposotata per unità di lunghezza decresce fino a raggiungere il minimo. Proseguendo oltre il minimo la particella continua a perdere energia per collisioni (quindi $\beta$ cala) ma la perdita per unità di lugnhezza in questa direzione sale come $1 /\beta^2$. In questo modo l'energia depositata avrà un massimo in corrispondenza del punto di arresto nel materiale della particella, tale massimo è chiamato Picco di Brag.

\subsection[\hspace{1mm} Scattering multiplo per particella in moto veloce nella materia]{Descrivere qualitativamente il fenomeno dello scattering multiplo da parte di una particella carica in moto veloce nella materia.
}\label{sec:4.a.28}
Nell'urto di una particella con un atomo bersaglio l'angolo di deflessione è, nel peggiore dei casi $\sim 10^{-3}$ rad, quindi trascurabile nella maggior parte dei contesti. Se invece la particella attraversa un materiale allora incontrerà diversi nuclei nel suo cammino, per ciascun nucleo avremo piccole deflessioni scatteranti.\\
La discussione sul fenomeno dello scattering multiplo è appunto la valutazione dell'angolo di deflessione all'uscita del mezzo. Anticipiamo che facendo numerose interazioni l'angolo di deflessione sarà distribuito gaussianamente (teorema del limite centrale). 

\subsection[\hspace{1mm} Definizione di angolo multiplo di scattering]{Definire l'angolo di multiplo scattering (rispetto alla direzione iniziale della particella) e definire la sua proiezione su un piano (che contiene la direzione iniziale della particella). Indicare i limiti delle due variabili cosí definite.
}\label{sec:4.a.29}
L'angolo multiplo di scattering $\theta$ è l'angolo formato tra la direzione iniziale e la direzione dopo lo scattering multiplo della particella che attraversa un mezzo. Possiamo ipotizzare questo angolo come $\theta \ll 1$ per una particella veloce.\\
Ipotizzando che la particella fosse inizialmente diretta lungo l'asse $\hat{z}$, possiamo definire le proiezioni dell'angolo di deflessione sui due piani che contengonola direzione iniziale tra loro ortogonali: $\theta_{x}$ e $\theta_{y}$ con $\theta^2=\theta^2_{x}+\theta^2_{y}$.\\
Come accennato sopra si impone una distribuzione gaussiana su questi angoli con r.m.s. pari a $\theta_0$ definita dal tipo di scattering preso come modello (per angoli piccoli l'ideale è lo scattering Rutherford).
Anche allo scarto quadratico medio su queste due variabili si impone $\theta_0\ll 1$, inoltre sono entrambe variabili a media nulla.

\subsection[\hspace{1mm} Espressione per l'angolo quadratico medio di scattering]{Spiegare il significato di ogni termine dell'espressione per l’angolo quadratico medio di multiplo scattering (rispetto alla direzione iniziale della particella)
\[
	\sqrt{\left<\theta^2_{\text{ms}}\right>}=\theta_0\sqrt{2}=z\frac{13.6\text{MeV}}{P\beta c}\sqrt{\frac{L}{X_0}}\left(1+0.0038\ln\left(\frac{L}{X_0}\right)  \right)  
.\] 
}\label{sec:4.a.30}
$\theta_0$ è l'angolo r.m.s. per i due angoli definiti sopra, $P$ è l'impulso iniziale della particella, $L$ è lo spessore del materiale attraversato,  $X_0$ la lunghezza di radiazione del materiale. Si arriva a questo risultato sfruttando per lo scarto quadratico sul singolo urto la sezione Mott e moltiplicando per il numero di possibili urti sulla base della lunghezza e densità del materale.

\subsection[\hspace{1mm} Distribuzione per il fenomeno di multiplo scattering]{Se non fosse sufficiente la approssimazione di piccoli angoli e distribuzione gaussiana, indicare quale fra le seguenti funzioni descriverebbe meglio il fenomeno del multiplo scattering:\\
	i) Bethe-Bloch,\\ 
	ii) Moliere,\\ 
	iii) Breit-Wigner, \\
	iv) Bohr.
}\label{sec:4.a.31}
La funzione sarebbe la Moliere.

\subsection[\hspace{1mm} Metodo di produzione di antiprotoni nell'esperimento di Segre]{Illustrare in modo qualitativo il metodo di produzione degli antiprotoni nell’esperimento di Segré et al.
}\label{sec:4.a.32}
Tramite l'acceleratore Bevatron si invia un fascio di protoni su una lamina di rame, l'urto rompe l'atomo in questione, si liberano $p, \pi^+, \overline{p}, \pi^-$. Gli antiprotoni liberati hanno impulso $p= 1.19$ MeV/c.
Successivamente le particelle vengono deviate dal campo del Bevatron per entrare nel sistema di rilevazione ideato per l'esperimento. Tra le particelle positive si ha in media circa 1 antiprotone ogni $10^{5}$ pioni. Con l'impulso di queste particelle si ha $\beta = 0.99$ per i pioni e $\beta=0.78$ per gli antiprotoni.

\subsection[\hspace{1mm} Metodo di separazione degli antiprotoni dal fondo di pioni nell'esperimento di Segre]{Spiegare qualitativamente il metodo di separazione degli antiprotoni dal fondo di pioni nell’esperimento di Segré et al. tramite contatori Cerenkov
}\label{sec:4.a.33}

Le particelle uscenti sono deflesse di $21^o$ dal campo magnetico del Bevatron e successivamente di altri $34^o$ dal campo magnetico di M1. Dopo M1 le particelle passano attraverso un collimatore Q1: un magnete quadripolare che focalizza il fascio ed uno scintillatore S1. 
\begin{figure}[H]
    \centering
    \incfig{segre}
    \caption{segre}
    \label{fig:segre}
\end{figure}
Dopo le particelle incontrano un secondo collimatore Q2 seguito da un altro magnete M2 che le deflette di altri $34^o$. Infine le particelle attraversano il secondo scintillatore S2, un primo contatore Cherenkov C1 che rileva particelle con $\beta>0.78$, un secondo contatore Cherenkov C2 con $0.75< \beta<0.78$ ed un ultimo scintillatore per rilevare le particelle che uscenti da C2 con angoli di scattering troppo grandi.
Il sistema di magneti permette di selezionare le particelle con un certo impulso, la velocità della particella invece è misurata in due modi indipendenti: tempo di volo e contatori Cherenkov.\\
Fare due misure indipendenti della velocità è indispensabile in quanto, utilizzando ad esempio il solo tempo di volo come controllo si otterrebbe una distribuzione di eventi del tipo:
\begin{figure}[H]
    \centering
    \incfig{distribuzione-pioni-antiprotoni}
    \caption{Distribuzione pioni-antiprotoni}
    \label{fig:distribuzione-pioni-antiprotoni}
\end{figure}

È quindi richiesta la coincidenza di eventi $S_1 * S_2 * C_2$, se scatta anche lo scintillatore $S_3$ allora il conteggio è annullato perchè vuol dire che la particella aveva un grande angolo di scattering (si evitano errori dovuti a particelle che non erano ben collimate con il gruppo). Con questo metodo di controlli incrociati si riesce ad escludere quasi tutti i $\pi$ dal conteggio (il 3 \% di questi attiva C2), per i pochi pioni che attivano C2 si utilizza come controllo C1: se si attiva quest'ultimo il conteggio viene annullato. In questo modo è stato possibile ridurre il fondo di pioni e distinguere il segnale dell'antiprotone tramite oscilloscopi (presumibilmente attivati dalle scintillazioni della strumentazione).

\subsection[\hspace{1mm} Esperimento di Anderson sulla scoperta del positrone]{Descrivere l’esperimento di Anderson sulla scoperta del positrone.
}\label{sec:4.a.34}
Nell'esperimento venne utizzata una camera a nebbia: un dispositivo isolato contenente vapore saturo. \\ 
Una particella carica e sufficientemente energetica che attraversa la camera ionizza il gas sulla sua traiettoria e gli atomi ionizzati formano dei nuclei di condensazione su cui si formano delle goccioline. Fotografando la camera è quindi possibile ricostruire la traiettoria della particella incidente. Se inoltre immergiamo il tutto in un campo magnetico possiamo misurare anche l'impulso a partire dal raggio di curvatura.
\[
	R = \frac{mV}{qB}
.\] 
Anderson utilizzo un campo magnetico di 1.7 T e registrò la traccia di una particella che attraversa 6 m di piombo. Le sue osservazioni furono che la particella prima di attraversare la lastra aveva un impulso $p_{i}=63$ MeV/c, dopo l'attraversamento invece $p_{f}=22.5$ MeV/c, dal verso di curvatura dedusse che la particella aveva carica positiva.\\
Si poteva quindi avanzare l'ipotesi che la particella fosse un protone non relativistico, se non fosse che in tal caso:
\[
	E_{i}= \frac{p^2}{2m_{p}}\approx 2.1 \text{ MeV}
.\] 
Quindi non avrebbe sufficiente energia per attraversare uno strato di 6 mm di piombo.\\
Ipotizzando invece che fosse un positrone relativistico questo perderebbe circa 7 MeV per ionizzazione e la sua energia si riduce di un fattore $1 /e$ per irraggiamento. Il risultato è quindi compatibile con l'impulso misurato nell'esperimento: era un positrone relativistico.

 % da revisionare

\section{Domande b}
\subsection[]{Dimostrare che all’interno del cono della radiazione Cherenkov vi sono due soluzioni per 
\[
	t'=t-\frac{nR}{c}
,\] 
nessuna soluzione all’esterno, ed una sola sul fronte d’onda.
}\label{sec:4.b.1}
Il problema è stato ampliamente discusso nella spiegazione del fenomeno alla domanda \hyperref[sec:4.a.6]{Domanda 4.a.6}.

\subsection[]{A partire dalla espressione
\[
	\frac{\mbox{d}^2 N_{\gamma}}{\mbox{d}E_{\gamma}\text{d}x} = z^2 \frac{\alpha}{\hbar c}\sin^2\theta_{c}
.\] 
valida per la radiazione Cherenkov, dimostrare che
\[
	N_{\gamma}=z^2 \frac{\alpha}{\hbar c} \int_{E_1}^{E_2}\left[ 1- \frac{1}{\beta^2 \epsilon_{r}\left( E \right)}\right]P_{\text{det}}\left(E\right) dE
.\] 
}\label{sec:4.b.2}
È necessario ricordare, dall'espressione dell'angolo Cherenkov che:
\[
	\sin^2\theta_{c}=1- \frac{c^2}{n^2\left(E\right)v^2}= 1- \frac{1}{\beta^2 \epsilon_{r}\left( E \right) }
.\] 
Successivamente integrando su $x$ si ottiene il fattore $L$ che è la lunghezza del tratto preso in considerazione, integrando anche nell'energia si ottiene anche il fattore $P\left( E \right)$ che è l'efficienza del fotorivelatore.

\subsection[]{Calcolare il numero di fotoni osservati da un fotorivelatore sensibile con efficienza del 30\% a luce fra 300nm e 600nm, al passaggio di un elettrone nei due casi seguenti: \\
	i) n=1.005 (gas), $\beta$=0.999, lunghezza=1m; \\
	ii) n=1.5 (solido trasparente), $\beta$=0.99, lunghezza=1cm.
}\label{sec:4.b.3}
Il conto da fare è l'integrale della espressione nella domanda precedente, il coefficiente di diffrazione è in questo caso indipentente dall'energia, mentre gli estremi di integrazione sono:
\begin{align*}
	&E_1= \hbar \omega_1= \hbar \frac{2\pi c}{\lambda_1} 	&E_2=\hbar \frac{2\pi c}{\lambda_2}
.\end{align*}
Ogni termine può uscire dall'integrale, ne risulta che:
\[
	N_{\gamma}=2\pi \alpha L \left( \frac{1}{\lambda_2}-\frac{1}{\lambda_1} \right) \left( 1- \frac{1}{\beta^2 n^2} \right) P_{\text{det}}
.\] 
Nel primo caso si rilevano 189 fotoni, nel secondo invece se ne rilevano 12533.

\subsection[]{Calcolare l’angolo di emissione della radiazione Cherenkov in funzione dell’impulso (e della massa) della particella e dell’indice di rifrazione.
}
\label{sec:4.b.4}
L'angolo Cherenkov vale \[
	\sin^2\theta_{c}= \frac{1}{n \beta}
.\] 
Sfruttiamo la definizione di $\gamma$ e del'impulso:
\begin{align*}
	&\gamma^2 = \frac{1}{1-\beta^2}		&p = m\gamma v
.\end{align*}
Manipolando le due espressioni si arriva all'equazione:
\[
	\frac{1}{1-\beta^2} = \frac{p^2}{m^2c^2\beta^2}  \implies \beta = \sqrt{\frac{p^2}{p^2+m^2c^2}} 
.\] 
Abbiamo quindi $\beta$ in funzione delle quantità richieste, basta adesso inserire nella prima equazione per concludere:
\[
	\sin^2\theta_{c}=\frac{1}{n} \sqrt{1+\left( \frac{mc}{p} \right)^2} 
.\] 

\subsection[]{Descrivere qualitativamente il principio di funzionamento dei rivelatori Cherenkov: \\
	i) a soglia\\
	ii) RICH\\ 
	iii) DIRC
}
\label{sec:4.b.5}
\paragraph{Rivelatori a soglia.}%
Questi rivelatori sono in grado di rilevare il numero di fotoni emessi ma non l'angolo di emissione.
\begin{figure}[H]
    \centering
    \incfig{rivelatori-checklist-a-soglia}
    \caption{Rivelatori Cherenkov a soglia}
    \label{fig:rivelatori-checklist-a-soglia}
\end{figure}
Il mezzo utilizzato nella camera è un gas ad alta pressione con $n \sim 1.001 - 1.01$, particolarmente adatto a particelle ultrarelativistiche. Questo rilevatore fornisce quindi risposta binaria a seconda che la particella superi o no la velocità della luce nel mezzo:
\begin{itemize}
	\item 0) Se $\beta < 1 /n$
	\item 1) se $\beta > 1 /n$
\end{itemize}
\paragraph{Rivelatore RICH}%
A differenza dei primi questi rivelatori sono in grado di misurare l'angolo di emissione $\theta_{c}$, l'indice di rifrazione del mezzo radiatore è tale da avere solitamente $n < 1.1$, la luce subisce quindi una rifrazione che gli permette comunque di uscire nella stessa direzione della particella (non superando quindi l'angolo limite di riflessione). Tale luce viaggia su superfici coniche e viene osservata su un piano contenente i fotorilevatori, quello che si osserva sono degli anelli: anelli Cherenkov. 
\begin{figure}[H]
	\centering
	\includegraphics[width=1\textwidth]{immagini/RICH.png}
	\caption{Rilevatore Cherenkov RICH.}
	\label{fig:immagini-RICH-png}
\end{figure}

\paragraph{Rilevatori DIRCH}%
Questo tipo di rivelatore funziona come il Rich: misura l'angolo $\theta_{c}$. L'indice di rifrazione del materiale è solitamente $n\sim 1.4$, per questo la luce viene riflessa internamente fino a raggiungere i lati del radiatore. Per evitare che non esca neanche al lato in cui vorremmo rilevare la radiazione si inserisce un distillato con $n \sim 1.33$ tra radiatore e fotorilevatori.
\begin{figure}[ht]
    \centering
    \incfig{rilevatori-cherenkov-dirch}
    \caption{Rilevatori Cherenkov DIRCH.}
    \label{fig:rilevatori-cherenkov-dirch}
\end{figure}

\subsection[]{Partendo dalla espressione\[
	\frac{\mbox{d} I_{\omega}}{\mbox{d} \Omega} = \frac{q^2}{4\pi^2 c}
	\left|\int\frac{\hat{n}\wedge\left[\left(\hat{n}-\bs{\beta}\right)\wedge\dot{\bs{\beta}}\right]}{\left(1-\hat{n}\cdot\bs{\beta}\right)^2}
	e^{i\omega\left(t'-\frac{\bs{r}'\cdot\hat{n}}{c}\right)}dt'\right|^2.\] 
dimostrare che l’energia persa per unità di frequenza nel caso non relativistico e’ approssimabile con \[
	I_{\omega}=
	\begin{cases}
		\frac{2q^2}{3\pi^2 c}\left| \Delta \bs{\beta} \right|^2  &\text{per } \omega< 1 /\tau\\  
		0 & \text{per } \omega > 1 /\tau
	\end{cases}.\] }
\label{sec:4.b.6}
Nel caso non relativistico la formula di partenza si scrive come :
\[
	\frac{\mbox{d} I_{\omega}}{\mbox{d} \Omega} \approx \frac{q^2}{4\pi^2 c}
	\left|\int\hat{n}\wedge\left[\hat{n}\wedge\dot{\bs{\beta}}\right]
	e^{i\omega t'}dt'\right|^2
.\] 
Se applichiamo questa formula ad una carica non relativistica che si avvicina ad un centro scatterante (nucleo) allora possiamo stimare il tempo di interazione \[
	\tau \approx \frac{b}{v}
.\] 
Con $v$ velocità della particella e $b$ parametro di impatto. Da questo tempo si ricava l'ampiezza di spettro di frequenza ottenibile in trasformata: \[
	\Delta \omega \approx \frac{1}{\tau}
.\] 
Sappiamo quindi che per $\omega\gg \frac{1}{\tau}$ la funzione $\frac{\mbox{d} I_{\omega}}{\mbox{d} \Omega}$ tenderà a zero (lemma di Rieman-Lebesgue), mentre per $\omega\ll \frac{1}{\tau}$ sarà l'espressione scritta sopra integrata nel dominio temporale della intera interazione:\[
	\frac{\mbox{d} I_{\omega}}{\mbox{d} \Omega} \approx 
	\begin{cases}
		\frac{q^2}{4\pi^2 c}
		\left|\int_{- \tau /2}^{\tau /2}\hat{n}\wedge\left[\hat{n}\wedge\dot{\bs{\beta}}\right]
		e^{i\omega t'}dt'\right|^2 		& \text{con } \omega \ll \frac{1}{\tau}\\
		0					& \text{con } \omega \gg \frac{1}{\tau}
	\end{cases}
.\] 
Il modello sussiste quindi nell'approssimare questa funzione con uno scalino "centrato" in $\frac{1}{\tau}$ , sostituiamo l'integrale già svolto:
\[
	\frac{\mbox{d} I_{\omega}}{\mbox{d} \Omega} \approx 
	\begin{cases}
		\frac{q^2}{4\pi^2 c^3} \left| \Delta v \right|^2\sin^2\theta			& \text{con } \omega < \frac{1}{\tau}\\
		0										& \text{con } \omega > \frac{1}{\tau}
	\end{cases}
.\] 
Dove è stato preso un sistema di riferimento in cui $\theta$ è l'angolo tra $\bs{v}$ ed il versore di osservazione del campo $\hat{n}$. \\
Integrando sull'angolo solido si ottiene in fine:
\[
	I_{\omega}=\frac{8}{3}\pi\frac{q^2}{4\pi^2c^3}\left|\Delta v\right|^2\left(1-\theta\left(\omega-\frac{1}{\tau}\right)\right)\theta\left(\omega\right)
.\] 
Che semplificando è l'espressione cercata (in cui si è fatto uso di distribuzioni).

\subsection[]{Partendo dalla espressione \[ 
	\frac{\mbox{d} I_{\omega}}{\mbox{d} \Omega} = \frac{q^2}{4\pi^2 c}
	\left|\int\frac{\hat{n}\wedge\left[\left(\hat{n}-\bs{\beta}\right)\wedge\dot{\bs{\beta}}\right]}{\left(1-\hat{n}\cdot\bs{\beta}\right)^2}
	e^{i\omega\left(t'-\frac{\bs{r}'\cdot\hat{n}}{c}\right)}dt'\right|^2.\] 
dimostrare che l’energia persa per unità di frequenza nel caso non relativistico ad un dato parametro di impatto b è approssimabile con
\[
	I_{\omega}=
	\begin{cases}
		\frac{8z^2Z^2\alpha \hbar c^2}{3\pi}\left( \frac{m_e}{M} \right) ^2 \frac{r^2_e}{V^2}\frac{1}{b ^2} & \text{con } \omega< \frac{V}{b}\\
		0 & \text{con } \omega > \frac{V}{b}
	\end{cases}
.\] 
}
\label{sec:4.b.7}
Si parte in realtà dalla espressione ottenuta nella domanda precedente (all'esame se beccate questa avete già fatto ambo):
\[
	I_{\omega}=
	\begin{cases}
		\frac{2q^2}{3\pi^2 c}\left| \Delta \bs{\beta} \right|  &\text{per } \omega< 1 /\tau\\  
		0 & \text{per } \omega > 1 /\tau
	\end{cases}.
\]
Il prossimo passo è calcolare $\left| \Delta \bs{v} \right| $ in funzione del parametro di impatto $b$, per farlo possiamo approssimare la traiettoria della particella scatterata ad una retta (approssimazione di angoli di scattering piccoli), sempre rispettata sperimentalmente in caso di Bremsstralhung. In questo modo la variazione di quantità di moto è solo perpendicolare alla direzione di incidenza della particella ed il suo modulo è dato da:
\[
	\left| \Delta \bs{p} \right| \approx \left| \int_{\infty}^{\infty} ze \bs{E}_{\bot} dt \right| 
.\] 
Nella approssimazione in cui la velocità della particella risulti invariata:
\[
	\left| \Delta \bs{p} \right| \approx \frac{ze}{v} \left| \int \bs{E}_{\bot} dx\right|= 
	\frac{ze}{2\pi b v} \left| \int 2\pi \bs{E}_{\bot}dx \right|= \frac{ze}{2\pi b v } 4\pi Ze
.\] 
Dove nell'ultimo passaggio si è applicato Gauss in CGS.\\
Quindi $\left| \Delta \bs{v} \right| = \left| \Delta \bs{p} \right| /M$, inserendo questo nella espressione di $I_{\omega}$ :
\[
	I_{\omega}=
	\begin{cases}
		\frac{2q^2}{3\pi^2 c^3}\frac{z^2q^2}{4 \pi^2b^2v ^2M^2}16 \pi^2 Z^2 e^2  &\text{per } \omega< 1 /\tau\\  
		0 & \text{per } \omega > 1 /\tau
	\end{cases}
.\] 
Inserendo il raggio classico dell'elettrone $r_e = q^2 / m_e c^2$ e la costante di struttura fine $\alpha=e^2 / \hbar c$ semplificando si ottiene l'espressione cercata.

\subsection[]{Dimostrare, a partire dalla espressione
\[
	\chi_{\omega} = \frac{16 z^4 Z^2 \alpha \hbar c^2}{3} \left( \frac{m_e}{M} \right)^2 \frac{r_e^2}{V^2}\ln\left[ \frac{MV^2}{\hbar \omega} \right] 
.\] 
non relativistica, che nel caso relativistico la sezione d’urto di irraggiamento per elettroni è approssimabile, in un modello, con 
\[
	\chi_{\omega} = \frac{16}{3}z^4 Z^2\alpha\hbar r_e^2\ln\left( \frac{195.5}{Z^{1 /3}} \right) 
.\] 
}
\label{sec:4.b.8}
Questo, e molto altro, è stato fatto nella \hyperref[sec:4.a.15]{Domanda 4.a.15}.

\subsection[]{Dimostrare che nel caso relativistico la perdita di energia per irraggiamento è approssimabile con 
\[
	\frac{\mbox{d} E_{\text{irr}}}{\mbox{d} x} = n_{\text{nuclei}} \frac{16}{3}z^2Z^2\alpha\left( \frac{m_e}{M} \right) ^2r_e^2\ln\left[ \frac{192}{Z^{1 /3}}\frac{M}{m_e} \right] 
.\] 
e quindi l'espressione approssimata per la lunghezza di radiazione per elettroni
\[
	X_0= \left( n_{\text{nuclei}} \frac{16}{3}\rho \frac{N_{A}}{A}  Z^2\alpha ^2r_e^2\ln\left[ \frac{192}{Z^{1 /3}}\right] \right)^{-1} 
.\] 
}
\label{sec:4.b.9}
Anche questa è stata ricavata nella \hyperref[sec:4.a.15]{Domanda 4.a.15}, si aggiunge solo che la lunghezza di radiazione, ottenuta la formula di cui sopra si deduce dal fatto che:
\[
	\frac{\mbox{d} E_{\text{irr}}}{\mbox{d} x} = \frac{E}{X_0}
.\] 
\paragraph{Nota}%
 È per l'energia persa che, nella equazione precedente, va il segno negativo al secondo termine.

\subsection[]{Valutare la lunghezza di radiazione del Piombo e del Silicio con il modello spiegato a lezione ed effettuare un confronto con i valori sperimentali reperibili su internet.
}
\label{sec:4.b.10}
Si applica la formula di Tsai semplificata:
\[
	\rho X_0 = \frac{1}{\frac{16}{3} \frac{N_{A}}{M_A}Z^2\alpha r_e^2 \ln \left(\frac{184}{Z^{1 /3}}\right)}
.\] 
\paragraph{Piombo: $Z=82$, $A=207$} Modello: $\rho X_0= 4.4$ g/cm$^2$ ; Dati: $\rho X_0= 6.37$ g/cm$^2$ 
\paragraph{Silicio: $Z=14$, $A=28$} Modello: $\rho X_0=17.5$ g/cm$^3$ ; Dati: $\rho X_0= 21.8$ g/cm$^2$

\subsection[]{Calcolare l'energia irraggiata da un elettrone di 60MeV che attraversi 5.6mm di Pb e calcolare il numero medio di fotoni emessi di energia fra 1eV e 1MeV
}
\label{sec:4.b.11}
La distanza percorsa è proprio la lunghezza di radiazione, quindi l'energia irraggiata dall'elettrone è: 
\[
	E_{\text{irr}}= E_{\text{in}}-E_{\text{lost}}= E\left( 1-\frac{1}{e} \right) \approx 38 \text{ MeV}
.\] 
Per il numero di fotoni in quel range di energia possiamo sfruttare le due formule:
\begin{align*}
	&\frac{\mbox{d} N_{\gamma}}{\mbox{d} \omega} = \frac{1}{\hbar \omega}\frac{\mbox{d} E_{\text{irr}}}{\mbox{d} \omega} = n_s \frac{\chi_{\omega}}{\hbar \omega}=
	\frac{1}{\omega X_0}\\
	&n \chi_{\omega}\frac{E}{\hbar} = n \frac{16}{3}Z^2\alpha r_e^2\left( \ln \frac{192}{Z^{1 /3}}  \right)E= \frac{E}{X_0}
.\end{align*}
scrivendo $n_s = n\cdot dx$ si ha:
 \[
	\frac{\mbox{d} N_{\gamma}}{\mbox{d}x \text{d}\omega}=n \frac{\chi_{\omega}}{\hbar \omega}=\frac{1}{\omega X_0} 
.\] 
Partiamo dall'integrare in $\omega$, dobbiamo dapprima collegarla prima all'energia dei fotoni emessi tramite:
\[
	\omega = \frac{E}{\hbar}
.\] 
Quindi:
\[
	\frac{\mbox{d} N_{\gamma}}{\mbox{d} x} = \frac{1}{X_0} \int_{E_1 /\hbar}^{E_2/\hbar} \frac{1}{E} dE   
.\] 
Dobbiamo prestare attenzione al fatto che qui E è dello stesso ordine di $X_0$, se cercassimo il numero di fotoni totali allora dovremmo tener di conto che l'energia sta diminuendo. Nel nostro caso invece i fotoni emessi a queste energie non risentono dell'effetto dell'ingresso in profondità perchè la particella ha sempre sufficiente energia per emettere tali frequnze in tutto il percorso (fantasticando), quindi:
\[
	N_{\gamma}= \frac{L}{X_0 } \ln\left( \frac{E_2}{E_1} \right) \approx 15
.\] 



\subsection[]{Calcolare il numero medio di fotoni emessi di energia fra 10 e 100 MeV per un elettrone di 1 GeV che attraversi 300 $\mu$m di Silicio o 1 mm di Piombo. Calcolare poi la probabilità che uno di questi fotoni effettui una interazione prima di uscire dal materiale.
}
\label{sec:4.b.12}
Si applica la formula dell'esercizio precedente ricordando che, per il silicio $X_0 = 9.36$ cm.
\begin{itemize}
	\item Si: $N_{\gamma}= 0.5$
	\item Pb: $N_{\gamma}\approx 2.9$
\end{itemize}
La probabilità che un fotone effettui interazione prima di uscire dal materiale può essere calcolata utilizzando l'equivalenza tra i processi fondamentali di formazione di coppie $e^+, e^-$ ed i processi come la Bremsstralhung in cui invece vengono rilasciati fotoni.
Quindi la lunghezza di irraggiamento $X_0$ può esser associata anche alla distanza in volo media di un fotone prima di produrre una coppia, da cui:
 \[
	P_{\text{int}}\approx \frac{\Delta l}{X_0}\cdot N_{\gamma} =
	\begin{cases}
		1.6 \cdot 10^{-3} & \text{Per il Silicio}\\
		50\% &\text{Per il Piombo}
	\end{cases}
.\] 
Sulla precisione di questa stima (ed il calcolo che invece andrebbe effettuato) vedi la domanda successiva.

\subsection[]{Utilizzando le tabelle che forniscono le sezioni d'urto di fotoni su atomi, calcolare la probabilità che un fotone da 10 MeV produca una coppia $e^+e^-$ in uno spessore di Piombo pari ad una lunghezza di radiazione.
}\label{sec:4.b.13}
Per fotoni incidenti con tale energia la sezione d'urto è $\sim 40$ b, quindi la probabilità di interazione è:
\[
	P = n \sigma \Delta l = \frac{\rho N_A}{M_A}\sigma \Delta l \approx \frac{\Delta l}{0.76 \text{ cm}} \approx 75 \%
.\] 
Quindi nella domanda precedente la probabilità ha un errore di stima di circa il $25 \%$.

\subsection[]{Calcolare l’energia minima e l'energia massima trasferibile in un singola collisione da una particella carica, di massa molto maggiore di quella dell'elettrone, in moto veloce attraverso la materia ad un singolo elettrone atomico.
}
\label{sec:4.b.14}
Nell'ipotesi di piccoli angoli di scattering la variazione di impulso dell'elettrone è interamente ortogonale alla direzione di incidenza del proiettile, quindi:
\begin{align*}
	\Delta \bs{p}&= \int_{-\infty}^{\infty} \bs{F}_{\bot} dt =\\
	&=  \int_{-\infty}^{\infty}-e \bs{E}_{\bot}dt =\\
	&= \frac{1}{v}\int_{-\infty}^{\infty}-e \bs{E}dx=\\
	&= \frac{-e}{2\pi b v} \frac{Ze}{\epsilon_0} \hat{n} 
.\end{align*}
I passaggi sono analoghi a quelli della \hyperref[sec:4.b.7]{Domanda 4.b.7}, $\hat{n}$ è il versore ortogonale a $v$.\\
L'energia trasferita ad un elettrone in funzione del parametro di impatto la si può calcolare come:
\[
	T\left( b \right)= \frac{\left| \Delta \bs{p} \right|^2 }{2m}= \frac{Z^2e^4}{4\pi^2\epsilon_0 v^2 \bs{r}}\frac{1}{2m}= 
	2z^2 \frac{m_e c^2}{\beta^2} \frac{r_e^2}{b ^2}
.\] 
Se si segue la logica del modello di Bohr per l'energia minima si ha che la durata dell'urto coulombiano è $\tau \sim \frac{b}{\gamma V}$, la richiesta di Bohr è che questo tempo sia minore del periodo di rotazione dell'elettrone atomico: $\tau< T_{e} \sim \frac{1}{\omega_{e}}$.\\
Da questa ultima relazione se ne ricava un $b_{\text{max}}$ che sostituito nella energia ci da:
\[
	T_{\text{min}}= 2Z^2 \frac{m_e r_e^2 \omega_e^2}{\beta^4 \gamma^4}
.\] 
Per l'energia massima trasferita invece si ha che $T_{\text{max}}=2m_ec^2\beta^2\gamma^2$ (conservazione di impulso ed energia con $M\gg m_e$)


\subsection[]{Indicare le ipotesi effettuate e dimostrare la seguente espressione approssimata
\[
	\frac{1}{\rho}\frac{\mbox{d} E_{\text{coll}}}{\mbox{d} x} = z^2 \frac{Z}{A\left( g \right) } 4 \pi \frac{m_e c^2}{\beta^2}N_A r_e^2\left( \ln \frac{c^2\beta^3\gamma^2}{z \omega_e r_e} \right) 
.\] 
 per la perdita di energia per collisioni (formula di Bohr)
}
\label{sec:4.b.15}
Consideriamo un tratto $\Delta x$ di materia e cerchiamo di calcolare l'energia media trasmessa in collisioni elettroniche in questo tratto, ci serve la concentrazione degli elettroni:
\[
	n_e =\rho \frac{Z}{M_A}N_A = \rho \frac{Z}{A\cdot 1\text{g}}N_A
.\] 
Integriamo sul volume cilindrico attorno gli scattering al variare di $b$ :
\[
	\Delta T_{\text{coll}} =T\left( b \right) 2 \pi b n_e \Delta x db
.\] 
Perciò in un tratto infinitesimo e mediando sui possibili parametri di impatto troviamo l'espressione della Stopping Power:
\[
	\left< \frac{\mbox{d} E_{\text{coll}}}{\mbox{d} x} \right> = n_e \int_{b_{\text{min}}}^{b_{\text{MAX}}} 2 \pi T\left( b \right) b db = 
	4 \pi n_e z^2 \frac{m_e c^2}{\beta^2} r_e^2 \ln\left( \frac{b_{\text{MAX}}}{b_{\text{min}}} \right) 
.\] 
Nel caso di Bhor le approssimazioni sono state spiegate nella domanda precedente, manca di ricavare il $b_{min}$ a partire dalla espressione (sopra) di $T_{\text{MAX}}$ :
\[
	b_{\text{min}}= \sqrt{2z^2 \frac{m_e c^2}{\beta^2}r_e^2 \ln\left( \frac{\beta^3\gamma^2c}{zr_e\omega_e} \right) } =
	\frac{zr_e}{\beta^2 \gamma}
.\] 	
Sostituendo i due parametri di impatto limite si ha la soluzione.

\subsection[]{Utilizzando la modellizzazione $I = ( 16 \text{ eV} )\cdot  Z^{0.9}$ , valutare il valore minimo dell’energia persa per collisioni, in MeV/g/cm$^2$, per i seguenti materiali:\\
i) Piombo \\
ii) Silicio \\
iii) aria a TPN.
}
\label{sec:4.b.16}
Dobbiamo applicare la formula di Bethe-Block nei pressi del minimo di energia persa per collisioni: $\gamma\beta \approx 3.5$.\\
Dalla correlazione tra $\gamma$ e $\beta$ questo significa che $\beta\approx 0.96$, quindi lo approssimeremo all'unità commettendo un errore del $4 \%$.\\
Inoltre si trascureranno anche i termini correttivi della formula e si lavorerà nella approssimazione $\gamma\ll M/m_e$:
 \[
	\left< \frac{1}{\rho} \frac{\mbox{d} E_{\text{coll}}}{\mbox{d} x}  \right> =
	z^2 \frac{Z}{A} \frac{K}{\beta^2} \left( \ln \frac{\left( 2 m_e c^2 \beta^2\gamma^2 \right)}{I} - \beta^2 - \frac{\delta}{2}  \right) 
.\] 
Dove $K$ vale:
\[
	K = N_A \frac{4\pi m_e c^2 r_e^2}{1 \text{ g}}= 0.307 \frac{\text{MeV}\cdot \text{cm}^2}{\text{g}}
.\] 
Nel nostro caso si considerano gli urti su elettroni $z=1$. Possiamo inoltre trascurare il termine di fermi, infatti questo è rilevante per alti valori di $\gamma\beta$ e non nei pressi del minimo. Quindi ci si riduce a :
\[
	\left< \frac{1}{\rho} \frac{\mbox{d} E_{\text{coll}}}{\mbox{d} x}  \right> =
	0.307 \frac{\text{MeV}\cdot \text{cm}^2}{\text{g}} \frac{Z}{A} \left( \ln\left( 7.5 \cdot 10^{5} \frac{1}{Z^{0.9}} \right) -1 \right) 
.\] 
\begin{table}[H]
	\centering
	\begin{tabular}{c|ccc}
				&	Z		& 		A		&	$E_{coll}$ $\frac{\text{MeV}\cdot \text{cm}^2}{\text{g}}$	\\
	\hline
	Piombo			&	82		&		207		&	1.04								\\
	Silicio			&	14		&		28		&	1.56								\\
	Aria a TPN (azoto)	&	7		&		14		&	1.66		
	\end{tabular}
\end{table}

\subsection[]{Nell'approssimazione di piccoli angoli e distribuzione gaussiana di varianza nota, calcolare il valor medio, il valore quadratico medio e la sigma per:\\
i) l'angolo di multiplo scattering rispetto alla direzione iniziale della particella\\
ii) la sua proiezione su un piano che contenga la direzione iniziale della particella.
}
\label{sec:4.b.17}
Assumiamo che la particella si muova inizialmente lungo l'asse $\hat{z}$. Sia $\theta$ l'angolo di multiplo scattering rispetto alla direzione $\hat{z}$ e $\theta_{x}$, $\theta_{y}$ le sue proiezioni sui piani $xz$ e $yz$ rispettivamente.\\
Se si assume che gli ultimi due angoli siano distribuiti gaussianamente per simmetria devono avere anche la stessa deviazione standard e media nulla:
\[
	P\left( \theta_{x} \right) = \frac{1}{\theta_0 \sqrt{2\pi} } e^{- \theta^2_{x} / 2\theta_0^2}
.\] 
Analogamente per $\theta_{y}$. Di questa distribuzione sappiamo che:
\[
	\left<\theta_{x} \right> = \left<\theta_{y} \right> = 0
.\] 
\[
	\left<\theta_{x}^2 \right> = \left< \theta_{y}^2 \right> = \theta_0^2
.\] 
\[
	\sigma_{\theta_{x}} = \sigma_{\theta_{y}} = \theta_0
.\] 
Considerando che geometricamente $\theta^2 = \theta_{x}^2+\theta_{y}^2$ possiamo cercare la distribuzione di $\theta$, la si trova imponendo la normalizzazione sulle due proiezioni:
\begin{align*}
	1 &= \int_{-\infty}^{\infty}d\theta_{x} \int_{-\infty}^{\infty}d\theta_{y} P\left( \theta_{x} \right) P\left( \theta_{y} \right) =\\
	  &=  \int_{-\infty}^{\infty}d\theta_{x} \int_{-\infty}^{\infty}d\theta_{y}  \frac{1}{2\pi \theta_0} \exp\left( - \frac{\theta_{x}^2+\theta_{y}^2}{2\theta_0^2} \right) \\
	  &= \int_{0}^{\infty} \frac{\theta}{\theta_0^2} e^{- \theta^2 /2\theta_0^2} d\theta
.\end{align*}
Abbiamo quindi la funzione di distribuzione su $\theta$ :
\[
	P\left( \theta \right) = \frac{\theta}{\theta_0^2} e^{- \theta^2 /2\theta_0^2}
.\] 
Possiamo quindi calcolare i valori richiesti anche per quest'ultima:
\begin{align*}
	& \left<\theta \right> = \int_{0}^{\infty} \frac{\theta^2}{\theta_0^2} e^{- \theta^2 /2\theta_0^2} d\theta = \theta_0 \sqrt{\frac{\pi}{2}}\\
	& \left<\theta^2 \right> = \int_{0}^{\infty}\frac{\theta^3}{\theta_0^2} e^{- \theta^2 /2\theta_0^2} d\theta = 2 \theta_0^2 \\
	& \sigma_\theta = \sqrt{\left<\theta^2 \right>- \left< \theta \right>^2} =\theta_0 \sqrt{ \frac{4-\pi}{2}} 
.\end{align*}
Inoltre il massimo di $P\left( \theta \right)$ è prorpio $\theta_0$

\subsection[]{Dimostrare l'espressione approssimata per l’angolo quadratico medio di multiplo scattering (proiezione su un piano):
\[
	\theta_0 = z \frac{\text{costante}}{P \beta c} \sqrt{\frac{L}{X_0}} 
.\] 
e confrontare il valore della costante ottenuta con la formula contenuta nel PDG.
}\label{sec:4.b.18}
Per effettuare il conto è necessario prima calcolare lo scarto quadratico medio sul singolo urto e successivamente sommare su tutti i centri scatteranti tenendo conto della composizione del materiale.\\
Partiamo dalla sezione d'urto Mott per il singolo urto e approssimiamola nel caso di angoli di scattering piccoli:
\[
	\frac{\mbox{d} \sigma }{\mbox{d} \Omega} = \left( \frac{zZ \alpha \hbar c}{2PV} \right)^2 \frac{1-\beta^2\sin^2\frac{\theta}{2}}{\sin^4 \frac{\theta}{2}}=
	\left( \frac{2zZ \alpha \hbar c}{PV} \right)^2 \frac{1}{\theta^4}
.\] 
Il termine d'innanzi alla frazione con $\theta$ deve avere le dimensioni di una distanza al quadrato, lo definiamo come:
\[
	l=\frac{2zZ \alpha \hbar c}{PV} \implies \frac{\mbox{d} \sigma}{\mbox{d} \Omega} = \frac{l^2}{\theta^4}
.\] 
Si può calcolare lo scarto quadratico medio su un urto conoscendo questa distribuzione approssimata:
\[
	\left< \theta^2_{\text{urto}} \right> = \frac{\int_{\theta_{\text{min}}}^{\theta_{\text{max}}} \theta^2\frac{\mbox{d} \sigma}{\mbox{d} \Omega} d\Omega }
	{\int_{\theta_{\text{min}}}^{\theta_{\text{max}}} \frac{\mbox{d} \sigma}{\mbox{d} \Omega} d\Omega}=
	\frac{\int_{\theta_{\text{min}}}^{\theta_{\text{max}}} \frac{l^2}{\theta^2} 2\pi \sin\theta d\theta}{\sigma_{\text{ms}}} \approx
	\frac{2\pi l^2}{\sigma_{\text{ms}}} \ln \frac{\theta_{\text{max}}}{\theta_{\text{min}}}
.\] 
Cerchiamo di riscrivere quest'ultima utilizzando il parametro di impatto, è noto che, per questo scattering:
\[
	\tan \frac{\theta}{2} = \frac{zZ \alpha \hbar c}{PV} \frac{1}{b} \implies \theta \approx \frac{l}{b}
.\]
Il vantaggio di usare $b$ è che sappiamo bene $b_{\text{max}}$ e $b_{\text{min}}$ :
\begin{align*}
	&b_{\text{min}}= R_{\text{nucleo}} = r_0 A^{1 /3}\\
	&b_{\text{max}}=R_{\text{atomo}}= 1.4 a_0 Z^{- 1/3} 
.\end{align*}
Si ha allora che:
\[
	\ln \frac{\theta_{\text{max}}}{\theta_{\text{min}}}= \ln \frac{b_{\text{max}}}{b_{\text{min}}}=
	\ln \frac{1.4\cdot 0.53 \cdot 10^{5}\text{ fm} }{1.4\text{ fm}\left( AZ \right)^{1 /3} } \approx 
	\ln \frac{5.3 \cdot 10^{4} }{\sqrt[3]{2}Z^{2 /3} }= 
	2\ln \frac{205}{Z^{1 /3}}
.\] 
Dove si è sfruttata anche l'approssimazione $A \approx 2Z$.
Quindi \[
	\left<\theta_{\text{urto}} \right> \approx \frac{4\pi l^2}{\sigma_{\text{ms}}} \ln \frac{205}{Z^{1/3}}
.\] 
L'angolo quadratico medio totale si ottiene quindi sommando in quadratura su tutti gli urti:
\[
	\left< \theta^2 \right> = N_{\text{urti}} \left<\theta^2_{\text{urto}} \right> = 4\pi nL l^2 \ln \frac{205}{Z^{1 /3}} 
.\] 
Ricordando adesso che la lunghezza di radiazione dalla formula di Tsai approssimata è:
\[
	X_0 \approx \frac{1}{4Z^2 n \alpha \hbar r_e^2 \ln \frac{184}{Z^{1 /3}}}
.\] 
si può concludere che :
\begin{align*}
	\sqrt{\left<\theta^2 \right>} &= \sqrt{\frac{X_0}{X_0} 4\pi nL l^2 \ln \frac{205}{Z^{1 /3}}} =\\
	&=  \sqrt{\frac{L}{X_0}} \sqrt{ \frac{4\pi n \ln \frac{205}{Z^{1 /3}} }{4Z^2 n \alpha \hbar r_e^2 \ln \frac{184}{Z^{1 /3}}} } \approx \\ 
	&\approx l \sqrt{\frac{L}{X_0}} \sqrt{\frac{\pi}{Z^2\alpha r_e^2}}=\\
	&= \frac{2zZ \alpha \hbar c}{PV}\frac{1}{r_e}\sqrt{\frac{L}{X_0}} \sqrt{\frac{\pi}{Z^2 \alpha}}=\\
	&= \frac{\sqrt{2} z m_e c^2}{PV} \sqrt{\frac{L}{X_0}} \sqrt{\frac{2\pi}{\alpha}} \approx \\
	&\approx z \sqrt{2} \frac{14.6 \text{ MeV}}{PV}\sqrt{\frac{L}{X_0}} 
.\end{align*}
Da confrontare con la costante del PDG che vale invece $13.6$ MeV.\\
Per ottenere $\theta_0$ basta dividere la precedente per la radice di 2.

\subsection[]{Utilizzando le opportune tabelle, anche reperibili su web, per le seguenti particelle: \\
i) elettrone 3.5MeV \\
ii) elettrone 100MeV \\
iii) pione di 1GeV \\
iv) muone di 45 GeV \\
v) protone da 7 TeV\\
Che attraversino:\\
a) 2 mmPb\\
b) 2 mm scintillatore\\
iii) 0.3 mm Silicio\\
iv) 1 m Aria\\
Indicare se siano rilevanti e, in caso affermativo, calcolare le seguenti quantità: \\
a) energia persa per irraggiamento \\
b) l’energia persa per collisioni \\
c) probabilita' di interazione forte con i nuclei\\
d) angolo quadratico medio di multiplo scattering.
}
\label{sec:4.b.19}
Alcune delle cose che ci serviranno sono nella seguente tabella:

\begin{figure}[H]
	\centering
	\includegraphics[width=0.8\textwidth]{immagini/materials.png}
	\caption{Tabella con proprietà di alcuni materiali.}
	\label{fig:}
\end{figure}

Dobbiamo ripassare, quantità per quantità, tutte le cose che ci serviranno sulle particelle:

\paragraph{Energia persa per irraggiamento}%
Questa è trascurabile per tutti tranne che per gli elettroni. Per questi ultimi dobbiamo distinguere tra elettroni aventi energia maggiore o minore dell'energia critica:
\[
	E_{c} \approx 98 \frac{\text{MeV}}{Z^{1 /3}}
.\] 
Se gli elettroni hanno una energia maggiore di quest'ultima allora si applica 
\[
	E_{\text{irr}}= E_0 e^{- x /X_0}
.\]
Altrimenti potremmo dare una stima dell'energia irraggiata mediante la formula per uno  schermaggio incompleto come spiegato sul Jackson, ma la risposta sarà comunque che questa energia è trascurabile rispetto a quella persa per collisioni.

\paragraph{Energia persa per collisioni}%
Se l'energia persa per irraggiamento è trascurabile è pratico vedere se l'energia con cui il proiettile incide è sufficiente ad attraversare il materiale.\\
Per far questo è necessario utilizzare \href{https://www.nist.gov/pml/stopping-power-range-tables-electrons-protons-and-helium-ions}{i dati ed i grafici} sullo stopping range dal PDG, una volta messi i parametri ed ottenuta la curva di Range si effettuano i seguenti passaggi:
\begin{itemize}
	\item Determinare il Range in ingresso $R\left( E_{\text{in}}\right)$ graficamente.
	\item Calcolare il Range di uscita come $R\left( E_{\text{out}} \right)= R\left( E_{\text{in}} \right) - \Delta x$.
	\item Determinare $E_{\text{out}}$ sulle $x$ in corrispondenza di $R\left( E_{\text{out}} \right)$,
\end{itemize}
Se l'energia trovata risulta graficamente prossima allo zero allora l'energia persa per collisioni è tutta l'energia incidente.\\
Se si ha invece una energia persa per irraggiamento non trascurabile rispetto all'energia in ingresso allora si applica la formula di Tsai approssimata come nella \hyperref[sec:4.b.16]{Domanda 4.b.16}.
\paragraph{Probabilità di interazione forte}%
Questa interazione avviene soltanto per adroni, quindi nel nostro caso il $\pi^{-}$ ed il $p$. Per queste particelle si calcola la sezione d'urto nucleare come l'area della circonferenza avente raggio la somma del raggio del proiettile e del raggio del nucleo bersagio:
\[
	\sigma=\pi\left( r_{p}+R \right)^2 = \pi\left( 1.25 + 1.25\cdot A^{1/3} + 2 \right)^2 \text{ fm}^2 
.\] 
successivamente il conto da fare è:
\[
	P_{\text{forte}}= \sigma \frac{\rho N_{A}}{A}x
.\] 
\paragraph{Angolo quadratico medio di multiplo scattering}%
Se la particella riesce ad uscire il conto da effettuare è quello della \hyperref[sec:4.b.18]{Domanda 4.b.18}

\subsection[]{Per un muone che attraversi, incidendo perpendicolarmente, una lastra di Ferro di 5cm di spessore in cui e' presente un campo magnetico di intensita' nota, calcolare il valore numerico del rapporto fra la deflessione angolare dovuta al campo magnetico e la dispersione quadratica media dovuta al multiplo scattering. Come sara’ la funzione di distribuzione dell’angolo in uscita? Quale e’ la dipendenza dall’energia del muone incidente?
}
\label{sec:4.b.20}
Ipotizziamo che l'energia del muone sia tale da garantire che oltrepasserà la lastra (possiamo tornarci dopo) e che la frazione di energia persa nella lastra sia molto inferiore dell'energia iniziale in modo da considerare la curvatura magnetica uniforme.\\
Calcoliamo per prima cosa il raggio di curvatura dovuto al solo campo magnetico:
\[
	R= \frac{mv}{eB}= \frac{P}{e B} = \frac{3}{10} \frac{P  \text{ [GeV/c]}}{B \text{ [T]}}
.\] 
Dove si è utilizzato l'esoterica conversione della carica elementare in unità gaussiane. Con un pò di geometria si calcola il seno dell'angolo di deflessione magnetic:
\[
	\sin\theta_{\text{m}}= \frac{x}{R} = \frac{50}{3} \cdot 10^{-3}\text{ [m]}  \frac{B \text{ [T]}}{P  \text{ [GeV/c]}}
.\] 
Approssimando adesso angoli piccoli (è necessario che $ \frac{P \text{ [GeV/c]}}{B \text{ [T]}}>\sim 1$ ) si ottiene:
\[
	\theta_{\text{m}}= \frac{x}{R} \approx 15 \text{ mrad}  \frac{\text{ [m] } B \text{ [T]}}{P  \text{ [GeV/c]}}
.\] 
Per l'angolo quadratico medio di multiplo scattering si ha invece:
\[
	\theta_0= \frac{13.6 \text{ [MeV]}}{PV} \sqrt{ \frac{5 \text{ [mm]}}{X_0}} 
.\] 
Nel ferro si ha $X_0 = 1.76$ cm, inoltre possiamo assumere $\beta \sim 1$ per la nostra stima, quindi:
\[
	\theta_0 \approx \frac{13.6 \cdot 10^{-3} \text{ [eV]}}{P \text{ [GeV/c] [m/s]}} \sqrt{ \frac{5 \cdot 10^{-3}\text{ [m]} }{1.76 \cdot 10^{-2}\text{ [m]}}}
	\approx 7.24 \text{ mrad} \cdot \frac{\text{[eV]}}{P\text{ [GeV/c] [m/s]} }
.\] 
Quindi il rapporto cercato (sempre per impulsi sufficienti a garantire l'uscita del muone e angoli di deflessione piccoli) è indipentente dall'impulso:
\[
	\frac{\theta_{m}}{\theta_0 \sqrt{2}} \approx 0.3\cdot B \text{ [T]}
.\]
La distribuzione angolare sarà in tal caso una Moliere centrata in $\theta_{m}$.

\subsection[]{Cercando i dati nelle apposite figure o tabelle si calcolino il valore (o i limiti) dell'energia degli elettroni emessi nello stato finale della reazione $\gamma$+C per energie del fotone incidente pari a: 1keV, 10keV, 100keV, 1MeV, 10MeV.
}
\label{sec:4.b.21}
La sezione d'urto di un fotone sull'atomo di carbonio segue questi andamenti:
\begin{figure}[H]
	\centering
	\includegraphics[width=0.6\textwidth]{immagini/cross-section-photon.png}
	\caption{Sezione d'urto di fotone su carbonio.}
	\label{fig:immagini-cross-section-photon-png}
\end{figure}
Ad 1 keV il processo dominante è il fotoelettrico. Per questo processo l'energia cinetica dell'elettrone prodotto è la stessa l'energia del fotone incidente (trascurando l'energia di legame dell'ordine di 1-10 eV). \\
A 10 keV domina ancora il fotoelettrico ma iniziano ad essere importanti anche il Rayleigh ($\sigma_{coh}$ in figura) ed il compton ($\sigma_{incoh}$ in figura).
Nel primo (che è uno scattering elastico) l'elettrone non acquista energia. Nel secondo invece l'energia dell'elettrone è compresa tra gli 0 ed i 9 keV a seconda dell'angolo di scattering:
\[
	E_{phot} = \hbar\omega= \frac{h}{\lambda}c 
.\] 
Il cambio di frequenza del Compton è noto e vale 
\[
	\Delta \lambda = \lambda'\left( 1-\cos\theta \right) 
.\] 
Con $\lambda' = 2.43$pm lunghezza d'onda Compton.\\
La lunghezza d'onda iniziale vale: $\lambda = hc /E = hc /10$keV$\approx 20$ pm. Qunidi al massimo questa può diventare $\lambda_{\text{fin}} = \lambda + 2\lambda'$ se l'urto è centrale. L'energia minima del fotone dopo l'urto è quindi:
\[
E^{\text{Compton}}_{\text{min}}= \frac{h}{\lambda_{\text{fin}}}c = \frac{\frac{197}{2\pi}\text{MeV}\cdot\text{fm}}{24.86 \text{pm}} \approx 1.3 \text{ keV}
.\] 
Per differenza (conservazione dell'energia) si ottiene che l'elettrone può avere energia massima di $E^{\text{el}}_{\text{max}} \approx 8.7$ keV, l'energia mininima si ha quando il coseno vale 1 e quindi la lunghezza d'onnda del fotone non cambia. In tal caso $E^{\text{el}}_{\text{min}}=0$.\\
A 100 keV ed a 1 MeV il dominante è il Compton mentre a 10 MeV inizia ad essere importante la produzione di coppie.
Per la produzione di coppie trascurando il rinculo del nucleo ci si aspetta che ciascuno dei prodotti abbia energia di circa 5 MeV.

\subsection[]{Ricavare la relazione tra angolo di scattering e cambio di frequenza nell'effetto Compton
}
\label{sec:4.b.22}
Definiamo k e k' i vettori d'onda del fotone prima e dopo l'urto e p, p' il 4-impulso dell'elettrone prima e dopo l'urto. Dalla comservazione del quadrimpulso si deve avere che:
\[
	\hbar\left( k- k \right)= p'-p	
.\] 
Facendo la norma quadra a destra e sinistra e ricordando che l'elettrone è comsiderato inizialmente fermo si ottiene:
\[
	2\hbar\left( 0 + 0 - 2\left( \omega\omega'- kk'\cos\theta\right) \right) = m^2+m^2- 2mE'
.\] 
In cui $\theta$ è l'angolo di scattering. Adesso sfruttiamo la conservazione dell'energia ($E' = \hbar\omega + m - \hbar\omega'$) per ottenere:
\[
	\frac{2\hbar\omega\omega'}{c}\left( 1-\cos\theta \right) = 2m_{e}c^2\hbar\left( \omega-\omega' \right) 
.\] 
Se passiamo alle lunghezze d'onda otteniamo la relazione cercata:
\[
	\Delta \lambda = \lambda_{c}\left( 1-\cos\theta \right) 
.\] 
Dove $\lambda_{c}= \frac{h}{m_{e}c}$ 



\subsection[]{Cercando i dati delle sezioni d’urto totali nelle apposite figure o tabelle (reperibili anche nella compilazione Particle Data Group http://pdg.lbl.gov ) si calcoli la probabilità di interazione di:\\
• un fotone da 100 eV che incida su 1 $\mu$m di grafite\\
• un fotone da 1 MeV che incida su 1 mm di grafite\\
• un fotone da 10 MeV che incida su 1 mm di Piombo\\
• un fotone da 50 KeV che incida su 1 $\mu$m di Piombo\\
• un neutrino da 100 GeV che incida su 1 km di grafite\\
• un protone da 100 GeV che incida su 1 cm di grafite\\
}
\label{sec:4.b.23}
Partiamo dal metodo da adottare: la probabilità cercata è data da
\[
	P = n_{s}\sigma
.\] Dove $n_{s}$ è la densità superficiale di bersagli. Questa quantità può essere espressa in funzione di variabili tipiche del materiale:
\[
	P = \frac{\rho}{A \text{[g]}}N_{A}L\sigma
.\] 
\subsection[]{Esprimere la sezione d'urto Rayleigh in funzione della sezione d'urto differenziale Thomson e del fattore di forma atomico F($\theta$).
}
\label{sec:4.b.24}
La sezione d'urto Rayleigh differenziale vale
\[
	\frac{\mbox{d} \sigma_{el}}{\mbox{d} \Omega}= \left.\frac{\mbox{d} \sigma_{el}}{\mbox{d} \Omega} \right|_{e} \left| Z F\left( \bs{q} \right)  \right|^2 \quad
		\text{ con }
	\quad
	\left.\frac{\mbox{d} \sigma_{el }}{\mbox{d} \Omega} \right|_{e}= 
		\frac{r_e^2\omega^4}{\left( \omega_0^2-\omega^2 \right)^2 + \omega^2\Gamma_{tot}^2\left( \omega \right)} \frac{1+\cos^2\theta}{2}
.\] 

Per ottenere quanto richiesto dal testo è necessario integrare sull'angolo solido e conoscere il fattore di forma dipendente dall'angolo di scattering $\theta$.

\subsection[]{Dimostrare, utilizzando il materiale distribuito, perchè nell’esperimento di Anderson sulla scoperta del positrone alcune tracce positive osservate non
possono essere nessuna delle particelle positive conosciute nel 1932.
}
\label{sec:4.b.25}
Riprendendo la trattazione della \hyperref[sec:4.a.34]{Domanda 4.a.34} l'impulso iniziale misurato da Anderson era $p_{i}= 63$ MeV/c, quello finale invece era $p_{f}=22.5$ MeV/c, se fosse stato un protone (unica particella con quella carica positiva conosciuta allora) sarebbe stato:
\[
	E_{i}= \frac{p^2}{2m_{p}}\approx 2.1 \text{ MeV}
.\] 
Un protone con quella energia ha un range in aria di circa $7$ mm, la traccia della foto di Anderson è lunga almeno $30$ cm, per questo non poteva essere un protone.


\subsection[]{Calcolare la velocità media dei pioni e antiprotoni nell'esperimento di Segrè (quantità di moto 1.19 Gev/c) dopo che hanno attraversato il contatore Cherenkov a quarzo (spessore 2.5", densità relativa 2.2).
}
\label{sec:4.b.26}
Per il calcolo possiamo utilizzare lo spessore relativo del materiale: 
\[
	\Delta x = x \text{ [pollici]}\cdot \rho_{rel}= (2.5 \ \cdot \ 2.54) \text{ [cm]} \ \cdot \ 2.2 \left[\text{g}/\text{cm}^3 \right] = 13.9\left[\text{g}/\text{cm}^2 \right]
\]
L'energia dell'antiprotone entrante è 
\[
	E_{\overline{p}} = \sqrt{ p_{in}^2+m_{p}^2} \approx 1.51 \text{ GeV}
.\] 
Lo stopping power per tale energia e per il diossido di silicio (quarzo) vale $dE /dx = 1.745$ MeV cm$^2$/g, da questo si ricava l'energia finale:
\[
	\Delta E = \frac{\mbox{d} E}{\mbox{d} x} \Delta x \approx 24.3 \text{ [MeV]} \implies E_{f} \approx 1.49 \text{ [GeV]}
.\] 
Quindi tornando indietro con le formule inverse si ricava facilmente che $\beta_{f}= 0.777$. Analogamente si può procedere per i pioni se solo esistessero le dannate tabelle.



 % da aggiustare e fare
\end{document}
