\subsection[ Spessore di un materiale espresso in due modi]{Quale e' la relazione fra lo spessore di un materiale, espresso in cm, e lo spessore espresso in g/cm$^2$ ?
} \label{sec:4.a.1}
La relazione tra $X$ [cm] e $X'$ [g/cm$^2$] consiste in:
\[
	X\rho = X'
.\] 
Dove $\rho$ è la densità del mezzo, il vantaggio di esprimere uno spessore in g/cm$^2$ è che in questa unità lo spessore risulta indipendente dallo stato (solido, liquido, gassoso) del materiale che si usa (solitamente per rilevare le radiazioni di particelle che attraversanola materia).
\subsection[ Effetto fotoelettrico e scattering Compton]{Spiegare qualitativamente l’effetto fotoelettrico e lo scattering Compton, indicandone le differenti caratteristiche
}\label{sec:4.a.2}
L'effetto fotoelettrico constiste nella interazione:
\[
	\ce{\ce{\gamma} +  \text{atomo} -> \ce{e^-} + \text{atomo}+}
.\] 
In pratica un fotone con energia $E$ cede tutta la sua energia ad un elettrone atomico che viene espulso dall'atomo con energia $E-E_{\text{lev}}$ dove $E_{\text{lev}}$ è l'energia del livello a cui si trova l'elettrone. La soglia di energia per far avvenire il processo è quindi $E_{\text{lev}}$, mentre il range di energia in cui il processo è il prevalente è tra $E^{\text{min}}_{\text{lev}}$ e $E^{\text{max}}_{\text{lev}}$. Per il carbonio l'energia di livello massima è $10$ keV mentre per il piombo arriva anche a $500$ keV.\\
L'effetto compton invece sussiste in un urto elastico tra un fotone ed un elettrone, considerato libero:
\[
	\ce{\ce{\gamma} + e- -> \ce{\gamma} + e- }
.\] 
In cui vi può essere uno scambio di energia tra il fotone e l'atomo. Questo è l'effetto prevalente per energie del fotone incidente dell'ordine dei MeV ed a differenza del Rayleigh la sua sezione d'urto va come $Z$.
\subsection[ Lunghezza d'onda Compton]{Dare l'espressione ed il valore numerico della lunghezza d'onda Compton.
}\label{sec:4.a.3}
Visto che nell'effetto Compton vi è scambio di energia tra il fotone e l'elettrone allora la frequenza del fotone dopo lo scattering risulta cambiata, applicando la conservazione dell'energia e dell'impulso si ottiene che:
\[
	\omega-\omega'= \frac{\hbar \omega\omega'}{mc^2}\left( 1-\cos\theta \right) 
.\]
Dove $\omega$ è la frequenza del fotone prima dello scattering, $\omega'$ la frequenza dopo lo scattering, $\theta$ l'angolo di scattering.
Esprimendo tale relazione in termini di $\lambda= 2\pi c/ \omega$ si ha che:
\[
	\lambda'-\lambda= \lambda_{c}\left( 1-\cos\theta \right) 
.\] 
Abbiamo quindi definito la lunghezza d'onda Compton dell'elettrone:
\[
	\lambda_{c}=\frac{h}{mc} =2.43 \cdot 10^{-12} \text{ m} =  2.43 \text{ pm}
.\] 
\subsection[ Scattering Rayleigh]{Spiegare qualitativamente lo scattering Rayleigh
}\label{sec:4.a.4}
Lo scattering Rayleigh corrisponde all'interazione elastica tra un fotone ed un atomo. È un effetto importante fino ad energie $\sim 1$ keV (sezione d'urto che va come $Z^2$), dopo la sua sezione d'urto decresce come il reciproco del quadrato dell'energia.

\subsection[ Creazione di coppie su nucleo]{Spiegare qualitativamente il fenomeno della creazione di coppie  da parte di un raggio gamma che incide su un atomo.
}\label{sec:4.a.5}
La creazione di coppie 
\[
	\ce{\ce{\gamma} -> e+ + e-} 
\]
è cinematicamente proibita, infatti prima della reazione il sistema del centro di massa non esiste, dopo si, ma la velocità di tale sistema non cambia dopo l'urto (o più in generale, il centro di massa prima dell'urto non era definito, quindi non può essere scritta una legge di conservazione).
C'è quindi bisogno di una particella spettatrice carica.
Supponiamo che vi sia un atomo spettatore, si ha:
\[	
	\ce{\ce{\gamma} + \text{atomo} -> e+ + e- +\text{atomo}} 
.\] 
Se $M$ è la massa di tale atomo allora l'energia di soglia per il fotone è (conservazione del modulo del quadrimpulso):
\begin{align*}
	\left( E_{\gamma}+M \right)^2-E^2_{\gamma}=\left( 2m_e + M \right)^2
.\end{align*}
\[
	E_{\gamma}\ge 2m_e\left( 1+ \frac{m_e}{M} \right) 
.\] 
In particolare $E_{\gamma}\ge 2m_e \approx 1$ MeV, da cui la necessità di avere un raggio $\gamma$.


\subsection[ Effetto Cherenkov e angolo Cherenkov]{Descrivere qualitativamente l’effetto Cherenkov e dimostrare tramite il principio di Huygens che la radiazione Cherenkov è emessa ad un solo angolo.
}\label{sec:4.a.6}
L'effetto Cherenkov consiste nella emissione di onde elettromagnetiche da parte di una particella carica che si muove a velocità $v$ superiore a quella della luce.
\[
	\beta c> \frac{c}{n}
.\] 
Con $n$ parte reale dell'indice di rifrazione. L'emissione è dovuta all'irraggiamento coerente di dipoli elettrici che sono messi in oscillazione dal campo elettromagnetico generato dalla particella nel suo passaggio.\\
Analizziamo nel dettaglio la direzione di propagazione di quest'onda utilizzando il principio di Huygens.\\
Ipotizziamo che al tempo $t'$ si propaghi dalla particella un'onda elettromagnetica con fronte d'onda sferico. Dopo un tempo $\Delta t$ tale fronte ha un raggio 
\[
	R = \frac{c}{n} \Delta t
.\] 
Tra il tempo in cui l'onda discussa ha iniziato a propagarsi ($t'$) ed il tempo in cui ha raggiunto il raggio $R$ ($t = t'+ \Delta t$) si propagano perturbazioni sferiche del tutto analoghe a quella sopra, che al tempo $t$ avranno un raggio intermedio tra $R$ e $0$.\\
Di conseguenza la perturbazione elettromagnetica si sviluppa su un fronte che è la sovrapposizione di tutte le superfici sferiche di onde formatesi nel tempo $\Delta t$ (effetto motoscafo).
Chiamiamo la direzione di propagazione dell'onda $\theta$, ci aspettiamo che il coseno di quest'angolo abbia proporzionalità inversa da $\beta$ e da $n$. In questo modo all'aumentare di $n$ l'effetto risulta più evidente (luce "più lenta") mentre al diminuire di $\beta$ (nei limiti in cui vale ancora $\beta > 1 /n$) l'effetto tenda a svanire (si torna in situazioni in cui le cose vanno più lentamente della luce). 
 \[
	\cos\theta = \frac{1}{\beta n}
.\] 
Ipotizziamo che in un punto $P$ vi sia un osservatore che rivela le onde elettromagnetiche, la situazione sarebbe questa:
\begin{figure}[H]
    \centering
    \incfig{cherenkov-huygens}
    \caption{Schema di osservazione dell'effetto Cherenkov}
    \label{fig:cherenkov-huygens}
\end{figure}
Nell'immagine l'osservatore in $P$ rileva l'onda arrivare al tempo $t$, quando la particella è "già oltre", in un certo senso possiamo pensare che l'osservatore sente prima il dolore del proiettile che il rimbombo della pistola.\\
Facendo riferimento alla figura \ref{fig:cherenkov-huygens} dimostriamo la relazione angolare citata sopra.
\begin{align*}
	&R = \left| \bs{R} \right| = \frac{c}{n} \Delta t	&\bs{\rho} + \bs{\beta} c \Delta t=\bs{R} 
.\end{align*}
Facendo il modulo quadro della relazione vettoriale si ottiene:
\[
	R^2=\frac{c^2}{n^2}\Delta t^2=\beta^2c^2\Delta t^2 + \rho^2 + 2\rho\beta c\Delta t \cos\alpha 
.\] 
Adesso dividiamo per $\rho^2$ questa espressione, ciò che si ottiene è una equazione di secondo grado nell'incognita $\frac{\Delta t}{\rho}$ con dei coefficenti interessanti:
\[
	c^2\left( \beta^2- \frac{1}{n^2} \right) \left( \frac{\Delta t^2}{\rho^2} \right) + 2\beta c \cos\alpha\left( \frac{\Delta t}{\rho} \right) +1=0
.\] 
I coefficenti sono interessanti perchè, per ipotesi noi abbiamo assunto che la particella viaggia ad una velocità tale che $\beta>\frac{1}{n}$, condizione che ritroviamo espressa nel segno del primo coefficente. Quindi la soluzione della equazione esiste se:
\[
	\frac{\Delta}{4}=\beta^2c^2\cos^2\alpha - c^2\left( \beta^2-\frac{1}{n^2} \right) \ge 0
.\] 
Quindi serve avere $\sin\alpha\le \frac{1}{\beta n}$ e, imponendo anche $\Delta t>0$ si deve avere $\cos\alpha<0$ (si prende il cono all'indietro, non quello in avanti).
Quindi abbiamo due soluzioni distinte per ogni punto interno al cono delimitato dai vincoli su $\alpha$.\\
Con il principio di Huygnes sappiamo però che le soluzioni interne al cono non troveranno sovrapposizione coerente di onde sferiche, cosa che invece si ottiene sul bordo del cono dove tutte le onde sferiche avanzano coerentemente a formare una unica perturbazione.
Ci concentriamo quindi sulla soluzione al bordo ($\sin \alpha = \frac{1}{\beta n}$).\\
Facciamo una considerazione geometrica su questa particolare scelta dell'angolo $\alpha$:
\begin{figure}[H]
    \centering
    \incfig{geometria-cherenkov}
    \caption{Situazione geometrica della radiazione cherenkov}
    \label{fig:geometria-cherenkov}
\end{figure}
Sfruttando un teorema sui triangoli:
\[
	\frac{R}{\sin\left( \pi-\alpha \right) }= \frac{\Delta x}{\sin\hat{P}}
.\]
E inserendo i valori di $\Delta x$, $R$ usati sopra si dimostra che l'angolo sotteso al punto $P$ è rettangolo per $\sin\alpha= \frac{1}{\beta n}$.\\
Quindi possiamo ricavare l'angolo Cherenkov semplicemente dalle proprietà dei triangoli rettangoli:
\[
	\theta_{c}=\frac{\pi}{2}-\left( \pi-\alpha \right) = \alpha-\frac{\pi}{2}
.\] 
In particolare
\[
	\sin\alpha =\sin\left( \theta_{c}+ \frac{\pi}{2} \right) = \cos\theta_{c}=\frac{1}{\beta n}
.\] 
Abbiamo quindi trovato l'angolo Cherenkov, indica la direzione dal quale l'osservatore in $P$ vede arrivare l'onda.



\subsection[ Situazioni di applicazione della formula di Frank-Tamm]{Descrivere la situazione in cui la legge 
\[
	\frac{\mbox{d} N_{\gamma}}{\mbox{d} E_{\gamma}\text{d}x} = z^2 \frac{\alpha}{\hbar c} \sin^2\theta_{c}
.\] 
inerente la radiazione Cherenkov, è applicabile e spiegare il significato e l'unità di misura di ogni grandezza fisica ivi indicata.
}\label{sec:4.a.7}
Quella espressa è detta la formula di Frank-Tamm, si applica quando una particella di carica $z$ (in unità di $e$) attraversa uno spessore $dx$ infinitesimo di materiale con indice di rifrazione $n\left( \omega \right)$ ed esprime il numero di fotoni emessi per unità di energia $E_{\gamma}= \hbar \omega$. 
Questa equazione si riscrive in termini dell'indice di rifrazione sostituendo l'equivalente dell'angolo Cherenkov discusso nella domanda precedente:
\[	
	\frac{\mbox{d} N_{\gamma}}{\mbox{d} E_{\gamma}\text{d}x} = z^2 \frac{\alpha}{\hbar c} \left( 1- \frac{1}{\beta^2 n^2} \right)  
.\] 
Ricordiamo adesso alcuni valori delle costanti in gioco:
\begin{align*}
	&\alpha = \frac{e^2}{\hbar c}= \frac{1}{137}	&\hbar c = 197 \text{ MeV}\cdot \text{fm}
.\end{align*}
Quindi \[
	\frac{\alpha}{\hbar c}= \frac{10^{-6}\cdot 10^{13}}{137\cdot 197} \left[ \frac{\text{fotoni}}{\text{eV}\cdot \text{cm}} \right] \approx 
	370 \left[ \frac{\text{fotoni}}{\text{eV}\cdot \text{cm}} \right]
.\] 
Ricordiamo anche che l'indice di rifrazione è in realtà $n\left( \omega \right) $, da cui deriva la dipendenza della formula da $E_{\gamma}$.

\subsection[ Situazioni di applicazione della formula di Frank-Tamm integrale]{Descrivere la situazione in cui la legge 
\[
	N_{\gamma}= z^2 \frac{\alpha}{\hbar c}L \int_{E_1}^{E_2} \left[ 1 - \frac{1}{\beta^2 \epsilon_{r}\left( E \right)}\right]P_{\text{det}}dE 	
.\] 
inerente la radiazione Cherenkov, è applicabile e spiegare il significato e l'unità di misura di ogni grandezza fisica ivi indicata.
}\label{sec:4.a.8}
Le quantità sono le stesse della domanda precedente, tuttavia questa esprime il numero di fotoni rilevati da un detector avente una efficienza $P_{\text{det}}$ nel range di energia tra $E_1$ ed $E_2$, il mezzo attraversato ha inoltre larghezza $L$.
\subsection[ Cause ed effetti della radiazione di frenamento]{Descrivere qualitativamente le cause e gli effetti del fenomeno della radiazione di frenamento da parte di una particella carica nella materia.
}\label{sec:4.a.9}
Una particella carica di massa $M$ e carica $q=ze$ che attravera un materiale subisce l'interazione coulombiana con i nuclei che lo compongono di carica $Ze$.\\
Per ogni interazione la particella accelera (a causa della repulsione o attrazione) e quindi irraggia: Radiazione di frenamento o Bremsstrahlung.
\begin{figure}[H]
    \centering
    \incfig{bremsstrahlung}
    \caption{Schema dello scattering coulombiano per la Bremsstrahlung}
    \label{fig:bremsstrahlung}
\end{figure}
La perdita di energia dovuta a questo irraggiamento è importante per elettroni (aventi una certa energia critica $E_{c}$) ed è invece trascurabile per le altre particelle. Inoltre la radiazione di frenamento avviene in modo indipendente su ciascun nucleo del materiale attraversato, è allora facile immaginare che darà luogo a processi di emissione incoerenti.\\
Nel sistema del laboratorio la radiazione può esssere rilevata in modi diversi a seconda della velocità del proiettile:
\paragraph{Particella non relativistica}%
La radiazione è proporzionale a $\sin^2\alpha$, con $\alpha$ angolo tra la media della accelerazione nel processo e la direzione di osservazione.
\begin{figure}[H]
    \centering
    \incfig{bremsstrahlung-non-relativistica}
    \caption{Emissione della radioazione nel caso di Bremsstrahlung non relativistica}
    \label{fig:bremsstrahlung-non-relativistica}
\end{figure}
\paragraph{Particella ultrarelativistica}%
La radiazione viene rilevata in avanti, con angolo di $\frac{1}{\gamma}$ rispetto alla direzione di $\bs{\beta}$ (si dimostra con una trasformazione dal sistema in cui la particella è ferma a quello del laboratorio).
\begin{figure}[ht]
    \centering
    \incfig{bremsstrahlung-relativistica}
    \caption{Emissione della radioazione nel caso di Bremsstrahlung relativistica}
    \label{fig:bremsstrahlung-relativistica}
\end{figure}

\subsection[\hspace{1mm} Importanza della Bremsstralhung per gli elettroni]{Spiegare perchè la perdita di energia per irraggiamento è significativa per elettroni e non per le altre particelle cariche.
}\label{sec:4.a.10}
La forza di Coulomb dipende dalla carica elettrica e non dalla massa della particella, quindi dall'equazione del moto si ottiene che:
\[
	a \sim \frac{1}{M}
.\]
Visto che l'irraggiamento dipende dall'accelerazione al quadrato si ha anche che:
\[
	I\sim \frac{1}{M^2}
.\] 
e banalmente questo è l'unico motivo per cui gli elettroni irraggiano maggiormente di Bremsstrahlung.

\subsection[\hspace{1mm} Situazione di applicazione della energia irraggiata per unità di frequenza e di angolo solido]{Descrivere la situazione in cui la legge
\[
	\frac{\mbox{d}I_{\omega}}{\mbox{d}\Omega}=
	\frac{q^2}{4\pi^2c}\left|\int\frac{\hat{n}\wedge\left[\left(\hat{n}-\bs{\beta}\right)\wedge\dot{\bs{\beta}}\right]}{\left(1-\hat{n}\cdot\bs{\beta}\right)^2}
	e^{i\omega\left(t'-\frac{\bs{r}'\cdot\hat{n}}{c}\right)}dt'\right|^2 
.\] 
è applicabile e spiegare il significato e l'unità di misura di ogni grandezza fisica ivi
indicata.
}\label{sec:4.a.11}
L'equazione descrive l'energia irraggiata per unità di frequenza ed angolo solido, deriva dalla più generale 
\[
	\frac{\mbox{d} I_{\omega}}{\mbox{d} \Omega} = \frac{\epsilon_0 c r^2}{\pi} \left| \bs{E}\left( \bs{r}, \omega \right)  \right| ^2
.\] 
Dove $\bs{E}\left( \bs{r}, \omega \right)$ è la trasformata del campo elettrico di radiazione (valida per ogni tipo di irraggiamento).
Nel caso in questione è quello di una carica puntiforme accelerata in moto relativistico, tale equazione è stata dimostrata nella \hyperref[sec:3.a.14]{Domanda 3.a.14}.
\subsection[\hspace{1mm} Energia irraggiata per unità di frequenza per Bremsstralhung non relativistica]{Descrivere la situazione in cui la legge
\[
	I_{\omega}\left( b \right) = 
	\begin{cases}
		\frac{8z^4Z^2 \alpha \hbar c^2}{3\pi} \left( \frac{m_e}{M} \right) ^2 \frac{r^2_e}{V^2} \frac{1}{b ^2}  & \text{per } \omega<\frac{V}{b}\\
		0 & \text{per } \omega> \frac{V}{b}
	\end{cases}
.\] 
applicabile e spiegare il significato e l'unità di misura di ogni grandezza fisica ivi indicata.
}\label{sec:4.a.12}
$I_{\omega}\left( b \right) $ indica l'energia irraggiata per unità di frequenza in funzione del parametro di impatto (J$\cdot$s). Tutte le altre quantità sono state ampliamente preserntate nel corso della lettura ma facciamo un ripassino su quelle che non vediamo da un pò:
\begin{align*}
	&r_e = k_0 \frac{e^2}{m_e c^2} \approx 2.82 \text{ fm}\\
	&m_e = 0.511 \text{MeV/c}^2 = 9.12 \cdot 10^{-31} \text{kg}
.\end{align*}
Detto questo la formula rappresenta $I_{\omega}$ per una Bremstrahlung nel caso non relativistico nella approssimazione che il moto complessivo della carica possa essere assunto come rettilineo, quindi angoli di scattering piccoli. In tal modo viene assunta la variazione di quantità di moto nella sola direzione ortogonale alla direzione di incidenza del proiettile ed i conti per la dimostrazione diventano pochi.\\
È stata inoltre approssimata la funzione ad uno scalino, quando in realtà i termini sopra esrpessi corrispondono all'andamento asintodico di quest'ultima.

\subsection[\hspace{1mm} Sezione d'urto di irraggiamento]{Definire la sezione d'urto di irraggiamento e spiegare il suo significato fisico.
}\label{sec:4.a.13}
La sezione d'urto di irraggiamento è definita come
\[
	\chi_{\omega}= \int_{b_{min}}^{b_{max}} 2\pi I_{\omega}b db \quad \left[ \frac{\text{J}}{\text{Hz}}\cdot \text{m}^2 \right] 
.\] 
ed è la grandezza che moltiplicata per il numero di nuclei per unità di superficie fornisce l'energia irraggiata per intervallo di frequenza $I_{\omega}$.

\subsection[\hspace{1mm} Sezione d'urto di irraggiamento per schermaggio incompleto]{Descrivere la situazione in cui la legge
	\[
		\chi_{\omega}=\frac{16 z^4 Z^2 \alpha \hbar c^2}{3} \left( \frac{m_e}{M} \right)^2 \frac{r_e^2}{V^2} \ln\left( \frac{MV^2}{\hbar \omega} \right) 
	.\] 
e spiegare il significato e l'unità di misura di ogni grandezza fisica ivi indicata.
}\label{sec:4.a.14}
Questa sezione d'urto di irraggiamento è quella derivante dalla equazione della domanda precedente integrata sul parametro di impatto. Questa è applicabile nel caso in cui: 
\[
	\frac{\hbar}{M V} < b < \frac{V}{\omega}
.\] 
Con $M$ la massa della particella incidente.\\
Saremmo portati ad pensare che il limite inferiore su $b$ sia il raggio nucleare, in realtà il principio di indeterminazione risulta più stringente per particelle lente:
\[
	\Delta x \Delta p = \hbar 
.\] 
L'indeterminazione sull'impulso della particella è sicuramente inferiore all'impulso iniziale di questa: $\Delta p \lesssim MV$ (siamo a distanze tali per cui la probabilità di schianto nel nucleo è alta), mentre sul $\Delta x$ l'indeterminazione è proprio il parametro di impatto $b$, quindi:
\[
	b MV \gtrsim \hbar \implies b \gtrsim \frac{\hbar}{MV}
.\] 
Come limite superiore invece saremmo portati a pensare che sia il raggio atomico, in realtà il tempo tipico di interazione stringe ancora il campo nel caso non relativistico:
\[
	\Delta t \approx \frac{b}{V} \implies \omega < \frac{V}{b} 
.\] 
Nel caso di elettroni relativistici il parametro di impatto inferiore è proprio la lunghezza d'onda Compton.

\subsection[\hspace{1mm} Sezione d'urto di irraggiamento relativistica]{Descrivere la situazione in cui la legge
\[
	\frac{\mbox{d}E^{\text{irr}}}{\mbox{d}x}=
	n_{\text{nuclei}}\frac{16}{3}z^4Z^2\alpha\left(\frac{m_e}{M}\right)^2r_e^2\ln\left(\frac{192}{Z^{1 /3}}\frac{M}{m_e}\right) E
.\] 
applicabile e spiegare il significato e l'unità di misura di ogni grandezza fisica ivi indicata.
}\label{sec:4.a.15}
La formula descrive l'energia irraggiata per unità di lunghezza da una particella ultrarelativistica ($V \sim c$) con energia iniziale $E$ e massa $M$, questa espressione si ottiene integrando la sezione d'urto di irraggiamento di tale particella nel dominio detto di Screening attivo, ricaviamo intanto tale sezione d'urto. Nel sistema $\Sigma'$ in cui la particella è a riposo si ha che durante lo Scattering (in cui il nucleo va incontro alla particella) $M$ irraggia in manera non relativistica, quindi la formula per la $\chi'_{\omega}$ è:
\[
	\chi'_{\omega}=\frac{16}{3}z^4Z^2\alpha\hbar c^2 \left( \frac{m_e}{M} \right)^2  \frac{r^2_e}{V^2}\ln\left( \frac{b_{\text{MAX}}}{b_{\text{min}}} \right) 
.\] 
È bene ricordare che per questa interazione avremo una larghezza spettrale di \[
	\omega'< \gamma \frac{c}{b}
\]
Come minimo del parametro di impatto si può usare il principio di indeterminazione come per la domanda precedente, c'è tuttavia da stare attenti al fatto che adesso $\Delta p \lesssim MV\gamma$, quindi:
\[
	b \gtrsim \frac{\hbar}{MV\gamma}
.\] 
Considerando la grandezza di $\gamma$ questa stima risulta decisamente inferiore del raggio nucleare, è quindi erronea.\\
IL tempo di interazione è $\tau \approx b /\left( c\gamma \right)$, possiamo allora pensare che il campo elettrico agisce sul proiettile in una regione di estensione $\tau \cdot c = b /\gamma$. Se questa fosse più piccola di $\Delta x$ allora non avrebbe senso parlare di interazione e irraggiamento (si presenterebbero effetti di interferenza), quindi poniamo:
\[
	\Delta x \gtrsim \frac{b}{\gamma} \implies b_{\text{min}} = \frac{\hbar}{MV}
.\] 
Come nel caso non relativistico.\\
Dobbiamo adesso trovare il limite superiore, si hanno due possibili casi:
\begin{align*}
	b^{\left( 1 \right) }_{\text{max}} = \gamma \frac{v}{\omega'}
.\end{align*}
Considerando il limite dovuto al tempo di interazione, oppure
\[
	b^{\left( 2 \right) }_{\text{max}}= a =1.4 \frac{a_0}{Z^{1 /3}}
.\] 
Che corrisponde al raggio atomico con:
\[
	a_0 = \frac{\hbar}{m_e c \alpha}
.\] 
Per valutare quale dei due sia quello giusto consideriamo il fatto che, nel sistema del laboratorio, tutta la radiazione viene emessa in avanti con un angolo rispetto alla velocità iniziale della particella:
\[
	\hat{\theta}=\frac{1}{\gamma} \implies \cos \hat{\theta}\approx 1-\frac{1}{2\gamma^2}
.\] 
Quindi la corrispondente frequenza nel sistema primato sarà 
\[
	\omega'=\gamma\omega\left(1-\beta\cos\hat{\theta}\right)\approx\gamma\omega\left(1-\left(1-\frac{1}{2\gamma^2}\right)\left(1-\frac{1}{2\gamma^2}\right)\right) 
	= \frac{\omega}{\gamma}
.\] 
Siccome il fotone irraggiato nel laboratorio avrà sicuramente energia inferiore all'energia cinetica dell'elettrone che lo ha partorito si ha (conservazione dell'energia):
\[
	\hbar\omega\le \left( \gamma-1 \right)Mc^2\implies\omega'\lesssim \frac{Mc^2}{\hbar}
.\] 
Questo ci aiuta a valutare chi tra i due parametri è più grande:
\[
	\frac{b^{\left( 1 \right)}_{\text{max}}}{b^{\left( 2 \right)}_{\text{max}}}=\frac{\gamma v /\omega'}{a}\gtrsim \frac{\gamma\hbar}{aMc}=
	=\gamma Z^{1 /3} \frac{m_e \alpha}{1.4 M}  \approx \gamma\frac{Z^{1 /3}}{192}\frac{m_e}{M}
.\] 
Vista la dipendenza diretta da $\gamma$ è facile vedere che, per elettroni ultrarelativistici si ha sempre $\gamma v /\omega' > a$, quindi il parametro giusto da scegliere è il secondo, ovvero il raggio atomico $a$.
Il caso in cui lavoriamo è detto screening attivo, espresso graficamente come segue:
\begin{figure}[H]
    \centering
    \incfig{screening-bremsstrahlung}
    \caption{Tipi di screening per la Bremsstrahlung.}
    \label{fig:screening-bremsstrahlung}
\end{figure}
In ogni caso si deve rispettare il fatto che la frazione calcolata poco fa deve essere maggiore di 1 per andare in Screening attivo, quindi:
\[
	\gamma > \frac{192M}{Z^{1 /3}m_e}
.\] 
Inserendo tutto nella sezione d'urto si ottiene l'espressione:
\[
	\chi'_{\omega}=\frac{16}{3}z^4Z^2\alpha\hbar c^2 \left( \frac{m_e}{M} \right)^2  \frac{r^2_e}{c^2}\ln\left( \frac{192\cdot M}{Z^{1 /3}\cdot m_e} \right) 
.\] 
Adesso basta notare che, come visto sopra, $\omega = \gamma \omega'$. Inoltre $dE_{\text{irr}} = \gamma dE_{\text{irr}}'$ che viene dal fatto che questa quantità infinitesima trasforma come un tempo:
\[
\frac{\mbox{d} E}{\mbox{d} t} = -\frac{2}{3}\frac{e^2}{c^3}\frac{\mbox{d} u^{\mu}}{\mbox{d} \tau} \frac{\mbox{d} u_{\mu}}{\mbox{d} \tau}\implies
\text{d}E \sim  G \text{d}t
.\]
Con G invariante di Lorentz.\\
Prendiamo allora la definizione di $\chi_{\omega}$:
\[
	\chi_{\omega}=\int_{b_{\text{min}}}^{b_{\text{max}}}2\pi b \frac{\mbox{d} E}{\mbox{d} w} db
.\] 
Con tutte le considerazioni fatte sopra e considerando in più che il parametro di impatto è invariante (essendo ortogonale al boost) otteniamo $\chi_{\omega}= \chi'_{\omega}$, quindi:
\[
	\frac{\mbox{d} E^{\text{irr}}}{\mbox{d} x} = n_{\text{nuclei}} \int_{0}^{\omega_{\text{max}}} \chi_{\omega} d\omega = 
	n_{\text{nuclei}} \int_{0}^{E /\hbar} d\omega \chi_{\omega} 
	= n_{\text{nuclei}} \frac{E}{\hbar} \chi_{\omega}
.\] 
Che è proprio l'espressione cercata (E è l'energia cinetica della particella!!).
\subsection[\hspace{1mm} Sezione d'urto di irraggiamento in funzione della sezione d'urto espressa in $m^2$]{Descrivere la situazione in cui la legge
\[
	\chi_{\omega} = \hbar \omega \frac{\text{d}\sigma_{y}}{\text{d}\omega}
.\] 
 è applicabile e spiegare il significato e l'unità di misura di ogni grandezza fisica ivi indicata.
}\label{sec:4.a.16}
La formula è una manipolazione algebrica della definizione di $\chi_{\omega}$:
\[
	\chi_{\omega}= \frac{1}{n} \frac{\mbox{d} E_{\text{irr}}}{\mbox{d} x \text{d}\omega} 
.\] 
Con $n$ densità dei centri scatteranti. Proviamo a manipolare l'espressione:
\[
	\chi_{\omega} = \hbar \omega\frac{\mbox{d}}{\mbox{d} \omega} \left( \frac{\mbox{d}E_{\text{irr}}}{\mbox{d}t}\frac{1}{n\cdot \hbar \omega \frac{\mbox{d} x}{\mbox{d} t} } \right)  
.\] 
Possiamo allora definire una corrente di particelle proiettili (l'inverso del secondo termine in parentesi) a moltiplicare la potenza irraggiata mediata (è mediata perchè $E_{\text{irr}}$ è l'energia irraggiata totale), questa altro non è che la sezione d'urto del processo di irraggiamento (definita come una superficie, non come quella finta sezione d'urto $\chi_{\omega}$).
\[
	\chi_{\omega}= \hbar \omega \frac{\mbox{d} }{\mbox{d} \omega} \left( \sigma_{y} \right) 
.\] 
\subsection[\hspace{1mm} Definizione di lunghezza di radiazione]{Dare la definizione di lunghezza di radiazione.
}\label{sec:4.a.17}
La lunghezza di radiazione $X_0$ è la distanza tipica che una particella percorre in un materiale. Tale distanza infatti subentra nella legge esponenziale con cui varia l'energia della particella:
\[
	E = E_0 e^{-x /X_0}
.\] 
Tale lunghezza è tipica del materiale che la particella percorre. Per lo Screening attivo la lunghezza di radiazione è l'inverso del coefficiente che moltiplica $E$ (l'equazione della domanda 4.a.15):
\[
	X_0^{\text{attivo}} = \frac{1}{n_{\text{nuclei}}\frac{16}{3}z^4Z^2\alpha\left(\frac{m_e}{M}\right)^2r_e^2\ln\left(\frac{192}{Z^{1 /3}}\frac{M}{m_e}\right)} 
.\] 

\subsection[\hspace{1mm} Definizione di energia Critica]{Dare la definizione di "Energia critica"
}\label{sec:4.a.18}
L'energia critica è l'energia oltre il quale l'energia dell'elettrone si attenua per il solo effetto della Bremstrahlung.\\
Nello screening attivo abbiamo visto esserci un limite inferiore a $\gamma$ per la particella incidente, questo implica che in tal caso l'energia della particella ha anc'essa un limite inferiore che corrisponde proprio alla sua energia critica:
\[
	E_{c} = Mc^2\gamma_{\text{lim}}= Mc^2 \frac{192}{Z^{1 /3}}
.\] 
\subsection[\hspace{1mm} Legge di attenuazione per irraggiamento dell'energia in un mezzo]{Descrivere la situazione in cui la legge \[
	E = E_0 e^{-x /X_0}
.\] 
è applicabile e spiegare il significato e l'unità di misura di ogni grandezza ivi indicata.
}\label{sec:4.a.19}
La legge è applicabile nel caso di Bremstrahlung con particella incidente di energia $E_0$ superiore alla energia $E_{c}$ e descrive l'andamento dell'energia all'interno del materiale in cui tutta l'energia persa va in irraggiamento.
\subsection[\hspace{1mm} Formula di Tsai]{Descrivere la situazione in cui la legge
	\[
		\frac{1}{\rho X_0} = 4 \alpha r^2_e\frac{Z^2}{A\left(g\right)}N_A\left[\ln\left(\frac{184}{Z^{1/3}}\right)-f\left(Z\right)+\frac{L'}{Z}\right] 
	.\] 
applicabile e spiegare il significato e l'unità di misura di ogni grandezza ivi indicata. [formula di Tsai non dimostrata a lezione]
}\label{sec:4.a.20}
La formula descrive in modo più preciso l'inverso della lunghezza di radiazione tenendo condo della struttura del materiale che il proiettile attraversa (con gli appositi coefficienti $f\left( Z\right) $ e $L'$).\\
Il termine $f\left( Z \right) $ corrisponde ad una serie in $a = \alpha Z$, i primi termini sono :
\[
	f\left( Z \right) = a^2\left[ \left( 1+a^2 \right)^{-1}+0.0202 + \ldots \right] 
.\] 
Il termine $L'$ invece dipende dal tipo di materiale, tuttavia per i materiali con $Z\ge 5$ vale:
\[
	L' = \ln\left( \frac{194}{Z^{2 /3}} \right) 
.\] 
Ricordiamo che $A\left( g \right) $ è la massa atomica del materiale attraversato in grammi e $N_A = 6.022 \cdot 10^{23} $ è il numero di avogadro.
\subsection[\hspace{1mm} Perdita di energia per collisioni]{Descrivere qualitativamente il meccanismo della perdita di energia per collisioni da parte di una particella carica nella materia.
}\label{sec:4.a.21}
Una particella carica interagisce con gli elettroni atomici di un materiale cedendo loro energia tramite interazioni coulumbiane. Si ipotizza in queste interazioni che la variazione di impulso di ogni elettrone sia interamente ortogonale alla direzione di incidenza della particella facendo sostanzialmente uso di una approssimazione impulsiva. \\
L'effetto che il passaggio della particella provoca sul materiale può essere di vari tipi:
\begin{itemize}
	\item Il materiale si ionizza producendo coppie elettrone-ione nei gas o elettrone-lacuna nei semiconduttori.
	\item Il materiale emette fotoni di scintillazione (sostanzialmente dovuti al salto in alto e successivamente in basso in energia dell'elettrone partecipe dello scattering).
	\item Riscaldamento del materiale per agitazione termica nel passaggio.
\end{itemize}
Tale processo trova applicazione nella misura simultanea della quantità di moto e della massa di una pariticella ed allo studio del percorso residuo di particelle cariche per uso medico (adroterapia) o protettivo per la schermatura di radiazione (degli acceleratori di particelle).

\subsection[\hspace{1mm} Formula di Bethe-Bloch]{Spiegare il significato di ogni termine dell’espressione per la perdita di energia per collisioni (formula di Bethe-Bloch, non dimostrata):
	\[
		\frac{1}{\rho}\frac{\mbox{d} E_{\text{coll}}}{\mbox{d} x} = 
		z^2 \frac{Z}{A\left( g \right) }4\pi \frac{m_e c^2}{\beta^2} N_A r^2_e 
		\left( \frac{1}{2}\ln\left( \frac{2m_e c^2 \beta^2 \gamma^2}{I^2}T_{\text{Max}} \right) - \beta^2 + \frac{\delta}{2} \right) 
	.\] 
}\label{sec:4.a.22}
Abbiamo alcune quantità già viste nella formula di Tsai, in più c'è $I$ che è l'energia potenziale media di ionizzazione, $T_{\text{max}}$ che è l'energia massima trasferibile dalla particella all'elettrone:
\[
	T_{\text{Max}}= \frac{2m_e c^2 \beta\gamma}{1+ \frac{2m_e \gamma}{M}+\left( \frac{m_e}{M} \right) ^2}
.\] 
ed il fattore $\delta /2$ è il fattore di correzione della densità dovuto a fermi, quest'ultimo dipende da $\beta$ e dal tipo di materiale attraversato.

\subsection[\hspace{1mm} Andamento della Bethe-Bloch]{Disegnare qualitativamente la funzione di Bethe-Bloch indicando i valori dei punti significativi.
}\label{sec:4.a.23}
\begin{figure}[H]
    \centering
    \incfig{bethe}
    \caption{Andamento commentato della Bethe-Bloch.}
    \label{fig:bethe}
\end{figure}
Sull'asse delle $x$ abbiamo $\beta\gamma = \frac{\left| \bs{p} \right| }{Mc}$.
Si  aggiunge soltanto che il minimo della funzione si ottiene per:
\[
	\frac{1}{\rho}\left.\frac{\mbox{d} E_{\text{coll}} }{\mbox{d} x} \right|_{\text{min}} \sim 2 \frac{\text{MeV}}{g /\text{cm}^2} 
.\] 
Le particelle aventi impulso $\left| \bs{p} \right| \sim 3.5$ Mc si dicono al minimo di ionizzazione e sono importanti per l'effetto detto Picco di Brag.

\subsection[\hspace{1mm} Range o percorso residuo di una particella]{Definire il "percorso residuo" ("range") per una particella carica in un materiale.
}\label{sec:4.a.24}
Il percorso residuo $R$ (anche chiamato Range) è il percorso dopo il quale una particella attraversando un materiale esaurisce la sua energia e si ferma.\[
	R\left( E_0 \right) = \int_0^{R} dx = \int_0^{E_0} dE \cdot \frac{\mbox{d} x}{\mbox{d} E} =
	\int_0^{E_0}\frac{1}{\frac{\mbox{d} E}{\mbox{d} x} } dE
.\] 
Il calcolo precedente viene fatto numericamente poichè l'espressione per $\frac{\mbox{d} x}{\mbox{d} E}$ detto Stopping Range è complicata e soprattutto tiene conto non solo della energia persa in collisioni coulombiane ma anche quella persa in irraggiamento e in interazioni nucleari forti.
\subsection[\hspace{1mm} Stopping Power per i vari processi nella materia]{Definire gli "stopping power" totale, collision, radiative e nuclear; indicare per quali particelle ognuno di essi sia o meno rilevante.
}\label{sec:4.a.25}
Partiamo dal totale:
\[
	\left( \frac{\mbox{d} x}{\mbox{d} E} \right)_{\text{tot}}= \left( \frac{\mbox{d} x}{\mbox{d} E} \right)_{\text{collision}}+\left( \frac{\mbox{d} x}{\mbox{d} E} \right)_{\text{radiative}}+\left( \frac{\mbox{d} x}{\mbox{d} E} \right)_{\text{nuclear}}
.\] 
Il primo termine è l'inverso della formula di Bethe-Block, il secondo è l'inverso della formula di Tsai mentre il terzo è rilevante solo per interazioni forti, ovvero interessa gli adroni.

\subsection[\hspace{1mm} Calcolo del percorso residuo nota la curva di stopping power.]{Come si calcola il "percorso residuo" ("range"), nota la curva $\frac{\mbox{d} E}{\mbox{d} x}$, in funzione dell'energia E della particella?
}\label{sec:4.a.26}
Si calcola con l'integrale numerico della \hyperref[sec:4.a.24]{Domanda 4.a.24}. 


\subsection[\hspace{1mm} Picco di Brag]{Spiegare qualitatativamente il cosiddetto "picco di Bragg"
}\label{sec:4.a.27}
Quando una particella carica è rallentata da un materiale fino ad arrestarsi la distribuzione di energia depositata dipende in maniera marcata dalla forma dello stopping power $ST\left( E \right) = \left( \frac{\mbox{d} E}{\mbox{d} x}  \right)^{-1}$.\\
Se la particella entra nel mezzo con una energia superiore a quella del minimo di ionizzazione $\beta \gamma > 3.5$ allora l'energia deposotata per unità di lunghezza decresce fino a raggiungere il minimo. Proseguendo oltre il minimo la particella continua a perdere energia per collisioni (quindi $\beta$ cala) ma la perdita per unità di lugnhezza in questa direzione sale come $1 /\beta^2$. In questo modo l'energia depositata avrà un massimo in corrispondenza del punto di arresto nel materiale della particella, tale massimo è chiamato Picco di Brag.

\subsection[\hspace{1mm} Scattering multiplo per particella in moto veloce nella materia]{Descrivere qualitativamente il fenomeno dello scattering multiplo da parte di una particella carica in moto veloce nella materia.
}\label{sec:4.a.28}
Nell'urto di una particella con un atomo bersaglio l'angolo di deflessione è, nel peggiore dei casi $\sim 10^{-3}$ rad, quindi trascurabile nella maggior parte dei contesti. Se invece la particella attraversa un materiale allora incontrerà diversi nuclei nel suo cammino, per ciascun nucleo avremo piccole deflessioni scatteranti.\\
La discussione sul fenomeno dello scattering multiplo è appunto la valutazione dell'angolo di deflessione all'uscita del mezzo. Anticipiamo che facendo numerose interazioni l'angolo di deflessione sarà distribuito gaussianamente (teorema del limite centrale). 

\subsection[\hspace{1mm} Definizione di angolo multiplo di scattering]{Definire l'angolo di multiplo scattering (rispetto alla direzione iniziale della particella) e definire la sua proiezione su un piano (che contiene la direzione iniziale della particella). Indicare i limiti delle due variabili cosí definite.
}\label{sec:4.a.29}
L'angolo multiplo di scattering $\theta$ è l'angolo formato tra la direzione iniziale e la direzione dopo lo scattering multiplo della particella che attraversa un mezzo. Possiamo ipotizzare questo angolo come $\theta \ll 1$ per una particella veloce.\\
Ipotizzando che la particella fosse inizialmente diretta lungo l'asse $\hat{z}$, possiamo definire le proiezioni dell'angolo di deflessione sui due piani che contengonola direzione iniziale tra loro ortogonali: $\theta_{x}$ e $\theta_{y}$ con $\theta^2=\theta^2_{x}+\theta^2_{y}$.\\
Come accennato sopra si impone una distribuzione gaussiana su questi angoli con r.m.s. pari a $\theta_0$ definita dal tipo di scattering preso come modello (per angoli piccoli l'ideale è lo scattering Rutherford).
Anche allo scarto quadratico medio su queste due variabili si impone $\theta_0\ll 1$, inoltre sono entrambe variabili a media nulla.

\subsection[\hspace{1mm} Espressione per l'angolo quadratico medio di scattering]{Spiegare il significato di ogni termine dell'espressione per l’angolo quadratico medio di multiplo scattering (rispetto alla direzione iniziale della particella)
\[
	\sqrt{\left<\theta^2_{\text{ms}}\right>}=\theta_0\sqrt{2}=z\frac{13.6\text{MeV}}{P\beta c}\sqrt{\frac{L}{X_0}}\left(1+0.0038\ln\left(\frac{L}{X_0}\right)  \right)  
.\] 
}\label{sec:4.a.30}
$\theta_0$ è l'angolo r.m.s. per i due angoli definiti sopra, $P$ è l'impulso iniziale della particella, $L$ è lo spessore del materiale attraversato,  $X_0$ la lunghezza di radiazione del materiale. Si arriva a questo risultato sfruttando per lo scarto quadratico sul singolo urto la sezione Mott e moltiplicando per il numero di possibili urti sulla base della lunghezza e densità del materale.

\subsection[\hspace{1mm} Distribuzione per il fenomeno di multiplo scattering]{Se non fosse sufficiente la approssimazione di piccoli angoli e distribuzione gaussiana, indicare quale fra le seguenti funzioni descriverebbe meglio il fenomeno del multiplo scattering:\\
	i) Bethe-Bloch,\\ 
	ii) Moliere,\\ 
	iii) Breit-Wigner, \\
	iv) Bohr.
}\label{sec:4.a.31}
La funzione sarebbe la Moliere.

\subsection[\hspace{1mm} Metodo di produzione di antiprotoni nell'esperimento di Segre]{Illustrare in modo qualitativo il metodo di produzione degli antiprotoni nell’esperimento di Segré et al.
}\label{sec:4.a.32}
Tramite l'acceleratore Bevatron si invia un fascio di protoni su una lamina di rame, l'urto rompe l'atomo in questione, si liberano $p, \pi^+, \overline{p}, \pi^-$. Gli antiprotoni liberati hanno impulso $p= 1.19$ MeV/c.
Successivamente le particelle vengono deviate dal campo del Bevatron per entrare nel sistema di rilevazione ideato per l'esperimento. Tra le particelle positive si ha in media circa 1 antiprotone ogni $10^{5}$ pioni. Con l'impulso di queste particelle si ha $\beta = 0.99$ per i pioni e $\beta=0.78$ per gli antiprotoni.

\subsection[\hspace{1mm} Metodo di separazione degli antiprotoni dal fondo di pioni nell'esperimento di Segre]{Spiegare qualitativamente il metodo di separazione degli antiprotoni dal fondo di pioni nell’esperimento di Segré et al. tramite contatori Cerenkov
}\label{sec:4.a.33}

Le particelle uscenti sono deflesse di $21^o$ dal campo magnetico del Bevatron e successivamente di altri $34^o$ dal campo magnetico di M1. Dopo M1 le particelle passano attraverso un collimatore Q1: un magnete quadripolare che focalizza il fascio ed uno scintillatore S1. 
\begin{figure}[H]
    \centering
    \incfig{segre}
    \caption{segre}
    \label{fig:segre}
\end{figure}
Dopo le particelle incontrano un secondo collimatore Q2 seguito da un altro magnete M2 che le deflette di altri $34^o$. Infine le particelle attraversano il secondo scintillatore S2, un primo contatore Cherenkov C1 che rileva particelle con $\beta>0.78$, un secondo contatore Cherenkov C2 con $0.75< \beta<0.78$ ed un ultimo scintillatore per rilevare le particelle che uscenti da C2 con angoli di scattering troppo grandi.
Il sistema di magneti permette di selezionare le particelle con un certo impulso, la velocità della particella invece è misurata in due modi indipendenti: tempo di volo e contatori Cherenkov.\\
Fare due misure indipendenti della velocità è indispensabile in quanto, utilizzando ad esempio il solo tempo di volo come controllo si otterrebbe una distribuzione di eventi del tipo:
\begin{figure}[H]
    \centering
    \incfig{distribuzione-pioni-antiprotoni}
    \caption{Distribuzione pioni-antiprotoni}
    \label{fig:distribuzione-pioni-antiprotoni}
\end{figure}

È quindi richiesta la coincidenza di eventi $S_1 * S_2 * C_2$, se scatta anche lo scintillatore $S_3$ allora il conteggio è annullato perchè vuol dire che la particella aveva un grande angolo di scattering (si evitano errori dovuti a particelle che non erano ben collimate con il gruppo). Con questo metodo di controlli incrociati si riesce ad escludere quasi tutti i $\pi$ dal conteggio (il 3 \% di questi attiva C2), per i pochi pioni che attivano C2 si utilizza come controllo C1: se si attiva quest'ultimo il conteggio viene annullato. In questo modo è stato possibile ridurre il fondo di pioni e distinguere il segnale dell'antiprotone tramite oscilloscopi (presumibilmente attivati dalle scintillazioni della strumentazione).

\subsection[\hspace{1mm} Esperimento di Anderson sulla scoperta del positrone]{Descrivere l’esperimento di Anderson sulla scoperta del positrone.
}\label{sec:4.a.34}
Nell'esperimento venne utizzata una camera a nebbia: un dispositivo isolato contenente vapore saturo. \\ 
Una particella carica e sufficientemente energetica che attraversa la camera ionizza il gas sulla sua traiettoria e gli atomi ionizzati formano dei nuclei di condensazione su cui si formano delle goccioline. Fotografando la camera è quindi possibile ricostruire la traiettoria della particella incidente. Se inoltre immergiamo il tutto in un campo magnetico possiamo misurare anche l'impulso a partire dal raggio di curvatura.
\[
	R = \frac{mV}{qB}
.\] 
Anderson utilizzo un campo magnetico di 1.7 T e registrò la traccia di una particella che attraversa 6 m di piombo. Le sue osservazioni furono che la particella prima di attraversare la lastra aveva un impulso $p_{i}=63$ MeV/c, dopo l'attraversamento invece $p_{f}=22.5$ MeV/c, dal verso di curvatura dedusse che la particella aveva carica positiva.\\
Si poteva quindi avanzare l'ipotesi che la particella fosse un protone non relativistico, se non fosse che in tal caso:
\[
	E_{i}= \frac{p^2}{2m_{p}}\approx 2.1 \text{ MeV}
.\] 
Quindi non avrebbe sufficiente energia per attraversare uno strato di 6 mm di piombo.\\
Ipotizzando invece che fosse un positrone relativistico questo perderebbe circa 7 MeV per ionizzazione e la sua energia si riduce di un fattore $1 /e$ per irraggiamento. Il risultato è quindi compatibile con l'impulso misurato nell'esperimento: era un positrone relativistico.

