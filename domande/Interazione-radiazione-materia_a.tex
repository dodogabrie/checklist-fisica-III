\subsection[]{Quale e' la relazione fra lo spessore di un materiale, espresso in cm, e lo spessore espresso in g/cm$^2$ ?
} \label{sec:4.a.1}
La relazione tra $X$ [cm] e $X'$ [g/cm$^2$] consiste in:
\[
	X\rho = X'
.\] 
Dove $\rho$ è la densità del mezzo, il vantaggio di esprimere uno spessore in g/cm$^2$ è che in questa unità lo spessore risulta indipendente dallo stato (solido, liquido, gassoso) del materiale che si usa (solitamente per rilevare le radiazioni di particelle che attraversanola materia).
\subsection[]{Spiegare qualitativamente l’effetto fotoelettrico e lo scattering Compton, indicandone le differenti caratteristiche
}\label{sec:4.a.2}
L'effetto fotoelettrico constiste nella interazione:
\[
	\ce{\ce{\gamma} +  \text{atomo} -> \ce{e^-} + \text{atomo}+}
.\] 
In pratica un fotone con energia $E$ cede tutta la sua energia ad un elettrone atomico che viene espulso dall'atomo con energia $E-E_{\text{lev}}$ dove $E_{\text{lev}}$ è l'energia del livello a cui si trova l'elettrone. La soglia di energia per far avvenire il processo è quindi $E_{\text{lev}}$, mentre il range di energia in cui il processo è il prevalente è tra $E^{\text{min}}_{\text{lev}}$ e $E^{\text{max}}_{\text{lev}}$. Per il carbonio l'energia di livello massima è $10$ keV mentre per il piombo arriva anche a $500$ keV.\\
L'effetto compton invece sussiste in un urto elastico tra un fotone ed un elettrone, considerato libero:
\[
	\ce{\ce{\gamma} + e- -> \ce{\gamma} + e- }
.\] 
In cui vi può essere uno scambio di energia tra il fotone e l'atomo. Questo è l'effetto prevalente per energie del fotone incidente dell'ordine dei MeV ed a differenza del Rayleigh la sua sezione d'urto va come $Z$.
\subsection[]{Dare l'espressione ed il valore numerico della lunghezza d'onda Compton.
}\label{sec:4.a.3}
Visto che nell'effetto Compton vi è scambio di energia tra il fotone e l'elettrone allora la frequenza del fotone dopo lo scattering risulta cambiata, applicando la conservazione dell'energia e dell'impulso si ottiene che:
\[
	\omega-\omega'= \frac{\hbar \omega\omega'}{mc^2}\left( 1-\cos\theta \right) 
.\]
Dove $\omega$ è la frequenza del fotone prima dello scattering, $\omega'$ la frequenza dopo lo scattering, $\theta$ l'angolo di scattering.
Esprimendo tale relazione in termini di $\lambda= 2\pi c/ \omega$ si ha che:
\[
	\lambda'-\lambda= \lambda_{c}\left( 1-\cos\theta \right) 
.\] 
Abbiamo quindi definito la lunghezza d'onda Compton dell'elettrone:
\[
	\lambda_{c}=\frac{h}{mc}
.\] 
\subsection[]{Spiegare qualitativamente lo scattering Rayleigh
}\label{sec:4.a.4}
Lo scattering Rayleigh corrisponde all'interazione elastica tra un fotone ed un atomo. È un effetto importante fino ad energie $\sim 1$ keV (sezione d'urto che va come $Z^2$), dopo la sua sezione d'urto decresce come il reciproco del quadrato dell'energia.

\subsection[]{Spiegare qualitativamente il fenomeno della creazione di coppie  da parte di un raggio gamma che incide su un atomo.
}\label{sec:4.a.5}
La creazione di coppie 
\[
	\ce{\ce{\gamma} -> e+ + e-} 
\]
è cinematicamente proibita, infatti prima della reazione il sistema del centro di massa non esiste, dopo si, ma la velocità di tale sistema non cambia dopo l'urto (o più in generale, il centro di massa prima dell'urto non era definito, quindi non può essere scritta una legge di conservazione).
C'è quindi bisogno di una particella spettatrice carica.
Supponiamo che vi sia un atomo spettatore, si ha:
\[	
	\ce{\ce{\gamma} + \text{atomo} -> e+ + e- +\text{atomo}} 
.\] 
Se $M$ è la massa di tale atomo allora l'energia di soglia per il fotone è (conservazione del modulo del quadrimpulso):
\begin{align*}
	\left( E_{\gamma}+M \right)^2-E^2_{\gamma}=\left( 2m_e + M \right)^2
.\end{align*}
\[
	E_{\gamma}\ge 2m_e\left( 1+ \frac{m_e}{M} \right) 
.\] 
In particolare $E_{\gamma}\ge 2m_e \approx 1$ MeV, da cui la necessità di avere un raggio $\gamma$.


\subsection[]{Descrivere qualitativamente l’effetto Cherenkov e dimostrare tramite il principio di Huygens che la radiazione Cherenkov è emessa ad un solo angolo.
}\label{sec:4.a.6}
L'effetto Cherenkov consiste nella emissione di onde elettromagnetiche da parte di una particella carica che si muove a velocità $v$ superiore a quella della luce.
\[
	\beta c> \frac{c}{n}
.\] 
Con $n$ parte reale dell'indice di rifrazione. L'emissione è dovuta all'irraggiamento coerente di dipoli elettrici che sono messi in oscillazione dal campo elettromagnetico generato dalla particella nel suo passaggio.\\
Analizziamo nel dettaglio la direzione di propagazione di quest'onda utilizzando il principio di Huygens.\\
Ipotizziamo che al tempo $t'$ si propaghi dalla particella un'onda elettromagnetica con fronte d'onda sferico. Dopo un tempo $\Delta t$ tale fronte ha un raggio 
\[
	R = \frac{c}{n} \Delta t
.\] 
Tra il tempo in cui l'onda discussa ha iniziato a propagarsi ($t'$) ed il tempo in cui ha raggiunto il raggio $R$ ($t = t'+ \Delta t$) si propagano perturbazioni sferiche del tutto analoghe a quella sopra, che al tempo $t$ avranno un raggio intermedio tra $R$ e $0$.\\
Di conseguenza la perturbazione elettromagnetica si sviluppa su un fronte che è la sovrapposizione di tutte le superfici sferiche di onde formatesi nel tempo $\Delta t$ (effetto motoscafo).
Chiamiamo la direzione di propagazione dell'onda $\theta$, ci aspettiamo che il coseno di quest'angolo abbia proporzionalità inversa da $\beta$ e da $n$. In questo modo all'aumentare di $n$ l'effetto risulta più evidente (luce "più lenta") mentre al diminuire di $\beta$ l'effetto tenda a svanire (si torna in situazioni in cui le cose vanno più lentamente della luce). 
 \[
	\cos\theta = \frac{1}{\beta n}
.\] 
Ipotizziamo che in un punto $P$ vi sia un osservatore che rivela le onde elettromagnetiche, la situazione sarebbe questa:
\begin{figure}[H]
    \centering
    \incfig{cherenkov-huygens}
    \caption{Schema di osservazione dell'effetto Cherenkov}
    \label{fig:cherenkov-huygens}
\end{figure}
Nell'immagine l'osservatore in $P$ rileva l'onda arrivare al tempo $t$, quando la particella è "già oltre", in un certo senso possiamo pensare che l'osservatore sente prima il dolore del proiettile che il rimbombo della pistola.\\
Facendo riferimento alla figura \ref{fig:cherenkov-huygens} dimostriamo la relazione angolare citata sopra.
\begin{align*}
	&R = \left| \bs{R} \right| = \frac{c}{n} \Delta t	&\bs{\rho} + \bs{\beta} c \Delta t=\bs{R} 
.\end{align*}
Facendo il modulo quadro della relazione vettoriale si ottiene:
\[
	R^2=\frac{c^2}{n^2}\Delta t^2=\beta^2c^2\Delta t^2 + \rho^2 + 2\rho\beta c\Delta t \cos\alpha 
.\] 
Adesso dividiamo per $\rho^2$ questa espressione, ciò che si ottiene è una equazione di secondo grado nell'incognita $\frac{\Delta t}{\rho}$ con dei coefficenti interessanti:
\[
	c^2\left( \beta^2- \frac{1}{n^2} \right) \left( \frac{\Delta t^2}{\rho^2} \right) + 2\beta c \cos\alpha\left( \frac{\Delta t}{\rho} \right) +1=0
.\] 
I coefficenti sono interessanti perchè, per ipotesi noi abbiamo assunto che la particella viaggia ad una velocità tale che $\beta>\frac{1}{n}$, condizione che ritroviamo espressa nel segno del primo coefficente. Quindi la soluzione della equazione esiste se:
\[
	\frac{\Delta}{4}=\beta^2c^2\cos^2\alpha - c^2\left( \beta^2-\frac{1}{n^2} \right) \ge 0
.\] 
Quindi serve avere $\sin\alpha\le \frac{1}{\beta n}$ e, imponendo anche $\Delta t>0$ si deve avere $\cos\alpha<0$ (si prende il cono all'indietro, non quello in avanti).
Quindi abbiamo due soluzioni distinte per ogni punto interno al cono delimitato dai vincoli su $\alpha$.\\
Con il principio di Huygnes sappiamo però che le soluzioni interne al cono non troveranno sovrapposizione coerente di onde sferiche, cosa che invece si ottiene sul bordo del cono dove tutte le onde sferiche avanzano coerentemente a formare una unica perturbazione.
Ci concentriamo quindi sulla soluzione al bordo ($\sin \alpha = \frac{1}{\beta n}$).\\
Facciamo una considerazione geometrica su questa particolare scelta dell'angolo $\alpha$:
\begin{figure}[H]
    \centering
    \incfig{geometria-cherenkov}
    \caption{Situazione geometrica della radiazione cherenkov}
    \label{fig:geometria-cherenkov}
\end{figure}
Sfruttando un teorema sui triangoli:
\[
	\frac{R}{\sin\left( \pi-\alpha \right) }= \frac{\Delta x}{\sin\hat{P}}
.\]
E inserendo i valori di $\Delta x$, $R$ usati sopra si dimostra che l'angolo sotteso al punto $P$ è rettangolo per $\sin\alpha= \frac{1}{\beta n}$.\\
Quindi possiamo ricavare l'angolo Cherenkov semplicemente dalle proprietà dei triangoli rettangoli:
\[
	\theta_{c}=\frac{\pi}{2}-\left( \pi-\alpha \right) = \alpha-\frac{\pi}{2}
.\] 
In particolare
\[
	\sin\alpha =\sin\left( \theta_{c}+ \frac{\pi}{2} \right) = \cos\theta_{c}=\frac{1}{\beta n}
.\] 
Abbiamo quindi trovato l'angolo Cherenkov, indica la direzione dal quale l'osservatore in $P$ vede arrivare l'onda.



\subsection[]{Descrivere la situazione in cui la legge 
\[
	\frac{\mbox{d} N_{\gamma}}{\mbox{d} E_{\gamma}\text{d}x} = z^2 \frac{\alpha}{\hbar c} \sin^2\theta_{c}
.\] 
inerente la radiazione Cherenkov, è applicabile e spiegare il significato e l'unità di misura di ogni grandezza fisica ivi indicata.
}\label{sec:4.a.7}
Quella espressa è detta la formula di Frank-Tamm, si applica quando una particella di carica $z$ (in unità di $e$) attraversa uno spessore $dx$ infinitesimo di materiale con indice di rifrazione $n\left( \omega \right)$ ed esprime il numero di fotoni emessi per unità di energia $E_{\gamma}= \hbar \omega$. 
Questa equazione si riscrive in termini dell'indice di rifrazione sostituendo l'equivalente dell'angolo Cherenkov discusso nella domanda precedente:
\[	
	\frac{\mbox{d} N_{\gamma}}{\mbox{d} E_{\gamma}\text{d}x} = z^2 \frac{\alpha}{\hbar c} \left( 1- \frac{1}{\beta^2 n^2} \right)  
.\] 
Ricordiamo adesso alcuni valori delle costanti in gioco:
\begin{align*}
	&\alpha = \frac{e^2}{\hbar c}= \frac{1}{137}	&\hbar c = 197 \text{ MeV}\cdot \text{fm}
.\end{align*}
Quindi \[
	\frac{\alpha}{\hbar c}= \frac{10^{-6}\cdot 10^{13}}{137\cdot 197} \left[ \frac{\text{fotoni}}{\text{eV}\cdot \text{cm}} \right] \approx 
	370 \left[ \frac{\text{fotoni}}{\text{eV}\cdot \text{cm}} \right]
.\] 
Ricordiamo anche che l'indice di rifrazione è in realtà $n\left( \omega \right) $, da cui deriva la dipendenza della formula da $E_{\gamma}$.

\subsection[]{Descrivere la situazione in cui la legge 
\[
	N_{\gamma}= z^2 \frac{\alpha}{\hbar c}L \int_{E_1}^{E_2} \left[ 1 - \frac{1}{\beta^2 \epsilon_{r}\left( E \right)}\right]P_{\text{det}}dE 	
.\] 
inerente la radiazione Cherenkov, è applicabile e spiegare il significato e l'unità di misura di ogni grandezza fisica ivi indicata.
}\label{sec:4.a.8}
Le quantità sono le stesse della domanda precedente, tuttavia questa esprime il numero di fotoni rilevati da un detector avente una efficienza $P_{\text{det}}$ nel range di energia tra $E_1$ ed $E_2$, il mezzo attraversato ha inoltre larghezza $L$.
\subsection[]{Descrivere qualitativamente le cause e gli effetti del fenomeno della radiazione di frenamento da parte di una particella carica nella materia.
}\label{sec:4.a.9}
Una particella carica di massa $M$ e carica $q=ze$ che attravera un materiale subisce l'interazione coulombiana con i nuclei che lo compongono di carica $Ze$.\\
Per ogni interazione la particella accelera (a causa della repulsione o attrazione) e quindi irraggia: Radiazione di frenamento o Bremsstrahlung.
\begin{figure}[H]
    \centering
    \incfig{bremsstrahlung}
    \caption{Schema dello scattering columbiano per la Bremsstrahlung}
    \label{fig:bremsstrahlung}
\end{figure}
La perdita di energia dovuta a questo irraggiamento è importante per elettroni (aventi una certa energia critica $E_{c}$) ed è invece trascurabile per le altre particelle. Inoltre la radiazione di frenamento avviene in modo indipendente su ciascun nucleo del materiale attraversato, è allora facile immaginare che darà luogo a processi di emissione incoerenti.\\
Nel sistema del laboratorio la radiazione può esssere rilevata in modi diversi a seconda della velocità del proiettile:
\paragraph{Particella non relativistica}%
La radiazione è proporzionale a $\sin^2\alpha$, con $\alpha$ angolo tra la media della accelerazione nel processo e la direzione di osservazione.
\begin{figure}[H]
    \centering
    \incfig{bremsstrahlung-non-relativistica}
    \caption{Emissione della radioazione nel caso di Bremsstrahlung non relativistica}
    \label{fig:bremsstrahlung-non-relativistica}
\end{figure}
\paragraph{Particella ultrarelativistica}%
La radiazione viene rilevata in avanti, con angolo di $\frac{1}{\gamma}$ rispetto alla direzione di $\bs{\beta}$ (si dimostra con una trasformazione dal sistema in cui la particella è ferma a quello del laboratorio).
\begin{figure}[ht]
    \centering
    \incfig{bremsstrahlung-relativistica}
    \caption{Emissione della radioazione nel caso di Bremsstrahlung relativistica}
    \label{fig:bremsstrahlung-relativistica}
\end{figure}

\subsection[]{Spiegare perchè la perdita di energia per irraggiamento è significativa per elettroni e non per le altre particelle cariche.
}\label{sec:4.a.10}
La forza di Coulomb dipende dalla carica elettrica e non dalla massa della particella, quindi dall'equazione del moto si ottiene che:
\[
	a \sim \frac{1}{M}
.\]
Visto che l'irraggiamento dipende dall'accelerazione al quadrato si ha anche che:
\[
	I\sim \frac{1}{M^2}
.\] 
e banalmente questo è l'unico motivo per cui gli elettroni irraggiano maggiormente di Bremsstrahlung.

\subsection[]{Descrivere la situazione in cui la legge
\[
	\frac{\mbox{d}I_{\omega}}{\mbox{d}\Omega}=
	\frac{q^2}{4\pi^2c}\left|\int\frac{\hat{n}\wedge\left[\left(\hat{n}-\bs{\beta}\right)\wedge\dot{\bs{\beta}}\right]}{\left(1-\hat{n}\cdot\bs{\beta}\right)^2}
	e^{i\omega\left(t'-\frac{\bs{r}'\cdot\hat{n}}{c}\right)}dt'\right|^2 
.\] 
è applicabile e spiegare il significato e l'unità di misura di ogni grandezza fisica ivi
indicata.
}\label{sec:4.a.11}
L'equazione descrive l'energia irraggiata per unità di frequenza ed angolo solido, deriva dalla più generale 
\[
	\frac{\mbox{d} I_{\omega}}{\mbox{d} \Omega} = \frac{\epsilon_0 c r^2}{\pi} \left| \bs{E}\left( \bs{r}, \omega \right)  \right| ^2
.\] 
Dove $\bs{E}\left( \bs{r}, \omega \right)$ è la trasformata del campo elettrico di radiazione (valida per ogni tipo di irraggiamento).
Nel caso in questione è quello di una carica puntiforme accelerata in moto relativistico, tale equazione è stata dimostrata nella \hyperref[sec:3.a.14]{Domanda 3.a.14}.
\subsection[]{Descrivere la situazione in cui la legge
\[
	I_{\omega}\left( b \right) = 
	\begin{cases}
		\frac{8z^4Z^2 \alpha \hbar c^2}{3\pi} \left( \frac{m_e}{M} \right) ^2 \frac{r^2_e}{V^2} \frac{1}{b ^2}  & \text{per } \omega<\frac{V}{b}\\
		0 & \text{per } \omega> \frac{V}{b}
	\end{cases}
.\] 
applicabile e spiegare il significato e l'unità di misura di ogni grandezza fisica ivi indicata.
}\label{sec:4.a.12}
$I_{\omega}\left( b \right) $ indica l'energia irraggiata per unità di frequenza in funzione del parametro di impatto (J$\cdot$s). Tutte le altre quantità sono state ampliamente preserntate nel corso della lettura ma facciamo un ripassino su quelle che non vediamo da un pò:
\begin{align*}
	&r_e = k_0 \frac{e^2}{m_e c^2} \approx 2.82 \text{ fm}\\
	&m_e = 0.511 \text{MeV/c}^2 = 9.12 \cdot 10^{-31} \text{kg}
.\end{align*}
Detto questo la formula rappresenta $I_{\omega}$ per una Bremstrahlung nel caso non relativistico nella approssimazione che il moto complessivo della carica possa essere assunto come rettilineo, quindi angoli di scattering piccoli. In tal modo viene assunta la variazione di quantità di moto nella sola direzione ortogonale alla direzione di incidenza del proiettile ed i conti per la dimostrazione diventano pochi.\\
È stata inoltre approssimata la funzione ad uno scalino, quando in realtà i termini sopra esrpessi corrispondono all'andamento asintodico di quest'ultima.

\subsection[]{Definire la sezione d'urto di irraggiamento e spiegare il suo significato fisico.
}\label{sec:4.a.13}
La sezione d'urto di irraggiamento è definita come
\[
	\chi_{\omega}= \int_{b_{min}}^{b_{max}} 2\pi I_{\omega}b db \quad \left[ \frac{\text{J}}{\text{Hz}}\cdot \text{m}^2 \right] 
.\] 
ed è la grandezza che moltiplicata per il numero di nuclei per unità di superficie fornisce l'energia irraggiata per intervallo di frequenza $I_{\omega}$.

\subsection[]{Descrivere la situazione in cui la legge
	\[
		\chi_{\omega}=\frac{16 z^4 Z^2 \alpha \hbar c^2}{3} \left( \frac{m_e}{M} \right)^2 \frac{r_e^2}{V^2} \ln\left( \frac{MV^2}{\hbar \omega} \right) 
	.\] 
e spiegare il significato e l'unità di misura di ogni grandezza fisica ivi indicata.
}\label{sec:4.a.14}
Questa sezione d'urto di irraggiamento è quella derivante dalla equazione della domanda precedente integrata sul parametro di impatto. Questa è applicabile nel caso in cui $\omega$ dello scattering rispetti la relazione (stime dovute a considerazioni quantistiche) \[
	\frac{V}{a}< \omega < \frac{E_{c}}{\hbar}
.\] 
Ovvero in termini di $b$ :
\[
	\frac{\hbar}{M V} < b < \frac{V}{\omega}
.\] 
Con $a$ raggio atomico dell'atomo su cui si incide, $E_{c}$ energia critica di irraggiamento e $M$ la massa della particella incidente.\\
La prima disuguaglianza deriva dal fatto che la particella deve avere energia necessaria a produrre un fotone, abbiamo visto nella \hyperref[sec:3.a.15]{Domanda 3.a.15} che i campi di una particella in moto hanno tempi di interazione dell'ordine di ($\gamma \approx 1$)\[
	\Delta t \approx \frac{b}{V} 
.\] 
quindi la trasformata del campo sarà definita in una ampiezza spettrale di: \[
	\omega = \frac{1}{\Delta t} \approx \frac{V}{b}
.\] 
Questa da il limite inferiore alla prima relazione (su $\omega$) e ci aiuta per il limite inferiore della seconda (su $b$): l'energia cinetica della particella in arrivo avrà come limite inferiore l'energia necessaria a produrre un fotone con la frequenza definita sopra: \[
	MV^2>\hbar \omega \implies \frac{MV^2}{\hbar}>\omega \implies \frac{MV}{\hbar} > \frac{1}{b}
.\] 
\paragraph{Nota} La trattazione fatta sopra del limite inferiore è assolutamente grossolana ed erronea, da prendere solo come riferimento mnemonico.

Come limite superiore si ha proprio la relazione che lega $b$ alla frequenza che non può oltrepassare per motivi di tempi di interazione.\\
Nel caso di elettroni relativistici il parametro di impatto inferiore è proprio la lunghezza d'onda Compton.

\subsection[]{Descrivere la situazione in cui la legge
\[
	\frac{\mbox{d}E^{\text{irr}}}{\mbox{d}x}=
	n_{\text{nuclei}}\frac{16}{3}z^4Z^2\alpha\left(\frac{m_e}{M}\right)^2r_e^2\ln\left(\frac{192}{Z^{1 /3}}\frac{M}{m_e}\right) E
.\] 
applicabile e spiegare il significato e l'unità di misura di ogni grandezza fisica ivi indicata.
}\label{sec:4.a.15}
La formula descrive l'energia irraggiata per unità di lunghezza da una particella ultrarelativistica ($V \sim c$) con energia iniziale $E$ e massa $M$, questa espressione si ottiene integrando la sezione d'urto di irraggiamento di tale particella nel dominio detto di Screening attivo, ricaviamo intanto tale sezione d'urto. Nel sistema $\Sigma'$ in cui la particella è a riposo si ha che durante lo Scattering (in cui il nucleo va incontro alla particella) $M$ irraggia in manera non relativistica, quindi la formula per la $\chi'_{\omega}$ è:
\[
	\chi'_{\omega}=\frac{16}{3}z^4Z^2\alpha\hbar c^2 \left( \frac{m_e}{M} \right)^2  \frac{r^2_e}{V^2}\ln\left( \frac{b_{\text{MAX}}}{b_{\text{min}}} \right) 
.\] 
È bene ricordare che per questa interazione avremo una larghezza spettrale di \[
	\omega'< \gamma \frac{c}{b}
\] e che le frequenze hanno limite superiore dettato dal fatto che la particella non può irraggiare di più della sua energia a riposo:
\[
	\omega'< \frac{M c^2}{\hbar}
.\] 
Utiilizziamo nel nostro modello un equivalente della lunghezza d'onda Compton per la nostra particella come il minimo parametro di impatto (motivo abbozzato nella domanda precedente)
\[
	b_{\text{min}}= \frac{\hbar}{Mc}
.\] 
Come limite superiore possiamo mettere il raggio atomico, ovvero il limite oltre il quale non vi è più nessuna interazione coulombiana con il nucleo a causa dello schermaggio degli elettroni:
\[
	b_{\text{MAX}}=a= 1.4 \frac{a_0}{Z^{1 /3}}
.\] 
con $a_0$ raggio classico di Bhor:
 \[
	 a_0 = \frac{\hbar}{m_e c \alpha}
.\] 
Lo screening attivo citato sopra entra in gioco prorpio qui, infatti il limite superiore utilizzato vale soltanto se è vera la disuguaglianza che lega l'energia della particella a riposo alla energia che può emettere alla estremità dell'atomo \[
	Mc^2 < \hbar \omega'\left( a \right)  \implies \frac{M c^2}{\hbar} < \gamma \frac{c}{a}
.\]
\paragraph{Nota}%
Se questa ultima scritta non fosse rispettata allora non saremmo in screening attivo e dovremmo mettere limite massimo \[b_{\text{MAX}} = \gamma \frac{c}{\omega'}\]
Nel nostro caso invece si ha una condizione su $\gamma$
\[
	\gamma > \frac{M c^2 a}{\hbar c} = \frac{M c^2 \cdot 1.4}{\hbar c}\cdot \frac{\hbar}{c m_e \alpha}\cdot \frac{1}{Z^{1 /3}} = \frac{192.5 \cdot M}{Z^{1 /3}m_e}
.\] 
Inserendo tutto nella sezione d'urto si ottiene l'espressione:
\[
	\chi'_{\omega}=\frac{16}{3}z^4Z^2\alpha\hbar c^2 \left( \frac{m_e}{M} \right)^2  \frac{r^2_e}{c^2}\ln\left( \frac{192\cdot M}{Z^{1 /3}\cdot m_e} \right) 
.\] 
Adesso basta notare che, avendo $\omega = \gamma \omega'$ sarà $\chi_{\omega}= \chi'_{\omega}$ perchè la sezione trovata è indipendente da $\omega'$, se ne conclude che:
\[
	\frac{\mbox{d} E^{\text{irr}}}{\mbox{d} x} = n_{\text{nuclei}} \int_{0}^{\omega_{\text{max}}} \chi_{\omega} d\omega = 
	n_{\text{nuclei}} \int_{0}^{E /\hbar} d\omega \chi_{\omega} 
	= n_{\text{nuclei}} \frac{E}{\hbar} \chi_{\omega}
.\] 
Che è proprio l'espressione cercata.
\subsection[]{Descrivere la situazione in cui la legge
\[
	\chi_{\omega} = \hbar \omega \frac{\text{d}\sigma_{y}}{\text{d}\omega}
.\] 
 è applicabile e spiegare il significato e l'unità di misura di ogni grandezza fisica ivi indicata.
}\label{sec:4.a.16}
La formula è una manipolazione algebrica della definizione di $\chi_{\omega}$:
\[
	\chi_{\omega}= \frac{1}{n} \frac{\mbox{d} E_{\text{irr}}}{\mbox{d} x \text{d}\omega} 
.\] 
Con $n$ densità dei centri scatteranti. Proviamo a manipolare l'espressione:
\[
	\chi_{\omega} = \hbar \omega\frac{\mbox{d}}{\mbox{d} \omega} \left( \frac{\mbox{d}E_{\text{irr}}}{\mbox{d}t}\frac{1}{n\cdot \hbar \omega \frac{\mbox{d} x}{\mbox{d} t} } \right)  
.\] 
Possiamo allora definire una corrente di particelle proiettili (l'inverso del secondo termine in parentesi) a moltiplicare la potenza irraggiata mediata (è mediata perchè $E_{\text{irr}}$ è l'energia irraggiata totale), questa altro non è che la sezione d'urto del processo di irraggiamento (definita come una superficie, non come quella finta sezione d'urto $\chi_{\omega}$).
\[
	\chi_{\omega}= \hbar \omega \frac{\mbox{d} }{\mbox{d} \omega} \left( \sigma_{y} \right) 
.\] 
\subsection[]{Dare la definizione di lunghezza di radiazione.
}\label{sec:4.a.17}
\subsection[]{Dare la definizione di "Energia critica"
}\label{sec:4.a.18}
\subsection[]{Descrivere la situazione in cui la legge E = E0 
 è applicabile e spiegare il significato e l'unità di misura di ogni grandezza ivi indicata.
}\label{sec:4.a.19}
\subsection[]{Descrivere la situazione in cui la legge
applicabile e spiegare il significato e l'unità di misura di ogni grandezza ivi indicata. [formula di Tsai non dimostrata a lezione]
}\label{sec:4.a.20}
\subsection[]{Descrivere qualitativamente il meccanismo della perdita di energia per collisioni da parte di una particella carica nella materia.
}\label{sec:4.a.21}
\subsection[]{Spiegare il significato di ogni termine dell’espressione per la perdita di energia per collisioni (formula di Bethe-Bloch, non dimostrata):
}\label{sec:4.a.22}
\subsection[]{Disegnare qualitativamente la funzione di Bethe-Bloch indicando i valori dei punti significativi.
}\label{sec:4.a.23}
\subsection[]{Definire il "percorso residuo" ("range") per una particella carica in un materiale.
}\label{sec:4.a.24}
\subsection[]{Definire gli "stopping power" totale, collision, radiative e nuclear; indicare per quali particelle ognuno di essi sia o meno rilevante.
dE
}\label{sec:4.a.25}
\subsection[]{Come si calcola il "percorso residuo" ("range"), nota la curva 
	, in funzione dxdell'energia E della particella?
}\label{sec:4.a.26}
\subsection[]{Spiegare qualitatativamente il cosiddetto "picco di Bragg"
}\label{sec:4.a.27}
\subsection[]{Descrivere qualitativamente il fenomeno dello scattering multiplo da parte di una particella carica in moto veloce nella materia.
}\label{sec:4.a.28}
\subsection[]{Definire l'angolo di multiplo scattering (rispetto alla direzione iniziale della particella) e definire la sua proiezione su un piano (che contiene la direzione iniziale della particella). Indicare i limiti delle due variabili cosí definite.
}\label{sec:4.a.29}
\subsection[]{Spiegare il significato di ogni termine dell'espressione per l’angolo quadratico medio di multiplo scattering (rispetto alla direzione iniziale della particella)
}\label{sec:4.a.30}
\subsection[]{31.. Se non fosse sufficiente la approssimazione di piccoli angoli e distribuzione gaussiana, indicare quale fra le seguenti funzioni descriverebbe meglio il fenomeno del multiplo scattering: i) Bethe-Bloch, ii) Moliere, iii) Breit-Wigner, iv) Bohr.
}\label{sec:4.a.31}
\subsection[]{Illustrare in modo qualitativo il metodo di produzione degli antiprotoni nell’esperimento di Segré et al.
}\label{sec:4.a.32}
\subsection[]{Spiegare qualitativamente il metodo di separazione degli antiprotoni dal fondo di pioni nell’esperimento di Segré et al. tramite contatori Cerenkov
}\label{sec:4.a.33}
\subsection[]{Descrivere l’esperimento di Anderson sulla scoperta del positrone.
}\label{sec:4.a.34}
