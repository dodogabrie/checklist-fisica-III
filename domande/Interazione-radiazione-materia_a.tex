\subsection{Quale e' la relazione fra lo spessore di un materiale, espresso in cm, e lo spessore espresso in g/cm2 ?
}
\subsection{Spiegare qualitativamente l’effetto fotoelettrico e lo scattering Compton, indicandone le differenti caratteristiche
}
\subsection{Dare l'espressione ed il valore numerico della lunghezza d'onda Compton.
}
\subsection{Spiegare qualitativamente lo scattering Rayleigh
}
\subsection{Spiegare qualitativamente il fenomeno della creazione di coppie  da parte di un raggio gamma che incide su un atomo.
}
\subsection{Descrivere qualitativamente l’effetto Cherenkov e dimostrare tramite il principio di Huygens che la radiazione Cherenkov è emessa ad un solo angolo.
}
\subsection{Descrivere la situazione in cui la legge 
	, inerente la radiazione Cherenkov, è applicabile e spiegare il significato e l'unità di misura di ogni grandezza fisica ivi indicata.
}
\subsection{Descrivere la situazione in cui la legge 
	, inerente la radiazione Cherenkov, è applicabile e spiegare il significato e l'unità di misura di ogni grandezza fisica ivi indicata.
}
\subsection{Descrivere qualitativamente le cause e gli effetti del fenomeno della radiazione di frenamento da parte di una particella carica nella materia.
}
\subsection{Spiegare perchè la perdita di energia per irraggiamento è significativa per elettroni e non per le altre particelle cariche.
}
\subsection{Descrivere la situazione in cui la legge
applicabile e spiegare il significato e l'unità di misura di ogni grandezza fisica ivi
indicata.
}
\subsection{Descrivere la situazione in cui la legge
applicabile e spiegare il significato e l'unità di misura di ogni grandezza fisica ivi
indicata.
}
\subsection{Definire la sezione d'urto di irraggiamento e spiegare il suo significato fisico.
}
\subsection{Descrivere la situazione in cui la legge
e spiegare il significato e l'unità di misura di ogni grandezza fisica ivi indicata.
}
\subsection{Descrivere la situazione in cui la legge
applicabile e spiegare il significato e l'unità di misura di ogni grandezza fisica ivi
indicata.
}
\subsection{Descrivere la situazione in cui la legge
 è applicabile e spiegare il significato e l'unità di misura di ogni grandezza fisica ivi indicata.
}
\subsection{Dare la definizione di lunghezza di radiazione.
}
\subsection{Dare la definizione di "Energia critica"
}
\subsection{Descrivere la situazione in cui la legge E = E0 
 è applicabile e spiegare il significato e l'unità di misura di ogni grandezza ivi indicata.
}
\subsection{Descrivere la situazione in cui la legge
applicabile e spiegare il significato e l'unità di misura di ogni grandezza ivi indicata. [formula di Tsai non dimostrata a lezione]
}
\subsection{Descrivere qualitativamente il meccanismo della perdita di energia per collisioni da parte di una particella carica nella materia.
}
\subsection{Spiegare il significato di ogni termine dell’espressione per la perdita di energia per collisioni (formula di Bethe-Bloch, non dimostrata):
}
\subsection{Disegnare qualitativamente la funzione di Bethe-Bloch indicando i valori dei punti significativi.
}
\subsection{Definire il "percorso residuo" ("range") per una particella carica in un materiale.
}
\subsection{Definire gli "stopping power" totale, collision, radiative e nuclear; indicare per quali particelle ognuno di essi sia o meno rilevante.
dE
}
\subsection{Come si calcola il "percorso residuo" ("range"), nota la curva 
	, in funzione dxdell'energia E della particella?
}
\subsection{Spiegare qualitatativamente il cosiddetto "picco di Bragg"
}
\subsection{Descrivere qualitativamente il fenomeno dello scattering multiplo da parte di una particella carica in moto veloce nella materia.
}
\subsection{Definire l'angolo di multiplo scattering (rispetto alla direzione iniziale della particella) e definire la sua proiezione su un piano (che contiene la direzione iniziale della particella). Indicare i limiti delle due variabili cosí definite.
}
\subsection{Spiegare il significato di ogni termine dell'espressione per l’angolo quadratico medio di multiplo scattering (rispetto alla direzione iniziale della particella)
}
\subsection{31.. Se non fosse sufficiente la approssimazione di piccoli angoli e distribuzione gaussiana, indicare quale fra le seguenti funzioni descriverebbe meglio il fenomeno del multiplo scattering: i) Bethe-Bloch, ii) Moliere, iii) Breit-Wigner, iv) Bohr.
}
\subsection{Illustrare in modo qualitativo il metodo di produzione degli antiprotoni nell’esperimento di Segré et al.
}
\subsection{Spiegare qualitativamente il metodo di separazione degli antiprotoni dal fondo di pioni nell’esperimento di Segré et al. tramite contatori Cerenkov
}
\subsection{Descrivere l’esperimento di Anderson sulla scoperta del positrone.
}
