\subsection[]{ Definire le quantità $\beta$ e $\gamma$ per le trasformazioni di Lorentz.} 
Presi due sistemi di riferimento inerziali $O$ ed $O'$ si ha che $\beta$ è la velocità del sistema $O'$ rispetto ad $O$ (in unità di c) mentre:
\[
	\gamma = \frac{1}{\sqrt{1-\beta^2}}
.\] 

\subsection[]{ Definire il prodotto scalare di due 4-vettori.} 
Presi $x^{\mu} = (x^0,\boldsymbol{x})$ e $y^{\mu} = ( y^0, \boldsymbol{y})$ si definisce il loro prodotto come $x^{\mu}y_{\nu}$ come:
\[
x^{\mu}y_{\nu} = x^{0}y^{0}-\boldsymbol{x} \cdot \boldsymbol{y} 
.\] 
\subsection[]{ Definire il modulo di un 4-vettore.}
Se $x^{\mu}$ è un 4-vettore il suo modulo è definito secondo la metrica:
\[
|x|^2 = x^{\mu}x_{\mu}
.\] 
Dato che il tensore metrico non è definito positivo il modulo di un 4-vettore può esser positivo, negativo o nullo. 
\subsection[]{ Scrivere le trasformazioni di Lorentz per il boost lungo un asse (asse x).}
Per un boost lungo l'asse x si ha:
\[
\begin{cases}
	ct' = \gamma ( ct - \beta x) \\	
	x' = \gamma (x - \beta ct ) \\
	y' = y \\
	z' = z
\end{cases}
\]
o in forma matriciale:

\[
\left( \begin{array}{c} ct' \\ x' \\ y' \\ z' \end{array} \right)
= 
\begin{pmatrix}
	\gamma  & -\gamma \beta  & 0 & 0\\
	- \gamma \beta & \gamma & 0 & 0\\
	0 & 0 & 1 & \\
	0 & 0 & 0 & 1 \\
\end{pmatrix}
\left( \begin{array}{c} ct \\ x \\ y \\ z  \end{array} \right)
\] 

\subsection[]{ Definire le derivate in 4-dimensioni, la quadridivergenza, il differenziale di uno scalare di Lorentz, l’operatore di D’Alembert.}
Si definisce gli operatori $\partial^{\mu}$ e $\partial_{\mu}$ come:
\[
	\partial^{\mu} = \left( \frac{1}{c} \frac{\partial}{\partial t} , - \nabla  \right) ,\quad
	\partial_{\mu} = \left(\frac{1}{c} \frac{\partial}{\partial t} ,  \nabla  \right)
.\] 
Preso un generico campo tensoriale, la sua 4-divergenza (rispetto a qualche indice) è la contrazione tra l'operatore $\partial^{\mu}$ e l'indice stesso (se ques'ultimo è covariante) oppure tra l'operatore $\partial_{mu}$ e l'indice stesso (se quest'ultimo è controvariante). Ad esempio preso il campo vettoriale $v^{\mu} \equiv \left(v^{0}, \boldsymbol{v}\right)$ la sua 4-divergenza sarà:
\[
\partial_{\mu}v^{\mu} = \frac{1}{c} \frac{\partial v^{0}}{\partial t} + \nabla \boldsymbol{v} 
.\]
Se invece $\phi$ è un invariante di Lorentz il suo differenziale è:
 \[
	 d\phi = dx^{\mu}\partial_{\mu}\phi = \frac{\partial\phi}{\partial t}dt + \left(d \boldsymbol{x}\cdot \nabla \right)\phi  
.\]
L'operatore di D'Alembert invece è: 
\[
	\Box = \partial_{\mu}\partial^{\mu} = \frac{1}{c^{2}}\frac{\partial^{2}}{\partial t^{2}}    
.\] 
\subsection[]{ Definire il tempo proprio e dare la relazione (differenzale) fra tempo proprio e tempo nel sistema in cui si osserva il moto.}
Si sa che l'intervallo $ds^{2}$ è uno scalare di Lorentz, inoltre per un sistema solidale (o tangente) si ha $dx^{\mu} = \left( cd\tau, 0 \right)$ con $d\tau$ il tempo proprio infinitesimo. Si ha quindi: 
\[
		c^{2}d\tau^{2} = c^{2}d\tau^{2} - |dx|^{2} = c^{2}dt^{2} -|\boldsymbol{v}|^{2}dt^{2} \implies d\tau = \frac{dt}{\gamma} 
.\] 
\subsection[]{ Dare la definizione di invariante di Lorentz.} 
Un invariante di Lorentz è una grandezza che viene lasciata invariata dalle trasformazioni di Lorentz: è uguale in tutti i sistemi di riferimento inerziali.
\subsection[]{ Definire la 4-velocità ed il 4-impulso di un punto materiale di massa m, esprimere le loro unità di misura nei sistemi MKS e $\hbar$ = c = 1 , dimoststrare che il loro modulo è costante.} 
In MKS la 4-velocità ed il 4-impulso sono rispettivamente:
\[
	u^{\mu} = \frac{dx^{\mu}}{d\tau} = \left( \gamma c, \gamma \boldsymbol{v}  \right) \left( \frac{[m]}{[s]}\right) , \quad \quad 
	p^{\mu} = mu^{\mu} \left( \frac{[kg] \cdot [m]}{[s]} \right) 
.\]
Mentre in unità di $\hbar$ = c = 1 si ha che $u^{\mu}$ è adimensionale mentre $p^{\mu}$ di misura in $kg$.
Dimostriamo che il modulo di questi due è costante:
\[
	u^{\mu}u_{\mu} = \gamma^{2}\left( c^{2} - \boldsymbol{v^{2}} \right) = c^{2} 
\]
 \[
	p^{\mu}p_{\mu} = mc^2
\] 
\subsection[]{ Enunciare la legge di conservazione del 4-impulso.}
Per un sistema isolato (ovvero non sottoposto a forze esterne) il 4-impulso totale si conserva nel tempo.
\subsection[]{ Dare la definizione di 4-vettore covariante e controvariante; definire un tensore covariante di rango 2 e la sua traccia.} 
Un tensore covariante è il prodotto tensoriale di due quadrivettori covarianti. Di conseguenza un tensore covariante $F_{\mu \nu}$ trasforma sotto trasformazioni di Lorentz come:
\[
	F'_{\mu \nu} = \Lambda_{\mu}^{\alpha}\Lambda_{\nu}^{\beta} F_{\alpha \beta}  
.\]
La traccia di $F$ è 
\[
	F^{\mu}_{\mu} = g^{\mu \nu} F_{\mu \nu}
\] 
\subsection[]{Definire il tensore metrico $g_{\mu \nu}$}
Il tensore metrico si definisce come: $g_{\mu \nu} = $ diag$\left( 1,-1,-1,-1 \right) $ 
\subsection[]{Dare la definizione di tensore antisimmetrico di rango 2 ed indicare quali dei suoi elementi siano le componenti di un vettore polare e quali quelle di un vettore assiale tridimensionale.}
Un tensore antisimmetrico $F^{\mu \nu}$ di rango 2 è un tensore che cambia segno sotto scambio di indici:
\[
	F^{\mu \nu} = -F^{\nu \mu}
\]
Nel caso particolare in cui $F^{\mu \nu}$ è della forma:
\[
	F^{\mu \nu} = 
\left( 
\begin{array}{c|ccc}
	0 & v_{x} & v_{y} & v_{z} \\
	\hline
	-v_{x} & 0 & -w_{z} & w_{y} \\
	-v_{y} & w_{z} & 0 & -w_{x} \\
	-v_{z} & -w_{y} & w_{x} & 0
\end{array}
\right) 
\] 
Allora il vettore $\boldsymbol{v}$ è polare (invariante sotto parità) mentre il vettore $\boldsymbol{w}$ è assiale ("contro"variante sotto parità: cambia segno).
\subsection[]{ Definire quando una legge è scritta in modo relativisticamente covariante.}
Una legge fisica è scritta in modo relativisticamente covariante se è una uguaglianza tra due oggetti che trasformano allo stesso modo sotto cambi di sistema di riferimento, ossia se i due oggetti hanno gli stessi indici covarianti e controvarianti. 
\subsection[]{ Enunciare o ricavare la legge relativistica di composizione delle velocità.}
Supponiamo di avere un corpo puntiforme e siano $\boldsymbol{v}$ e $\boldsymbol{v'}$ le sue velocità nei sistemi inerziali $O$ e  $O'$.
Se  $\boldsymbol{w} = w \hat{x}$ è la velocità di $O'$ rispetto ad $O$ allora si ha 
\[
	v'_{x} = \frac{v_{x} - w}{1 - v_{x}w/c^{2}} \quad \quad
	v'_{y} = \frac{v_{y}}{\gamma \left( 1 - v_{x}w/c^{2} \right) } \quad \quad
	v'_{z} = \frac{v_{z}}{\gamma \left( 1 - v_{x}w/c^{2} \right) }
\]
con $\gamma = \frac{1}{\sqrt{1 - w^{2}/c^{2}}}$

\subsection[]{ Dimostrare che il modulo di un 4-vettore ed il prodotto di due 4-vettori sono invarianti di Lorentz.} 
È sufficiente dimostrare che il prodotto di due 4-vettori è invariante:
\[
	x'^{\mu} y'_{\mu} = \Lambda^{\mu}_{\alpha} g_{\mu \beta} \Lambda^{\beta}_{\gamma} x^{\alpha} y^{\gamma}
.\]
Inoltre il gruppo di Lorentz può esser definito come il gruppo che lascia invariato la metrica di Minkowsky, quindi:
\[
	\Lambda^{\mu}_{\alpha} g_{\mu \beta} \Lambda^{\beta}_{\gamma} = g_{\alpha \gamma} \implies x'^{\mu}y'_{\mu} = x^{\mu}y_{\mu}
\]  
\subsection[]{ Spiegare il paradosso dei gemelli.} 
Consideriamo i gemelli Bob e Alice. Supponiamo che la prima rimanga sulla terra (supposta sistema inerziale) e che Bob parta per una stella lontana a velocità costante. Per Alice l'orologio di Bob è rallentato dunque lei pensa che al ritorno di Bob ella sarà più giovane di lui.\\ 
Dal punto di vista di Bob è invece l'orologio di Alice ad essere rallentato (che nel suo sistema si allontana da lei alla velocità della nave) quindi pensa in maniera opposta ad Alice: crede che sarà lui il più giovane al suo ritorno.\\
Il paradosso nasce dalla erroneità della seconda affermazione (quella fatta da Bob): il sistema di Bob non può essere inerziale perchè dovrà necessariamente accelerare per tornare indietro. Quindi al ritorno Bob è più vecchio di Alice e, facendo un diagramma di Minkowsky, si vede che l'invecchiamento di Bob è tutto dovuto alla fase di accelerazione e decelerazione della nave.
\subsection[]{Dimostrare che l'operatore di D'Alembert è un invariante di Lorentz.} 
Se si considera l'operatore di D'Alembert come il prodotto di quadrivettori allora la dimostrazione è già stata effettuata nel punto $(1.a.15)$.
\subsection[]{Dimostrare che la 4-accelerazione e la 4-velocità sono perpendicolari.} 
Visto che $u^{\mu}u_{\mu} = c^{2}$ possiamo derivare a destra e sinistra rispetto al tempo proprio ottenendo:
\[
	0 = \frac{\mbox{d} u^{\mu}u_{\mu}}{\mbox{d} \tau} = u^{\mu}a_{\mu} + a^{\mu}u_{\mu} = 2 u^{\mu}a_{\mu}
\] 
Da quest'ultima relazione si evince che 4-velocità e 4-accelerazione sono perpendicolari.
\subsection[]{ Quanto valgono in MKS e in CGS le costanti: c, $\epsilon_0$,  $\mu_0$, $e^2/4\pi$,  $\hbar$?} 
In MKS si ha:
\[
	c = 3 \cdot 10^{8} \quad \text{m/s}
\] 
\[
	\epsilon_0 = 8.854 \cdot 10^{-12} \quad \text{F/m}
\]
\[
	\mu_0 = 4\pi \cdot 10^{-7} \quad \text{H/m}
\] 
\[
	\frac{e^{2}}{4\pi} = 2.04 \cdot 10^{-39} \quad \text{C}^{2}
\] 
\[
	\hbar = 1.05 \cdot 10^{-34} \quad \text{J} \cdot \text{s}
\] 
In CGS invece:
\[
	c = 3 \cdot 10^{10} \quad \text{cm/s}
\] 
\[
	\epsilon_0 = \frac{1}{4\pi}
\]
\[
	\mu_0 = \frac{4\pi}{c^{2}} 
\] 
\[
	\frac{e^{2}}{4\pi} = 1.83 \cdot 10^{-20} \quad esu^{2} \quad \text{(con 1 esu = 1 Am/c = $10^{-8}$ cm/c)} 
\] 
\[
	\hbar = 1.05 \cdot 10^{-27} \quad \text{erg} \cdot \text{s}
\]

\subsection[]{Quanto vale entro il 5\% la costante $\hbar c$ in eV-nm e in MeV-fm?}
La costante $\hbar c$ vale:
\[
	\hbar c = 197 \text{ eV/nm} = 197 \text{ MeV/fm}
\]  

\subsection[]{Spiegare la differenza tra le seguenti categorie di fotoni: infrarossi - visibili - ultravioletti - raggi X - raggi $\gamma$. } 
La differenza tra le categorie sta nella energia (o equivalentemente nella frequenza):
\begin{table}[H]
	\centering
	\label{tab: fotoni}
	\begin{tabular}{ccc}
		Fotoni & Frequenza [Hz] & Energia [eV] \\
		\hline
		infrarosso & $5 \cdot 10^{11} - 4 \cdot 10^{14}$ & $2 \cdot 10^{-3} \sim 1.5$  \\
		visibile & $4 \cdot 10^{14} - 8 \cdot 10^{14}$ & $1.5 \sim  3$ \\
		ultravioletto & $8 \cdot 10^{14} - 3 \cdot 10^{17}$ & $3 \sim 10^3$ \\
		raggi X & $3 \cdot 10^{17} - 5 \cdot 10^{19}$ & $ 10^3 \sim 2 \cdot 10^{5} $ \\
		raggi $\gamma$ & $\ge 5 \cdot 10^{19}$ & $\ge 10^5$
	\end{tabular}
\end{table}

\subsection[]{ Quanto vale la massa del fotone?}
Il fotone ha massa nulla.
\subsection[]{ Quanto valgono, entro il 5\%, la carica elettrica dell’elettrone e del protone (in MKSA)?}
La carica dell'elettrone vale 
\[
	e = -1.602 \cdot 10^{-19} \quad \text{C}
\]
 mentre quella del protone vale l'opposto. 

\subsection[]{ Quanto vale, entro il 5\%, la costante di struttura fine ($\alpha$)?}
\[
	\alpha = \frac{e^{2}}{4\pi \epsilon_0 \hbar c} = 7.29 \cdot 10^{-3} \approx \frac{1}{137}
\] 	
\subsection[]{ Quanto valgono, entro il 10\%, la massa dell’elettrone e del protone (in MKS e in MeV/$c^2$)?}
\[
m_{e} = 0.511 \text{ MeV/$c^{2}$} = 9.11 \cdot 10^{-31} \text{ kg}
\] 
\[
m_{p} = 938 \text{ MeV/$c^{2}$} = 1.67 \cdot 10^{-27} \text{ kg}
\] 
\subsection[]{Dire se la differenza fra la massa del neutrone e la somma della massa del protone e dell’elettrone sia: 1 MeV; 10 MeV; 100 MeV oppure negativa.}
\[
	m_n - \left( m_p + m_e \right) \approx 1 \text{ Mev} 
\] 
\subsection[]{Quanto è l’ordine di grandezza dell’energia media di legame di un elettrone all’interno di un atomo?}
L'energia di legame di un elettrone all'interno di un atomo varia tra 1 e 100 eV, tale energia è tendenzialmente più vicina ad 1 eV.
\subsection[]{Spiegare la differenza fra ottica fisica ed ottica geometrica}
L'ottica geometrica studia i fenomeni ottici assumendo che la luce si propaghi mediante raggi rettilinei (riflessione, rifrazione).\\ 
L'ottica fisica è la branca dell'ottica che studia i fenomeni in cue emerge la natura ondulatoria della luce (interferenza, diffrazione).\\
Nel limite in cui le dimensioni lineari deglio oggetti studiati siano molto maggiori della lunghezza d'onda della luce incidente l'ottica fisica è approssimata sempre meglio dall'ottica geometrica .
\subsection[]{ Esprimere tutte le relazioni fra campo elettrico, magnetico direzione di propagazione di un’onda e.m. piana.}
I campi ed il vettore d'onda formano una terna ortogonale; in particolare si ha (in CGS):
\[
	\boldsymbol{B} = \hat{k} \wedge \boldsymbol{E} \quad \quad 
	\boldsymbol{E} = \boldsymbol{B} \wedge \overline{k} 
\] 
In particolare i cambi sono trasversali, ovvero:
\[
	\hat{k} \cdot \boldsymbol{E} = 0 \quad \quad 
	\hat{k} \cdot \boldsymbol{B} = 0
\] 

\subsection[]{ Dare la definizione di onda piana elettromagnetica monocromatica e delle seguenti quantita’: ampiezza, frequenza angolare, vettore d’onda, frequenza, periodo, lunghezza
d’onda. Scrivere le relazioni esistenti fra le grandezze sopra definite.}
Un onda piana monocromatica è un onda le cui componenti dei campi sono della forma 
\[
	f\left( \boldsymbol{r}, t \right) = f_{0} e^{i \boldsymbol{k} \cdot \boldsymbol{r} - i \omega t }
\] 
L'ampiezza è il modulo dei campi, $\omega$ è la frequenza angolare, $\boldsymbol{k}$ è il vettore d'onda, $f = \omega/2\pi$ è la frequenza, $T = 1/f$ è il periodo, $\lambda = 2\pi/\boldsymbol{|k|}$ è la lunghezza d'onda. Nel vuoto si ha la relazione $\omega = k c$.

\subsection[]{ Definire la relazione di dispersione, la velocità di fase e la velocità di gruppo per un’onda e.m. e spiegarne il loro significato fisico. }
Dalle equazioni di Maxwell in MKS
\[
	\nabla \boldsymbol{E} = \frac{\rho}{\epsilon_0} \quad \quad \quad \quad 
	\nabla \boldsymbol{B} = 0 
\] 
\[
	\nabla \times \boldsymbol{E} = - \frac{\partial \boldsymbol{B} }{\partial t} \quad \quad \quad
	\nabla \times \boldsymbol{B} = \mu_0 \boldsymbol{J} + \mu_0 \epsilon_0 \frac{\partial \boldsymbol{E} }{\partial t} 
\] 
si ricava che, per campi monocromatici:	
\[
	\nabla^{2}\boldsymbol{E} - \frac{\epsilon\left( \omega \right) \mu \left( \omega \right) \omega^{2}}{c^{2}}\boldsymbol{E} = 0 
\]
\[
	\nabla^{2}\boldsymbol{B} - \frac{\epsilon\left( \omega \right) \mu \left( \omega \right) \omega^{2}}{c^{2}}\boldsymbol{B} = 0
\] 
Quindi esplicitando anche il laplaciano delle equazioni si trova la relazione funzionale che lega $\omega$ a k: la relazione di dispersione
\[
	c^{2}k^{2} = \epsilon \left( \omega \right) \mu \left( \omega \right) \omega^{2}
\] 
La velocità di gruppo e di fase possono allora essere definite come
\[
v_{f} = \frac{\omega}{k} \quad \quad 
v_{g} = \frac{\partial \omega}{\partial k} 
\]
La prima rappresenta la velocità con cui si propaga la fase dell'onda mentre la seconda quella con cui si propaga l'inviluppo del pacchetto. 
\subsection[]{ Definire la polarizzazione di un’onda e.m.}
La polarizzazione di un'onda EM è la direzione in cui oscilla il campo elettrico. Questa può essere lineare (se la direzione di oscillazione non varia nel tempo) , circolare o ellittica.
\subsection[]{ In un sistema Oxyz scrivere l‘espressione del campo elettrico e del campo magnetico di un’onda e.m. piana monocromatica, polarizzata linearmente lungo y e che si propaga lungo x, sia utilizzando il formalismo reale, sia utilizzando il formalismo complesso complesso.}
In CGS:
\[
	\boldsymbol{E} = E_{0} \hat{y} e^{ikx - i\omega t} = E_{0} \hat{y} \cos\left( kx - \omega t \right)   
\] 
\[
	\boldsymbol{B} = E_{0} \hat{z} e^{ikx - i\omega t} = E_{0} \hat{z} \cos\left( kx - \omega t \right)  
\] 
\subsection[]{ Enunciare e spiegare il principio di Huygens.}
Ogni elemento di un fronte d'onda si può considerare come sorgente secondaria di onde sferiche in fase con l'onda primaria e di ampiezza proporzionale all'ampiezza dell'onda primaria e all'area dell'elemento di fronte d'onda. \\
La distribuzione angolare di ampiezza è data dal fattore di obliquità:
\[
	f\left( \theta \right) = \frac{1 + \cos\left( \theta \right) }{2} 
\] 
\subsection[]{ Definire e calcolare l'impedenza del vuoto, e chiarire il suo significato fisico. }
Consideriamo un'onda e.m. monocromatica polarizzata linearmente che propaga (nel vuoto) nella direzione $\hat{z}$, i campi saranno in MKS:
\[
	\boldsymbol{E} = E_{0} \hat{x} \cos\left( \omega t - kz \right) \quad \quad
	\boldsymbol{B} = \frac{E_{0}}{c} \hat{y} \cos\left( \omega t - kz \right) 
\] 
Da cui il vettore di Poynting:
\[
	\boldsymbol{S} = \frac{E_{0}^2}{c \mu_{0}}\cos^2\left( \omega t - kz \right) = \sqrt{\frac{\mu_0}{\epsilon_0}}E_0^2 \cos^2\left( \omega t - kz \right)   
\] 
Nell'ultima espressione il termine sotto radice ha le dimensioni di $\sqrt{\frac{H}{F}} = \Omega$: è una resistenza.  
\[
	Z_{0} = \sqrt{\frac{\mu_{0}}{\epsilon_0}} \approx 120 \pi \text{ }\Omega
\]
Questa è chiamata impedenza del vuoto e rappresenta la resistività $\rho$ superficiale di un materiale cheassorbe senza riflettere le onde e.m. piane (tipicamente nella regione di microonde: $\sim 3 \text{GHz} < f < \sim 300 \text{GHz}$) 

\subsection[]{ Definire - in CGS e in MKSA - per un sistema di cariche e correnti elettriche: momento di dipolo elettrico; momento di quadrupolo elettrico;  momento di dipolo magnetico.}
\[
	\boldsymbol{p} = \int \rho \left( \boldsymbol{r} \right) d^{3}r     
\] 
\[
	\Q = \int \rho \left( r \right) \left( 3 \boldsymbol{r} \otimes \boldsymbol{r} -  r^2 \I  \right) d^3 r 
\] 
\[
	\boldsymbol{m} = \frac{1}{2[c]}\int \boldsymbol{r} \wedge \boldsymbol{J}\left(\boldsymbol{r}\right) d^3r
\] 
Dove $\rho$ e  $\boldsymbol{J}$ sono le densità di carica e di corrente, le quantità tra parentesi [ ] sono quelle da aggiungere per il sistema CGS.

\subsection[]{ Calcolare, a partire dalle EDM, la velocità delle onde elettromagnetiche in un mezzo omogeneo, lineare ed isotropo.}
In un mezzo così descritto vale la relazione:
\[
	\nabla^{2}\boldsymbol{E} - \frac{\epsilon\left( \omega \right) \mu \left( \omega \right) \omega^{2}}{c^{2}}\boldsymbol{E} = 0 
\] 
Quindi la velocità di fase cercata è (considerando la relazione di dispersione trovata sopra):
\[
v_{f} = \frac{\omega}{k} =  \frac{c}{\sqrt{\epsilon \mu} }
\] 
\subsection[]{ Esprimere la densita’ di energia di un’onda e.m. piana in funzione dei campi elettrico e/o magnetico.}
In unità MKS:
\[
	u = \frac{1}{2} \left( \boldsymbol{E}\boldsymbol{D} + \boldsymbol{B}\boldsymbol{H} \right) = \epsilon |\boldsymbol{E}|^2    
\] 
In CGS invece:
\[
	u = \frac{\boldsymbol{E}\boldsymbol{D} + \boldsymbol{B}\boldsymbol{H}}{8\pi} = \frac{\epsilon}{4\pi} |\boldsymbol{E}|^2 
\]
\subsection[]{ Dare la definizione ed esprimere il vettore di Poynting di un’onda e.m. piana in funzione del campo elettrico e/o magnetico.}
\[
	\boldsymbol{S} = \frac{c}{[4\pi]}\boldsymbol{E} \wedge \boldsymbol{H} = \frac{c}{[4\pi]}Z_{0} E_0^2 \hat{k}     
\] 
Le quantità in [ ] indicano i pezzi da aggiungere in CGS.

\subsection[]{ Esprimere la pressione (di radiazione) che un campo e.m. esercita su una superficie piana.}
Supponendo la superficie perfettamente riflettente si ottiene:
\[
	p = \frac{2I}{c} 
.\] 
con I = $\left<S\right>$ è l'intensità dell'onda. 
\subsection[]{ Dare la definizione di interferenza e diffrazione; di interferenza costruttiva e distruttiva.}
L'interferenza è un fenomeno in cui le intensità di due onde coerenti non si sommano linearmente. La diffrazione è un fenomeno in cui un fascio di radiazione si allarga (emette onde sferiche) dopo aver superato una fenditura o un ostacolo.

