\subsection[ Quadricorrente e quadripotenziale del campo E.M.]{Dare la definizione di quadri-corrente e di quadri-potenziale del campo elettromagnetico.}
\label{sec:3.a.1}
\[
	j^{\mu}=\left( c \rho, \boldsymbol{j} \right) \quad \quad 
	A^{\mu}=\left( \varphi, \boldsymbol{A}\right) 
.\] 

\subsection[ Definizione del tensore dei campi E.M.]{Dare la definizione del tensore del campo elettromagnetico e scriverne le componenti.}
\label{seq:3.a.2}
Il tensore discusso è il tensore antisimmetrico di rango 2
\[
F^{\mu \nu} = \partial^{\mu}A^{\nu} - \partial^{\nu}A^{\mu}  
.\] 
avente componenti:
\[
F^{\mu \nu}=
\left(
\begin{array}{c|ccc}
	0 & -Ex & -E_y & -E_z \\
	\hline
	Ex & 0 & -B_z & B_y \\
	E_y & B_z & 0 & -B_x \\
	E_z  & -B_y & B_x & 0 \\
\end{array}
\right)
.\] 

\subsection[ Densità di energia del campo elettromagnetico, vettore di Poynting, tensore degli sforzi]{Dare la definizione della "densità di energia" del campo elettromagnetico, del "vettore di Poynting" e del "tensore degli sforzi di Maxwell"}
Trattiamo qua di leggi di conservazione provenienti da equazioni di continuità:
\label{sec:3.a.3}
\[
	\frac{\partial}{\partial t} \left(\text{dentità di energia}\right) + \nabla \cdot \left( \text{vettore} \right) = \left( \text{densità di potenza} \right) 
.\] 
Si può dimostrare (vedi Fisica II, Teorema di Poynting) che la densità di energia nell'equazione nel caso di campo elettromagnetico è:
\[
\rho_{E}=\epsilon_0 \frac{E^2}{2}+\epsilon_0c^2 \frac{B^2}{2} 
.\] 
Mentre il vettore, detto Vettore di Poynting:
\[
\boldsymbol{S}= \frac{\boldsymbol{E}\times \boldsymbol{B}}{\mu_0}
.\] 
Il tensore di Maxwell deriva invece da un'altra legge di conservazione: quella dell'impulso del campo elettromagnetico. Tale tensore è così definito:
\[
	T_{i,j}=\left( \epsilon_0 \frac{E^2+c^2B^2}{2}\delta_{ij}-\epsilon_0\left( E_iE_j+c^2B_iB_j \right)  \right) 
.\] 
\subsection[ Equazioni di Maxwell in forma covariante]{Scrivere le equazioni di Maxwell (sia quelle non omogenee che quelle omogenee) in forma covariante.}
\label{sec:3.a.4}
\[
	\partial_{\mu}F^{\mu \nu} = \frac{j^{\mu}}{\epsilon_0c} \quad \quad \text{Disomogenee}
.\]  
\[
	\partial_{\mu} \tilde{F}^{\mu \nu}= \partial_{\mu}\frac{1}{2} \epsilon^{\mu\nu\alpha\beta}F_{\alpha\beta} = 0 \quad \quad \text{Omogenee}
.\]  
Dove è stato definito il tensore duale, per chi ancora si chiede come cavolo sia fatto si fa una piccola digressione:
\begin{align*}
	\tilde{F}^{0i}&= \frac{1}{2}\epsilon^{0i\alpha\beta}F_{\alpha\beta}=\frac{1}{2}\epsilon^{0ijk}F_{jk}=\frac{1}{2}\epsilon^{ijk}F_{jk}=
	-\frac{1}{2}\epsilon^{ijk}\epsilon_{jkl}B^{L}= -B^{i}\\
	\tilde{F}^{ij}&=\frac{1}{2}\epsilon^{ijkl}F_{kl}=\frac{1}{2}\epsilon^{ij0k}F_{0k}+ \frac{1}{2}\epsilon^{ijk 0}F_{k 0}=
	-\epsilon^{ijk}F_{0k}= \epsilon^{ijk}E_{k}
.\end{align*}
Quindi questo tensore è lo stesso di quello dei campi se si manda $\ce{ E -> B}$ e $\ce{ B -> - E}$.

\subsection[ Equazione di continuità per la 4-corrente]{Scrivere l'equazione di continuità per la quadri-corrente in forma covariante (e verificarne la consistenza con le equazioni di Maxwell)}
\label{sec:3.a.5}
Prendiamo le equazioni di Maxwell disomogenee (in forma covariante) e sfruttiamo l'antisimmetricità del tensore elettromagnetico:
\[
	\partial_{\mu}\partial_{\nu}F^{\mu\nu}=0
.\] 
facendo la stesa operazione anche al di là dell'uguale si trova quindi l'equazione richiesta
\[
	\partial_{\mu}j^{\mu}= \frac{\partial \rho}{\partial t} + \nabla\cdot \boldsymbol{j}=0
.\] 
\subsection[ Gauge di Lorentz e di Coulomb]{Dare la definizione di "gauge di Lorenz" e di "gauge di Coulomb".}
\label{sec:3.a.6}
Nella Gauge di Lorenz è richiesto:
\[
	\partial_{\mu}A^{\mu}=0
.\] 
Mentre nella Gauge di Columb:
\[
	\nabla\cdot \boldsymbol{A}=0
.\] 
Le due condizioni si equivalgono nel caso elettrostatico.

\subsection[ Trasformazioni di Lorentz dei campi]{Scrivere la legge di trasformazione di Lorentz del campo elettrico e del campo magnetico (distinguendo fra componenti parallele e componenti perpendicolari al "boost").}
\label{sec:3.a.7}
\begin{align*}
	&\boldsymbol{E}'_{\|} = \boldsymbol{E}_{\|}\\
	&\boldsymbol{E}'_{\bot}=\gamma\left( \boldsymbol{E}_{\bot}+ \boldsymbol{\beta} \wedge \boldsymbol{B} \right)\\
	&\boldsymbol{B}'_{\|}=\boldsymbol{B}_{\|}\\
	&\boldsymbol{B}'_{\bot}=\gamma\left( \boldsymbol{B}_{\bot}-\boldsymbol{\beta}\wedge \boldsymbol{E}  \right) 
.\end{align*}

\subsection[ Quadrivettore densità di forza]{Dare la definizione del quadri-vettore "densità di forza di Lorentz" .}
\label{sec:3.a.8}
\[
	f^{\mu}=\frac{\mbox{d} p^{\mu}}{\mbox{d} t \text{d}^3r} 
.\] 
Il fatto che questo sia un quadrivettore deriva dal fatto che il denominatore è un invariante relativistico, si ha inoltre che:
\[
	f^{\mu}=\frac{1}{c}F^{\mu\nu}j_{\nu}=\left(\frac{\boldsymbol{j}\cdot\boldsymbol{E}}{c},\rho\boldsymbol{E}+\frac{1}{c}\boldsymbol{j}\wedge\boldsymbol{B}\right)
.\] 

\subsection[ Effetto doppler relativistico]{Ricavare le espressioni dell’effetto Doppler relativistico (calcolo della frequenza e dell’angolo misurati dal rivelatore nel caso di moto relativo fra sorgente e rivelatore stesso).}
\label{sec:3.a.9}
Si sa che esiste il 4-vettore:
\[
	k^{\mu}=\left( \frac{\omega}{c}, \boldsymbol{k} \right) 
.\]
Consideriamo un moto bidimensionale e mettiamoci nel sistema di una sorgente di onde elettromagnetiche che si muove (la sorgente) a velocità $\boldsymbol{v} =c\beta \hat{x}$, in tale sistema il quadrivettore $k^{\mu}$ lo definiamo come:
\[
	k^{\mu}=\frac{\omega}{c}\left( 1, \ \cos\theta \cdot \hat{x}, \ \sin\theta\cdot  \hat{y} \right) 
.\] 
Invece nel sistema del rivelatore questo quadrivettore è:
\[
	k^{\mu}_{R} = \frac{\omega}{c}\left( 1, \ \cos\theta_{R}\cdot \hat{x}, \ \sin\theta_{R} \cdot \hat{y} \right) 
.\] 
Legati dalla trasformazione di Lorentz tra i due sistemi:
\begin{align*}
	\omega_{R}&=\gamma\omega\left( 1 +\beta\cos\theta \right)\\ 
	\tan\theta_{R}&=\frac{\sin\theta}{\gamma\left( \cos\theta+\beta \right) }
.\end{align*}

\subsection[\hspace{1mm} Potenziali ritardati]{Scrivere l’espressione per i potenziali ritardati ( $\phi$ ed $A$ ) per una qualunque distribuzione di cariche ( $\rho$ ) e correnti ( $j$ ).}
\label{sec:3.b.10}
Potenziale scalare:
\[
	\varphi\left( \boldsymbol{r},t \right) = \int 
	\frac{\rho\left( \boldsymbol{r}', t-\left| \boldsymbol{r}-\boldsymbol{r}' \right| /c  \right) }{\left| \boldsymbol{r}-\boldsymbol{r}' \right| } d^3r'
.\] 
Potenziale vettore:
\[
	\boldsymbol{A}\left( \boldsymbol{r},t \right) =\frac{1}{c}\int 
	\frac{\boldsymbol{j}\left( \boldsymbol{r}', t-\left| \boldsymbol{r}-\boldsymbol{r}' \right| /c  \right) }{\left| \boldsymbol{r}-\boldsymbol{r}' \right| }d^3r'
.\] 

\subsection[\hspace{1mm} Campo elettrico generato da una carica puntiforme in moto arbitrario]{Spiegare tutti i termini dell’espressione 
\label{sec:3.a.11}
	\[
	\boldsymbol{E} = \left[ \frac{q}{R^2}\frac{\hat{n}- \boldsymbol{\beta}}{\gamma^2 \left( 1 - \hat{n} \cdot \boldsymbol{\beta} \right)^3 } + \frac{q}{Rc} \frac{\hat{n} \wedge \left[ \left( \hat{n} -\boldsymbol{\beta}  \right) \wedge \dot{\boldsymbol{\beta}} \right] }{\left( 1 - \hat{n} \boldsymbol{\beta}\right)^3 } \right]_{t' = t - R/c}  
\] 
per il campo elettrico generato da una carica puntiforme in moto arbitrario.}
Sia $\boldsymbol{s}\left( t \right) $ la legge oraria della carica q e sia $\boldsymbol{r}$ la posizione dell'osservatore, allora possiamo definire le quantità dell'espressione:
\[
	\boldsymbol{R}=\boldsymbol{r}-\boldsymbol{s}\left( t' \right) \quad \quad \hat{n}=\boldsymbol{R}/R
.\] 
Il primo termine della equazione nella richiesta è il campo generato da una carica che si muove in moto rettilineo uniforme e non è un termine radiativo (va giu come $R^{-2}$), il secondo termine è invece radiativo e dipende dalla accelerazione della carica.
\subsection[\hspace{1mm} Solido di radiazion e diagramma di radiazione]{Dare la definizione di “solido di radiazione” e di “diagramma di radiazione” per una carica accelerata.}
\label{sec:3.a.12}
\paragraph{Solido di radiazione}
\label{par:Solido di radiazione.}
Preso un oggetto che irraggia si dice solido di radiazione la figura tridimensiona costruita posizionando nell'origine la sorgente di radiazione e nelle diverse direzioni spaziali frecce di lunghezza proporzionale all'intensità del campo elettrico nella direzione indicata dalla freccia stessa.

\paragraph{Diagramma di radiazione.}%
\label{par:Diagramma di radiazione.}
Sezione planare del solido di radiazione in direzioni opportune a comprenderne la forma. 

\subsection[\hspace{1mm} Campo magnetico in funzione del campo elettrico per carica puntiforme]{Quanto vale il campo magnetico generato da una carica puntiforme in moto arbitrario se è noto il campo elettrico?}
\label{sec:3.a.13}
Vale in generale che (in CGS):
\[
	\bs{B}= \hat{n} \wedge \bs{E} 
.\]
in MKSA basta dividere per c.
\subsection[\hspace{1mm} Distribuzione angolare di energia irraggiata per unità di frequenza]{Spiegare tutti i termi della espressione 
\[
	\frac{\mbox{d} I_{\omega}}{\mbox{d} \Omega} = \frac{q^2}{4 \pi^2 c} \left| \int{ \frac{\hat{n} \wedge \left[ \left( \hat{n}-\boldsymbol{\beta}\right) \wedge \dot{\boldsymbol{\beta}} \right]}{\left( 1- \hat{n} \boldsymbol{\beta}\right)^2} e^{i\omega \left( t' - \frac{\boldsymbol{r'} \cdot \hat{n}}{c} \right) }  } dt' \right|^2
\] 
}

\label{sec:3.a.14}
Parliamo della famigerata Radiazione di sincrotrone.
Quella scritta sopra è la distribuzione angolare dell'energia irraggiata per unità di frequenza $I_{\omega}$ di una carica $q$ in moto con velocità  $\boldsymbol{\beta}$. Inoltre $\hat{n}$ è la direzione di osservazione e il sistema usato è il CGS. 
Anche se non richiesto possiamo dimostrarla a partire dalla formula del campo di radiazione della \hyperref[sec:3.a.11]{Domanda 3.a.11}.:
\[
	\boldsymbol{E}= \left. \frac{q}{cr}
	\frac{\hat{n}\wedge\left[\left(\hat{n}-\boldsymbol{\beta}\right)\wedge\dot{\boldsymbol{\beta}}\right]}
	{\left(1-\hat{n}\cdot \boldsymbol{\beta}\right)^3} \right|_{t=t_{\text{rit}}}
\]
ricordiamo che $\boldsymbol{B}=\hat{n}\wedge \boldsymbol{E}$, e che come conseguenza la densità di energia irraggiatà $\varepsilon$ è proporzionale a $\left| \boldsymbol{E} \right| ^2$, quindi l'energia irraggiata per unità di angolo solido si esprime come:
\begin{align*}
	\frac{\mbox{d} \varepsilon}{\mbox{d} \Omega} =& \int_{-\infty}^{\infty}r^2 \left[ \boldsymbol{S}\cdot\hat{n}\right] dt =\\
	=& \int r^2 \frac{c}{4\pi}\left| \boldsymbol{E} \right|^2 dt = \\
	=& \frac{q^2}{4\pi c} \int \left[ \left. \frac{ \hat{n} \wedge \left( \left( \hat{n}-\boldsymbol{\beta} \right)\wedge\dot{\boldsymbol{\beta}}\right)}
		{\left( 1- \hat{n}\cdot \boldsymbol{\beta} \right)^3 } \right|_{t = t_{\text{rit}}} \right]^2 dt
.\end{align*}
È utile adesso avvalersi dell'itentità di parseval sulla trasformata di Fourier:
\[
	\int_{-\infty}^{\infty} \left| x\left( t \right)  \right|^2 dt = 2\pi \int_{-\infty}^{\infty} \left| \hat{x}\left( \omega \right)\right|^2d\omega 
.\] 
Portiamo allora l'integranda nel dominio delle frequenze (si moltiplica per $2\pi$ per ottenere l'dentità sopra, quindi nella seguente dobbiamo dividere per $2\pi$):
\[
	\frac{\mbox{d} \varepsilon}{\mbox{d} \Omega} = \frac{1}{2\pi}  \frac{q^2}{4\pi c} 2 \cdot\int_0^{\infty} \left| \hat{f}\left( \omega \right) \right|^2 d \omega 
.\] 
con $\hat{f}\left( \omega \right) $ trasformata dell'integranda nell'equazione sopra, l'integrale va solo sulle frequenze positive perchè la funzione che trasformiamo è reale ($\hat{f}\left( -\omega \right) = \hat{f}^*\left( \omega \right) $), da questo viene il fattore 2.\\
Adesso la funzione integranda si avvicina molto all'oggetto che stavamo cercando (un oggetto definito per unità di frequenza), infatti:
\[
	\frac{\mbox{d} I_{\omega}}{\mbox{d} \Omega} = \frac{q^2}{4\pi^2 c} \left| \hat{f}\left( \omega \right)  \right|^2 =  \frac{q^2}{4\pi^2 c} 
	\left| \int_{-\infty}^{\infty} \left. \frac{ \hat{n} \wedge \left( \left( \hat{n}-\boldsymbol{\beta} \right)\wedge\dot{\boldsymbol{\beta}}\right)}
		{\left( 1- \hat{n}\cdot \boldsymbol{\beta} \right)^3 } \right|_{t = t_{\text{rit}}} e^{-i\omega t} dt \right|^2 
.\]
Basta adesso cambiare variabili: 
\[
	\ce{\ce{t} -> \ce{t_{\text{rit}}}} = t' =  t - \frac{R\left( t' \right) }{c} \implies t\left( t' \right) = t' + \frac{R\left( t' \right) }{c}  
.\] 
\[
	\ce{\ce{dt} -> \ce{dt_{\text{rit}}}} = dt' = d \left(t - \frac{R\left( t' \right) }{c} \right)
.\] 
con $R\left( t \right)$ distanza tra particella e punto di osservazione, è bene ricordare che per distanze "grandi" come quelle dei campi di radiazione si ha:
\[
	R\left( t' \right) \approx x - \hat{n}\cdot \boldsymbol{r}\left( t' \right)  
.\] 
Con $x$ distanza del punto di osservazione dall'origine e $\boldsymbol{r}\left( t \right)$ traiettoria della particella. In questo modo è evidente che lo Jacobiano del cambio di variabili sia:
\[
	\frac{dt}{dt'}= 1-\hat{n} \cdot \boldsymbol{\beta}
.\] 
rimettendo tutto dentro si ottiene l'espressione nella richiesta (avendo cura di notare che all'esponenziale ci sarà una fase che proviene da R, quella è irrilevante essendoci un modulo quadro.).


\subsection[\hspace{1mm} Tempo di interazione per carica in movimento con osservatore]{Una carica elettrica Q si muove con velocità costante (relativistica) di modulo V su una retta, a distanza b da tale retta si trova un osservatore che misura il campo elettrico e magnetico generato dalla carica. Quanto è l'ordine di grandezza del tempo in cui l'osservatore misura un campo elettrico che sia almeno la metà del campo elettrico massimo misurato?}
\label{sec:3.a.15}
È evidentemente necessario per rispondere alla domanda parametrizzare il campo elettrico in funzione del tempo nel riferimento dell'osservatore. Per far questo ci mettiamo da prima nel riferimento della carica (O') e poi con una trasformazione di Lorentz dei campi andiamo nel sistema dell'osservatore:
\begin{align*}
&									&&						&\boldsymbol{E}_{\|}=&\boldsymbol{E}'_{\|} \\
& \bs{E}'= e \frac{\bs{R}'}{R'^3}, \quad \quad \quad \bs{B}'= \ 0	 &\ce{->[\quad \text{Lorentz}\quad]}&		&\bs{E}_{\bot}=&\gamma\bs{E}'_{\bot}\\ 
&									&&						&\bs{B}\ =&\gamma\bs{\beta}\wedge \bs{E} 
.\end{align*}
Tuttavia come è ben noto i campi sono espressi ancora in funzione delle variabili del sistema solidale alla carica (t',x',y',z'), sarà necessario un altro boost per cambiare parametrizzazione.
\begin{align*}
	&x\left( t \right) = v\cdot t	&&					&x'=&\gamma\left( x-vt \right) \\
	&y\left( t \right) = 0		&\ce{ ->[\quad \text{Lorentz}] \quad}&	&y'=&y \\
	&z\left( t \right) = 0		&&					&z'=&z
.\end{align*}
Rimane la dipendenza del campo elettrico dalla distanza dalla carica anche dopo la trasformazione di Lorentz, tale distanza è tuttavia parametrizzata con le variabili primate: 
\begin{figure}[ht]
    \centering
    \incfig{moto-carica}
    \caption{Carica in moto e punto di osservazione dei campi}
    \label{fig:moto-carica}
\end{figure}
\begin{align*}
	R'& = \sqrt{x'^2 + y'^2 + z'^2} =\\ 
	&=\sqrt{\gamma^2\left( x-vt \right)^2 + y^2 + z^2} =\\ 
	&=\gamma \sqrt{\left( x-vt \right)^2 + \left( 1-\beta^2 \right)\left( y^2+z^2 \right)}=\gamma R^{*} 
.\end{align*}	
Inseriamola nelle componenti dei campi: 
\begin{align*}
	&E_x = e \frac{\gamma\left( x-vt \right) }{\gamma^3\left(R^{*}\right)^3} = e \frac{x-vt}{\gamma^2 \left(R^*\right)^3}\\
	&E_y = e \gamma \frac{y}{\gamma^3\left(R^*\right)^3} = e \frac{y}{\gamma^2\left(R^*\right)^3} \\ 
	&E_z = e \gamma \frac{z}{\gamma^3\left(R^*\right)^3} = e \frac{z}{\gamma^2 \left(R^*\right)^3} 
.\end{align*}
In maniera più compatta si può scrivere:
\[
	\bs{E}=e \frac{\bs{R}_i}{\gamma^2 \left( R^* \right)^3}
.\] 
Con $\bs{R}_i$ defito dalle equazioni precedenti.\\
Adesso con riferimento alla Figura \ref{fig:moto-carica} calcoliamo il campo visto dal punto $P=\left( 0,b,0 \right) $:
\begin{align*}
	&E_x = -e \frac{\gamma vt}{\left( \gamma^2v^2t^2 + b^2 \right)^{3 /2} }\\
	&E_y = e \frac{b\gamma }{\left( \gamma^2v^2t^2 + b^2 \right)^{3 /2}}\\
	&E_z = 0
.\end{align*}
Se si prova a graficare queste funzioni ci si rende conto che il contributo più significativo al campo misurato da $P$ è quello lungo $y$ (alternativamente si può dimostrare che il modulo del campo elettrico è massimo quando si annulla $E_x$ ovvero in  $t=0$), quindi per abbozzare l'ordine di grandezza richiesto senza perdersi in conti basta notare che $E_y$ ha un solo massimo: per $t=0$, quindi possiamo richiedere che il campo valga circa la metà rispetto a questo massimo. La richiesta si traduce nell'avere il denominatore che valga il doppio rispetto al massimo, ovvero:
\[
	\gamma^2v^2t^2 \sim b^2 \implies t \sim \frac{b}{\gamma v}
.\] 
\subsection[\hspace{1mm} Principio di Babinet]{Enunciare il principio di Babinet.}
\label{sec:3.a.16}
La figura di diffrazione generata da un corpo opaco è la stessa che si ottiene con lo schermo complementare: ottenuto sostituendo il corpo con una apertura.

\subsection[\hspace{1mm} Fattore di forma]{Definire il fattore di forma per un'onda che incide su su sistema.}
\label{sec:3.a.17}
Il fattore di forma per un sistema caratterizzato da una densità di carica $\rho$ è definito:
\[
	F\left(\bs{q} \right) = \frac{\int\rho\left( \bs{r} \right) e^{-i \bs{q}\cdot \bs{r}} d^3r}{\int\rho\left( \bs{r} \right) d^3r}
.\] 
Dove $\bs{q}$ rappresenta la differenza tra il vettore d'onda entrante e uscente.\\
Tale quantità può essere definita anche per sistemi discreti facendo un attento uso delle $\delta$
\subsection[\hspace{1mm} Funzionamento di un acceleratore elettrostatico]{Spiegare qualitativamente il funzionamento di un acceleratore elettrostatico.}
\label{sec:3.a.18}
In un acceleratore elettrostatico le particelle vengono accelerate da campi statici. Le particelle vengono fatte passare in un tubo a vuoto in cui è presente un campo elettrico costante.\\ 
L'energia che viene fornita alla particella è modesta: 1-10 MeV. Questa energia è limitata dal fatto che le particelle passano nel tubo una sola volta. 

\subsection[\hspace{1mm} Energie per unità di lunghezza per drift tube e cavità superconduttrici]{Quali sono, approssimativamente, le energie per unità di lunghezza che attualmente si ottengono nell’accelerazione di protoni con la tecnica dei "drift tube"? E delle cavità superconduttrici?}
\label{sec:3.a.19}
Con i drift tube si riesce a fare $ 100 \frac{kV}{m}$, mentre con le cavità superconduttrici si arriva a $ 50 \frac{MV}{m}$. 


\subsection[\hspace{1mm} Funzionamento di un acceleratore lineare]{Spiegare qualitativamente il funzionamento di un acceleratore lineare, indicando le differenze importanti fra l’accelerazione di elettroni e di protoni.}
\label{sec:3.a.20}
Un acceleratore lineare consiste in un tubo a vuoto in cui sono presenti degli elettroni in cascata tra i quali viene stabilita una d.d.p. oscillante. \\
A differenza di un acceleratore elettrostatico si può sempre aggiungere uno stadio per accelerare le particelle. È quindi possibile raggiungere energie più elevate rispetto ai primi. \\
Una differenza importante tra protoni ed elettroni è la forma che i drift tube devono avere per accelerare gli uni o gli altri essendo gli elettroni molto meno massivi dei protoni. 
Un'altra grande differenza è la potenza irraggiata dai due: condideriamo l'accelerazione di una particella di carica q e massa m, visto che la traiettoria è rettilinea si ha:
\[
	qE=m \gamma^3a	
.\]
La potenza irraggiata è quindi (CGS):
\[
	P= \frac{2q^2}{3c^3}\gamma^{6}a^{6}=\frac{2q^4E^2}{3m^2c^3}	
.\]
In particolare la potenza irraggiata da un elettrone è circa $3 \cdot 10^{6} $ volte quella irraggiata da un protone.
\subsection[\hspace{1mm} Funzionamento di acceleratore cicolare]{Spiegare qualitativamente il funzionamento di un acceleratore circolare, indicando le differenze importanti fra l’accelerazione di elettroni e di protoni.}
\label{sec:3.a.21}
Un acceleratore circolare moderno è solitamente il sincrotrone, quest'ultimo consiste nel tenere le particelle in moto circolare mediante l'ausilio di magneti e cavità superconduttrici che si alternano tra loro. I campi generati da questi due ultimi sono regolati in modo da tenere le particelle lungo la traiettoria prestabilita.
Qui la grande differenza tra l'accelerazione di protoni ed elettroni consiste nella energia persa nell'irraggiamento, per arrivarci è necessario avvalersi della Formula di Larmor relativistica (MKSA):
\[
	P=I= \frac{q^2}{6\pi \epsilon_0 } \gamma^6\left[  \frac{\left|\dot{\bs{\beta}} \right|^2}{c}-  \left| \bs{\beta}\wedge \dot{\bs{\beta}}  \right| ^2  \right] 
.\] \label{eq:Larmor}
Nel caso di un acceleratore lineare conta soltanto il primo termine, nel circolare invece pesa anche il secondo. Analizziamo proprio questo secondo caso, ipotizzando un raggio di curvatura R:
\begin{align*}
	I= &\frac{q^2}{6\pi \epsilon_0 c^3} \gamma^6 \left( a^2-a^2\beta^2  \right)=\\
	=&\frac{q^2\gamma^4}{6\pi \epsilon_0 c^3}\left| \bs{a} \right|^2 =\\
	=& \frac{q^2\gamma^4}{6\pi \epsilon_0c^3}\left( \frac{\beta^2c^2}{R} \right) ^2=\\
	=&\frac{q^2\gamma^4\beta^4c}{6\pi \epsilon_0R^2}
.\end{align*}
Se la particella trascorre un tempo $T_B$ sotto l'effetto curvatore del campo magnetico $B$ per ogni giro e $R_B$ è il raggio della circonferenza formata dalle sole zone in cui vi è l'effeto di questo campo si ha:
\[
	T_B = \frac{2\pi R_B}{\beta c}
.\] 
In quanto sotto l'effetto di $B$ non vi è accelerazione tangenziale. Allora l'energia persa per irraggiamento è data da:
\[
	\Delta E = I\cdot T_B
.\] 
Che con un pò di algebra e sostituzioni si può anche esprimere come:
\[
	\Delta E = \frac{4\pi}{3}\left( \frac{r_e}{R_B} \right) m_ec^2\beta^3\gamma^4  	
.\] 
La perdita di energia per irraggiamento è sostanzialmente dovuta alla velocità della particella, essendo meno massosi gli elettroni (a parità di energia dei protoni) hanno $\beta$ e $\gamma$ molto maggiori dei protoni, il loro irraggiamento in curva è tale da avere un nome per la notevole potenza emessa: luce di sincrotrone.(da rivedere perchè secondo me è un pò stupido\ldots)\\
L'energia persa per irraggiamento è quindi un fattore che limita molto le energie ottenibili in acceleratori circolari se si studiano collisioni di elettroni o positroni. Nel caso dei protoni il fattore limitante principale è l'intensità del campo magnetico che siamo in grado di produrre per far sterzare le particelle.


\subsection[\hspace{1mm} Solido di radiaziome per carica in moto circolare]{Effettuare un disegno, qualitativo, del solido di radiazione per una carica in un acceleratore lineare o circolare.}
Per l'acceleratore circolare si ha la distribuzione angolare (CGS):
\[
	\frac{\mbox{d} I}{\mbox{d} \Omega} = \frac{q^2a^2}{4\pi c^3\left( 1-\beta\cos\theta \right) ^3} \left[ 1-\frac{\sin^2\theta\sin^2\varphi}{\gamma^2\left(1-\beta\cos\theta \right) ^2} \right] 	
.\]
Che è piccata in un doppio cono stondato attorno a $\beta$:
\begin{figure}[H]
    \centering
    \incfig{distribuzione-sincrotrone}
    \caption{Distribuzione Angolare della radiazione di sincrotrone}
    \label{fig:distribuzione-sincrotrone}
\end{figure}

