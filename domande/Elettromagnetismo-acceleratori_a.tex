\subsection*{3.a.1 Dare la definizione di quadri-corrente e di quadri-potenziale del campo elettromagnetico.}

\subsection*{3.a.2 Dare la definizione del tensore del campo elettromagnetico e scriverne le componenti.}

\subsection*{3.a.3 Dare la definizione della "densità di energia" del campo elettromagnetico, del "vettore di Poynting" e del "tensore degli sforzi di Maxwell"}

\subsection*{3.a.4 Scrivere le equazioni di Maxwell (sia quelle non omogenee che quelle omogenee) in forma covariante.}

\subsection*{3.a.5 Scrivere l'equazione di continuità per la quadri-corrente in forma covariante (e verificarne la consistenza con le equazioni di Maxwell)}

\subsection*{3.a.6 Dare la definizione di "gauge di Lorenz" e di "gauge di Coulomb".}

\subsection*{3.a.7 Scrivere la legge di trasformazione di Lorentz del campo elettrico e del campo magnetico (distinguendo fra componenti parallele e componenti perpendicolari al "boost").}

\subsection*{3.a.8 Dare la definizione del quadri-vettore "densità di forza di Lorentz" .}

\subsection*{3.a.9 Ricavare le espressioni dell’effetto Doppler relativistico (calcolo della frequenza e dell’angolo misurati dal rivelatore nel caso di moto relativo fra sorgente e rivelatore stesso).}

\subsection*{3.a.10 Scrivere l’espressione per i potenziali ritardati ( $\phi$ ed $A$ ) per una qualunque distribuzione di cariche ( $\rho$ ) e correnti ( $j$ ).}
vai giu
\subsection*{3.a.11 Spiegare tutti i termini dell’espressione 
\[
	\boldsymbol{E} = \left[ \frac{q}{R^2}\frac{\hat{n}- \boldsymbol{\beta}}{\gamma^2 \left( 1 - \hat{n} \cdot \boldsymbol{\beta} \right)^3 } + \frac{q}{Rc} \frac{\hat{n} \wedge \left[ \left( \hat{n} -\boldsymbol{\beta}  \right) \wedge \dot{\boldsymbol{\beta}} \right] }{\left( 1 - \hat{n} \boldsymbol{\beta}\right)^3 } \right]_{t' = t - R/c}  
\] 
per il campo elettrico generato da una carica puntiforme in moto arbitrario.}

\subsection*{3.a.12 Dare la definizione di “solido di radiazione” e di “diagramma di radiazione” per una carica accelerata.}

\subsection*{3.a.13 Quanto vale il campo magnetico generato da una carica puntiforme in moto arbitrario se è noto il campo elettrico?}

\subsection*{3.a.14 Spiegare tutti i termi della espressione 
\[
	\frac{\mbox{d} I_{\omega}}{\mbox{d} \Omega} = \frac{q^2}{4 \pi^2 c} \left| \int{ \frac{\hat{n} \wedge \left[ \left( \hat{n}-\boldsymbol{\beta}\right) \wedge \dot{\boldsymbol{\beta}} \right]]}{\left( 1- \hat{n} \boldsymbol{\beta}\right)^2} e^{i\omega \left( t' - \frac{\boldsymbol{r'} \cdot \hat{n}}{c} \right) }  }  \right|^2
\] 
}

\subsection*{3.a.15 Una carica elettrica Q si muove con velocità costante (relativistica) di modulo V su una retta, a distanza b da tale retta si trova un osservatore che misura il campo elettrico e magnetico generato dalla carica. Quanto è l'ordine di grandezza del tempo in cui l'osservatore misura un campo elettrico che sia almeno la metà del campo elettrico massimo misurato?}

\subsection*{3.a.16 Enunciare il principio di Babinet.}

\subsection*{3.a.17 Definire il fattore di forma per un'onda che incide su su sistema.}

\subsection*{3.a.18 Spiegare qualitativamente il funzionamento di un acceleratore elettrostatico.}

\subsection*{3.a.19 Quali sono, approssimativamente, le energie per unità di lunghezza che attualmente si ottengono nell’accelerazione di protoni con la tecnica dei "drift tube"? E delle cavità superconduttrici?}
\subsection*{3.a.20 Spiegare qualitativamente il funzionamento di un acceleratore lineare, indicando le differenze importanti fra l’accelerazione di elettroni e di protoni.}

\subsection*{3.a.21 Spiegare qualitativamente il funzionamento di un acceleratore circolare, indicando le differenze importanti fra l’accelerazione di elettroni e di protoni.}

\subsection*{3.a.22 Effettuare un disegno, qualitativo, del solido di radiazione per una carica in un acceleratore lineare o circolare.}

